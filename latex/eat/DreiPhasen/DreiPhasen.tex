
%==========================================================
% Titel:    Elektrizit�tslehre, Elektrische Maschinen
% Autor:    Adrian Freihofer
% Erstellt: 7.11.2001
% Ge�ndert:
%==========================================================


\index{Dreiphasen} 
\index{Drehstrom}

\Hauptbox{
 \Bildbox{
   \input eat/DreiPhasen/DreiSin
}
 \Formelbox{
   \footnotesize Maschensatz: \normalsize
   \[\underline{U}_1 + \underline{U}_2 + \underline{U}_3 = 0\]
   \[\underline{I}_1 + \underline{I}_2 + \underline{I}_3 = 0\] 
  }}{
 \Groessenbox{
  $\underline{U}$       & Spannung (komplex)     & $[V]$ \\
  $\underline{I}$       & Strom (komplex)        & $[A]$ \\            
 }}


\section{Sternschaltung }

\index{Sternschaltung} 

\Hauptbox{
 \Bildbox{
   \input eat/DreiPhasen/stern.pstex_t
}
 \Formelbox{
   \footnotesize Strang-Sternspannungen: \normalsize \\
   $\underline{U}_{Str 1} = \underline{U}_{1} = \underline{U}_1 - \underline{U}_2$ \\
   $\underline{U}_{Str 2} = \underline{U}_{2} = \underline{V}_1 - \underline{V}_2$ \\ 
   $\underline{U}_{Str 3} = \underline{U}_{3} = \underline{W}_1 - \underline{W}_2$ \\
   \footnotesize Aussenleiterspannungen: \normalsize \\
   $ \underline{U}_{12} = \underline{U}_1 - \underline{U}_2 $ \hspace{0.1cm} 
   \scriptsize $ \angle (\underline{U}_1 , \underline{U}_2 ) = 120 ^{\circ} $\\ \normalsize
   $ \underline{U}_{23} = \underline{U}_2 - \underline{U}_3 $ \\
   $ \underline{U}_{31} = \underline{U}_3 - \underline{U}_1 $ 
   \[U = U_{Str} \sqrt{3} \]
   \[I = I_{Str} \]
  
  }}{
 \Groessenbox{
  $\underline{U}$       & Spannung                 & $[V]$ \\
  $ U_{Str} $           & Strangspan"-nung         & $[V] $ \\
  $\underline{I}$       & Strom                    & $[A]$ \\            
 }}
\vfill


\section{Dreieckschaltung }

\index{Dreieckschaltung} 

\Hauptbox{
 \Bildbox{
   \input eat/DreiPhasen/dreieck.pstex_t
}
 \Formelbox{
   \footnotesize Strang-Sternspannungen: \normalsize \\
   $U_{Str 1} = \underline{U}_{1} = \underline{U}_1 - \underline{U}_2$ \\
   $U_{Str 2} = \underline{U}_{2} = \underline{V}_1 - \underline{V}_2$ \\ 
   $U_{Str 3} = \underline{U}_{3} = \underline{W}_1 - \underline{W}_2$ \\
   \footnotesize Aussenleiterspannungen: \normalsize \\
   $ \underline{U}_{12} = \underline{U}_1 $ \hspace{0.1cm} 
   \scriptsize $ \angle (\underline{U}_1 , \underline{U}_2 ) = 120 ^{\circ} $\\ \normalsize
   $ \underline{U}_{23} = \underline{U}_2 $ \\
   $ \underline{U}_{31} = \underline{U}_3 $ 
   \[U = U_{Str} \]
   \[I = 2 I_{Str} \cos (30 ^{\circ} ) =  I_{Str} \sqrt{3}\]
  
  }}{
 \Groessenbox{
  $\underline{U}$       & Spannung                 & $[V]$ \\
  $ U_{Str} $           & Strangspan"-nung         & $[V] $ \\
  $\underline{I}$       & Strom                    & $[A]$ \\            
 }}



\subsection{Leistungen bei Stern- und Dreieckschaltung }

\index{Leistung bei Sternschaltung} 

\Hauptbox{
 \Bildbox{
   \input eat/DreiPhasen/leist1.pstex_t
}
 \Formelbox{
   \[S_{Str} = U_{Str} I_{Str} \]
   \[S = 3 S_{Str}  = \sqrt{3} U I \]
   \[P = S \cos(\varphi) = \sqrt{3} U I \cos (\varphi ) \]
   \[Q = S \sin (\varphi) = \sqrt{3} U I \sin (\varphi) \]
   \[W = P t = \sqrt{3} U I \cos (\varphi ) t \]
   \[W_b = Q t = \sqrt{3} U I \sin (\varphi ) t \]
  }}{
 \Groessenbox{
  $U$       & Spannung                  & $[V]$ \\
  $I$       & Strom                     & $[A]$ \\  
  $S$       & Scheinleitung             & $[VA]$ \\
  $P $      & Wirkleistung              & $[W]$ \\  
  $Q$       & Blindleistung             & $[Var]$ \\    
  $W$       & Wirkarbeit                & $[Ws]$ \\ 
  $W_b$     & Blindarbeit               & $[Vars]$ \\    
  $t$       & Zeit                      & $[s]$ \\           
 }}

\vfill



%%% Local Variables:
%%% mode: latex
%%% TeX-master: "../../../FoSaHSR"
%%% End: