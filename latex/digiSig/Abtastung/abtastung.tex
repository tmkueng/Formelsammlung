%==========================================================
% Titel:    Digitale Signalverarbeitung Abtasten der Signale
% Autor:    Adrian Freihofer
% Erstellt: 11.1.2003
% Ge�ndert:
%==========================================================


\section{Ideale Abtastung} 

\index{Abtastung!Ideale} 

\Hauptbox{
 \Bildbox{
	\begin{footnotesize}Zeitbereich:\end{footnotesize}\\
	\input digiSig/Abtastung/idealAbt.pstex_t 
}
 \Formelbox{
	\[ s_a(t) = s(t) T \delta_p(t) \] 
	\[ s_a(t) = T s(t) \sum_{m=-\infty}^\infty \delta (t-mT) \]
	\[ s_a(t) = T \sum_{m=-\infty}^\infty s(mT) \delta(t-mT) \]
  }
\Bildbox{
	\begin{footnotesize}Frequenzbereich:\end{footnotesize}\\
	\input digiSig/Abtastung/idealAbtSpec.pstex_t 
}
 \Formelbox{
	\begin{small}\[ S_a(\omega) = S(\omega) \ast \sum_{k=-\infty}^\infty \delta (\omega - k \omega_c) \]\end{small}
	\[ S_a(\omega) = \sum_{k=-\infty}^\infty S (\omega - k \omega_c) \]
	\[\omega_c = \frac{2 \pi}{T} \]
  }
}{
 \Groessenbox{
	$s$       & Signal                 & $[...]$ \\
	$s_a$     & s abgetastet           & $[...]$ \\
	$S$       & Spektrum von s         & $[...]$ \\
	$t$       & Zeit                   & $[s]$ \\
	$T$       & Periode                & $[s]$ \\
	$m$       & m-te Periode           & $[1]$ \\
	$\omega$  & Kreisfrequenz          & $[\frac{1}{s}]$ \\
	$\omega_c$& Abtastfrequenz         & $[\frac{1}{s}]$ \\
 }}

\section{Flat Top Sampling}

\index{Abtastung!Flat Top Sampling} 
\index{Abtastung!Rechteckpuls} 
\index{Abtastung!Real} 

\Hauptbox{
\Bildbox{
	\input digiSig/Abtastung/flatTop.pstex_t 
	\input digiSig/Abtastung/flatTopPlot.pstex_t \\ 
	\begin{footnotesize}Signal wird verzerrt durch $ G_\tau(\omega)$\end{footnotesize}
}
 \Formelbox{\label{flatTop}
	\[ s_a(t) = \sum_{m=-\infty}^\infty s(mT) r_\tau(t-mT) \]
	\[ S_a(\omega) = G_\tau(\omega) \sum_{k=-\infty}^\infty S(\omega - k \omega_c) \]
	\[ G_\tau(\omega) = \frac{1}{T} R_\tau(\omega) = \frac{\tau}{T} \frac{\sin \left(\frac{\tau}{2} \omega \right)}{\frac{\tau}{2} \omega} \]
	\begin{footnotesize}Je k�rzer die Abtast-Pulse desto breiter die $ \frac{\sin(x)}{x} $ Kurve.\end{footnotesize}
  }
}{
 \Groessenbox{
	$s$       & Signal                 & $[...]$ \\
	$s_a$     & s abgetastet           & $[...]$ \\
	$S$       & Spektrum von s         & $[...]$ \\
	$t$       & Zeit                   & $[s]$ \\
	$T$       & Periode                & $[s]$ \\
	$\tau$    & Rechteckbreite         & $[s]$ \\
	$m$       & m-te Periode           & $[1]$ \\
	$\omega$  & Kreisfrequenz          & $[\frac{1}{s}]$ \\
 }}
\vfill


\section{Sample and Hold}

\index{Sample and Hold} 
\index{Abtastung!Sample and Hold} 
\index{Abtastung!Rechteckpuls} 
\index{Abtastung!Real} 

\Hauptbox{
\Bildbox{
	\input digiSig/Abtastung/sampleHold.pstex_t 
}
 \Formelbox{
	Entspricht Flat Top Sampling (S. \pageref{flatTop}) bei $\tau = T$ \\
	Die $ \frac{\sin(x)}{x} $ Kurve hat die Nulldurg�nge bei $ k \frac{2 \pi}{T}, k = \{ 1,2,...\} $
  }
}{
 \Groessenbox{
	$T$       & Periode                & $[s]$ \\
	$\tau$    & Rechteckbreite         & $[s]$ \\
 }}


\section{Abtasttheorem}

\index{Abtasttheorem} 
\index{Abtastung!Abtasttheorem} 

\Hauptbox{
\Bildbox{
	\begin{footnotesize}Problem:\end{footnotesize}\\
	\input digiSig/Abtastung/abtastTheo.pstex_t \\
	\begin{footnotesize}$\Rightarrow $ Rekonstruktion ist nicht m�glich.\end{footnotesize}
}
 \Formelbox{
	\[ \omega_c > 2 \omega_{max} \]
	\begin{footnotesize}$\Rightarrow $ Praktisch muss immer ein analoger Tiefpass vorgeschaltet werden.\end{footnotesize}
  }
}{
 \Groessenbox{
	$\omega_c$& Abtastfrequenz         & $[\frac{1}{s}]$ \\
	$\omega_{max}$& max Frequenz in $s(t)$   & $[\frac{1}{s}]$ \\
 }}


\section{Rekonstruktion}

\index{Abtastung!Rekonstruktion} 

\Hauptbox{
\Bildbox{
	\begin{small}Ist das Abtasttheorem erf�llt, so ist das urspr�ngliche Signal exakt reproduzierbar.\end{small}\\ \\
	\input digiSig/Abtastung/rekonst.pstex_t \\ \\
}
 \Formelbox{
	\[ s_r(t) = T \sum_{m=-\infty}^{\infty} s(mT) h_r(t-mT) \]
	\begin{footnotesize}\[ s_r(t) = T \sum_{m=-\infty}^{\infty} s(mT) \delta(t-mT) \ast h_r(t) \]\end{footnotesize}
	\[ h_r (t) = \frac{\omega_c}{2 \pi} \frac{\sin\left(\frac{\omega_c}{2} t \right)}{\frac{\omega_c}{2}t}\]
  }
}{
 \Groessenbox{
	$s_r$      & Signal rekonst.        & $[...]$ \\
	$h_r$      & Stossantw. Rekonstruktions-Tiefpass  & $[...]$ \\
	$t$        & Zeit                   & $[s]$ \\
	$T$        & Periode                & $[s]$ \\
	$m$        & m-te Periode           & $[1]$ \\
	$\omega_c$ & Abtastfrequenz         & $[\frac{1}{s}]$ \\
 }}
\vfill

\subsection{Interpolation}

\index{Abtastung!Interpolation} 

\Hauptbox{
\Bildbox{
	\input digiSig/Abtastung/interpol.pstex_t
}
 \Formelbox{
	\[ s_i(t) = \sum_{m=-\infty}^{\infty} s(mT) h_i(t-mT) \]
	\begin{small}\[ S_i(\omega) = \frac{1}{T} \sum_{k=-\infty}^{\infty} S(\omega - k \omega_c) H_i(\omega) \]\end{small}
  }
\Bildbox{
	\input digiSig/Abtastung/interpolOrdNull.pstex_t 
}
 \Formelbox{
	\begin{footnotesize}Halteglied nullter Ordnung \end{footnotesize} 
	\[ h_i (t) = \mbox{Rechteck, } h=1, \tau = T \]	
	\[ H_i (\omega) = T \frac{\sin\left(\frac{T}{2} \omega \right)}{\frac{T}{2}\omega} e^{-j \frac{T}{2} \omega}\]
  }
\Bildbox{
	\input digiSig/Abtastung/interpolLin.pstex_t
}
 \Formelbox{
	\begin{footnotesize}Lineare Interpolation \end{footnotesize} 
	\[ h_i (t) = \mbox{Dreieck, } h=1, \tau = 2T \]	
	\[ H_i (\omega) = T \left (\frac{\sin\left(\frac{T}{2} \omega \right)}{\frac{T}{2}\omega} \right)^2 e^{-jT \omega}\]
  }
}{
 \Groessenbox{
	$s_i$      & Signal interpoliert    & $[...]$ \\
	$h_i$      & Interpolatios"-funktion & $[1]$ \\
	$t$        & Zeit                   & $[s]$ \\
	$T$        & Periode                & $[s]$ \\
	$\tau$     & Pulsbreite             & $[s]$ \\
	$m,k$      & m,k-te Periode         & $[1]$ \\
	$\omega$  & Kreisfrequenz           & $[\frac{1}{s}]$ \\
	$\omega_c$ & Abtastfrequenz         & $[\frac{1}{s}]$ \\
 }}


\section{Energie und Leistung bandbegrenzter Signale}

\index{Abtastung!Energie} 
\index{Abtastung!Leistung} 
\index{Energie} 
\index{Leistung} 

\Hauptbox{
\Bildbox{
	\begin{small}Falls das Abtasttheorem, $ T < \frac{1}{2} f_{max} $ eingehalten wird, hat das abgetastete Signal die selbe Energie bzw. Leistung wie das Original. Siehe Parsevalsches Theorem S. \pageref{parseval}\end{small}
}
 \Formelbox{
	\[ W =  \int_{-\infty}^{\infty} s^2(t) dt \] 
	\[ W = T\sum_{m=-\infty}^{\infty} s^2(mT) \]
	\[ P = \frac{1}{T_{per}}\int_0 ^{T_{per}} s^2(t) dt \]
	\[ P = \frac{1}{N}\sum_{m=0} ^{N-1} s^2(mT) \]
	\[ N = \frac{T_{per}}{T} \]
  }
}{
 \Groessenbox{
	$W$        & Energie                & $[Ws]$ \\
	$P$        & Leistung               & $[W]$ \\
	$s$        & Signal                 & $[...]$ \\
	$t$        & Zeit                   & $[s]$ \\
	$T$        & Periode                & $[s]$ \\
	$T_{per}$  & Periodenin"-ter"-vall  & $[s]$ \\
	$m$        & m-te Periode           & $[1]$ \\
	$\omega_c$ & Abtastfrequenz         & $[\frac{1}{s}]$ \\
	$N$        & Abtastwerte"-zahl        & $[1]$ \\
 }}
\vfill




%%% Local Variables:
%%% mode: latex
%%% TeX-master: "../../../FoSaHSR"
%%% End: