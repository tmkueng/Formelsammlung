%%% Local Variables: 
%%% mode: latex
%%% TeX-master: "../FoSaHSR"
%%% End: 

\part{Mathematik} \index{Mathematik}
%\twocolumn

\chapter{Grundlagen}

\setlength{\unitlength}{1mm}
\section{Allgemeines}
\subsection{Binome}     \index{Binom}

\[ {n \choose k}= \frac{{n!}}{{k!\left( {n - k} \right)!}}\]
\[\left( {a + b} \right)^n  = \sum\limits_{k = 0}^n {{n \choose k}a^{n - k} b^k } \]
\[\left( {a - b} \right)^n  = \sum\limits_{k = 0}^n {\left( { - 1} \right)^k {n \choose k}a^{n - k} b^k } \]
\[\left( {a^2  - b^2 } \right) = \left( {a - b} \right)\left( {a + b} \right)\]
\[\left( {a^3  \pm b^3 } \right) = \left( {a \pm b} \right)\left( {a^2  \mp ab + b^2 } \right)\]
%\fontsize{7}{8}\selectfont
\begin{scriptsize}
\begin{center}
\[
\begin{array}{c}        \index{Pascal Dreieck}

1 \\
1 \quad 1 \\
1 \quad 2 \quad 1 \\
1 \quad 3 \quad 3 \quad 1 \\
1 \quad 4 \quad 6 \quad 4 \quad 1\\
1 \quad 5 \quad 10 \quad 10 \quad 5 \quad 1\\
1 \quad 6 \quad 15 \quad 20 \quad 15 \quad 6 \quad 1 \\
1 \quad 7 \quad 21 \quad 35 \quad 35 \quad 21 \quad 7 \quad 1 \\
1 \quad 8 \quad 28 \quad 56 \quad 70 \quad 56 \quad 28 \quad 8 \quad 1 \\
1 \quad 9 \quad 36 \quad 84 \quad 126 \quad 126 \quad 84 \quad 36 \quad 9 \quad 1 \\
1 \quad 10 \quad 45 \quad 120 \quad 210 \quad 252 \quad 210 \quad 120 \quad 45 \quad 10 \quad 1\\
1 \quad 11 \quad 55 \quad 165 \quad 330 \quad 462 \quad 462 \quad 330 \quad 165 \quad 55 \quad 11 \quad 1\\
1 \quad 12 \quad 66 \quad 220 \quad 495 \quad 792 \quad 924 \quad 792 \quad 495 \quad 220 \quad 66 \quad 12 \quad 1 \\
\end{array}
\]
\end{center}
\end{scriptsize}

\subsection{Faktorzerlegungen}                                                          \index{Faktorzerlegungen}

\[a^2 - b^2 = (a + b)(a - b)\]
\[a^3 - b^3 = (a - b)(a^2 + ab + b^2)\]
\[a^3 + b^3 = (a + b)(a^2 - ab + b^2)\]
\[a^n - b^n = (a - b)(a^{n-1} + a^{n-2}b + ... + ab^{n-2} + b^{n-1})\]
f"ur n gerade:

\[a^n - b^n = (a + b)(a^{n-1} - a^{n-2}b + - ... + ab^{n-2} - b^{n-1})\]
f"ur n ungerade:

\[a^n + b^n = (a + b)(a^{n-1} - a^{n-2}b + - ... - ab^{n-2} + b^{n-1})\]

\[s^2 + 1 = (s-j)(s+j)\]

\subsection{Quadratische Gleichung}             \index{Quadratische Gleichung}

\[ a x ^2  + b x  + c = 0\quad \quad  x  = \frac{{ - b \pm \sqrt {b^2  - 4ac} }}{{2a}} \]

\subsection{Arithmetische Folge}                \index{Folge!Aritmetische}

\[a_{n + 1}  - a_n  = d,\quad d\;{\mathrm const}{\mathrm .}\]
\[a_n  = a_1  + \left( {n - 1} \right)d\]
\[s_n  = n\frac{{a_1  + a_n }}{2} = a_1 n + \frac{{n\left( {n - 1} \right)}}{2}d\]

\subsection{Geometrische Folge}                 \index{Folge!Geometrische}

\[{{a_{n + 1} } \mathord{\left/
 {\vphantom {{a_{n + 1} } {a_n }}} \right.
 \kern-\nulldelimiterspace} {a_n }} = q,\quad q\;{\mathrm const}{\mathrm .}\]
\[a_n  = a_1 q^{n - 1} \]
\[s_n  = a_1 \frac{{1 - q^n }}{{1 - q}}\]
\[s = \mathop {\lim }\limits_{n \to \infty } s_n  = \frac{{a_1 }}{{1 - q}},\quad {\mathrm falls }\left| q \right| < 1\]

\subsection{Partialbruchzerlegung}      \index{Partialbruchzerlegung}

\[r\left( z \right) = \frac{{r_1 \left( z \right)}}{{\left( {z - a} \right)\left( {z - b} \right)^3 \left( {\left( {z - c} \right)^2  + d^2 } \right)^3 }}\]
\[r\left( z \right) = \frac{\alpha }{{z - a}} + \frac{{\beta _1 }}{{z - b}} + \frac{{\beta _2 }}{{\left( {z - b} \right)^2 }} + \frac{{\beta _3 }}{{\left( {z - b} \right)^3 }} + \]
\[\frac{{\gamma _1 z + \delta _1 }}{{\left( {z - c} \right)^2  + d^2 }} + \frac{{\gamma _2 z + \delta _2 }}{{\left( {\left( {z - c} \right)^2  + d^2 } \right)^2 }} + \frac{{\gamma _3 z + \delta _3 }}{{\left( {\left( {z - c} \right)^2  + d^2 } \right)^3 }}\]


\section{Matrizen und Determinanten}      \index{Matrizen und Determinanten}

\subsection{2 $\times$ 2 Matrizen}        \index{2 $\times$ 2 Matrizen!Matrizen und Determinanten}

\[\mbox{det}\left[\begin{array}{ll}
 a_{11} & a_{12} \\ 
 a_{21} & a_{22} \\
\end{array} \right] = a_{11}a_{22} - a_{12}a_{21}
\]

\[
\left[\begin{array}{ll}
 a_{11} & a_{12} \\ 
 a_{21} & a_{22} \\
\end{array} \right]
\left[\begin{array}{ll}
 c_{11} & c_{12} \\ 
 c_{21} & c_{22} \\
\end{array} \right] =
\left[\begin{array}{ll}
 a_{11}c_{11} + a_{12}c_{21} & a_{11}c_{12} + a_{12}c_{22} \\ 
 a_{21}c_{11} + a_{22}c_{21} & a_{21}c_{12} + a_{22}c_{22} \\
\end{array} \right] \]
Achtung: $ A B \neq B A $ ! \\ \\
Inverse: (falls $ a_{11} a_{22} - a_{12} a_{21} \neq 0 $ )
\[ A^{-1} = \left[\begin{array}{ll}
 a_{11} & a_{12} \\ 
 a_{21} & a_{22} \\
\end{array} \right] ^{-1} = \frac{1}{a_{11} a_{22} - a_{12} a_{21}} \left[\begin{array}{ll}
 a_{22} & -a_{12} \\ 
 -a_{21} & a_{11} \\
\end{array} \right] \]

\subsection{3 $\times$ 3 Matrizen}        \index{3 $\times$ 3 Matrizen!Matrizen und Determinanten}

\[\mbox{det} \left[\begin{array}{lll}
 a_{11} & a_{12} & a_{13}\\ 
 a_{21} & a_{22} & a_{23}\\
 a_{31} & a_{32} & a_{33}\\
\end{array} \right] = \]

\[= a_{11}\, 
\mbox{det}\left[\begin{array}{ll}
 a_{22} & a_{23} \\ 
 a_{32} & a_{33} \\
\end{array} \right] 
- a_{12}\,
\mbox{det}\left[\begin{array}{ll}
 a_{21} & a_{23} \\ 
 a_{31} & a_{33} \\
\end{array} \right] 
+ a_{13}\,
\mbox{det}\left[\begin{array}{ll}
 a_{21} & a_{22} \\ 
 a_{31} & a_{32} \\
\end{array} \right]\]

\subsection{Transponierte einer Matrix} \index{Transponierte}
                                        \index{Matrix!Transponierte}
\[ 
\left[ \begin{array}{ll}
 a_{11} & a_{12} \\ 
 a_{21} & a_{22} \\
 a_{31} & a_{32} \\ 
 a_{41} & a_{42} \\
\end{array} \right]  
\mbox{\hspace{1cm}} 
A^T =
\left[ \begin{array}{llll}
 a_{11} & a_{21} & a_{31} & a_{41} \\
 a_{21} & a_{22} & a_{32} & a_{42} \\
\end{array} \right] 
\]  

\[ ( A \cdot B ) ^T = B^T \cdot A^T \]
\[ ( A \cdot B \cdot C) ^T = C^T \cdot B^T \cdot A^T \]
\[ ( A^T ) ^{-1} = (A^{-1} )^T \]


\section{Vektorrechnung}                  \index{Vektorrechnung}

\subsection{Grundlagen}                   \index{Grundlagen!Vektorrechnung}

\subsubsection{Skalarprodukt}             \index{Skalarprodukt!Vektorrechnung}

\[\vec{x}\cdot\vec{y} = x y \cos \alpha\]
\[\vec{x}\bot \vec{y}\Leftrightarrow \vec{x}\cdot\vec{y} = 0\]

\subsubsection{Skalare Projektion von $\vec{b}$ auf $\vec{a}$} \index{Skalare Projektion !Vektorrechnung}

\[b _a = \vec{b}\,\vec{e} _a\]

\subsubsection{Vektorielle Projektion von $\vec{b}$ auf $\vec{a}$} \index{Vektorielle Projektion !Vektorrechnung}

\[\vec{b} _a = b _a\,\vec{e} _a = (\vec{b}\,\vec{e} _a)\vec{e} _a \]

\subsubsection{Vektorprodukt}             \index{Vektorprodukt!Vektorrechnung}

\[|\vec{a} \times \vec{b}| = a \, b \sin \alpha\]
\[ \vec a \times \vec b = \left[ a_1 , a_2 , a_3 \right ] \times \left[ b_1 , b_2 , b_3 \right ] = 
\left[ det\left[ \begin{array}{cc}
a_2 & a_3 \\ 
b_2 & b_3
\end{array}  \right ] , - det\left[ \begin{array}{cc}
a_3 & a_1 \\ 
b_3 & b_1
\end{array}  \right ], det\left[ \begin{array}{cc}
a_1 & a_2 \\ 
b_1 & b_2
\end{array}  \right ] \right ] \]
\[\vec{a} = \lambda \cdot \vec{b}\Leftrightarrow \vec{a} \times \vec{b} = 0\]
\[\vec{a} \times \vec{b} = - (\vec{b} \times \vec{a})\]

\subsubsection{Steigung eines Vektors}

\[\vec{x}=(x_1,x_2,x_3)\]
\[\tan\alpha = \frac{x_3}{\sqrt{x_1^2+x_2^2}}\]

\subsection{Lineare Abbildungen}                   \index{Lineare Abbildungen!Vektorrechnung}

\subsubsection{Drehung der XY-Ebene um den Ursprung mit Drehwinkel $\varphi$}

\[\left[\begin{array}{l}
\tilde{x} \\ 
\tilde{y}\\

\end{array} \right] = 
\left[\begin{array}{rr}
\cos\varphi & -\sin\varphi  \\ 
\sin\varphi & \cos\varphi\\
\end{array}\right] 
\left[\begin{array}{l}
x \\ 
y \\
\end{array} \right] 
\] 


\subsubsection{Spiegelung der XY-Ebene an der Geraden g durch den Ursprung mit den Steigungswinkel $\varphi$}

\[\left[\begin{array}{l}
\tilde{x} \\ 
\tilde{y}\\
\end{array} \right] = 
\left[\begin{array}{rr}
\cos 2\varphi & \sin 2\varphi  \\ 
\sin 2\varphi & -\cos 2\varphi\\
\end{array}\right] 
\left[\begin{array}{l}
x \\ 
y \\
\end{array} \right]\]

\subsubsection{Drehung des Raumes um die X-Achse}
\[\left[\begin{array}{l}
\tilde{x} \\ 
\tilde{y}\\
\tilde{z}\\
\end{array} \right] = 
\left[\begin{array}{ccc}
1 & 0 & 0 \\ 
0 & \cos\varphi & -\sin\varphi \\
0 & \sin\varphi & \cos\varphi \\
\end{array}\right] 
\left[\begin{array}{l}
x \\ 
y \\
z \\
\end{array} \right] 
\] 

\subsubsection{Drehung des Raumes um die Y-Achse}
\[\left[\begin{array}{l}
\tilde{x} \\ 
\tilde{y}\\
\tilde{z}\\
\end{array} \right] = 
\left[\begin{array}{ccc}
\cos\varphi  & 0 & \sin\varphi  \\ 
0 & 1 & 0 \\
-\sin\varphi & 0 & \cos\varphi \\
\end{array}\right] 
\left[\begin{array}{l}
x \\ 
y \\
z \\
\end{array} \right] 
\] 


\subsubsection{Drehung des Raumes um die Z-Achse}

\[\left[\begin{array}{l}
\tilde{x} \\ 
\tilde{y}\\
\tilde{z}\\
\end{array} \right] = 
\left[\begin{array}{ccc}
\cos\varphi & -\sin\varphi & 0 \\ 
\sin\varphi & \cos\varphi & 0 \\
0 & 0 & 1 \\
\end{array}\right] 
\left[\begin{array}{l}
x \\ 
y \\
z \\
\end{array} \right] 
\] 


\section{Trigonometrie}                  \index{Trigonometrie}

\subsection{Komplementwinkel}            \index{Trigonometrie!Komplementwinkel}

\[\sin \alpha = \cos (\frac{\pi}{2} - \alpha) \quad \quad \cos \alpha = \sin (\frac{\pi}{2} -\alpha )\]
\[\tan \alpha = \cot (\frac{\pi}{2} - \alpha) \quad \quad \cot \alpha = \tan (\frac{\pi}{2} -\alpha )\]
 

\subsection{Sinussatz}             \index{Trigonometrie!Sinussatz}

\[\sin \alpha = \sin (\pi - \alpha)\]
\[\frac{a}{\sin \alpha} = \frac{b}{\sin \beta} = \frac{c}{\sin \gamma} = 2r\]
wobei r = Umkreisradius

\subsection{Cosinussatz}             \index{Trigonometrie!Cosinussatz}
\[a^2 = b^2 + c^2 - 2bc \cos \alpha\]
\[b^2 = a^2 + c^2 - 2ac \cos \beta\]
\[c^2 = a^2 + b^2 - 2ab \cos \gamma\]

\section{Goniometerie}                  \index{Goniometrie}
\subsection{Serien (L�sungsmengen)}
\[\alpha _1  = \arcsin x,\,\,\,\,\alpha _2  = \pi  - \alpha _1 \]
\[\alpha _{1n}  = \alpha _1  + n2\pi ,\,\,\,\,\alpha _{2n}  = \alpha _2  + n2\pi \]
\[ \pm \alpha  = \arccos x,\,\,\,\,\alpha _n  =  \pm \alpha  + n2\pi \]
\[\alpha _0  = \arctan x,\,\,\,\,\alpha _n  = \alpha _0  + n\pi ,\,\,\,\,n \in {\mathbb{Z}}\]



\subsection{Potenzen}                   \index{Goniometrie!Potenzen}


\[1 + \tan ^2 \alpha = \frac{1}{\cos ^2 \alpha}\]

\[\sin ^2 \alpha  + \cos ^2 \alpha  = 1\]
\[\sin ^3 \alpha  = \frac{1}{4}\left( {3\sin \alpha  - \sin 3\alpha } \right)\]
\[\cos ^3 \alpha  = \frac{1}{4}\left( {3\cos \alpha  + \cos 3\alpha } \right)\]
\[\sin ^4 \alpha  = \frac{1}{8}\left( {\cos 4\alpha  - 4\cos 2\alpha  + 3} \right)\]
\[\cos ^4 \alpha  = \frac{1}{8}\left( {\cos 4\alpha  + 4\cos 2\alpha  + 3} \right)\]

\subsection{Additionstheoreme}          \index{Goniometrie!Additionstheoreme}
                                        \index{Additionstheoreme}

\[\sin \left( {\alpha  \pm \beta } \right) = \sin \alpha \cos \beta  \pm \cos \alpha \sin \beta \]
\[\cos \left( {a \pm \beta } \right) = \cos \alpha \cos \beta  \mp \sin \alpha \sin \beta \]
\[\tan \left( {\alpha  \pm \beta } \right) = \frac{{\tan \alpha  \pm \tan \beta }}{{1 \mp \tan \alpha \tan \beta }}\]

\subsection{Doppelwinkel}       \index{Goniometrie!Doppelwinkel}
                                \index{Doppelwinkel}

\[\sin 2\alpha  = 2\sin \alpha \cos \alpha \]
\[\cos 2\alpha  = \cos ^2 \alpha  - \sin ^2 \alpha =  2 cos�(\alpha)-1 = 1 - sin�(\alpha) \]
\[\tan 2\alpha  = \frac{{2\tan \alpha }}{{1 - \tan ^2 \alpha }}\]

\subsection{Dreifachwinkel}     \index{Goniometrie!Dreifachwinkel}
                                \index{Dreifachwinkel}

\[\sin 3\alpha  = 3\sin \alpha  - 4\sin ^3 \alpha \]
\[\cos 3\alpha  = 4\cos ^3 \alpha  + 3\cos \alpha \]
\[\tan 3\alpha  = \frac{{3\tan \alpha  - \tan ^3 \alpha }}{{1 - 3\tan ^2 \alpha }}\]

\subsection{Halbwinkel}         \index{Goniometrie!Halbwinkel}
                                \index{Halbwinkel}

\[\sin ^2 \frac{\alpha }{2} = \frac{{1 - \cos \alpha }}{2}\]
\[\cos ^2 \frac{\alpha }{2} = \frac{{1 + \cos \alpha }}{2}\]
\[\tan ^2 \frac{\alpha }{2} = \frac{{1 - \cos \alpha }}{{1 + \cos \alpha }}\]

\subsection{Summen und Produkte}        \index{Goniometrie!Summe und Produkte}


\[\sin \alpha  + \sin \beta  = 2\sin \frac{ \alpha  + \beta }{2}\cos \frac{\alpha  - \beta }{2}\]
\[\sin \alpha  - \sin \beta  = 2\cos \frac{ \alpha  + \beta }{2}\sin \frac{\alpha  - \beta }{2}\]
\[\cos \alpha  + \cos \beta  = 2\cos \frac{ \alpha  + \beta }{2}\cos \frac{\alpha  - \beta }{2}\]
\[\cos \alpha  - \cos \beta  =  - 2\sin \frac{ \alpha  + \beta}{2}\sin \frac{\alpha  - \beta }{2}\]
\[\sin \alpha \sin \beta  = \frac{1}{2}\left( {\cos( \alpha  - \beta)   - \cos( \alpha  + \beta) } \right)\]
\[\cos \alpha \cos \beta  = \frac{1}{2}\left( {\cos( \alpha  - \beta)  + \cos( \alpha  + \beta) } \right)\]
\[\sin \alpha \cos \beta  = \frac{1}{2}\left( {\sin( \alpha  - \beta)  + \sin( \alpha  + \beta) } \right)\]

\subsection{Genaue Funktionswerte}      \index{Goniometrie!Genaue Funktionswerte}
                                        \index{Sinuswerte}
                                        \index{Cosinuswerte}
                                        \index{Tangenswerte}

\begin{tabular}{l|ccccc} 
  $\alpha$ & $0$ & $\frac{\pi}{6}$ & $\frac{\pi}{4}$ & $\frac{\pi}{3}$ & $\frac{\pi}{2}$\\
  \hline
  $\sin\alpha$ & $0$ & $\frac{1}{2}$ & $\frac{\sqrt{2}}{2}$ & $\frac{\sqrt{3}}{2}$ & $1$\\
  $\cos\alpha$ & $1$ & $\frac{\sqrt{3}}{2}$ & $\frac{\sqrt{2}}{2}$ & $\frac{1}{2}$ & $0$\\
  $\tan\alpha$ & $0$ & $\frac{\sqrt{3}}{3}$ & $1$ & $\sqrt{3}$ & $-$\\
  $\cot\alpha$ & $-$ & $\sqrt{3}$ & $1$ & $\frac{\sqrt{3}}{3}$ & $0$\\
\end{tabular}



\section{Logarithmen}   \index{Goniometrie!Logarithmen}
                                
\[\log \left( {u \cdot v} \right) = \log u + \log v\]
\[\log \left( {\frac{u}{v}} \right) = \log u - \log v\]
\[\log \left( {u^k } \right) = k\log u\]
\[\log \sqrt[k]{u} = \frac{1}{k}\log u\]
\[\log _b r = \frac{{\log _a r}}{{\log _a b}}\]

\section{Komplexe Zahlen}       \index{Komplexe Zahlen}
\subsection{Allgemeines}

\[j^2  =  - 1,\;\;\;\frac{1}{j} =  - j,\;\;\;\;\left( { - 1} \right)^j  = \left( {e^{j\pi } } \right)^j  = e^{ - \pi } \]
\[\underline z  \in {\mathbb{C}},\,\,\,\overline {\underline z } :{\mathrm konjugiertkomplex}\]
\[{\mathrm karthesisch:}\,\,\underline z  = a + jb,\;\;\;\;\underline {\overline z }  = a - jb\,\]
\[{\mathrm polar:}\,\,\underline z  = r \cdot e^{j\varphi } ,\,\,\,\,\underline {\overline z }  = r \cdot e^{ - j\varphi } \]
\[\underline z  = r\left( {\cos \varphi  + j\sin \varphi } \right) = r \cdot e^{j\varphi }  = a + jb\]
\[a = r\cos \varphi ,\;\;\;\;b = r\sin \varphi \]
\[r = \left| {\underline z } \right| = \sqrt {a^2  + b^2 } ,\,\,\,\,\varphi  = \left \{ \begin{array}{lll}
    I. & \mbox {Quadrant} & \arctan \frac{b}{a} \\
    II. & \mbox {Quadrant} & \arctan \frac{b}{a} + \pi \\
    III. & \mbox {Quadrant} & \arctan \frac{b}{a} + \pi \\
    IV. & \mbox {Quadrant} & \arctan \frac{b}{a} + 2 \pi \\
    \end{array} \right.\]

\subsection{Rechenregeln}       \index{Komplexe Zahlen}

\[\left( {a_1  + jb_1 } \right) \pm \left( {a_2  + jb_2 } \right) = a_1  \pm a_2  + j\left( {b_1  \pm b_2 } \right)\]
\[\left( {a_1  + jb_1 } \right)\left( {a_2  + jb_2 } \right) = \left( {a_1 a_2  - b_1 b_2 } \right) + j\left( {a_1 b_2  + b_1 a_2 } \right)\]
\[\underline z _1  \cdot \underline z _2  = r_1 r_2  \cdot e^{j\left( {\varphi _1  + \varphi _2 } \right)} \]
\[\frac{{\left( {a_1  + jb_1 } \right)}}{{\left( {a_2  + jb_2 } \right)}} = \frac{{\left( {a_1  + jb_1 } \right)\left( {a_2  - jb_2 } \right)}}{{\left( {a_2^2  + b_2^2 } \right)}}\]
\[\frac{{\underline z _1 }}{{\underline z _2 }} = \frac{{r_1 }}{{r_2 }} \cdot e^{j\left( {\varphi _1  - \varphi _2 } \right)} \]
\[\sqrt[n]{{\underline z }} = \sqrt[n]{r}\left( {\cos \frac{\varphi }{n} + j\sin \frac{\varphi }{n}} \right)\]
\[\sqrt[n]{{\underline z }} = e^{\frac{1}{n}\ln \underline z } \mbox{\small $+(n-1)$ weitere L�sungen gleichm�ssig verteilt auf einem Kreis mit Radius $\sqrt[n]{r}$}  \]

\subsection{Euler}                      \index{Komplexe Zahlen!Euler}
                                        \index{Euler}

\[e^{ \pm jkt}  = \cos kt \pm j\sin kt\]
\[e^{ \pm jk\pi }  = \left( { - 1} \right)^k ,\;\;\;\;e^{t + j2\pi }  = e^t \]
\[\cos kt = \frac{1}{2}\left( {e^{jkt}  + e^{ - jkt} } \right)\]
\[\sin kt = \frac{1}{{2j}}\left( {e^{jkt}  - e^{ - jkt} } \right)\]
\[\cosh kt = \frac{1}{2}\left( {e^{kt}  + e^{ - kt} } \right)\]
\[\sinh kt = \frac{1}{2}\left( {e^{kt}  - e^{ - kt} } \right)\]

\section{Ableiten}                      \index{Ableiten}
                                        \index{Differenzieren}
\subsection{Rechenregeln}               \index{Ableiten!Rechenregeln}
                                        \index{Differenzieren!Rechenregeln}


\[\left( {\lambda f} \right)' = \lambda f'\]
\[\left( {f \pm g} \right)' = f' \pm g'\]
\[\left( {f \cdot g} \right)' = f' \cdot g + f \cdot g'\]
\[\left(\frac{f}{g}\right)' = \frac{{g \cdot f' - f \cdot g'}}{{g^2 }}\]
\[f'^{ - 1}  = \frac{1}{{f' \circ f^{ - 1} }}\]
\[\left( {f \circ g} \right)' = \left( {f' \circ g} \right) \cdot g'\]

\subsubsection{Elementare Funktionen}      \index{Ableiten!Elementare Funktionen}
                                           \index{Differenzieren!Elementare Funktionen}


\[\,{\mathop{\mathrm pot'}\nolimits} _k  x  = k\,{\mathop{\mathrm pot}\nolimits} _{k - 1}  x \]
\[\sin' k x = k\cos k x \]
\[\cos' k x  =  - k\sin k x \]
\[\exp' k x  = k\exp k x \]
\[\log'  x  = \frac{1}{ x }\]
\[\ln' \left| f \right| = \frac{{f'}}{f}\]
\[\left(a^{k x }\right)'  = \left( {k\ln a} \right)a^{k x } \]
\[\tan'  x  = \frac{1}{{\cos ^2  x }} = 1 + \tan ^2  x \]
\[\cot'  x  =  - \frac{1}{{\sin ^2  x }} =  - 1 - \cot ^2  x \]
\[\sqrt  x'   = \frac{1}{{2\sqrt  x  }}\]
\[\arcsin'  x  = \frac{1}{{\sqrt {1 -  x ^2 } }}\]
\[\arccos'  x  =  - \frac{1}{{\sqrt {1 -  x ^2 } }}\]
\[\arctan'  x  = \frac{1}{{1 +  x ^2 }}\]
\[{\mathop{\mathrm arccot'}\nolimits} \, x  =  - \frac{1}{{1 +  x ^2 }}\]
\[{\mathop{\mathrm cosh'\, x}} = {\mathop{\mathrm sinh x}} = \frac{e^z - e^{-z}}{2}\]
\[{\mathop{\mathrm sinh'\, x}} = {\mathop{\mathrm cosh x}} = \frac{e^z + e^{-z}}{2}\]
\[{\mathop{\mathrm arcosh'\, x}} =  \frac{1}{\sqrt{1+x^2}}\]
\[{\mathop{\mathrm arsinh'\, x}} =  \frac{1}{\sqrt{x^2-1}}\]


\subsubsection{Satz von Bernoulli und de l'Hospital}       \index{Ableiten!Bernoulli, de l'Hospital}
                                                        \index{Differenzieren!Bernoulli, de l'Hospital}
                                                        \index{Bernoulli}
                                                        \index{Hospital}
\[
\mathop {\lim }\limits_{t \to t_0 } \frac{{f(x)}}{{g(x)}} = \mathop {\lim }\limits_{t \to t_0 } \frac{{f'(x)}}{{g'(x)}}
\]

\noindent \textbf{Beispiel:}

\[
\mathop {\lim }\limits_{t \to \infty } \frac{t}{{e^t }} = \mathop {\lim }\limits_{t \to \infty } \frac{1}{{e^t }} = 0
\]



\section{Integrieren}                                    \index{Aufleiten}
                                                        \index{Integrieren}

\subsection{Rechenregeln}                               \index{Aufleiten!Rechenregeln}
                                                        \index{Integrieren!Rechenregeln}

\[\int {\lambda f}  = \lambda \int f \]
\[\int {\left( {f \pm g} \right) = \int {f \pm \int g } } \]
\[\int {f \cdot g' = f \cdot g - \int {f' \cdot g} } \]


\subsection{Substitution}                               \index{Integrieren!Substitution}

\[\int{f(x)\,dx} \]
\noindent Aufstellen der Substitutionsgleichung:
\[ u= g(x) , \; \frac{du}{dx} = g'(x), \; dx = \frac{du}{g'(x)} \qquad \mbox{ bzw. } \; x=h(u) , \; \frac{dx}{du} = h'(u), \; dx = \frac{h'(u)}{du} \]
\begin{footnotesize}( $u = g(x) $ bzw. $ x = h(u) $ m�ssen monotone Funktionen sein)\end{footnotesize}\\
Substitution:
\[\int f(x) \, dx = \int \varphi(d) \, du \]
Integration:
\[ \int \varphi(u) \, du = \Phi(u) \]
R�cksubstitutuion:
\[ \int f(x) \, dx = \int \varphi(u) \, du = \Phi(u) = \Phi(g(x)) = F(x) \]
\textbf{Beispiel:}
\[ \int _0 ^2 x \sqrt{3 x^2 +4} \, dx \]
\[ \mbox{Subst: } u = 3x^2 + 4 \: \Leftrightarrow \: u' =\frac{du}{dx} = 6x \]
\begin{small}Die neuen Grenzen erhalten wir durch Einsetzten der urspr�nglichen Grenzen in die Substitutionsgleichung, die R�cksubstition entf�llt:\end{small}
\[ \begin{array}{lcl}
2 & \mapsto & 16 \\ 
0 & \mapsto & 4
\end{array} \]
\[ \Rightarrow \int _4 ^{16} \sqrt{u} \, du \]

\subsection{S�tze}                                      \index{Aufleiten!S�tze}
                                                        \index{Integrieren!S�tze}

\[\int\limits_{a}^{b} {f} =  - \int\limits_{b}^{a}{ f } \]
\[\int\limits_{a}^{b}{f(t)\,dt} = -\int\limits_{-a}^{-b}{f(-t)\,dt}\]
\[\int\limits_{a}^{b} {f = \int\limits_a^c {f + \int\limits_c^b f } } \]
\[\int\limits_{a}^{b} {f\left(  t  \right)}  = \int\limits_{a + c}^{b + c} f \left( { t  - c} \right)\]
\[f\;{\mathrm stetig} \Rightarrow \int\limits_a^b {f = \int {f\left( b \right) - \int {f\left( a \right)} } }  = F\left( b \right) - F\left( a \right)\]
\[f\;{\mathrm stetig}\;{\mathrm in}\;\left[ {a,b} \right] \Rightarrow \exists \;\xi  \in \left[ {a,b} \right]\;{\mathrm mit}\;\int\limits_a^b {f = \left( {b - a} \right)f\left( \xi  \right)} \]



\subsection{Integration rationaler Funktionen}                 \index{Integrieren!Integration rationaler Funktionen}


Rationale Funktionen k"onnen integriert werden, indem man Division der Polynome durchf"uhrt\\
Beispiel:\quad$\int{\frac{x^2}{x^2+1}}$

\[x^2 : (x^2+1) = 1 + \frac{1}{x^2+1}\]
\[\int{1+\frac{1}{x^2+1}}\,dx=x+\arctan x\]

\subsection{Rationalisierungsformeln}                   \index{Rationalisierungsformeln} 

\subsubsection{F�r Rationale Funktionen von $\sin{x}$ und $\cos{x}$}

\begin{itemize}

\item Beispiel
\[ \int \frac{1 + \cos{x}}{\sin{x}}\,dx\]
\item Substitution
\[ u=\tan{x/2} \quad \] \[ \Rightarrow \quad dx=\frac{2}{1+u^2}\,du \quad \sin{x}=\frac{2u}{1+u^2} \quad \cos{x} \frac{1-u^2}{1+u^2}\]

\end{itemize}

\subsubsection{Weitere Rationalisierungsformeln siehe {Papula Seite 148}} 

\subsection{Spezielle Integrale}                        \index{Aufleiten!Spezielle Integrale}
                                                        \index{Integrieren!Spezielle Integrale}

\[\int {{\mathop{\mathrm pot}\nolimits} _k  = \frac{1}{{k + 1}}} \,{\mathop{\mathrm pot}\nolimits} _{k + 1} \]
\[\int {\exp k x\,dx = \frac{1}{k}} \exp k x \]
\[\int {a^{c x } }\,dx  = \frac{1}{{c\ln a}}a^{c x } \]
\[\int {\frac{1}{ x }}\,dx  = \ln  x \]
\[\int {\ln \left|  x  \right|}\,dx  =  x \left( {\ln \left|  x  \right| - 1} \right)\]
\[\int {\frac{1}{{ x \ln  x }}}\,dx  = \ln \left| {\ln \left|  x  \right|} \right|\]
\[\int {\log _a } \left|  x  \right|\,dx =  x \left( {\log _a \left|  x  \right| - \log _a {\mathrm e}} \right)\]
\[\int { x ^k } \ln  x \,dx  = \frac{{ x ^{k + 1} }}{{k + 1}}\left( {\ln  x  - \frac{1}{{k + 1}}} \right),\quad k \ne  - 1,\quad x > 0\]
\[\int {\frac{{\ln  x }}{ x }\,dx = \frac{1}{2}} \left( {\ln  x } \right)^2 \]
\[\int {\sin \left( {a x  + b} \right)\,dx =  - \frac{1}{a}\cos \left( {a x  + b} \right)} \]
\[\int {\cos \left( {a x  + b} \right)\,dx = \frac{1}{a}\sin \left( {a x  + b} \right)} \]
\[\int {\tan  x \,dx =  - \ln \left| {\cos  x } \right|} \]
\[\int {\cot  x\,dx  = \ln \left| {\sin  x } \right|} \]
\[\int {\frac{1}{{\sin  x }}}\,dx  = \ln \left| {\tan \frac{ x }{2}} \right|\]
\[\int {\frac{1}{{\cos  x }}}\,dx  = \ln \left| {\tan \left( {\frac{ x }{2} + \frac{\pi }{4}} \right)} \right|\]
\[\int {\sin ^2  x }\,dx  = \frac{1}{2}\left( { x  - \sin  x \cos  x } \right)\]
\[\int {\cos ^2  x }\,dx  = \frac{1}{2}\left( { x  + \sin  x \cos  x } \right)\]
\[\int {\tan ^2  x }\,dx  = \tan  x  -  x \]
\[\int {\cot ^2  x \,dx =  - \cot  x  -  x } \]
\[\int {\frac{1}{{ x ^2 }}} \sin \frac{1}{ x }\,dx = \cos \frac{1}{ x }\]
\[\int {\arcsin  x \,dx =  x \arcsin  x  + \sqrt {1 -  x ^2 } } \]
\[\int {\arccos  x \,dx = }  x \arccos  x  - \sqrt {1 -  x ^2 } \]
\[\int {\arctan }  x \,dx =  x \arctan  x  - \frac{1}{2}\ln \left( {1 +  x ^2 } \right)\]
\[\int {{\mathop{\mathrm arccot}\nolimits} \, x \,dx =  x \,{\mathop{\mathrm arccot}\nolimits}  x  + } \frac{1}{2}\ln \left( {1 +  x ^2 } \right)\]
\[\int {\left( {a x  + b} \right)^k } \,dx = \frac{{\left( {a x  + b} \right)^{k + 1} }}{{a\left( {k + 1} \right)}},\quad k \ne 1\]
\[\int {\left( {a x ^p  + b} \right)^k  x ^{p - 1} } \,dx = \frac{{\left( {a x ^p  + b} \right)^{k + 1} }}{{ap\left( {k + 1} \right)}},\quad k \ne 1,\quad ap \ne 0\]
\[\int {\frac{1}{{a x  + b}}} \,dx = \frac{1}{a}\ln \left| {a x  + b} \right|\]
\[\int {\frac{{a x  + b}}{{c x  + d}}} \,dx = \frac{{a x  + b}}{c} - \frac{{ad - bc}}{{c^2 }}\ln \left| {c x  + d} \right|\]
\[\int {\frac{{ x ^{p - 1} }}{{a x ^p  + b}} \,dx= \frac{1}{{ap}}\ln \left| {a x ^p  + b} \right|,\quad ap \ne 0} \]
\[\int {\frac{{a x  + b}}{{c x  + d}} \,dx= \frac{{a x  + b}}{c}}  - \frac{{ad - bc}}{{c^2 }}\ln \left| {c x  + d} \right|\]
\[\int {\frac{1}{{ x ^2  + a^2 }}} \,dx = \frac{1}{a}\arctan \frac{ x }{a}\]
\[\int {\frac{1}{{ x ^2  - a^2 }}} \,dx = \frac{1}{{2a}}\ln \left| {\frac{{ x  - a}}{{ x  + a}}} \right|\]
\[\int {\frac{{ x ^2 }}{{ x ^2  + a^2 }}} \,dx =  x  - a\arctan \frac{ x }{a}\]
\[\int {\frac{{2 x }}{{1 -  x ^2 }} \,dx=  - \ln \left| {1 -  x ^2 } \right|} \]
\[\int {\sqrt { x ^2  \pm a^2 } } \,dx = \frac{ x }{2}\sqrt { x ^2  \pm a^2 }  \pm \frac{{a^2 }}{2}\ln \left( { x  + \sqrt { x ^2  \pm a^2 } } \right)\]
\[\int {\frac{1}{{\sqrt { x ^2  \pm a^2 } }}} \,dx = \ln \left( { x  + \sqrt { x ^2  \pm a^2 } } \right)\]
\[\int {\sqrt {a^2  -  x ^2 } }  \,dx= \frac{ x }{2}\sqrt {a^2  -  x ^2 }  + \frac{{a^2 }}{2}\arcsin \frac{ x }{{\left| a \right|}}\]
\[\int {\frac{1}{{\sqrt {a^2  -  x ^2 } }}}\,dx = \arcsin \frac{ x }{{\left| a \right|}}\]
\[\int {e^{c x } \sin \left( {a x  + b} \right) \,dx= \frac{{e^{c x } }}{{a^2  + c^2 }}\left( {c\sin \left( {a x  + b} \right) - a\cos \left( {a x  + b} \right)} \right)} \]
\[\int {e^{c x } } \cos \left( {a x  + b} \right)\,dx = \frac{{e^{c x } }}{{a^2  + c^2 }}\left( {c\cos \left( {a x  + b} \right) + a\sin \left( {a x  + b} \right)} \right)\]
\[\int {\exp _k } \sin _l \,dx = \frac{{\exp _k }}{{l^2  - k^2 }}\left( {jk\sin _l  - l\cos _l } \right)\]
\[\int {\exp _k } \cos _l \,dx = \frac{{\exp _k }}{{l^2  - k^2 }}\left( {jk\cos _l  - l\sin _l } \right)\]
\[\int { x ^n } \sin k x \,dx =  - \frac{{ x ^n }}{k}\cos k x  + \frac{n}{k}\int { x ^{n - 1} } \cos k x\,dx \quad n \in {\mathbb{N}}\]
\[\int { x ^n } \cos k x \,dx =  + \frac{{ x ^n }}{k}\sin k x  - \frac{n}{k}\int { x ^{n - 1} } \sin k x\,dx \quad n \in {\mathbb{N}}\]


\chapter{Fourierreihen}                                       \index{Fourierreihe}
                                                        \index{Fourier!Reihen}
                                                        \index{Reihen!Fourier}

\section{Bezeichungen}                                  \index{Fourier!Bezeichnungen}
                                                        \index{Fourier!trig. Polynome}
                                                        \index{Fourier!Exp. Polynome}
\noindent
Vektorraum der trigonometrischen Polynome: $\mathbb{P}\subset\mathbb{V}$
\noindent
\[\mathbb{P}=\{ a_0 \cos_0 + \sum\limits_{k=1}^n a_k \cos_k + b_k \sin_k | n \in N^* \}\]
Vektorraum der Exponentialpolynome: $\mathbb{E}\subset\mathbb{V}$
\[\mathbb{E}=\{ \sum\limits_{k=-n}^m c_k \exp_k | c_k \in \mathbb{C}\}\]

\[\sin _k  = \sin kt\]
\[\cos _k  = \cos kt\]
\[\exp _k  = e^{jkt}\]

\section{Skalarprodukt}        \index{Fourier!Skalarprodukt}

\subsection{Eigenschaften}

\[\left[ {a,b} \right] = \left[ {b,a} \right]\]
\[\left[ {a + b,c} \right] = \left[ {a,c} \right] + \left[ {b,c} \right]\]
\[\left[ {\lambda a,b} \right] = \lambda \left[ {a,b} \right]\]
\[\left[ {a,a} \right] \ge 0\quad \quad \left[ {a,a} \right] = 0 \Leftrightarrow a = 0\]

\subsection{Definitionen in $\mathbb{P}$ und $\mathbb{E}$}\[\left[ {f,g} \right] = \frac{1}{\pi }\int\limits_{ - \pi }^\pi  {f \cdot g\quad \quad f,g \in } \,{\mathbb{P}}\mbox{ und }{\mathrm STF}\]
\[\left[ {f,g} \right] = \frac{1}{{2\pi }}\int\limits_0^{2\pi}{f \cdot \overline g } \quad \quad f,g \in {\mathbb{E}}\]
\[\left[ {f,g} \right] = \overline {\left[ {g,f} \right]} \quad \quad f,g \in {\mathbb{E}}\]


\subsection{F�r orthonormierte Basis}           \index{Fourier!Orthonormierte Basis}
                                                \index{Orthonormierte Basis}

\[\left[ {\cos _k ,\sin _l } \right] = 0\quad \quad \quad \quad \quad k \in {\mathbb{N}}_0 ,\;l \in {\mathbb{N}}\]
\[\left[ {\cos _k ,\cos _l } \right] = \left\{ {\begin{array}{*{20}c}
   {1\quad k = l}  \\
   {0\quad k \ne l}  \\
\end{array}} \right.\quad \quad k,l \in {\mathbb{N}}_0 \]
\[\left[ {\sin _k ,\sin _l } \right]\; = \left\{ {\begin{array}{*{20}c}
   {1\quad k = l}  \\
   {0\quad k \ne l}  \\
\end{array}} \right.\quad \quad k,l \in {\mathbb{N}}\]
\[\left[ {\exp _k ,\exp _l } \right] = \left\{ {\begin{array}{*{20}c}
   {1\quad k = l}  \\
   {0\quad k \ne l}  \\
\end{array}} \right.\quad \quad k,l \in {\mathbb{Z}}\]

\section{Norm in $\mathbb{P}$ und $\mathbb{E}$}         \index{Fourier!Norm in $\mathbb{P}$ und $\mathbb{E}$}
                                                        \index{Norm in $\mathbb{P}$ und $\mathbb{E}$}
\[\left\| p \right\| = \sqrt {\left[ {p,p} \right]} \quad \quad p \in{\mathbb{P}\mbox{ und }}{\mathrm STF}\]
\[\left\| p \right\|^2  = \left[ {p,p} \right] = a_0^2  + \sum\limits_{k = 1}^n {a_k^2 }  + b_k^2 \]
\[\left\| e \right\| = \sqrt {\left[ {e,e} \right]} \quad \quad e \in {\mathbb{E}}\]
\[\left\| e \right\|^2  = \left[ {e,e} \right] = \sum\limits_{k =  - n}^n {\left| {c_k } \right|^2 } \]

\section{Cauchy-Schwarzsche Ungleichung}                \index{Fourier!Cauchy-Schwarzsche}
                                                        \index{Cauchy-Schwarzsche}
\[
\left[ {f,g} \right]^2  \le \left[ {f,f} \right] \cdot \left[ {g,g} \right]
\]


\section{Abstand}

\[
{\mathrm d}\left( {f,g} \right) = \left\| {f - g} \right\|
\]

\section{Fourierreihe reell}                            \index{Fourier!Reihe reel}
                                                        
\subsection{Fourierkoeffizienten}                       \index{Fourier!Koeffizienten}

\[a_k  = \left[ {f,\cos _k } \right] = \frac{1}{\pi }\int\limits_{-\pi}^{\pi} {f(t)\cos kt\;dt} \quad f \in {\mathrm STF}\]
\[b_k  = \left[ {f,\sin _k } \right] = \frac{1}{\pi }\int\limits_{-\pi}^{\pi} {f(t)\sin kt\;dt} \quad b_0  = 0\]
gerade Funktion: \\ $g(-t) = g(t) \Rightarrow b_k = 0$\\
ungerade Funktion: \\ $ u(-t) = -u(t) \Rightarrow a_k = 0$

\subsection{Fourierreihe der Funktion $f \in \mathbb{P}$}       \index{Fourier!Funktion $f \in \mathbb{P}$}

\[f = \sum\limits_{k = 0}^\infty  {\left( {a_k \cos _k  + b_k \sin _k } \right)} \]
\[\cos _0  = \frac{1}{{\sqrt 2 }}\quad \quad a_0  = \left[ {f,\cos _0 } \right]\quad \quad b_0  = 0\]

\section{Fourierreihe komplex}          \index{Fourier!Reihe komplex}
\subsection{Fourierkoeffizienten}       \index{Fourier!Koeffizienten komplex}

\[c_0  = \frac{{a_0 }}{{\sqrt 2 }}\quad \quad c_{ - k}  = \overline {c_k } \]
\[c_k  = \frac{1}{2}\left( {a_k  - jb_k } \right)\quad \quad a_k  = 2\,{\mathop{\mathrm Re}\nolimits} \left( {c_k } \right) = c_k  + c_{ - k} \]
\[c_{ - k}  = \frac{1}{2}\left( {a_k  + jb_k } \right)\quad \quad b_k  =  - 2\,{\mathop{\mathrm Im}\nolimits} \left( {c_k } \right) = j\left( {c_k  - c_{ - k} } \right)\]
\[c_k  = \left[ {f,\exp _k } \right] = \frac{1}{{2\pi }}\int\limits_{-\pi}^{\pi} {f\exp _{ - k} }  = \frac{1}{{2\pi }}\int\limits_{-\pi}^{\pi} {f(t)\,e^{ - jkt} \,dt} \]


\subsection{Fourierreihe der Funktion $f \in \mathbb{E}$}       \index{Fourier!Funktion $f \in \mathbb{E}$}

\[
f = \sum\limits_{k =  - \infty }^\infty  {c_k } \exp _k  = c_0  + \sum\limits_{k = 1}^\infty  {\left( {c_k \exp _k  + c_{ - k} \exp _{ - k} } \right)} 
\]

\section{Parsevalsches Theorem}         \index{Parsevalsches Theorem}
\label{parseval} 
\[\left\| {f - \sum\limits_{k = 0}^n {\left( {a_k \cos _k  + b_k \sin _k } \right)} } \right\|^2  = \left\| f \right\|^2  - \sum\limits_{k = 0}^n {\left( {a_k^2  + b_k^2 } \right)}  = \left\| f \right\|^2  - \sum\limits_{k =  - n}^n {\left| {c_k } \right|^2 } \]
Leistung periodischer Signale:
\[ \frac{1}{T_0} \int_{-\frac{T_0}{2}}^{\frac{T_0}{2}} | f(t)|^2 dt = \sum _{n=-\infty}^{\infty} |c_n|^2 \]

\section{Durchgang durch LTI-System}

\[ \mbox{gegeben:} \quad f(t)=c_k e^{jkt} + c_{-k} e^{-jkt} = a_k\cos_k + b_{k}\sin_k \mbox{;} \quad H(\omega) \]
\[ \mbox{gesucht:} \quad T(f(t))= \tilde{f}\]

\[\tilde{f} = \sum\limits_{-\infty}^{\infty} c_k e^{jkt} H(k)\]
\[\tilde{a}_k = \mbox{Re}(H(k)(a_k -jb_k))\]
\[\tilde{b}_{k} = -\mbox{Im}(H(k)(a_k -jb_k))\]
\[\tilde{f}= \tilde{a}_k\cos_k + \tilde{b}_{k}\sin_k\] 

\section{Fourierkoeffizienten wichtiger periodischer Signale}
  \begin{tabular}[t]{ll}
    \begin{minipage}[t][\totalheight][c]{0.55\textwidth}
	Periodische Rechteckfolge\\		\index{Fourier!Rechtecksignal}
    	\input{Mathe/perRecht.pstex_t}
  \end{minipage} & \begin{minipage}[t][\totalheight][c]{0.4\textwidth}\vspace{0.3cm}
	\[c_n = a_n = \frac{ \tau }{T} \frac{ \sin \left( \frac{n\pi\tau}{T} \right) }{ \frac{n\pi\tau}{T} } \]
	\[b_n = 0 \]
  \end{minipage}
  \end{tabular} \\
  \begin{tabular}[t]{ll}
    \begin{minipage}[t][\totalheight][c]{0.55\textwidth}
	Doppelweggleichgerichtete $\cos$-Schwingung\\	\index{Fourier!Doppelweg Gleichgerichtet}
    	\input{Mathe/doppelCos.pstex_t}
  \end{minipage} & \begin{minipage}[t][\totalheight][c]{0.4\textwidth}\vspace{0.3cm}
	\[c_n = a_n = \frac{ 2 }{\pi} (-1)^{n+1} \frac{1}{4 n^2-1} \]
	\[b_n = 0 \]
  \end{minipage}
  \end{tabular}\\
  \begin{tabular}[t]{ll}
    \begin{minipage}[t][\totalheight][c]{0.55\textwidth}
	Einweggleichgerichtete $\cos$-Schwingung\\	\index{Fourier!Einweg Gleichgerichtet}
    	\input{Mathe/einwegCos.pstex_t}
  \end{minipage} & \begin{minipage}[t][\totalheight][c]{0.4\textwidth}\vspace{0.3cm}
	\[c_n = a_n = \frac{ 1 }{\pi} \frac{ \cos \left( \frac{n\pi}{2} \right) }{ 1-n^2 } \]
	\[b_n = 0 \]
  \end{minipage}
  \end{tabular}\\ 
  \begin{tabular}[t]{ll}
    \begin{minipage}[t][\totalheight][c]{0.55\textwidth}
	Folge von Raised-Cosine-Impulsen\\	\index{Fourier!Cosine Folge}
    	\input{Mathe/raisedCos.pstex_t}
  \end{minipage} & \begin{minipage}[t][\totalheight][c]{0.4\textwidth}\vspace{0.3cm}
	\[c_n = a_n = \frac{ \tau }{2T} \frac{ \sin \left( \frac{n\pi\tau}{T} \right) }{ \frac{n\pi\tau}{T} } \frac{1}{1-\left( \frac{n \tau}{T} \right)^2} \]
	\[b_n = 0 \]
  \end{minipage}
  \end{tabular}\\
  \begin{tabular}[t]{ll}
    \begin{minipage}[t][\totalheight][c]{0.55\textwidth}
	Dreieckschwingung DC-frei\\		\index{Fourier!Dreieck Folge}
    	\input{Mathe/dreieck.pstex_t}
  \end{minipage} & \begin{minipage}[t][\totalheight][c]{0.4\textwidth}\vspace{0.3cm}
	\[c_n = a_n = \frac{ 2 [1-(-1)^n] }{(n \pi)^2}, \qquad c_0 = 0 \]
	\[b_n = 0 \]
  \end{minipage}
  \end{tabular}\\
  \begin{tabular}[t]{ll}
    \begin{minipage}[t][\totalheight][c]{0.55\textwidth}
	S�gezahnschwingung DC-frei\\	\index{Fourier!S�gezahn Signal}
    	\input{Mathe/saegezahn.pstex_t}
  \end{minipage} & \begin{minipage}[t][\totalheight][c]{0.4\textwidth}\vspace{0.3cm}
	\[c_n = -jb_n , \qquad c_0 = 0 \]
	\[a_n = 0, \qquad b_n = \frac{(-1)^{n+1}}{n \pi} \]
  \end{minipage}
  \end{tabular}






\chapter{Fouriertransformation}         \index{Fourier!Tranformation}
\begin{picture}(60,25)
\thicklines
\put(10,0){$F$}
\put(10,20){$f$}
\put(40,0){$\tilde{F}=F \cdot H$}
\put(40,20){$\tilde{f}=f*h$}
\put(7,10){${\cal F}$}
\put(42,10){${\cal F}^{-1}$}
\put(13,1){\vector(1,0){26}}
\put(11,18){\vector(0,-1){14}}
\put(13,21){\vector(1,0){26}}
\put(41,4){\vector(0,1){14}}
%\put(62,10){$G=\frac{1}{{\mathrm cp}(s)}$}
\end{picture}


\section{Fouriertransformation}

\[{\cal F}(f(t)) = F(\omega ),\quad \quad {\cal F}^{ - 1} (F(\omega )) = f(t)\]
\[F(\omega ) = \int\limits_{ - \infty }^\infty  {f(t)e^{ - j\omega t} } \,dt\]
\[f(t) = \frac{1}{{2\pi }}\int\limits_{ - \infty }^\infty  {F(\omega )} \,e^{j\omega t} \,d\omega \]
Wichtig: Sonderf�lle beachten! (Division durch 0 in der L�sung separat behandeln)

\section{Fourier-Cosinustransformation}         \index{Fourier!Cosinustransformation}
\noindent F�r gerade Funktionen

\[{\cal F}_c (f(t)) = F_c \left( \omega  \right),\quad \quad {\cal F}_c^{ - 1} (F_c (\omega )) = f(t)\]
\[F_c (\omega ) = \int\limits_{0 }^\infty  {f(t)\cos \omega t} \,dt\]
\[f(t) = \frac{1}{\pi }\int\limits_0^\infty  {F_c (\omega )} \,\cos{\omega t} \,d\omega \]
\[F = 2F_c \]

\section{Fourier-Sinustransformation}
\noindent F�r ungerade Funktionen

\[{\cal F}_s (f(t)) = F_s (\omega ),\quad \quad {\cal F}_s^{ - 1}(F_s (\omega )) = f(t)\]
\[F_s (\omega ) = \int\limits_{0}^\infty  {f(t)\sin {\omega t}\,dt}\]
\[f(t) = \frac{j}{\pi }\int\limits_0^\infty  {F_s (\omega )} \,\sin{\omega t} \,d\omega \]
\[F =  - 2jF_s \]

\section{Faltung}               \index{Fourier!Faltung}

\[\left( {f * g} \right)(t) = \int\limits_{ - \infty }^\infty  {f(\tau )}  \cdot g(t - \tau )\,d\tau \]
\[f * g = g * f\quad \quad \left( {f * g} \right) * k = f * \left( {g * k} \right)\]
\[{\cal F}\left( {f * g} \right) = {\cal F}(f) \cdot {\cal F}(g) = F \cdot G\]

\subsection{Fallunterscheidung bei Definitionsbereichen}
\label{faltungFourier}
\begin{tabular}[t]{ll}
 \begin{minipage}[t][\totalheight][c]{5.5cm}
        $p(t) = (f \ast g)$\\
        $D(g(t)) = [a | b ]$\\
        $D(f(t)) = [c | d ]$\\ \\
        \setlength{\unitlength}{4144sp}%
%
\begingroup\makeatletter\ifx\SetFigFont\undefined%
\gdef\SetFigFont#1#2#3#4#5{%
  \reset@font\fontsize{#1}{#2pt}%
  \fontfamily{#3}\fontseries{#4}\fontshape{#5}%
  \selectfont}%
\fi\endgroup%
\begin{picture}(2172,2433)(1,-1852)
\thinlines
{\color[rgb]{0,0,0}\put(541,-1681){\vector( 0, 1){2250}}
}%
{\color[rgb]{0,0,0}\put(541,-1681){\vector( 1, 0){1620}}
}%
{\color[rgb]{0,0,0}\multiput(811,-511)(120.00000,0.00000){5}{\line( 1, 0){ 60.000}}
\multiput(1351,-511)(0.00000,-120.00000){2}{\line( 0,-1){ 60.000}}
\multiput(1351,-691)(-120.00000,0.00000){5}{\line(-1, 0){ 60.000}}
\multiput(811,-691)(0.00000,120.00000){2}{\line( 0, 1){ 60.000}}
}%
{\color[rgb]{0,0,0}\multiput(811,-691)(120.00000,0.00000){5}{\line( 1, 0){ 60.000}}
\multiput(1351,-691)(-83.07692,-83.07692){7}{\line(-1,-1){ 41.538}}
\multiput(811,-1231)(0.00000,120.00000){5}{\line( 0, 1){ 60.000}}
}%
{\color[rgb]{0,0,0}\multiput(1351,-511)(-120.00000,0.00000){5}{\line(-1, 0){ 60.000}}
\multiput(811,-511)(83.07692,83.07692){7}{\line( 1, 1){ 41.538}}
\multiput(1351, 29)(0.00000,-120.00000){5}{\line( 0,-1){ 60.000}}
}%
{\color[rgb]{0,0,0}\put(496,-781){\line( 1, 0){ 45}}
}%
{\color[rgb]{0,0,0}\put(496,-1501){\line( 1, 0){ 45}}
}%
{\color[rgb]{0,0,0}\multiput(811,-511)(-126.00000,0.00000){3}{\line(-1, 0){ 63.000}}
}%
{\color[rgb]{0,0,0}\put(496,-511){\line( 1, 0){ 45}}
}%
{\color[rgb]{0,0,0}\multiput(811,-691)(-126.00000,0.00000){3}{\line(-1, 0){ 63.000}}
}%
{\color[rgb]{0,0,0}\put(541,-691){\line(-1, 0){ 45}}
}%
{\color[rgb]{0,0,0}\multiput(811,-1231)(-126.00000,0.00000){3}{\line(-1, 0){ 63.000}}
}%
{\color[rgb]{0,0,0}\put(541,-1231){\line(-1, 0){ 45}}
}%
{\color[rgb]{0,0,0}\multiput(1351, 29)(-114.00000,0.00000){8}{\line(-1, 0){ 57.000}}
}%
{\color[rgb]{0,0,0}\put(541, 29){\line(-1, 0){ 45}}
}%
{\color[rgb]{0,0,0}\multiput(811,164)(108.00000,54.00000){3}{\line( 2, 1){ 54.000}}
}%
{\color[rgb]{0,0,0}\multiput(1351,164)(-108.00000,54.00000){3}{\line(-2, 1){ 54.000}}
}%
{\color[rgb]{0,0,0}\put(1081,-1411){\circle*{180}}
}%
{\color[rgb]{0,0,0}\put(1081,-601){\circle*{180}}
}%
{\color[rgb]{0,0,0}\put(946,-871){\circle*{180}}
}%
{\color[rgb]{0,0,0}\put(1216,-376){\circle*{180}}
}%
{\color[rgb]{0,0,0}\put(991,-106){\circle*{180}}
}%
{\color[rgb]{0,0,0}\put(541,-1501){\line( 1, 1){990}}
}%
{\color[rgb]{0,0,0}\put(541,-781){\line( 1, 1){990}}
}%
{\color[rgb]{0,0,0}\put(811,-1681){\line( 0, 1){1890}}
}%
{\color[rgb]{0,0,0}\put(1351,-1681){\line( 0, 1){1890}}
}%
\put(901,-916){\makebox(0,0)[lb]{\smash{\SetFigFont{10}{12.0}{\rmdefault}{\mddefault}{\updefault}{\color[rgb]{0,0,0}II}%
}}}
\put(766,-1816){\makebox(0,0)[lb]{\smash{\SetFigFont{10}{12.0}{\rmdefault}{\mddefault}{\updefault}{\color[rgb]{0,0,0}$c$}%
}}}
\put(361,-1546){\makebox(0,0)[lb]{\smash{\SetFigFont{10}{12.0}{\rmdefault}{\mddefault}{\updefault}{\color[rgb]{0,0,0}$a$}%
}}}
\put(361,-826){\makebox(0,0)[lb]{\smash{\SetFigFont{10}{12.0}{\rmdefault}{\mddefault}{\updefault}{\color[rgb]{0,0,0}$b$}%
}}}
\put(766,344){\makebox(0,0)[lb]{\smash{\SetFigFont{10}{12.0}{\rmdefault}{\mddefault}{\updefault}{\color[rgb]{0,0,0}$a \leq \tau \leq b$}%
}}}
\put(1531, 74){\makebox(0,0)[lb]{\smash{\SetFigFont{10}{12.0}{\rmdefault}{\mddefault}{\updefault}{\color[rgb]{0,0,0}$t \leq d + \tau$}%
}}}
\put(1531,-646){\makebox(0,0)[lb]{\smash{\SetFigFont{10}{12.0}{\rmdefault}{\mddefault}{\updefault}{\color[rgb]{0,0,0}$c + \tau \leq t$}%
}}}
\put(2071,-1816){\makebox(0,0)[lb]{\smash{\SetFigFont{10}{12.0}{\rmdefault}{\mddefault}{\updefault}{\color[rgb]{0,0,0}$\tau$}%
}}}
\put(361,479){\makebox(0,0)[lb]{\smash{\SetFigFont{10}{12.0}{\rmdefault}{\mddefault}{\updefault}{\color[rgb]{0,0,0}$t$}%
}}}
\put(  1,-736){\makebox(0,0)[lb]{\smash{\SetFigFont{10}{12.0}{\rmdefault}{\mddefault}{\updefault}{\color[rgb]{0,0,0}$a+d$}%
}}}
\put(  1,-556){\makebox(0,0)[lb]{\smash{\SetFigFont{10}{12.0}{\rmdefault}{\mddefault}{\updefault}{\color[rgb]{0,0,0}$c+b$}%
}}}
\put(  1,-16){\makebox(0,0)[lb]{\smash{\SetFigFont{10}{12.0}{\rmdefault}{\mddefault}{\updefault}{\color[rgb]{0,0,0}$b+d$}%
}}}
\put(  1,-1276){\makebox(0,0)[lb]{\smash{\SetFigFont{10}{12.0}{\rmdefault}{\mddefault}{\updefault}{\color[rgb]{0,0,0}$a+c$}%
}}}
\put(1056,-1461){\makebox(0,0)[lb]{\smash{\SetFigFont{10}{12.0}{\rmdefault}{\mddefault}{\updefault}{\color[rgb]{0,0,0}I}%
}}}
\put(1006,-651){\makebox(0,0)[lb]{\smash{\SetFigFont{10}{12.0}{\rmdefault}{\mddefault}{\updefault}{\color[rgb]{0,0,0}III}%
}}}
\put(1146,-431){\makebox(0,0)[lb]{\smash{\SetFigFont{10}{12.0}{\rmdefault}{\mddefault}{\updefault}{\color[rgb]{0,0,0}IV}%
}}}
\put(936,-166){\makebox(0,0)[lb]{\smash{\SetFigFont{10}{12.0}{\rmdefault}{\mddefault}{\updefault}{\color[rgb]{0,0,0}V}%
}}}
\put(1306,-1816){\makebox(0,0)[lb]{\smash{\SetFigFont{10}{12.0}{\rmdefault}{\mddefault}{\updefault}{\color[rgb]{0,0,0}$d$}%
}}}
\end{picture}

        \vspace{1ex}
  \end{minipage} &
 \begin{minipage}[t][\totalheight][c]{5.4cm}
        \begin{tabular}{|l|ll|}
        \multicolumn{3}{l}{1. Fall: $c+b < a+d$}  \\ \hline
        I   & $t < a+c$:             & $p(t) = 0$  \\ 
        II  & $a+c \leq t \leq b+c$: & $p(t) = \int_a ^{t-c} {f(\tau )}  \cdot g(t - \tau )\,d\tau$  \\ 
        III & $b+c \leq t \leq a+d$: & $p(t) = \int_{a} ^{b} {f(\tau )}  \cdot g(t - \tau )\,d\tau$ \\ 
        IV  & $a+d \leq t \leq b+d$: & $p(t) = \int_{t-d} ^b {f(\tau )}  \cdot g(t - \tau )\,d\tau$ \\ 
        V   & $b+d < t$:             & $p(t) = 0$  \\ \hline 
        \multicolumn{3}{l}{}\\
        \multicolumn{3}{l}{2. Fall: $c+b > a+d$} \\ \hline
        I   & $t < a+c$: & $p(t) = 0$ \\
	II  & $a+c \leq t \leq a+d$: & $p(t) = \int_a ^{t-c} {f(\tau )}  \cdot g(t - \tau )\,d\tau$  \\
        III & $a+d \leq t \leq b+c$: & $p(t) = \int_{t-d} ^{t-c} {f(\tau )}  \cdot g(t - \tau )\,d\tau$  \\
	IV  & $b+c \leq t \leq b+d$: & $p(t) = \int_{t-d} ^b {f(\tau )}  \cdot g(t - \tau )\,d\tau$ \\  
        V   & $b+d < t$:             & $p(t) = 0$  \\  \hline  
        \multicolumn{3}{l}{}\\
        \multicolumn{3}{l}{3. Fall: $c+b = a+d$} \\ \hline
        III & $a+d = t = b+c$: & $p(t) = p(a+d)$  \\ \hline
        \end{tabular} 
 \end{minipage}   \\
\end{tabular}


\section{Eigenschaften}

\begin{displaymath}
\begin{array}{lp{5em}l}
t \mapsto f(t)& & \omega \mapsto \overline{F(-\omega)}\\
t \mapsto f(-t)& & \omega \mapsto {F(-\omega)}\\
t \mapsto f(at)& & \omega \mapsto \frac{1}{\mid a \mid}F(\frac{\omega}{a})\\
t \mapsto f(t-t_0)& & \omega \mapsto F(\omega) e^{-jwt_0}\\
t \mapsto e^{j\omega_0 t}f(t)& & \omega \mapsto F(\omega-\omega_0)\\
t \mapsto F(t)& & \omega \mapsto 2\pi f(-\omega)\\
t \mapsto f^{(n)}(t)& & \omega \mapsto (j\omega)^{n}F(\omega)\\
t \mapsto (-jt)^{n}f(t)& & \omega \mapsto F^{(n)}(\omega)\\
t \mapsto \int\limits_{-\infty}^t{f(\tau)\,d\tau} & &\omega \mapsto \frac{1}{j\omega}F(\omega)
\end{array}
\end{displaymath}

\section{Fouriertransformationen mit Diracdelta}   \index{Fourier!Diracdelta}
                                                   \index{Diracdelta}
\begin{displaymath}
\begin{array}{lp{5em}l}
Funktion                   & & Fourier-Transformierte \\
t \mapsto \delta(t)        & & \omega \mapsto 1 \\
t \mapsto 1                & & \omega \mapsto 2 \pi \delta(\omega) \\
t \mapsto \delta(t-t_0)    & & \omega \mapsto e^{-j \omega t_0}\\
t \mapsto e^{j \omega_0 t} & & \omega \mapsto 2 \pi \delta(\omega - \omega_0) \\
t \mapsto \sin(\omega_0 t) & & \omega \mapsto j \pi (\delta(\omega + \omega_0) - \delta(\omega - \omega_0))\\
t \mapsto \cos(\omega_0 t) & & \omega \mapsto \pi (\delta(\omega + \omega_0) + \delta(\omega - \omega_0))\\
t \mapsto \delta^{(n)}(t)  & & \omega \mapsto (j\omega)^{n}\\
t \mapsto sign(t)          & & \omega \mapsto \frac{2}{j \omega}\\
t \mapsto \frac{1}{\pi t}      & & \omega \mapsto -j \pi sign(\omega) \\
us                         & & \omega \mapsto \frac{1}{j \omega} +\pi \delta(\omega)
\end{array}
\end{displaymath}

Faltung mit Dirac:\\
\[(f(t) \ast \delta(t_0)) = \int_{-\infty}^{\infty} f(t) \delta(t_0 -t) dt = f(t_0) \]
\vfill

\section{Fouriertransformationen wichtiger Impulse} \index{Fourier!Impulse}
							\index{Impulse}
  \begin{tabular}[t]{ll}				\index{Fourier!Rechteck-Impuls}
    \begin{minipage}[t][\totalheight][c]{0.4\textwidth}
	Rechteckimpuls\\
    	\input{Mathe/rechteckPuls.pstex_t}
  \end{minipage} & \begin{minipage}[t][\totalheight][c]{0.4\textwidth}\vspace{1cm}
	\[S(\omega) = h T \frac{\sin \left( \frac{T \omega}{2}\right) } {\left( \frac{T \omega}{2}\right)}\]
  \end{minipage}
  \end{tabular} \\  					\index{Fourier!Dreieck-Impuls}
  \begin{tabular}[t]{ll}
    \begin{minipage}[t][\totalheight][c]{0.4\textwidth}
	Dreieckimpuls\\
    	\input{Mathe/dreieckImpuls.pstex_t}
  \end{minipage} & \begin{minipage}[t][\totalheight][c]{0.4\textwidth}\vspace{1cm}
	\[ S(\omega) = \frac{hT}{2} \left[ \frac{\sin \left( \frac{T \omega}{4}\right)  }{\frac{T \omega}{4}} \right] ^2 \]
  \end{minipage}
  \end{tabular}\\					\index{Fourier!Cosinus-Impuls}
  \begin{tabular}[t]{ll}
    \begin{minipage}[t][\totalheight][c]{0.4\textwidth}
	Cosinusimpuls\\
    	\input{Mathe/cosinusImpuls.pstex_t}
  \end{minipage} & \begin{minipage}[t][\totalheight][c]{0.4\textwidth}\vspace{1cm}
	\[ S(\omega) = \frac{2hT}{\pi} \frac{\cos \left( \frac{T \omega}{2}\right)  }{1-\left( \frac{T \omega}{\pi}\right) ^2} \]
  \end{minipage}
  \end{tabular}\\ 					\index{Fourier!Cosine-Impuls}
  \begin{tabular}[t]{ll}
    \begin{minipage}[t][\totalheight][c]{0.4\textwidth}
	Raised-Cosine-Impuls\\
    	\input{Mathe/raisedCosImp.pstex_t}
  \end{minipage} & \begin{minipage}[t][\totalheight][c]{0.4\textwidth}\vspace{1cm}
	\[ S(\omega) = \frac{hT}{2} \frac{\sin \left( \frac{T \omega}{2}\right)  }{\frac{T \omega}{2}\left[ 1- \left( \frac{T \omega}{2 \pi}\right)^2 \right] } \]
  \end{minipage}
  \end{tabular}\\					\index{Fourier!Cosine-Impuls}
  \begin{tabular}[t]{ll}
    \begin{minipage}[t][\totalheight][c]{0.4\textwidth}
	Gauss-Impuls\\
    	\input{Mathe/gaussImp.pstex_t}
  \end{minipage} & \begin{minipage}[t][\totalheight][c]{0.4\textwidth}\vspace{1cm}
	\[ S(\omega) = h \tau \sqrt{\pi} e^{\frac{-\omega^2\tau ^2}{4}} \]
  \end{minipage}
  \end{tabular}\\

\chapter{Laplace}         \index{Laplace}
\begin{picture}(60,25)
\thicklines
\put(10,0){$F$}
\put(10,20){$f$}
\put(40,0){$\tilde{F}=F \cdot G$}
\put(40,20){$\tilde{f}=f*g$}
\put(7,10){${\cal L}$}
\put(42,10){${\cal L}^{-1}$}
\put(13,1){\vector(1,0){26}}
\put(11,18){\vector(0,-1){14}}
\put(13,21){\vector(1,0){26}}
\put(41,4){\vector(0,1){14}}
\put(62,10){$G=\frac{1}{{\mathrm cp}(s)}$}
\end{picture}

\section{Laplacetransformation}

\[{\cal L}\,(f(t)) = F(s),\quad \quad {\cal L}^{ - 1} \,(F(s)) = f(t),\quad \quad s \in {\mathbb{C}}\]
\[F(s) = \int\limits_0^\infty  {f(t)\,} e^{ - st} \,dt\]
\[f(t) = \frac{1}{{2\pi j}}\int\limits_{x - j\infty }^{x + j\infty } {F(s)} \,e^{st} \,ds,\quad {\mathrm falls}\;t \ge 0\]
\[f(t) = 0,\quad {\mathrm falls}\;t < 0\]

\vfill

\section{Rechenregeln}                          \index{Laplacetransformation!Rechenregeln}

\begin{displaymath}
\begin{array}{lll}

t \mapsto f(at) && s \mapsto \frac{1}{a}F\left( {\frac{s}{a}} \right) \quad a > 0 \\

t \mapsto \frac{1}{a}f\left(\frac{t}{a}\right) && (s \mapsto F(a\,s)) \\
t \mapsto {u(t - a) \cdot f(t - a)} && s \mapsto e^{ - as} F(s) \quad a > 0 \\
t \mapsto f(t + a)) && s \mapsto e^{as} \left( {F(s) - \int\limits_0^a {f(t)\,e^{ - st} \,dt} }  \right)\quad a > 0 \\

t \mapsto e^{ - bt} f(t) && s \mapsto F(s + b) \quad c \in {\mathbb{C}} \\
t \mapsto f'(t) && s \mapsto s\,F(s) - f(0)\\
t \mapsto f^{(2)}(t) && s \mapsto s^2 F(s) - s\,f(0) - f'(0)\\

t \mapsto f^{(3)}(t) && s \mapsto s^3 F(s) - s^2 f(0) - sf'(0) - f^{(2)}(0)\\
t \mapsto f^{(n)}(t) && s \mapsto s^n F(s) - \sum\limits_{k = 0}^{n - 1} {s^{n-1-k} } f^{(k)} (0)\\
t \mapsto - t\,f(t) && s \mapsto F'(s)\\
t \mapsto + t^2 f(t) && s \mapsto F^{(2)}(s)\\
t \mapsto - t^3 f(t) && s \mapsto F^{(3)}(s)\\
t \mapsto \left( { - 1} \right)^n t^n f(t) && s \mapsto F^{(n)}(s)\\
t \mapsto \int\limits_0^t f (\tau )\,d\tau  && s \mapsto \frac{1}{s}F(s)\\

\end{array}
\end{displaymath}

\section{Spezielle Laplacetransformationen}             \index{Laplacetransformation!Spezielle}

\[{\cal L}\,(\delta (t)) = 1\]
\[{\cal L}\,(u(t)) = \frac{1}{s} \quad {\mathop{\mathrm Re}\nolimits} \,(s) > 0\]
\[{\cal L}\,(e^{at})  = \frac{1}{{s - a}}\quad {\mathop{\mathrm Re}\nolimits} \,(s) > {\mathop{\mathrm Re}\nolimits} \,(a)\]
\[{\cal L}\,(t^n)  = \frac{{n!}}{{s^{n + 1} }}\]
\[{\cal L}\,(t^n e^{at})  = \frac{{n!}}{{\left( {s - a} \right)^{n + 1} }}\]

\[{\cal L}(\sin at) = \frac{a}{{s^2  + a^2 }}\]
\[{\cal L}(\cos at) = \frac{s}{{s^2  + a^2 }}\]
\[{\cal L}\,\left(\frac{1}{d}e^{ct} \sin ct\right) = \frac{1}{{\left( {s - c} \right)^2  + d^2 }}\]
\[{\cal L}\,\left(e^{ct} \left( {\frac{c}{d}\sin dt + \cos dt} \right)\right) = \frac{s}{{\left( {s - c} \right)^2  + d^2 }}\]


\section{Faltung}                               \index{Laplacetransformation!Faltung}
\[\left( {f * g} \right)\left( t \right) = \int\limits_0^t {f(\tau )}  \cdot g(t - \tau )\;d\tau \]
\[f * g = g * f\quad \quad f(t) = g(t) = 0\quad {\mathrm falls}\;t < 0\]
\[{\cal L}\left( {f * g} \right) = {\cal L}(f) \cdot {\cal L}(g) = F \cdot G\]
Die Fallunterscheidung bei eingeschr�nkten Definitionsbereichen der Funktionen ist die selbe wie bei der Fourier-Theorie in Abschnitt \ref{faltungFourier} auf S. \pageref{faltungFourier} \\
\noindent \textbf{Beispiel:}\\
\noindent Geg:\quad$g(t)=u(t)-u(t-5)$ und $ f(t)=u(t-2)-u(t-6)$\\
Ges:\quad$\tilde{f}=(f*g)(t)$
\[ \tilde{f} = \int\limits_0^t {f(\tau)} \cdot g(t-\tau)\,d\tau\]
\[g(t-\tau) = 1 \quad \mbox{falls}\quad 0 \leq t - \tau\leq 5 \quad \Leftrightarrow \quad \tau \leq t \leq 5 + \tau\]
\input Mathe/Faltung.tex

\section{Periodische Funktionen}                \index{Laplacetransformation!Periodische Funktionen}
$f$ auf einer Periode $T$ vorgeben.

\[
F(s) = \int\limits_0^T {f(t)\,} e^{ - st} \,dt
\]

\noindent Periodische Fortsetzung:

\[
F_{per} (s) = F(s)\frac{1}{{1 - e^{ - sT} }}
\]



\chapter{Differentialgleichungen}               \index{Differential Gleichungen}
\section{1. Ordnung}                             \index{Differential Gleichungen!1.Ordnung}
\subsection{Homogene} \label{homogeneDgl}        \index{Differential Gleichungen!Homogene}
\subsubsection{Separierbar}

Praktisches Vorgehen beim L"osen der separierbaren Differentialgleichungen:
\[y' = \frac{g(x)}{h(y)} \quad \Leftrightarrow \quad \frac{dy}{dx} = \frac{g(x)}{h(y)}\]
\[\Leftrightarrow \quad h(y)\,dy = g(x)\,dx \quad \Leftrightarrow \quad \int{h(y)\,dy} = \int{g(x)\,dx}\]
\[\Leftrightarrow \quad H(y) = G(x) + c\]


\noindent Wenn durch ein Ausdruck, der die unbekannte Funktion enth"alt zu dividieren ist, so ist zu pr"ufen ob sein Verschwinden eine L"osung der DGL ergibt.


\subsubsection{Substitution}

Gegeben:
 \[\quad y'(x) = (x + y(x))^2 \]
Substitution:
\[\quad z = x + y(x) \Rightarrow z' = 1 + y'(x) \quad \Leftrightarrow \quad y'(x) = z' - 1\]
Einsetzten:
\[\quad z' - 1 = z^2 \quad  \Leftrightarrow  \quad \frac{dz}{dx} - 1 = z^2 \]
\[\Leftrightarrow \quad \frac{1}{1 - z^2}\,dz = dx \quad \Rightarrow \quad \mbox{separierbar}\]

\subsection{Partikul"are}                        \index{Differential Gleichungen!Partikul"are}

DGL:\quad $y' + y = q$

\subsubsection{Ansatz}

\noindent Ansatz f�r partikul�re L�sung: <<�hnlich>> wie die St�rfunktion ($q$), jedoch nicht in der homogenen L�sung enthalten.

\begin{tabular}{ll}
  St�rfunktion & Ansatz \\ \hline
  $\sin t$, $\cos t$ & $a \sin t + b \cos t$ \\
  $e^{-t}$ & $a\,e^{-t}$ \\
  $t\,e^{-t}$ & $ a\,e^{-t} + bt\,e^{-t}$ \\
  $t$ & $at+b$ \\   
\end{tabular}\\
Ansatz in DGL einsetzten und Koeffizientenvergleich durchf"uhren.

\subsubsection{Variation der Konstanten}
Homogene Lsg:\quad$y_h = c\,p(x)$\\
Ansatz:\quad$y_p = g(x)\,p(x)$ \hspace{2cm} (c wird durch $ g(x) $ ersetzt)  \\
Ansatz in DGL einsetzten und nach $g(x)$ aufl"osen



\subsection{L"osung}

Gesamtl"osungsmenge:\quad$y = y_h + y_p$


\section{H"ohere Ordnung}              \index{Differential Gleichungen!H"ohere Ordnung}

\subsection{Homogen, linear mit konstanten Koeffizienten}          \index{Differential Gleichungen!Homogen, linear, konst}

DGL:\quad $y^{(4)} + 6 y^{(3)} + 22 y'' + 30 y' + 13 y = 0$\\
$\Rightarrow$ \quad charakteristisches Polynom: $p(t) = t^4 + 6 t^3 + 22 t^2 + 30 t + 13$ 
\[\Leftrightarrow \quad p(t) = (t+1)^2(t+2-3j)(t+2+3j)\]
\[\mathbb{N}(p) = \{-1; -1; -2+3j; -2-3j \} \]
\noindent Aus den Nullstellen des charakteristischen Polynoms ergeben sich die L"osungen.\\
\noindent Ordnung DGL = Anzahl L"osungen
\[y_1(t) = e^{-t} \quad \quad y_2(t) = t\,e^{-t} \quad \quad y_3(t) = e^{t(-2+3j)} \quad \quad y_4 = e^{t(-2 - 3j)}\]
\noindent Linearkombinationen aus L�sungen komplexer Nullstellen ergibt reelle L"osungen:
\[\frac{1}{2}\left( {y_{3}(t) + y_{4}(t) } \right) = e^{-2t} \cos 3t \] 
\[\frac{1}{{2j}}\left( {y_{3}(t)  - y_{4}(t) } \right) = e^{-2t} \sin 3t\]

\subsection{Partikul"are}                        \index{Differential Gleichungen!Partikul"are}

\subsubsection{Ansatz}
\noindent $\Rightarrow$ Siehe \ref{homogeneDgl} Homogene S. \pageref{homogeneDgl}

\subsubsection{Variation der Konstanten}
St"orfunktion: \quad $q(x)$ \\
Homogene Lsg:\quad $y_1(t)\quad y_2(t)$ \\
Ansatz:\quad $y_p = g_1(t)\,y_1(t) + g_2(t)\,y_2(t)$ \\
$\Rightarrow \quad$ Gleichungssystem:
\begin{displaymath}
\begin{array}{l}
g_1'(t)\,y_1(t) + g_2'(t)\,y_2(t) = 0 \\
g_1'(t)\,y_1'(t) + g_2'(t)\,y_2'(t) = q(x) \\
\end{array}
\end{displaymath}
Dieses Gleichungsystem liefert $g_1(t)$ und $g_2(t)$ 

\section{Laplace}
\subsection{Lineare �bertragung}                \index{Laplace!Lineare �bertragung}

Uebertragungsfunktion:\quad $G(s) = \frac{1}{{{\mathop{\mathrm cp}\nolimits} (s)}}$\\
Stossantwort:\quad $g(t) = {\cal L}^{-1}(G(s)) = \tilde{u}'$\\
wobei cp = Charakteristisches Polynom und $\tilde{u}$ = Sprungantwort
\[y(0) = 0,\;y'(0) = 0,\;y^{(2)}(0) = 0,\; \ldots ,\;y^{(n)}(0) = 0\]
\[a_n y^{(n)} + a_{n - 1} y^{(n - 1)} +  \ldots  + a_1 y' + a_0 y = q\]
\[ \Downarrow {\cal L}\]
\[Y(s) \cdot {\mathop{\mathrm cp}\nolimits} (s) = F(s)\quad  \Leftrightarrow \quad Y(s) = \frac{{F(s)}}{{{\mathop{\mathrm cp}\nolimits} (s)}} = F(s) \cdot G(s)\]
\[ \Downarrow {\cal L}^{ - 1} \]
\[y(t) = \left( {f * g} \right)\left( t \right)\]
\noindent \textbf{Beispiel:}
\[y^{(2)} + 5y' + 6y = u,\quad \quad y(0) = 0,\quad y'(0) = 0\]
\[ \Downarrow {\cal L}\]
\[Y(s) \cdot \left( {s^2  + 5s + 6} \right) = Y(s) \cdot \left( {s + 2} \right)\left( {s + 3} \right) = \frac{1}{s}\]
\[Y(s) = \frac{1}{{s\left( {s + 2} \right)\left( {s + 3} \right)}} = \frac{\alpha }{s} + \frac{\beta }{{s + 2}} + \frac{\gamma }{{s + 3}}\]
\[1 = \alpha \left( {s + 2} \right)\left( {s + 3} \right) + \beta \left( {s + 3} \right)s + \gamma \left( {s + 2} \right)s\]

\[
\begin{array}{llll}
  s=0:  & 1=6\alpha & \Rightarrow & \alpha=\frac{1}{6}\\
  s=-2: & 1=-2\beta & \Rightarrow & \beta=-\frac{1}{2}\\
  s=-3: & 1=3\gamma & \Rightarrow & \gamma=\frac{1}{3}
\end{array}
\]

\[Y(s) = \frac{1}{6}\frac{1}{s} - \frac{1}{2}\frac{1}{{s + 2}} + \frac{1}{3}\frac{1}{{s + 3}}\]
\[ \Downarrow {\cal L}^{ - 1} \]
\[y(t) = \frac{1}{6}u(t) - \frac{1}{2}e^{ - 2t} u(t) + \frac{1}{3}e^{ - 3t} u(t)\]
\[y(t) = \left( {\frac{1}{6} - \frac{1}{2}e^{ - 2t}  + \frac{1}{3}e^{ - 3t} } \right)u(t)\]

\subsection{Nichtlineare �bertragung}           \index{Laplace!Nichtlineare �bertragung}

\noindent \textbf{Beispiel:}
\noindent Geg:\quad $g(t) = 1 - \cos t$\\
Ges:\quad $\tilde{v}$ auf $v=\sin t$ \quad \\
mit $\tilde{v}''(0)=1,\quad \tilde{v}'(0)=0,\quad \tilde{v}(0)=0$\\  
\[g(t) = 1 - \cos t\]
\[ \Downarrow {\cal L}\]
\[G(s) = \frac{1}{s} - \frac{s}{s^2 + 1} = \frac{1}{s^3 + s} = \frac{1}{cp(s)}\]
\[\Rightarrow \quad \mbox{DGL:}\quad y^{(3)} + y' = \sin t\]
\[ \Downarrow {\cal L}\]
\[s^3Y(s) - 1 +s\,Y(s) = \frac{1}{s^2 +1} \quad \Leftrightarrow \quad Y(s)(s^3 + s) = \frac{1}{s^2 + 1} + 1\]
\[\Leftrightarrow \quad Y(s) = \frac{1}{s^3 + s}\,\frac{1}{s^2 + 1} + \frac{1}{s^3 + 1}\]
\[ \Downarrow {\cal L}^{ - 1} \]
\[y(t) = (g * \sin)(t) + g(t)\]
\vfill

\section{�bersicht Laplace und Fourier}

\input{Mathe/fourLapUeber.pstex_t} 


\chapter{Funktionsdiskussion}                                 \index{Funktionsdiskussion}



\section{Funktionen mit einer Variablen}                 \index{Funktionsdiskussion!Funktionen mit einer Variablen}

\subsection{Zu beantwortende Fragen} \index{Kurvendiskussion}
                                 \index{Funktionsdiskussion}

\begin{enumerate}
\item Definitiondbereich $ D(f) $
\item Bild von $ f $
\item Hat der Graph von $ f $, $ G(f) $ Symmetrien? \\
Gerade $ f(-x) = f(x) $ oder Ungerade $ f(-x) = -f(x) $
\item Gibt es Polstellen? 
\item Gibt es Gebiete der Koordinatenebene wo der Graph keine Punkte hat? \\ (Achtung beim k�rzen)
\item Gibt es Schranken f�r die Funktionswerte?
\item Welches sind die Nullstellen von $ f $? 
\item Welches sind die Nullstellen der Ableitungen von $ f $?
\item Wo steigt $ f $ , wo f�llt $ f $?
\item Gibt es Grenzwerte f�r Argumente gegen $ \pm \infty $?
\item Gibt es Asymptoten?
\begin{footnotesize} \[ m = \lim _{|x| \rightarrow \infty} \left( \frac{f(x)}{x}\right) \qquad q = \lim _{|x| \rightarrow \infty} \left( f(x) - m x \right) \]
Asymptote: $ mx + q $ \\ \\
Bei Br�chen mit Polynomen ergibt eine Division mit Rest die Asymptote:\\
Beispiel: 
\[ (x^3 - 4x^2 - 17x + 60) \div (x^2 -4) = \underbrace{x - 4}_{Assymptote} + \underbrace{\frac{44-13x}{x^2-4}}_{Rest} \] \end{footnotesize}

Die Nullstellen des Z�hlerpolynoms im Rest ergeben die Schnittpunkte zwischen der Asymptote und der Funktion.
\item Gibt es absolute Maximal- oder Minimalstellen?
\end{enumerate} 

\subsection{Gerade (2-Punkte-Form)}

\[ y = \frac{y_2 - y_1}{x_2 - x_1}(x - x_1) + y_1\]

\subsection{Abstand eines Punktes von einer Geraden}
\noindent Gegeben: Gerade $Ax + By + C = 0$, Punkt $P = (p_1,p_2)$
\[d = \left|\frac{Ap_1 + Bp_2 + C}{\sqrt{A^2 + B^2}}\right| \quad (A^2 + B^2 \not= 0)\]

\section{Funktionen mit mehreren Variablen}                 \index{Funktionsdiskussion!Funktionen mit mehreren Variablen}

\subsection{Bezeichnungen}                             \index{Funktionsdiskussion!Bezeichnungen}
\[f_1(x,y) = \frac{\partial f}{\partial x}\]
\[f_2(x,y) = \frac{\partial f}{\partial y}\]

\subsubsection{Richtungsvektoren an die Parameterlinien}
Richtungsvektor an die Abszissenlinie:\quad $(1,0,f_1(x,y))$\\
Richtungsvektor an die Ordinatenlinie:\quad $(0,1,f_2(x,y))$

\subsubsection{Tangentialebene}
\[\varepsilon:\quad\vec{p}=(p_1,p_2,p_3)=(x,y,f(x,y))+\alpha (1,0,f_1(x,y))+\beta (0,1,f_2(x,y))\]
\[\vec{n_\varepsilon}=(f_1(x,y),f_2(x,y),-1)\]

\subsubsection{Gradient}
Wir betrachten die Funktion $f:(x_1,x_2)\mapsto f(x_1,x_2)$. Sie ist in einer gewissen Umgebung U von $(x_0,y_0)$ definiert.\\
Die Richtung des st"arksten Anstiegs von $f$ in $(x_0,y_0)$ ist
\[\hbox{grad}f(x_0,y_0)=(f_1(x_0,y_0),f_2(x_0,y_0)) = \vec{v}\]
($\Rightarrow$ Richtung der Fallgeraden in der Grundrissebene)\\
Richtungsvektor der Fallgerade der Tangentialebene:
\[(f_1(x_0,y_0),f_2(x_0,y_0),f_1(x_0,y_0)^2 + f_2(x_0,y_0)^2)\]

\subsubsection{Richtungsableitung}
\[D_{\vec{v}} f(x_0,y_0) = \hbox{grad}f(x_0,y_0)\cdot\vec{e_v}\]
wobei $\vec{e_v}$ der Einheitsvektor in Richtung $\vec{v}$ ist

\subsubsection{Totales Differential}
\[df = h \cdot f_1(x,y) + k \cdot f_2(x,y)\]
\noindent wobei h und k die Inkremente sind 

\subsubsection{Kettenregel}
\noindent Vollst"andig differenzierbare Funktionen:
\[f:(x_1,x_2) \mapsto f(x_1,x_2)\]
\[u:(y_1,y_2) \mapsto u(y_1,y_2)\]
\[v:(y_1,y_2) \mapsto v(y_1,y_2)\]
\[\tilde{f}:(y_1,y_2) \mapsto f(u(y_1,y_2),v(y_1,y_2))\]
\noindent Dann sind
\[\tilde{f_1}(y_1,y_2)=f_1(u(y_1,y_2),v(y_1,y_2))\cdot u_1(y_1,y_2) + f_2(u(y_1,y_2),v(y_1,y_2))\cdot v_1(y_1,y_2)\]
\[\tilde{f_2}(y_1,y_2)=f_1(u(y_1,y_2),v(y_1,y_2))\cdot u_2(y_1,y_2) + f_2(u(y_1,y_2),v(y_1,y_2))\cdot v_2(y_1,y_2)\]

\section{Kegelschnitte}

\subsection{Kreis}

\[(x - x_0)^2 + (y - y_0)^2 = r^2\]
\[M = (x_0, y_0)\]

\subsection{Ellipse}

\[\frac{(x - x_0)^2}{a^2} + \frac{(y - y_0)^2}{b^2} = 1\]
\[M = (x_0, y_0)\]


\subsection{Hyperbel}

\[\frac{(x - x_0)^2}{a^2} - \frac{(y - y_0)^2}{b^2} = 1\]
\[M = (x_0, y_0)\]

\subsection{Parabel}

\[(y - y_0)^2 = 2p(x - x_0)\]
\[S = (x_0,y_0)\]


%\onecolumn

%%% Local Variables: 
%%% mode: latex
%%% TeX-master: "../FoSaHSR"
%%% End: 
