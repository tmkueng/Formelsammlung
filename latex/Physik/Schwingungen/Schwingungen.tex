%==========================================================
% Titel:    Physik, Schwingungen 
% Autor:    Adrian Freihofer
% Erstellt: 7.11.2001
% Ge�ndert:
%==========================================================


\chapter{Schwingungen}                                          \index{Schwingungen}


\section{Freie Schwingungen}                                    \index{Schwingungen!freie Schwingung}

\subsection{Unged�mpfte, harmonische Schwingung}

\index{Schwingungen!Unged�mpfte Schwingung}
\index{Schwingungen!Harmonische Schwingung}
                                                                
\Hauptbox{
  \Bildbox{
    \scriptsize Zeitfunktion: \normalsize
    % GNUPLOT: LaTeX picture
\setlength{\unitlength}{0.240900pt}
\ifx\plotpoint\undefined\newsavebox{\plotpoint}\fi
\sbox{\plotpoint}{\rule[-0.200pt]{0.400pt}{0.400pt}}%
\begin{picture}(600,360)(60,0)
\font\gnuplot=cmr10 at 10pt
\gnuplot
\scriptsize
\sbox{\plotpoint}{\rule[-0.200pt]{0.400pt}{0.400pt}}%
\put(100.0,82.0){\rule[-0.200pt]{4.818pt}{0.400pt}}
\put(80,82){\makebox(0,0)[r]{-5}}
\put(519.0,82.0){\rule[-0.200pt]{4.818pt}{0.400pt}}
\put(100.0,106.0){\rule[-0.200pt]{4.818pt}{0.400pt}}
%\put(80,106){\makebox(0,0)[r]{-4}}
\put(519.0,106.0){\rule[-0.200pt]{4.818pt}{0.400pt}}
\put(100.0,130.0){\rule[-0.200pt]{4.818pt}{0.400pt}}
\put(80,130){\makebox(0,0)[r]{-3}}
\put(519.0,130.0){\rule[-0.200pt]{4.818pt}{0.400pt}}
\put(100.0,153.0){\rule[-0.200pt]{4.818pt}{0.400pt}}
%\put(80,153){\makebox(0,0)[r]{-2}}
\put(519.0,153.0){\rule[-0.200pt]{4.818pt}{0.400pt}}
\put(100.0,177.0){\rule[-0.200pt]{4.818pt}{0.400pt}}
\put(80,177){\makebox(0,0)[r]{-1}}
\put(519.0,177.0){\rule[-0.200pt]{4.818pt}{0.400pt}}
\put(100.0,201.0){\rule[-0.200pt]{4.818pt}{0.400pt}}
%\put(80,201){\makebox(0,0)[r]{0}}
\put(519.0,201.0){\rule[-0.200pt]{4.818pt}{0.400pt}}
\put(100.0,225.0){\rule[-0.200pt]{4.818pt}{0.400pt}}
\put(80,225){\makebox(0,0)[r]{1}}
\put(519.0,225.0){\rule[-0.200pt]{4.818pt}{0.400pt}}
\put(100.0,249.0){\rule[-0.200pt]{4.818pt}{0.400pt}}
%\put(80,249){\makebox(0,0)[r]{2}}
\put(519.0,249.0){\rule[-0.200pt]{4.818pt}{0.400pt}}
\put(100.0,272.0){\rule[-0.200pt]{4.818pt}{0.400pt}}
\put(80,272){\makebox(0,0)[r]{3}}
\put(519.0,272.0){\rule[-0.200pt]{4.818pt}{0.400pt}}
\put(100.0,296.0){\rule[-0.200pt]{4.818pt}{0.400pt}}
%\put(80,296){\makebox(0,0)[r]{4}}
\put(519.0,296.0){\rule[-0.200pt]{4.818pt}{0.400pt}}
\put(100.0,320.0){\rule[-0.200pt]{4.818pt}{0.400pt}}
\put(80,320){\makebox(0,0)[r]{5}}
\put(519.0,320.0){\rule[-0.200pt]{4.818pt}{0.400pt}}
\put(100.0,82.0){\rule[-0.200pt]{0.400pt}{4.818pt}}
\put(100,41){\makebox(0,0){0}}
\put(100.0,300.0){\rule[-0.200pt]{0.400pt}{4.818pt}}
\put(210.0,82.0){\rule[-0.200pt]{0.400pt}{4.818pt}}
\put(210,41){\makebox(0,0){5}}
\put(210.0,300.0){\rule[-0.200pt]{0.400pt}{4.818pt}}
\put(319.0,82.0){\rule[-0.200pt]{0.400pt}{4.818pt}}
\put(319,41){\makebox(0,0){10}}
\put(319.0,300.0){\rule[-0.200pt]{0.400pt}{4.818pt}}
\put(429.0,82.0){\rule[-0.200pt]{0.400pt}{4.818pt}}
\put(429,41){\makebox(0,0){15}}
\put(429.0,300.0){\rule[-0.200pt]{0.400pt}{4.818pt}}
\put(539.0,82.0){\rule[-0.200pt]{0.400pt}{4.818pt}}
\put(539,41){\makebox(0,0){20}}
\put(539.0,300.0){\rule[-0.200pt]{0.400pt}{4.818pt}}
\put(100.0,82.0){\rule[-0.200pt]{105.755pt}{0.400pt}}
\put(539.0,82.0){\rule[-0.200pt]{0.400pt}{57.334pt}}
\put(100.0,320.0){\rule[-0.200pt]{105.755pt}{0.400pt}}
\put(100.0,82.0){\rule[-0.200pt]{0.400pt}{57.334pt}}

\put(100,201){\usebox{\plotpoint}}
\put(99.67,203){\rule{0.400pt}{0.482pt}}
\multiput(99.17,203.00)(1.000,1.000){2}{\rule{0.400pt}{0.241pt}}
\put(100.0,201.0){\rule[-0.200pt]{0.400pt}{0.482pt}}
\put(101,207.67){\rule{0.241pt}{0.400pt}}
\multiput(101.00,207.17)(0.500,1.000){2}{\rule{0.120pt}{0.400pt}}
\put(101.0,205.0){\rule[-0.200pt]{0.400pt}{0.723pt}}
\put(102,213.67){\rule{0.241pt}{0.400pt}}
\multiput(102.00,213.17)(0.500,1.000){2}{\rule{0.120pt}{0.400pt}}
\put(102.0,209.0){\rule[-0.200pt]{0.400pt}{1.204pt}}
\put(103,218.67){\rule{0.241pt}{0.400pt}}
\multiput(103.00,218.17)(0.500,1.000){2}{\rule{0.120pt}{0.400pt}}
\put(103.0,215.0){\rule[-0.200pt]{0.400pt}{0.964pt}}
\put(104,224.67){\rule{0.241pt}{0.400pt}}
\multiput(104.00,224.17)(0.500,1.000){2}{\rule{0.120pt}{0.400pt}}
\put(104.0,220.0){\rule[-0.200pt]{0.400pt}{1.204pt}}
\put(104.67,230){\rule{0.400pt}{0.482pt}}
\multiput(104.17,230.00)(1.000,1.000){2}{\rule{0.400pt}{0.241pt}}
\put(105.0,226.0){\rule[-0.200pt]{0.400pt}{0.964pt}}
\put(106,234.67){\rule{0.241pt}{0.400pt}}
\multiput(106.00,234.17)(0.500,1.000){2}{\rule{0.120pt}{0.400pt}}
\put(106.0,232.0){\rule[-0.200pt]{0.400pt}{0.723pt}}
\put(107,240.67){\rule{0.241pt}{0.400pt}}
\multiput(107.00,240.17)(0.500,1.000){2}{\rule{0.120pt}{0.400pt}}
\put(107.0,236.0){\rule[-0.200pt]{0.400pt}{1.204pt}}
\put(108,244.67){\rule{0.241pt}{0.400pt}}
\multiput(108.00,244.17)(0.500,1.000){2}{\rule{0.120pt}{0.400pt}}
\put(108.0,242.0){\rule[-0.200pt]{0.400pt}{0.723pt}}
\put(109,250.67){\rule{0.241pt}{0.400pt}}
\multiput(109.00,250.17)(0.500,1.000){2}{\rule{0.120pt}{0.400pt}}
\put(109.0,246.0){\rule[-0.200pt]{0.400pt}{1.204pt}}
\put(110,254.67){\rule{0.241pt}{0.400pt}}
\multiput(110.00,254.17)(0.500,1.000){2}{\rule{0.120pt}{0.400pt}}
\put(110.0,252.0){\rule[-0.200pt]{0.400pt}{0.723pt}}
\put(111,259.67){\rule{0.241pt}{0.400pt}}
\multiput(111.00,259.17)(0.500,1.000){2}{\rule{0.120pt}{0.400pt}}
\put(111.0,256.0){\rule[-0.200pt]{0.400pt}{0.964pt}}
\put(112,263.67){\rule{0.241pt}{0.400pt}}
\multiput(112.00,263.17)(0.500,1.000){2}{\rule{0.120pt}{0.400pt}}
\put(112.0,261.0){\rule[-0.200pt]{0.400pt}{0.723pt}}
\put(113,268.67){\rule{0.241pt}{0.400pt}}
\multiput(113.00,268.17)(0.500,1.000){2}{\rule{0.120pt}{0.400pt}}
\put(113.0,265.0){\rule[-0.200pt]{0.400pt}{0.964pt}}
\put(114,273.67){\rule{0.241pt}{0.400pt}}
\multiput(114.00,273.17)(0.500,1.000){2}{\rule{0.120pt}{0.400pt}}
\put(114.0,270.0){\rule[-0.200pt]{0.400pt}{0.964pt}}
\put(115,277.67){\rule{0.241pt}{0.400pt}}
\multiput(115.00,277.17)(0.500,1.000){2}{\rule{0.120pt}{0.400pt}}
\put(115.0,275.0){\rule[-0.200pt]{0.400pt}{0.723pt}}
\put(116,279){\usebox{\plotpoint}}
\put(116,281.67){\rule{0.241pt}{0.400pt}}
\multiput(116.00,281.17)(0.500,1.000){2}{\rule{0.120pt}{0.400pt}}
\put(116.0,279.0){\rule[-0.200pt]{0.400pt}{0.723pt}}
\put(117.0,283.0){\rule[-0.200pt]{0.400pt}{0.723pt}}
\put(117.0,286.0){\usebox{\plotpoint}}
\put(118.0,286.0){\rule[-0.200pt]{0.400pt}{0.964pt}}
\put(118.0,290.0){\usebox{\plotpoint}}
\put(119.0,290.0){\rule[-0.200pt]{0.400pt}{0.723pt}}
\put(119.0,293.0){\usebox{\plotpoint}}
\put(120,295.67){\rule{0.241pt}{0.400pt}}
\multiput(120.00,295.17)(0.500,1.000){2}{\rule{0.120pt}{0.400pt}}
\put(120.0,293.0){\rule[-0.200pt]{0.400pt}{0.723pt}}
\put(121,298.67){\rule{0.241pt}{0.400pt}}
\multiput(121.00,298.17)(0.500,1.000){2}{\rule{0.120pt}{0.400pt}}
\put(121.0,297.0){\rule[-0.200pt]{0.400pt}{0.482pt}}
\put(122,300){\usebox{\plotpoint}}
\put(122,301.67){\rule{0.241pt}{0.400pt}}
\multiput(122.00,301.17)(0.500,1.000){2}{\rule{0.120pt}{0.400pt}}
\put(122.0,300.0){\rule[-0.200pt]{0.400pt}{0.482pt}}
\put(123,304.67){\rule{0.241pt}{0.400pt}}
\multiput(123.00,304.17)(0.500,1.000){2}{\rule{0.120pt}{0.400pt}}
\put(123.0,303.0){\rule[-0.200pt]{0.400pt}{0.482pt}}
\put(124.0,306.0){\rule[-0.200pt]{0.400pt}{0.482pt}}
\put(124.0,308.0){\usebox{\plotpoint}}
\put(125,309.67){\rule{0.241pt}{0.400pt}}
\multiput(125.00,309.17)(0.500,1.000){2}{\rule{0.120pt}{0.400pt}}
\put(125.0,308.0){\rule[-0.200pt]{0.400pt}{0.482pt}}
\put(126,311){\usebox{\plotpoint}}
\put(126,311){\usebox{\plotpoint}}
\put(126.0,311.0){\usebox{\plotpoint}}
\put(126.0,312.0){\usebox{\plotpoint}}
\put(127.0,312.0){\rule[-0.200pt]{0.400pt}{0.482pt}}
\put(127.0,314.0){\usebox{\plotpoint}}
\put(128,314.67){\rule{0.241pt}{0.400pt}}
\multiput(128.00,314.17)(0.500,1.000){2}{\rule{0.120pt}{0.400pt}}
\put(128.0,314.0){\usebox{\plotpoint}}
\put(129,316){\usebox{\plotpoint}}
\put(129,316){\usebox{\plotpoint}}
\put(129.0,316.0){\usebox{\plotpoint}}
\put(129.0,317.0){\usebox{\plotpoint}}
\put(130.0,317.0){\usebox{\plotpoint}}
\put(130.0,318.0){\usebox{\plotpoint}}
\put(131.0,318.0){\usebox{\plotpoint}}
\put(131.0,319.0){\usebox{\plotpoint}}
\put(132.0,319.0){\usebox{\plotpoint}}
\put(136,318.67){\rule{0.241pt}{0.400pt}}
\multiput(136.00,319.17)(0.500,-1.000){2}{\rule{0.120pt}{0.400pt}}
\put(132.0,320.0){\rule[-0.200pt]{0.964pt}{0.400pt}}
\put(137,319){\usebox{\plotpoint}}
\put(137,319){\usebox{\plotpoint}}
\put(137,319){\usebox{\plotpoint}}
\put(137.0,319.0){\usebox{\plotpoint}}
\put(138.0,318.0){\usebox{\plotpoint}}
\put(138.0,318.0){\usebox{\plotpoint}}
\put(139.0,317.0){\usebox{\plotpoint}}
\put(139.0,317.0){\usebox{\plotpoint}}
\put(140,314.67){\rule{0.241pt}{0.400pt}}
\multiput(140.00,315.17)(0.500,-1.000){2}{\rule{0.120pt}{0.400pt}}
\put(140.0,316.0){\usebox{\plotpoint}}
\put(141,315){\usebox{\plotpoint}}
\put(141,315){\usebox{\plotpoint}}
\put(141.0,314.0){\usebox{\plotpoint}}
\put(141.0,314.0){\usebox{\plotpoint}}
\put(142.0,312.0){\rule[-0.200pt]{0.400pt}{0.482pt}}
\put(142.0,312.0){\usebox{\plotpoint}}
\put(143.0,310.0){\rule[-0.200pt]{0.400pt}{0.482pt}}
\put(143.0,310.0){\usebox{\plotpoint}}
\put(144.0,308.0){\rule[-0.200pt]{0.400pt}{0.482pt}}
\put(144.0,308.0){\usebox{\plotpoint}}
\put(145.0,305.0){\rule[-0.200pt]{0.400pt}{0.723pt}}
\put(145.0,305.0){\usebox{\plotpoint}}
\put(146,301.67){\rule{0.241pt}{0.400pt}}
\multiput(146.00,302.17)(0.500,-1.000){2}{\rule{0.120pt}{0.400pt}}
\put(146.0,303.0){\rule[-0.200pt]{0.400pt}{0.482pt}}
\put(147,302){\usebox{\plotpoint}}
\put(147,298.67){\rule{0.241pt}{0.400pt}}
\multiput(147.00,299.17)(0.500,-1.000){2}{\rule{0.120pt}{0.400pt}}
\put(147.0,300.0){\rule[-0.200pt]{0.400pt}{0.482pt}}
\put(148,299){\usebox{\plotpoint}}
\put(148.0,297.0){\rule[-0.200pt]{0.400pt}{0.482pt}}
\put(148.0,297.0){\usebox{\plotpoint}}
\put(149,292.67){\rule{0.241pt}{0.400pt}}
\multiput(149.00,293.17)(0.500,-1.000){2}{\rule{0.120pt}{0.400pt}}
\put(149.0,294.0){\rule[-0.200pt]{0.400pt}{0.723pt}}
\put(150,288.67){\rule{0.241pt}{0.400pt}}
\multiput(150.00,289.17)(0.500,-1.000){2}{\rule{0.120pt}{0.400pt}}
\put(150.0,290.0){\rule[-0.200pt]{0.400pt}{0.723pt}}
\put(151.0,286.0){\rule[-0.200pt]{0.400pt}{0.723pt}}
\put(151.0,286.0){\usebox{\plotpoint}}
\put(152,280.67){\rule{0.241pt}{0.400pt}}
\multiput(152.00,281.17)(0.500,-1.000){2}{\rule{0.120pt}{0.400pt}}
\put(152.0,282.0){\rule[-0.200pt]{0.400pt}{0.964pt}}
\put(153,277.67){\rule{0.241pt}{0.400pt}}
\multiput(153.00,278.17)(0.500,-1.000){2}{\rule{0.120pt}{0.400pt}}
\put(153.0,279.0){\rule[-0.200pt]{0.400pt}{0.482pt}}
\put(154,272.67){\rule{0.241pt}{0.400pt}}
\multiput(154.00,273.17)(0.500,-1.000){2}{\rule{0.120pt}{0.400pt}}
\put(154.0,274.0){\rule[-0.200pt]{0.400pt}{0.964pt}}
\put(155,268.67){\rule{0.241pt}{0.400pt}}
\multiput(155.00,269.17)(0.500,-1.000){2}{\rule{0.120pt}{0.400pt}}
\put(155.0,270.0){\rule[-0.200pt]{0.400pt}{0.723pt}}
\put(156,263.67){\rule{0.241pt}{0.400pt}}
\multiput(156.00,264.17)(0.500,-1.000){2}{\rule{0.120pt}{0.400pt}}
\put(156.0,265.0){\rule[-0.200pt]{0.400pt}{0.964pt}}
\put(157,259.67){\rule{0.241pt}{0.400pt}}
\multiput(157.00,260.17)(0.500,-1.000){2}{\rule{0.120pt}{0.400pt}}
\put(157.0,261.0){\rule[-0.200pt]{0.400pt}{0.723pt}}
\put(158,254.67){\rule{0.241pt}{0.400pt}}
\multiput(158.00,255.17)(0.500,-1.000){2}{\rule{0.120pt}{0.400pt}}
\put(158.0,256.0){\rule[-0.200pt]{0.400pt}{0.964pt}}
\put(159,249.67){\rule{0.241pt}{0.400pt}}
\multiput(159.00,250.17)(0.500,-1.000){2}{\rule{0.120pt}{0.400pt}}
\put(159.0,251.0){\rule[-0.200pt]{0.400pt}{0.964pt}}
\put(160,244.67){\rule{0.241pt}{0.400pt}}
\multiput(160.00,245.17)(0.500,-1.000){2}{\rule{0.120pt}{0.400pt}}
\put(160.0,246.0){\rule[-0.200pt]{0.400pt}{0.964pt}}
\put(161,239.67){\rule{0.241pt}{0.400pt}}
\multiput(161.00,240.17)(0.500,-1.000){2}{\rule{0.120pt}{0.400pt}}
\put(161.0,241.0){\rule[-0.200pt]{0.400pt}{0.964pt}}
\put(162,234.67){\rule{0.241pt}{0.400pt}}
\multiput(162.00,235.17)(0.500,-1.000){2}{\rule{0.120pt}{0.400pt}}
\put(162.0,236.0){\rule[-0.200pt]{0.400pt}{0.964pt}}
\put(162.67,229){\rule{0.400pt}{0.482pt}}
\multiput(162.17,230.00)(1.000,-1.000){2}{\rule{0.400pt}{0.241pt}}
\put(163.0,231.0){\rule[-0.200pt]{0.400pt}{0.964pt}}
\put(164,224.67){\rule{0.241pt}{0.400pt}}
\multiput(164.00,225.17)(0.500,-1.000){2}{\rule{0.120pt}{0.400pt}}
\put(164.0,226.0){\rule[-0.200pt]{0.400pt}{0.723pt}}
\put(165,218.67){\rule{0.241pt}{0.400pt}}
\multiput(165.00,219.17)(0.500,-1.000){2}{\rule{0.120pt}{0.400pt}}
\put(165.0,220.0){\rule[-0.200pt]{0.400pt}{1.204pt}}
\put(166,213.67){\rule{0.241pt}{0.400pt}}
\multiput(166.00,214.17)(0.500,-1.000){2}{\rule{0.120pt}{0.400pt}}
\put(166.0,215.0){\rule[-0.200pt]{0.400pt}{0.964pt}}
\put(166.67,208){\rule{0.400pt}{0.482pt}}
\multiput(166.17,209.00)(1.000,-1.000){2}{\rule{0.400pt}{0.241pt}}
\put(167.0,210.0){\rule[-0.200pt]{0.400pt}{0.964pt}}
\put(167.67,202){\rule{0.400pt}{0.482pt}}
\multiput(167.17,203.00)(1.000,-1.000){2}{\rule{0.400pt}{0.241pt}}
\put(168.0,204.0){\rule[-0.200pt]{0.400pt}{0.964pt}}
\put(169,197.67){\rule{0.241pt}{0.400pt}}
\multiput(169.00,198.17)(0.500,-1.000){2}{\rule{0.120pt}{0.400pt}}
\put(169.0,199.0){\rule[-0.200pt]{0.400pt}{0.723pt}}
\put(170,191.67){\rule{0.241pt}{0.400pt}}
\multiput(170.00,192.17)(0.500,-1.000){2}{\rule{0.120pt}{0.400pt}}
\put(170.0,193.0){\rule[-0.200pt]{0.400pt}{1.204pt}}
\put(171,186.67){\rule{0.241pt}{0.400pt}}
\multiput(171.00,187.17)(0.500,-1.000){2}{\rule{0.120pt}{0.400pt}}
\put(171.0,188.0){\rule[-0.200pt]{0.400pt}{0.964pt}}
\put(172,180.67){\rule{0.241pt}{0.400pt}}
\multiput(172.00,181.17)(0.500,-1.000){2}{\rule{0.120pt}{0.400pt}}
\put(172.0,182.0){\rule[-0.200pt]{0.400pt}{1.204pt}}
\put(172.67,176){\rule{0.400pt}{0.482pt}}
\multiput(172.17,177.00)(1.000,-1.000){2}{\rule{0.400pt}{0.241pt}}
\put(173.0,178.0){\rule[-0.200pt]{0.400pt}{0.723pt}}
\put(174,170.67){\rule{0.241pt}{0.400pt}}
\multiput(174.00,171.17)(0.500,-1.000){2}{\rule{0.120pt}{0.400pt}}
\put(174.0,172.0){\rule[-0.200pt]{0.400pt}{0.964pt}}
\put(175,165.67){\rule{0.241pt}{0.400pt}}
\multiput(175.00,166.17)(0.500,-1.000){2}{\rule{0.120pt}{0.400pt}}
\put(175.0,167.0){\rule[-0.200pt]{0.400pt}{0.964pt}}
\put(176,159.67){\rule{0.241pt}{0.400pt}}
\multiput(176.00,160.17)(0.500,-1.000){2}{\rule{0.120pt}{0.400pt}}
\put(176.0,161.0){\rule[-0.200pt]{0.400pt}{1.204pt}}
\put(177,154.67){\rule{0.241pt}{0.400pt}}
\multiput(177.00,155.17)(0.500,-1.000){2}{\rule{0.120pt}{0.400pt}}
\put(177.0,156.0){\rule[-0.200pt]{0.400pt}{0.964pt}}
\put(177.67,150){\rule{0.400pt}{0.482pt}}
\multiput(177.17,151.00)(1.000,-1.000){2}{\rule{0.400pt}{0.241pt}}
\put(178.0,152.0){\rule[-0.200pt]{0.400pt}{0.723pt}}
\put(179,144.67){\rule{0.241pt}{0.400pt}}
\multiput(179.00,145.17)(0.500,-1.000){2}{\rule{0.120pt}{0.400pt}}
\put(179.0,146.0){\rule[-0.200pt]{0.400pt}{0.964pt}}
\put(180,140.67){\rule{0.241pt}{0.400pt}}
\multiput(180.00,141.17)(0.500,-1.000){2}{\rule{0.120pt}{0.400pt}}
\put(180.0,142.0){\rule[-0.200pt]{0.400pt}{0.723pt}}
\put(181,135.67){\rule{0.241pt}{0.400pt}}
\multiput(181.00,136.17)(0.500,-1.000){2}{\rule{0.120pt}{0.400pt}}
\put(181.0,137.0){\rule[-0.200pt]{0.400pt}{0.964pt}}
\put(182,131.67){\rule{0.241pt}{0.400pt}}
\multiput(182.00,132.17)(0.500,-1.000){2}{\rule{0.120pt}{0.400pt}}
\put(182.0,133.0){\rule[-0.200pt]{0.400pt}{0.723pt}}
\put(183,126.67){\rule{0.241pt}{0.400pt}}
\multiput(183.00,127.17)(0.500,-1.000){2}{\rule{0.120pt}{0.400pt}}
\put(183.0,128.0){\rule[-0.200pt]{0.400pt}{0.964pt}}
\put(184.0,124.0){\rule[-0.200pt]{0.400pt}{0.723pt}}
\put(184.0,124.0){\usebox{\plotpoint}}
\put(185,118.67){\rule{0.241pt}{0.400pt}}
\multiput(185.00,119.17)(0.500,-1.000){2}{\rule{0.120pt}{0.400pt}}
\put(185.0,120.0){\rule[-0.200pt]{0.400pt}{0.964pt}}
\put(186,114.67){\rule{0.241pt}{0.400pt}}
\multiput(186.00,115.17)(0.500,-1.000){2}{\rule{0.120pt}{0.400pt}}
\put(186.0,116.0){\rule[-0.200pt]{0.400pt}{0.723pt}}
\put(187,111.67){\rule{0.241pt}{0.400pt}}
\multiput(187.00,112.17)(0.500,-1.000){2}{\rule{0.120pt}{0.400pt}}
\put(187.0,113.0){\rule[-0.200pt]{0.400pt}{0.482pt}}
\put(188,107.67){\rule{0.241pt}{0.400pt}}
\multiput(188.00,108.17)(0.500,-1.000){2}{\rule{0.120pt}{0.400pt}}
\put(188.0,109.0){\rule[-0.200pt]{0.400pt}{0.723pt}}
\put(189,104.67){\rule{0.241pt}{0.400pt}}
\multiput(189.00,105.17)(0.500,-1.000){2}{\rule{0.120pt}{0.400pt}}
\put(189.0,106.0){\rule[-0.200pt]{0.400pt}{0.482pt}}
\put(190.0,102.0){\rule[-0.200pt]{0.400pt}{0.723pt}}
\put(190.0,102.0){\usebox{\plotpoint}}
\put(191,98.67){\rule{0.241pt}{0.400pt}}
\multiput(191.00,99.17)(0.500,-1.000){2}{\rule{0.120pt}{0.400pt}}
\put(191.0,100.0){\rule[-0.200pt]{0.400pt}{0.482pt}}
\put(192,95.67){\rule{0.241pt}{0.400pt}}
\multiput(192.00,96.17)(0.500,-1.000){2}{\rule{0.120pt}{0.400pt}}
\put(192.0,97.0){\rule[-0.200pt]{0.400pt}{0.482pt}}
\put(193,96){\usebox{\plotpoint}}
\put(193.0,94.0){\rule[-0.200pt]{0.400pt}{0.482pt}}
\put(193.0,94.0){\usebox{\plotpoint}}
\put(194.0,92.0){\rule[-0.200pt]{0.400pt}{0.482pt}}
\put(194.0,92.0){\usebox{\plotpoint}}
\put(195,88.67){\rule{0.241pt}{0.400pt}}
\multiput(195.00,89.17)(0.500,-1.000){2}{\rule{0.120pt}{0.400pt}}
\put(195.0,90.0){\rule[-0.200pt]{0.400pt}{0.482pt}}
\put(196,89){\usebox{\plotpoint}}
\put(196,89){\usebox{\plotpoint}}
\put(196.0,88.0){\usebox{\plotpoint}}
\put(196.0,88.0){\usebox{\plotpoint}}
\put(197.0,86.0){\rule[-0.200pt]{0.400pt}{0.482pt}}
\put(197.0,86.0){\usebox{\plotpoint}}
\put(198.0,85.0){\usebox{\plotpoint}}
\put(198.0,85.0){\usebox{\plotpoint}}
\put(199.0,84.0){\usebox{\plotpoint}}
\put(199.0,84.0){\usebox{\plotpoint}}
\put(200.0,83.0){\usebox{\plotpoint}}
\put(201,81.67){\rule{0.241pt}{0.400pt}}
\multiput(201.00,82.17)(0.500,-1.000){2}{\rule{0.120pt}{0.400pt}}
\put(200.0,83.0){\usebox{\plotpoint}}
\put(202,82){\usebox{\plotpoint}}
\put(202,82){\usebox{\plotpoint}}
\put(202,82){\usebox{\plotpoint}}
\put(205,81.67){\rule{0.241pt}{0.400pt}}
\multiput(205.00,81.17)(0.500,1.000){2}{\rule{0.120pt}{0.400pt}}
\put(202.0,82.0){\rule[-0.200pt]{0.723pt}{0.400pt}}
\put(206,83){\usebox{\plotpoint}}
\put(206,83){\usebox{\plotpoint}}
\put(206,83){\usebox{\plotpoint}}
\put(206,83){\usebox{\plotpoint}}
\put(206.0,83.0){\usebox{\plotpoint}}
\put(207.0,83.0){\usebox{\plotpoint}}
\put(207.0,84.0){\usebox{\plotpoint}}
\put(208.0,84.0){\usebox{\plotpoint}}
\put(208.0,85.0){\usebox{\plotpoint}}
\put(209,85.67){\rule{0.241pt}{0.400pt}}
\multiput(209.00,85.17)(0.500,1.000){2}{\rule{0.120pt}{0.400pt}}
\put(209.0,85.0){\usebox{\plotpoint}}
\put(210,87){\usebox{\plotpoint}}
\put(210,87){\usebox{\plotpoint}}
\put(210.0,87.0){\usebox{\plotpoint}}
\put(210.0,88.0){\usebox{\plotpoint}}
\put(211.0,88.0){\rule[-0.200pt]{0.400pt}{0.482pt}}
\put(211.0,90.0){\usebox{\plotpoint}}
\put(212.0,90.0){\rule[-0.200pt]{0.400pt}{0.482pt}}
\put(212.0,92.0){\usebox{\plotpoint}}
\put(213,93.67){\rule{0.241pt}{0.400pt}}
\multiput(213.00,93.17)(0.500,1.000){2}{\rule{0.120pt}{0.400pt}}
\put(213.0,92.0){\rule[-0.200pt]{0.400pt}{0.482pt}}
\put(214,95){\usebox{\plotpoint}}
\put(214,95.67){\rule{0.241pt}{0.400pt}}
\multiput(214.00,95.17)(0.500,1.000){2}{\rule{0.120pt}{0.400pt}}
\put(214.0,95.0){\usebox{\plotpoint}}
\put(215,98.67){\rule{0.241pt}{0.400pt}}
\multiput(215.00,98.17)(0.500,1.000){2}{\rule{0.120pt}{0.400pt}}
\put(215.0,97.0){\rule[-0.200pt]{0.400pt}{0.482pt}}
\put(216,101.67){\rule{0.241pt}{0.400pt}}
\multiput(216.00,101.17)(0.500,1.000){2}{\rule{0.120pt}{0.400pt}}
\put(216.0,100.0){\rule[-0.200pt]{0.400pt}{0.482pt}}
\put(217,103){\usebox{\plotpoint}}
\put(217,104.67){\rule{0.241pt}{0.400pt}}
\multiput(217.00,104.17)(0.500,1.000){2}{\rule{0.120pt}{0.400pt}}
\put(217.0,103.0){\rule[-0.200pt]{0.400pt}{0.482pt}}
\put(218,107.67){\rule{0.241pt}{0.400pt}}
\multiput(218.00,107.17)(0.500,1.000){2}{\rule{0.120pt}{0.400pt}}
\put(218.0,106.0){\rule[-0.200pt]{0.400pt}{0.482pt}}
\put(219,111.67){\rule{0.241pt}{0.400pt}}
\multiput(219.00,111.17)(0.500,1.000){2}{\rule{0.120pt}{0.400pt}}
\put(219.0,109.0){\rule[-0.200pt]{0.400pt}{0.723pt}}
\put(220,114.67){\rule{0.241pt}{0.400pt}}
\multiput(220.00,114.17)(0.500,1.000){2}{\rule{0.120pt}{0.400pt}}
\put(220.0,113.0){\rule[-0.200pt]{0.400pt}{0.482pt}}
\put(221.0,116.0){\rule[-0.200pt]{0.400pt}{0.964pt}}
\put(221.0,120.0){\usebox{\plotpoint}}
\put(222,123.67){\rule{0.241pt}{0.400pt}}
\multiput(222.00,123.17)(0.500,1.000){2}{\rule{0.120pt}{0.400pt}}
\put(222.0,120.0){\rule[-0.200pt]{0.400pt}{0.964pt}}
\put(223,127.67){\rule{0.241pt}{0.400pt}}
\multiput(223.00,127.17)(0.500,1.000){2}{\rule{0.120pt}{0.400pt}}
\put(223.0,125.0){\rule[-0.200pt]{0.400pt}{0.723pt}}
\put(224.0,129.0){\rule[-0.200pt]{0.400pt}{0.964pt}}
\put(224.0,133.0){\usebox{\plotpoint}}
\put(225,135.67){\rule{0.241pt}{0.400pt}}
\multiput(225.00,135.17)(0.500,1.000){2}{\rule{0.120pt}{0.400pt}}
\put(225.0,133.0){\rule[-0.200pt]{0.400pt}{0.723pt}}
\put(226,141.67){\rule{0.241pt}{0.400pt}}
\multiput(226.00,141.17)(0.500,1.000){2}{\rule{0.120pt}{0.400pt}}
\put(226.0,137.0){\rule[-0.200pt]{0.400pt}{1.204pt}}
\put(227,145.67){\rule{0.241pt}{0.400pt}}
\multiput(227.00,145.17)(0.500,1.000){2}{\rule{0.120pt}{0.400pt}}
\put(227.0,143.0){\rule[-0.200pt]{0.400pt}{0.723pt}}
\put(228,150.67){\rule{0.241pt}{0.400pt}}
\multiput(228.00,150.17)(0.500,1.000){2}{\rule{0.120pt}{0.400pt}}
\put(228.0,147.0){\rule[-0.200pt]{0.400pt}{0.964pt}}
\put(228.67,155){\rule{0.400pt}{0.482pt}}
\multiput(228.17,155.00)(1.000,1.000){2}{\rule{0.400pt}{0.241pt}}
\put(229.0,152.0){\rule[-0.200pt]{0.400pt}{0.723pt}}
\put(230,160.67){\rule{0.241pt}{0.400pt}}
\multiput(230.00,160.17)(0.500,1.000){2}{\rule{0.120pt}{0.400pt}}
\put(230.0,157.0){\rule[-0.200pt]{0.400pt}{0.964pt}}
\put(231,166.67){\rule{0.241pt}{0.400pt}}
\multiput(231.00,166.17)(0.500,1.000){2}{\rule{0.120pt}{0.400pt}}
\put(231.0,162.0){\rule[-0.200pt]{0.400pt}{1.204pt}}
\put(232,170.67){\rule{0.241pt}{0.400pt}}
\multiput(232.00,170.17)(0.500,1.000){2}{\rule{0.120pt}{0.400pt}}
\put(232.0,168.0){\rule[-0.200pt]{0.400pt}{0.723pt}}
\put(233,176.67){\rule{0.241pt}{0.400pt}}
\multiput(233.00,176.17)(0.500,1.000){2}{\rule{0.120pt}{0.400pt}}
\put(233.0,172.0){\rule[-0.200pt]{0.400pt}{1.204pt}}
\put(234,181.67){\rule{0.241pt}{0.400pt}}
\multiput(234.00,181.17)(0.500,1.000){2}{\rule{0.120pt}{0.400pt}}
\put(234.0,178.0){\rule[-0.200pt]{0.400pt}{0.964pt}}
\put(235,187.67){\rule{0.241pt}{0.400pt}}
\multiput(235.00,187.17)(0.500,1.000){2}{\rule{0.120pt}{0.400pt}}
\put(235.0,183.0){\rule[-0.200pt]{0.400pt}{1.204pt}}
\put(236,191.67){\rule{0.241pt}{0.400pt}}
\multiput(236.00,191.17)(0.500,1.000){2}{\rule{0.120pt}{0.400pt}}
\put(236.0,189.0){\rule[-0.200pt]{0.400pt}{0.723pt}}
\put(237,197.67){\rule{0.241pt}{0.400pt}}
\multiput(237.00,197.17)(0.500,1.000){2}{\rule{0.120pt}{0.400pt}}
\put(237.0,193.0){\rule[-0.200pt]{0.400pt}{1.204pt}}
\put(238,202.67){\rule{0.241pt}{0.400pt}}
\multiput(238.00,202.17)(0.500,1.000){2}{\rule{0.120pt}{0.400pt}}
\put(238.0,199.0){\rule[-0.200pt]{0.400pt}{0.964pt}}
\put(239,208.67){\rule{0.241pt}{0.400pt}}
\multiput(239.00,208.17)(0.500,1.000){2}{\rule{0.120pt}{0.400pt}}
\put(239.0,204.0){\rule[-0.200pt]{0.400pt}{1.204pt}}
\put(240,214.67){\rule{0.241pt}{0.400pt}}
\multiput(240.00,214.17)(0.500,1.000){2}{\rule{0.120pt}{0.400pt}}
\put(240.0,210.0){\rule[-0.200pt]{0.400pt}{1.204pt}}
\put(241,219.67){\rule{0.241pt}{0.400pt}}
\multiput(241.00,219.17)(0.500,1.000){2}{\rule{0.120pt}{0.400pt}}
\put(241.0,216.0){\rule[-0.200pt]{0.400pt}{0.964pt}}
\put(241.67,225){\rule{0.400pt}{0.482pt}}
\multiput(241.17,225.00)(1.000,1.000){2}{\rule{0.400pt}{0.241pt}}
\put(242.0,221.0){\rule[-0.200pt]{0.400pt}{0.964pt}}
\put(243,229.67){\rule{0.241pt}{0.400pt}}
\multiput(243.00,229.17)(0.500,1.000){2}{\rule{0.120pt}{0.400pt}}
\put(243.0,227.0){\rule[-0.200pt]{0.400pt}{0.723pt}}
\put(244,235.67){\rule{0.241pt}{0.400pt}}
\multiput(244.00,235.17)(0.500,1.000){2}{\rule{0.120pt}{0.400pt}}
\put(244.0,231.0){\rule[-0.200pt]{0.400pt}{1.204pt}}
\put(245,239.67){\rule{0.241pt}{0.400pt}}
\multiput(245.00,239.17)(0.500,1.000){2}{\rule{0.120pt}{0.400pt}}
\put(245.0,237.0){\rule[-0.200pt]{0.400pt}{0.723pt}}
\put(246,245.67){\rule{0.241pt}{0.400pt}}
\multiput(246.00,245.17)(0.500,1.000){2}{\rule{0.120pt}{0.400pt}}
\put(246.0,241.0){\rule[-0.200pt]{0.400pt}{1.204pt}}
\put(247,249.67){\rule{0.241pt}{0.400pt}}
\multiput(247.00,249.17)(0.500,1.000){2}{\rule{0.120pt}{0.400pt}}
\put(247.0,247.0){\rule[-0.200pt]{0.400pt}{0.723pt}}
\put(248,255.67){\rule{0.241pt}{0.400pt}}
\multiput(248.00,255.17)(0.500,1.000){2}{\rule{0.120pt}{0.400pt}}
\put(248.0,251.0){\rule[-0.200pt]{0.400pt}{1.204pt}}
\put(249,260.67){\rule{0.241pt}{0.400pt}}
\multiput(249.00,260.17)(0.500,1.000){2}{\rule{0.120pt}{0.400pt}}
\put(249.0,257.0){\rule[-0.200pt]{0.400pt}{0.964pt}}
\put(250,264.67){\rule{0.241pt}{0.400pt}}
\multiput(250.00,264.17)(0.500,1.000){2}{\rule{0.120pt}{0.400pt}}
\put(250.0,262.0){\rule[-0.200pt]{0.400pt}{0.723pt}}
\put(251,269.67){\rule{0.241pt}{0.400pt}}
\multiput(251.00,269.17)(0.500,1.000){2}{\rule{0.120pt}{0.400pt}}
\put(251.0,266.0){\rule[-0.200pt]{0.400pt}{0.964pt}}
\put(252,273.67){\rule{0.241pt}{0.400pt}}
\multiput(252.00,273.17)(0.500,1.000){2}{\rule{0.120pt}{0.400pt}}
\put(252.0,271.0){\rule[-0.200pt]{0.400pt}{0.723pt}}
\put(253,277.67){\rule{0.241pt}{0.400pt}}
\multiput(253.00,277.17)(0.500,1.000){2}{\rule{0.120pt}{0.400pt}}
\put(253.0,275.0){\rule[-0.200pt]{0.400pt}{0.723pt}}
\put(254,281.67){\rule{0.241pt}{0.400pt}}
\multiput(254.00,281.17)(0.500,1.000){2}{\rule{0.120pt}{0.400pt}}
\put(254.0,279.0){\rule[-0.200pt]{0.400pt}{0.723pt}}
\put(255,285.67){\rule{0.241pt}{0.400pt}}
\multiput(255.00,285.17)(0.500,1.000){2}{\rule{0.120pt}{0.400pt}}
\put(255.0,283.0){\rule[-0.200pt]{0.400pt}{0.723pt}}
\put(256,288.67){\rule{0.241pt}{0.400pt}}
\multiput(256.00,288.17)(0.500,1.000){2}{\rule{0.120pt}{0.400pt}}
\put(256.0,287.0){\rule[-0.200pt]{0.400pt}{0.482pt}}
\put(257,292.67){\rule{0.241pt}{0.400pt}}
\multiput(257.00,292.17)(0.500,1.000){2}{\rule{0.120pt}{0.400pt}}
\put(257.0,290.0){\rule[-0.200pt]{0.400pt}{0.723pt}}
\put(258,296.67){\rule{0.241pt}{0.400pt}}
\multiput(258.00,296.17)(0.500,1.000){2}{\rule{0.120pt}{0.400pt}}
\put(258.0,294.0){\rule[-0.200pt]{0.400pt}{0.723pt}}
\put(259,298){\usebox{\plotpoint}}
\put(259.0,298.0){\rule[-0.200pt]{0.400pt}{0.482pt}}
\put(259.0,300.0){\usebox{\plotpoint}}
\put(260.0,300.0){\rule[-0.200pt]{0.400pt}{0.723pt}}
\put(260.0,303.0){\usebox{\plotpoint}}
\put(261,304.67){\rule{0.241pt}{0.400pt}}
\multiput(261.00,304.17)(0.500,1.000){2}{\rule{0.120pt}{0.400pt}}
\put(261.0,303.0){\rule[-0.200pt]{0.400pt}{0.482pt}}
\put(262,306){\usebox{\plotpoint}}
\put(262.0,306.0){\rule[-0.200pt]{0.400pt}{0.482pt}}
\put(262.0,308.0){\usebox{\plotpoint}}
\put(263.0,308.0){\rule[-0.200pt]{0.400pt}{0.482pt}}
\put(263.0,310.0){\usebox{\plotpoint}}
\put(264,311.67){\rule{0.241pt}{0.400pt}}
\multiput(264.00,311.17)(0.500,1.000){2}{\rule{0.120pt}{0.400pt}}
\put(264.0,310.0){\rule[-0.200pt]{0.400pt}{0.482pt}}
\put(265,313){\usebox{\plotpoint}}
\put(265,313){\usebox{\plotpoint}}
\put(265.0,313.0){\usebox{\plotpoint}}
\put(265.0,314.0){\usebox{\plotpoint}}
\put(266.0,314.0){\rule[-0.200pt]{0.400pt}{0.482pt}}
\put(266.0,316.0){\usebox{\plotpoint}}
\put(267.0,316.0){\usebox{\plotpoint}}
\put(267.0,317.0){\usebox{\plotpoint}}
\put(268.0,317.0){\usebox{\plotpoint}}
\put(268.0,318.0){\usebox{\plotpoint}}
\put(269.0,318.0){\usebox{\plotpoint}}
\put(270,318.67){\rule{0.241pt}{0.400pt}}
\multiput(270.00,318.17)(0.500,1.000){2}{\rule{0.120pt}{0.400pt}}
\put(269.0,319.0){\usebox{\plotpoint}}
\put(271,320){\usebox{\plotpoint}}
\put(271,320){\usebox{\plotpoint}}
\put(271,320){\usebox{\plotpoint}}
\put(271,320){\usebox{\plotpoint}}
\put(274,318.67){\rule{0.241pt}{0.400pt}}
\multiput(274.00,319.17)(0.500,-1.000){2}{\rule{0.120pt}{0.400pt}}
\put(271.0,320.0){\rule[-0.200pt]{0.723pt}{0.400pt}}
\put(275,319){\usebox{\plotpoint}}
\put(275,319){\usebox{\plotpoint}}
\put(275,319){\usebox{\plotpoint}}
\put(275,319){\usebox{\plotpoint}}
\put(275.0,319.0){\usebox{\plotpoint}}
\put(276.0,318.0){\usebox{\plotpoint}}
\put(276.0,318.0){\usebox{\plotpoint}}
\put(277.0,317.0){\usebox{\plotpoint}}
\put(277.0,317.0){\usebox{\plotpoint}}
\put(278.0,315.0){\rule[-0.200pt]{0.400pt}{0.482pt}}
\put(278.0,315.0){\usebox{\plotpoint}}
\put(279.0,314.0){\usebox{\plotpoint}}
\put(279.0,314.0){\usebox{\plotpoint}}
\put(280.0,312.0){\rule[-0.200pt]{0.400pt}{0.482pt}}
\put(280.0,312.0){\usebox{\plotpoint}}
\put(281.0,310.0){\rule[-0.200pt]{0.400pt}{0.482pt}}
\put(281.0,310.0){\usebox{\plotpoint}}
\put(282,306.67){\rule{0.241pt}{0.400pt}}
\multiput(282.00,307.17)(0.500,-1.000){2}{\rule{0.120pt}{0.400pt}}
\put(282.0,308.0){\rule[-0.200pt]{0.400pt}{0.482pt}}
\put(283,307){\usebox{\plotpoint}}
\put(283,304.67){\rule{0.241pt}{0.400pt}}
\multiput(283.00,305.17)(0.500,-1.000){2}{\rule{0.120pt}{0.400pt}}
\put(283.0,306.0){\usebox{\plotpoint}}
\put(284,305){\usebox{\plotpoint}}
\put(284,301.67){\rule{0.241pt}{0.400pt}}
\multiput(284.00,302.17)(0.500,-1.000){2}{\rule{0.120pt}{0.400pt}}
\put(284.0,303.0){\rule[-0.200pt]{0.400pt}{0.482pt}}
\put(285.0,299.0){\rule[-0.200pt]{0.400pt}{0.723pt}}
\put(285.0,299.0){\usebox{\plotpoint}}
\put(286,295.67){\rule{0.241pt}{0.400pt}}
\multiput(286.00,296.17)(0.500,-1.000){2}{\rule{0.120pt}{0.400pt}}
\put(286.0,297.0){\rule[-0.200pt]{0.400pt}{0.482pt}}
\put(287,291.67){\rule{0.241pt}{0.400pt}}
\multiput(287.00,292.17)(0.500,-1.000){2}{\rule{0.120pt}{0.400pt}}
\put(287.0,293.0){\rule[-0.200pt]{0.400pt}{0.723pt}}
\put(288,292){\usebox{\plotpoint}}
\put(288,288.67){\rule{0.241pt}{0.400pt}}
\multiput(288.00,289.17)(0.500,-1.000){2}{\rule{0.120pt}{0.400pt}}
\put(288.0,290.0){\rule[-0.200pt]{0.400pt}{0.482pt}}
\put(289,284.67){\rule{0.241pt}{0.400pt}}
\multiput(289.00,285.17)(0.500,-1.000){2}{\rule{0.120pt}{0.400pt}}
\put(289.0,286.0){\rule[-0.200pt]{0.400pt}{0.723pt}}
\put(290,281.67){\rule{0.241pt}{0.400pt}}
\multiput(290.00,282.17)(0.500,-1.000){2}{\rule{0.120pt}{0.400pt}}
\put(290.0,283.0){\rule[-0.200pt]{0.400pt}{0.482pt}}
\put(291,276.67){\rule{0.241pt}{0.400pt}}
\multiput(291.00,277.17)(0.500,-1.000){2}{\rule{0.120pt}{0.400pt}}
\put(291.0,278.0){\rule[-0.200pt]{0.400pt}{0.964pt}}
\put(292,272.67){\rule{0.241pt}{0.400pt}}
\multiput(292.00,273.17)(0.500,-1.000){2}{\rule{0.120pt}{0.400pt}}
\put(292.0,274.0){\rule[-0.200pt]{0.400pt}{0.723pt}}
\put(293,273){\usebox{\plotpoint}}
\put(293,268.67){\rule{0.241pt}{0.400pt}}
\multiput(293.00,269.17)(0.500,-1.000){2}{\rule{0.120pt}{0.400pt}}
\put(293.0,270.0){\rule[-0.200pt]{0.400pt}{0.723pt}}
\put(294,263.67){\rule{0.241pt}{0.400pt}}
\multiput(294.00,264.17)(0.500,-1.000){2}{\rule{0.120pt}{0.400pt}}
\put(294.0,265.0){\rule[-0.200pt]{0.400pt}{0.964pt}}
\put(295,259.67){\rule{0.241pt}{0.400pt}}
\multiput(295.00,260.17)(0.500,-1.000){2}{\rule{0.120pt}{0.400pt}}
\put(295.0,261.0){\rule[-0.200pt]{0.400pt}{0.723pt}}
\put(296,253.67){\rule{0.241pt}{0.400pt}}
\multiput(296.00,254.17)(0.500,-1.000){2}{\rule{0.120pt}{0.400pt}}
\put(296.0,255.0){\rule[-0.200pt]{0.400pt}{1.204pt}}
\put(297,249.67){\rule{0.241pt}{0.400pt}}
\multiput(297.00,250.17)(0.500,-1.000){2}{\rule{0.120pt}{0.400pt}}
\put(297.0,251.0){\rule[-0.200pt]{0.400pt}{0.723pt}}
\put(298,244.67){\rule{0.241pt}{0.400pt}}
\multiput(298.00,245.17)(0.500,-1.000){2}{\rule{0.120pt}{0.400pt}}
\put(298.0,246.0){\rule[-0.200pt]{0.400pt}{0.964pt}}
\put(299,239.67){\rule{0.241pt}{0.400pt}}
\multiput(299.00,240.17)(0.500,-1.000){2}{\rule{0.120pt}{0.400pt}}
\put(299.0,241.0){\rule[-0.200pt]{0.400pt}{0.964pt}}
\put(299.67,234){\rule{0.400pt}{0.482pt}}
\multiput(299.17,235.00)(1.000,-1.000){2}{\rule{0.400pt}{0.241pt}}
\put(300.0,236.0){\rule[-0.200pt]{0.400pt}{0.964pt}}
\put(301,229.67){\rule{0.241pt}{0.400pt}}
\multiput(301.00,230.17)(0.500,-1.000){2}{\rule{0.120pt}{0.400pt}}
\put(301.0,231.0){\rule[-0.200pt]{0.400pt}{0.723pt}}
\put(302,223.67){\rule{0.241pt}{0.400pt}}
\multiput(302.00,224.17)(0.500,-1.000){2}{\rule{0.120pt}{0.400pt}}
\put(302.0,225.0){\rule[-0.200pt]{0.400pt}{1.204pt}}
\put(303,217.67){\rule{0.241pt}{0.400pt}}
\multiput(303.00,218.17)(0.500,-1.000){2}{\rule{0.120pt}{0.400pt}}
\put(303.0,219.0){\rule[-0.200pt]{0.400pt}{1.204pt}}
\put(303.67,213){\rule{0.400pt}{0.482pt}}
\multiput(303.17,214.00)(1.000,-1.000){2}{\rule{0.400pt}{0.241pt}}
\put(304.0,215.0){\rule[-0.200pt]{0.400pt}{0.723pt}}
\put(305,207.67){\rule{0.241pt}{0.400pt}}
\multiput(305.00,208.17)(0.500,-1.000){2}{\rule{0.120pt}{0.400pt}}
\put(305.0,209.0){\rule[-0.200pt]{0.400pt}{0.964pt}}
\put(306,202.67){\rule{0.241pt}{0.400pt}}
\multiput(306.00,203.17)(0.500,-1.000){2}{\rule{0.120pt}{0.400pt}}
\put(306.0,204.0){\rule[-0.200pt]{0.400pt}{0.964pt}}
\put(307,196.67){\rule{0.241pt}{0.400pt}}
\multiput(307.00,197.17)(0.500,-1.000){2}{\rule{0.120pt}{0.400pt}}
\put(307.0,198.0){\rule[-0.200pt]{0.400pt}{1.204pt}}
\put(308,191.67){\rule{0.241pt}{0.400pt}}
\multiput(308.00,192.17)(0.500,-1.000){2}{\rule{0.120pt}{0.400pt}}
\put(308.0,193.0){\rule[-0.200pt]{0.400pt}{0.964pt}}
\put(309,185.67){\rule{0.241pt}{0.400pt}}
\multiput(309.00,186.17)(0.500,-1.000){2}{\rule{0.120pt}{0.400pt}}
\put(309.0,187.0){\rule[-0.200pt]{0.400pt}{1.204pt}}
\put(309.67,181){\rule{0.400pt}{0.482pt}}
\multiput(309.17,182.00)(1.000,-1.000){2}{\rule{0.400pt}{0.241pt}}
\put(310.0,183.0){\rule[-0.200pt]{0.400pt}{0.723pt}}
\put(311,175.67){\rule{0.241pt}{0.400pt}}
\multiput(311.00,176.17)(0.500,-1.000){2}{\rule{0.120pt}{0.400pt}}
\put(311.0,177.0){\rule[-0.200pt]{0.400pt}{0.964pt}}
\put(312,169.67){\rule{0.241pt}{0.400pt}}
\multiput(312.00,170.17)(0.500,-1.000){2}{\rule{0.120pt}{0.400pt}}
\put(312.0,171.0){\rule[-0.200pt]{0.400pt}{1.204pt}}
\put(313,164.67){\rule{0.241pt}{0.400pt}}
\multiput(313.00,165.17)(0.500,-1.000){2}{\rule{0.120pt}{0.400pt}}
\put(313.0,166.0){\rule[-0.200pt]{0.400pt}{0.964pt}}
\put(314,159.67){\rule{0.241pt}{0.400pt}}
\multiput(314.00,160.17)(0.500,-1.000){2}{\rule{0.120pt}{0.400pt}}
\put(314.0,161.0){\rule[-0.200pt]{0.400pt}{0.964pt}}
\put(315,154.67){\rule{0.241pt}{0.400pt}}
\multiput(315.00,155.17)(0.500,-1.000){2}{\rule{0.120pt}{0.400pt}}
\put(315.0,156.0){\rule[-0.200pt]{0.400pt}{0.964pt}}
\put(316,149.67){\rule{0.241pt}{0.400pt}}
\multiput(316.00,150.17)(0.500,-1.000){2}{\rule{0.120pt}{0.400pt}}
\put(316.0,151.0){\rule[-0.200pt]{0.400pt}{0.964pt}}
\put(316.67,145){\rule{0.400pt}{0.482pt}}
\multiput(316.17,146.00)(1.000,-1.000){2}{\rule{0.400pt}{0.241pt}}
\put(317.0,147.0){\rule[-0.200pt]{0.400pt}{0.723pt}}
\put(318,139.67){\rule{0.241pt}{0.400pt}}
\multiput(318.00,140.17)(0.500,-1.000){2}{\rule{0.120pt}{0.400pt}}
\put(318.0,141.0){\rule[-0.200pt]{0.400pt}{0.964pt}}
\put(319,134.67){\rule{0.241pt}{0.400pt}}
\multiput(319.00,135.17)(0.500,-1.000){2}{\rule{0.120pt}{0.400pt}}
\put(319.0,136.0){\rule[-0.200pt]{0.400pt}{0.964pt}}
\put(320,130.67){\rule{0.241pt}{0.400pt}}
\multiput(320.00,131.17)(0.500,-1.000){2}{\rule{0.120pt}{0.400pt}}
\put(320.0,132.0){\rule[-0.200pt]{0.400pt}{0.723pt}}
\put(321,126.67){\rule{0.241pt}{0.400pt}}
\multiput(321.00,127.17)(0.500,-1.000){2}{\rule{0.120pt}{0.400pt}}
\put(321.0,128.0){\rule[-0.200pt]{0.400pt}{0.723pt}}
\put(322,122.67){\rule{0.241pt}{0.400pt}}
\multiput(322.00,123.17)(0.500,-1.000){2}{\rule{0.120pt}{0.400pt}}
\put(322.0,124.0){\rule[-0.200pt]{0.400pt}{0.723pt}}
\put(323.0,119.0){\rule[-0.200pt]{0.400pt}{0.964pt}}
\put(323.0,119.0){\usebox{\plotpoint}}
\put(324,114.67){\rule{0.241pt}{0.400pt}}
\multiput(324.00,115.17)(0.500,-1.000){2}{\rule{0.120pt}{0.400pt}}
\put(324.0,116.0){\rule[-0.200pt]{0.400pt}{0.723pt}}
\put(325,110.67){\rule{0.241pt}{0.400pt}}
\multiput(325.00,111.17)(0.500,-1.000){2}{\rule{0.120pt}{0.400pt}}
\put(325.0,112.0){\rule[-0.200pt]{0.400pt}{0.723pt}}
\put(326,107.67){\rule{0.241pt}{0.400pt}}
\multiput(326.00,108.17)(0.500,-1.000){2}{\rule{0.120pt}{0.400pt}}
\put(326.0,109.0){\rule[-0.200pt]{0.400pt}{0.482pt}}
\put(327.0,105.0){\rule[-0.200pt]{0.400pt}{0.723pt}}
\put(327.0,105.0){\usebox{\plotpoint}}
\put(328,100.67){\rule{0.241pt}{0.400pt}}
\multiput(328.00,101.17)(0.500,-1.000){2}{\rule{0.120pt}{0.400pt}}
\put(328.0,102.0){\rule[-0.200pt]{0.400pt}{0.723pt}}
\put(329,101){\usebox{\plotpoint}}
\put(329.0,99.0){\rule[-0.200pt]{0.400pt}{0.482pt}}
\put(329.0,99.0){\usebox{\plotpoint}}
\put(330.0,96.0){\rule[-0.200pt]{0.400pt}{0.723pt}}
\put(330.0,96.0){\usebox{\plotpoint}}
\put(331.0,94.0){\rule[-0.200pt]{0.400pt}{0.482pt}}
\put(331.0,94.0){\usebox{\plotpoint}}
\put(332,90.67){\rule{0.241pt}{0.400pt}}
\multiput(332.00,91.17)(0.500,-1.000){2}{\rule{0.120pt}{0.400pt}}
\put(332.0,92.0){\rule[-0.200pt]{0.400pt}{0.482pt}}
\put(333,91){\usebox{\plotpoint}}
\put(333,88.67){\rule{0.241pt}{0.400pt}}
\multiput(333.00,89.17)(0.500,-1.000){2}{\rule{0.120pt}{0.400pt}}
\put(333.0,90.0){\usebox{\plotpoint}}
\put(334,89){\usebox{\plotpoint}}
\put(334,89){\usebox{\plotpoint}}
\put(334.0,88.0){\usebox{\plotpoint}}
\put(334.0,88.0){\usebox{\plotpoint}}
\put(335.0,86.0){\rule[-0.200pt]{0.400pt}{0.482pt}}
\put(335.0,86.0){\usebox{\plotpoint}}
\put(336.0,85.0){\usebox{\plotpoint}}
\put(336.0,85.0){\usebox{\plotpoint}}
\put(337.0,84.0){\usebox{\plotpoint}}
\put(337.0,84.0){\usebox{\plotpoint}}
\put(338.0,83.0){\usebox{\plotpoint}}
\put(338.0,83.0){\usebox{\plotpoint}}
\put(339.0,82.0){\usebox{\plotpoint}}
\put(339.0,82.0){\rule[-0.200pt]{0.964pt}{0.400pt}}
\put(343.0,82.0){\usebox{\plotpoint}}
\put(343.0,83.0){\rule[-0.200pt]{0.482pt}{0.400pt}}
\put(345.0,83.0){\usebox{\plotpoint}}
\put(345.0,84.0){\usebox{\plotpoint}}
\put(346,84.67){\rule{0.241pt}{0.400pt}}
\multiput(346.00,84.17)(0.500,1.000){2}{\rule{0.120pt}{0.400pt}}
\put(346.0,84.0){\usebox{\plotpoint}}
\put(347,86){\usebox{\plotpoint}}
\put(347,86){\usebox{\plotpoint}}
\put(347,86){\usebox{\plotpoint}}
\put(347,85.67){\rule{0.241pt}{0.400pt}}
\multiput(347.00,85.17)(0.500,1.000){2}{\rule{0.120pt}{0.400pt}}
\put(348,87){\usebox{\plotpoint}}
\put(348,87){\usebox{\plotpoint}}
\put(348,87.67){\rule{0.241pt}{0.400pt}}
\multiput(348.00,87.17)(0.500,1.000){2}{\rule{0.120pt}{0.400pt}}
\put(348.0,87.0){\usebox{\plotpoint}}
\put(349,89){\usebox{\plotpoint}}
\put(349,89){\usebox{\plotpoint}}
\put(349.0,89.0){\usebox{\plotpoint}}
\put(349.0,90.0){\usebox{\plotpoint}}
\put(350,91.67){\rule{0.241pt}{0.400pt}}
\multiput(350.00,91.17)(0.500,1.000){2}{\rule{0.120pt}{0.400pt}}
\put(350.0,90.0){\rule[-0.200pt]{0.400pt}{0.482pt}}
\put(351,93){\usebox{\plotpoint}}
\put(351,93.67){\rule{0.241pt}{0.400pt}}
\multiput(351.00,93.17)(0.500,1.000){2}{\rule{0.120pt}{0.400pt}}
\put(351.0,93.0){\usebox{\plotpoint}}
\put(352,95){\usebox{\plotpoint}}
\put(352.0,95.0){\rule[-0.200pt]{0.400pt}{0.482pt}}
\put(352.0,97.0){\usebox{\plotpoint}}
\put(353,98.67){\rule{0.241pt}{0.400pt}}
\multiput(353.00,98.17)(0.500,1.000){2}{\rule{0.120pt}{0.400pt}}
\put(353.0,97.0){\rule[-0.200pt]{0.400pt}{0.482pt}}
\put(354,100){\usebox{\plotpoint}}
\put(354,101.67){\rule{0.241pt}{0.400pt}}
\multiput(354.00,101.17)(0.500,1.000){2}{\rule{0.120pt}{0.400pt}}
\put(354.0,100.0){\rule[-0.200pt]{0.400pt}{0.482pt}}
\put(355,105.67){\rule{0.241pt}{0.400pt}}
\multiput(355.00,105.17)(0.500,1.000){2}{\rule{0.120pt}{0.400pt}}
\put(355.0,103.0){\rule[-0.200pt]{0.400pt}{0.723pt}}
\put(356,107){\usebox{\plotpoint}}
\put(356,108.67){\rule{0.241pt}{0.400pt}}
\multiput(356.00,108.17)(0.500,1.000){2}{\rule{0.120pt}{0.400pt}}
\put(356.0,107.0){\rule[-0.200pt]{0.400pt}{0.482pt}}
\put(357,110){\usebox{\plotpoint}}
\put(357.0,110.0){\rule[-0.200pt]{0.400pt}{0.723pt}}
\put(357.0,113.0){\usebox{\plotpoint}}
\put(358,115.67){\rule{0.241pt}{0.400pt}}
\multiput(358.00,115.17)(0.500,1.000){2}{\rule{0.120pt}{0.400pt}}
\put(358.0,113.0){\rule[-0.200pt]{0.400pt}{0.723pt}}
\put(359,119.67){\rule{0.241pt}{0.400pt}}
\multiput(359.00,119.17)(0.500,1.000){2}{\rule{0.120pt}{0.400pt}}
\put(359.0,117.0){\rule[-0.200pt]{0.400pt}{0.723pt}}
\put(360,123.67){\rule{0.241pt}{0.400pt}}
\multiput(360.00,123.17)(0.500,1.000){2}{\rule{0.120pt}{0.400pt}}
\put(360.0,121.0){\rule[-0.200pt]{0.400pt}{0.723pt}}
\put(361,127.67){\rule{0.241pt}{0.400pt}}
\multiput(361.00,127.17)(0.500,1.000){2}{\rule{0.120pt}{0.400pt}}
\put(361.0,125.0){\rule[-0.200pt]{0.400pt}{0.723pt}}
\put(362,131.67){\rule{0.241pt}{0.400pt}}
\multiput(362.00,131.17)(0.500,1.000){2}{\rule{0.120pt}{0.400pt}}
\put(362.0,129.0){\rule[-0.200pt]{0.400pt}{0.723pt}}
\put(363,136.67){\rule{0.241pt}{0.400pt}}
\multiput(363.00,136.17)(0.500,1.000){2}{\rule{0.120pt}{0.400pt}}
\put(363.0,133.0){\rule[-0.200pt]{0.400pt}{0.964pt}}
\put(364,141.67){\rule{0.241pt}{0.400pt}}
\multiput(364.00,141.17)(0.500,1.000){2}{\rule{0.120pt}{0.400pt}}
\put(364.0,138.0){\rule[-0.200pt]{0.400pt}{0.964pt}}
\put(365,145.67){\rule{0.241pt}{0.400pt}}
\multiput(365.00,145.17)(0.500,1.000){2}{\rule{0.120pt}{0.400pt}}
\put(365.0,143.0){\rule[-0.200pt]{0.400pt}{0.723pt}}
\put(366,151.67){\rule{0.241pt}{0.400pt}}
\multiput(366.00,151.17)(0.500,1.000){2}{\rule{0.120pt}{0.400pt}}
\put(366.0,147.0){\rule[-0.200pt]{0.400pt}{1.204pt}}
\put(367,155.67){\rule{0.241pt}{0.400pt}}
\multiput(367.00,155.17)(0.500,1.000){2}{\rule{0.120pt}{0.400pt}}
\put(367.0,153.0){\rule[-0.200pt]{0.400pt}{0.723pt}}
\put(368,161.67){\rule{0.241pt}{0.400pt}}
\multiput(368.00,161.17)(0.500,1.000){2}{\rule{0.120pt}{0.400pt}}
\put(368.0,157.0){\rule[-0.200pt]{0.400pt}{1.204pt}}
\put(369,165.67){\rule{0.241pt}{0.400pt}}
\multiput(369.00,165.17)(0.500,1.000){2}{\rule{0.120pt}{0.400pt}}
\put(369.0,163.0){\rule[-0.200pt]{0.400pt}{0.723pt}}
\put(370,171.67){\rule{0.241pt}{0.400pt}}
\multiput(370.00,171.17)(0.500,1.000){2}{\rule{0.120pt}{0.400pt}}
\put(370.0,167.0){\rule[-0.200pt]{0.400pt}{1.204pt}}
\put(371,176.67){\rule{0.241pt}{0.400pt}}
\multiput(371.00,176.17)(0.500,1.000){2}{\rule{0.120pt}{0.400pt}}
\put(371.0,173.0){\rule[-0.200pt]{0.400pt}{0.964pt}}
\put(371.67,182){\rule{0.400pt}{0.482pt}}
\multiput(371.17,182.00)(1.000,1.000){2}{\rule{0.400pt}{0.241pt}}
\put(372.0,178.0){\rule[-0.200pt]{0.400pt}{0.964pt}}
\put(372.67,188){\rule{0.400pt}{0.482pt}}
\multiput(372.17,188.00)(1.000,1.000){2}{\rule{0.400pt}{0.241pt}}
\put(373.0,184.0){\rule[-0.200pt]{0.400pt}{0.964pt}}
\put(374,192.67){\rule{0.241pt}{0.400pt}}
\multiput(374.00,192.17)(0.500,1.000){2}{\rule{0.120pt}{0.400pt}}
\put(374.0,190.0){\rule[-0.200pt]{0.400pt}{0.723pt}}
\put(375,198.67){\rule{0.241pt}{0.400pt}}
\multiput(375.00,198.17)(0.500,1.000){2}{\rule{0.120pt}{0.400pt}}
\put(375.0,194.0){\rule[-0.200pt]{0.400pt}{1.204pt}}
\put(376,203.67){\rule{0.241pt}{0.400pt}}
\multiput(376.00,203.17)(0.500,1.000){2}{\rule{0.120pt}{0.400pt}}
\put(376.0,200.0){\rule[-0.200pt]{0.400pt}{0.964pt}}
\put(377,209.67){\rule{0.241pt}{0.400pt}}
\multiput(377.00,209.17)(0.500,1.000){2}{\rule{0.120pt}{0.400pt}}
\put(377.0,205.0){\rule[-0.200pt]{0.400pt}{1.204pt}}
\put(377.67,214){\rule{0.400pt}{0.482pt}}
\multiput(377.17,214.00)(1.000,1.000){2}{\rule{0.400pt}{0.241pt}}
\put(378.0,211.0){\rule[-0.200pt]{0.400pt}{0.723pt}}
\put(378.67,220){\rule{0.400pt}{0.482pt}}
\multiput(378.17,220.00)(1.000,1.000){2}{\rule{0.400pt}{0.241pt}}
\put(379.0,216.0){\rule[-0.200pt]{0.400pt}{0.964pt}}
\put(380,224.67){\rule{0.241pt}{0.400pt}}
\multiput(380.00,224.17)(0.500,1.000){2}{\rule{0.120pt}{0.400pt}}
\put(380.0,222.0){\rule[-0.200pt]{0.400pt}{0.723pt}}
\put(381,230.67){\rule{0.241pt}{0.400pt}}
\multiput(381.00,230.17)(0.500,1.000){2}{\rule{0.120pt}{0.400pt}}
\put(381.0,226.0){\rule[-0.200pt]{0.400pt}{1.204pt}}
\put(382,236.67){\rule{0.241pt}{0.400pt}}
\multiput(382.00,236.17)(0.500,1.000){2}{\rule{0.120pt}{0.400pt}}
\put(382.0,232.0){\rule[-0.200pt]{0.400pt}{1.204pt}}
\put(383,240.67){\rule{0.241pt}{0.400pt}}
\multiput(383.00,240.17)(0.500,1.000){2}{\rule{0.120pt}{0.400pt}}
\put(383.0,238.0){\rule[-0.200pt]{0.400pt}{0.723pt}}
\put(384,246.67){\rule{0.241pt}{0.400pt}}
\multiput(384.00,246.17)(0.500,1.000){2}{\rule{0.120pt}{0.400pt}}
\put(384.0,242.0){\rule[-0.200pt]{0.400pt}{1.204pt}}
\put(385,250.67){\rule{0.241pt}{0.400pt}}
\multiput(385.00,250.17)(0.500,1.000){2}{\rule{0.120pt}{0.400pt}}
\put(385.0,248.0){\rule[-0.200pt]{0.400pt}{0.723pt}}
\put(386,255.67){\rule{0.241pt}{0.400pt}}
\multiput(386.00,255.17)(0.500,1.000){2}{\rule{0.120pt}{0.400pt}}
\put(386.0,252.0){\rule[-0.200pt]{0.400pt}{0.964pt}}
\put(387,260.67){\rule{0.241pt}{0.400pt}}
\multiput(387.00,260.17)(0.500,1.000){2}{\rule{0.120pt}{0.400pt}}
\put(387.0,257.0){\rule[-0.200pt]{0.400pt}{0.964pt}}
\put(388,265.67){\rule{0.241pt}{0.400pt}}
\multiput(388.00,265.17)(0.500,1.000){2}{\rule{0.120pt}{0.400pt}}
\put(388.0,262.0){\rule[-0.200pt]{0.400pt}{0.964pt}}
\put(389.0,267.0){\rule[-0.200pt]{0.400pt}{0.723pt}}
\put(389.0,270.0){\usebox{\plotpoint}}
\put(390,273.67){\rule{0.241pt}{0.400pt}}
\multiput(390.00,273.17)(0.500,1.000){2}{\rule{0.120pt}{0.400pt}}
\put(390.0,270.0){\rule[-0.200pt]{0.400pt}{0.964pt}}
\put(391,278.67){\rule{0.241pt}{0.400pt}}
\multiput(391.00,278.17)(0.500,1.000){2}{\rule{0.120pt}{0.400pt}}
\put(391.0,275.0){\rule[-0.200pt]{0.400pt}{0.964pt}}
\put(392,281.67){\rule{0.241pt}{0.400pt}}
\multiput(392.00,281.17)(0.500,1.000){2}{\rule{0.120pt}{0.400pt}}
\put(392.0,280.0){\rule[-0.200pt]{0.400pt}{0.482pt}}
\put(393.0,283.0){\rule[-0.200pt]{0.400pt}{0.964pt}}
\put(393.0,287.0){\usebox{\plotpoint}}
\put(394,289.67){\rule{0.241pt}{0.400pt}}
\multiput(394.00,289.17)(0.500,1.000){2}{\rule{0.120pt}{0.400pt}}
\put(394.0,287.0){\rule[-0.200pt]{0.400pt}{0.723pt}}
\put(395,291){\usebox{\plotpoint}}
\put(395.0,291.0){\rule[-0.200pt]{0.400pt}{0.723pt}}
\put(395.0,294.0){\usebox{\plotpoint}}
\put(396.0,294.0){\rule[-0.200pt]{0.400pt}{0.723pt}}
\put(396.0,297.0){\usebox{\plotpoint}}
\put(397,299.67){\rule{0.241pt}{0.400pt}}
\multiput(397.00,299.17)(0.500,1.000){2}{\rule{0.120pt}{0.400pt}}
\put(397.0,297.0){\rule[-0.200pt]{0.400pt}{0.723pt}}
\put(398,301){\usebox{\plotpoint}}
\put(398.0,301.0){\rule[-0.200pt]{0.400pt}{0.482pt}}
\put(398.0,303.0){\usebox{\plotpoint}}
\put(399.0,303.0){\rule[-0.200pt]{0.400pt}{0.723pt}}
\put(399.0,306.0){\usebox{\plotpoint}}
\put(400,307.67){\rule{0.241pt}{0.400pt}}
\multiput(400.00,307.17)(0.500,1.000){2}{\rule{0.120pt}{0.400pt}}
\put(400.0,306.0){\rule[-0.200pt]{0.400pt}{0.482pt}}
\put(401,309){\usebox{\plotpoint}}
\put(401,309.67){\rule{0.241pt}{0.400pt}}
\multiput(401.00,309.17)(0.500,1.000){2}{\rule{0.120pt}{0.400pt}}
\put(401.0,309.0){\usebox{\plotpoint}}
\put(402,311){\usebox{\plotpoint}}
\put(402,311.67){\rule{0.241pt}{0.400pt}}
\multiput(402.00,311.17)(0.500,1.000){2}{\rule{0.120pt}{0.400pt}}
\put(402.0,311.0){\usebox{\plotpoint}}
\put(403,313){\usebox{\plotpoint}}
\put(403.0,313.0){\usebox{\plotpoint}}
\put(403.0,314.0){\usebox{\plotpoint}}
\put(404.0,314.0){\rule[-0.200pt]{0.400pt}{0.482pt}}
\put(404.0,316.0){\usebox{\plotpoint}}
\put(405.0,316.0){\usebox{\plotpoint}}
\put(405.0,317.0){\usebox{\plotpoint}}
\put(406.0,317.0){\usebox{\plotpoint}}
\put(406.0,318.0){\usebox{\plotpoint}}
\put(407.0,318.0){\usebox{\plotpoint}}
\put(407.0,319.0){\usebox{\plotpoint}}
\put(408.0,319.0){\usebox{\plotpoint}}
\put(408.0,320.0){\rule[-0.200pt]{0.964pt}{0.400pt}}
\put(412.0,319.0){\usebox{\plotpoint}}
\put(412.0,319.0){\rule[-0.200pt]{0.482pt}{0.400pt}}
\put(414.0,318.0){\usebox{\plotpoint}}
\put(414.0,318.0){\usebox{\plotpoint}}
\put(415.0,317.0){\usebox{\plotpoint}}
\put(415.0,317.0){\usebox{\plotpoint}}
\put(416,314.67){\rule{0.241pt}{0.400pt}}
\multiput(416.00,315.17)(0.500,-1.000){2}{\rule{0.120pt}{0.400pt}}
\put(416.0,316.0){\usebox{\plotpoint}}
\put(417,315){\usebox{\plotpoint}}
\put(417,315){\usebox{\plotpoint}}
\put(417,312.67){\rule{0.241pt}{0.400pt}}
\multiput(417.00,313.17)(0.500,-1.000){2}{\rule{0.120pt}{0.400pt}}
\put(417.0,314.0){\usebox{\plotpoint}}
\put(418,313){\usebox{\plotpoint}}
\put(418,313){\usebox{\plotpoint}}
\put(418,310.67){\rule{0.241pt}{0.400pt}}
\multiput(418.00,311.17)(0.500,-1.000){2}{\rule{0.120pt}{0.400pt}}
\put(418.0,312.0){\usebox{\plotpoint}}
\put(419,311){\usebox{\plotpoint}}
\put(419.0,310.0){\usebox{\plotpoint}}
\put(419.0,310.0){\usebox{\plotpoint}}
\put(420.0,307.0){\rule[-0.200pt]{0.400pt}{0.723pt}}
\put(420.0,307.0){\usebox{\plotpoint}}
\put(421.0,305.0){\rule[-0.200pt]{0.400pt}{0.482pt}}
\put(421.0,305.0){\usebox{\plotpoint}}
\put(422.0,302.0){\rule[-0.200pt]{0.400pt}{0.723pt}}
\put(422.0,302.0){\usebox{\plotpoint}}
\put(423,298.67){\rule{0.241pt}{0.400pt}}
\multiput(423.00,299.17)(0.500,-1.000){2}{\rule{0.120pt}{0.400pt}}
\put(423.0,300.0){\rule[-0.200pt]{0.400pt}{0.482pt}}
\put(424.0,296.0){\rule[-0.200pt]{0.400pt}{0.723pt}}
\put(424.0,296.0){\usebox{\plotpoint}}
\put(425.0,293.0){\rule[-0.200pt]{0.400pt}{0.723pt}}
\put(425.0,293.0){\usebox{\plotpoint}}
\put(426.0,289.0){\rule[-0.200pt]{0.400pt}{0.964pt}}
\put(426.0,289.0){\usebox{\plotpoint}}
\put(427.0,285.0){\rule[-0.200pt]{0.400pt}{0.964pt}}
\put(427.0,285.0){\usebox{\plotpoint}}
\put(428,280.67){\rule{0.241pt}{0.400pt}}
\multiput(428.00,281.17)(0.500,-1.000){2}{\rule{0.120pt}{0.400pt}}
\put(428.0,282.0){\rule[-0.200pt]{0.400pt}{0.723pt}}
\put(429.0,277.0){\rule[-0.200pt]{0.400pt}{0.964pt}}
\put(429.0,277.0){\usebox{\plotpoint}}
\put(430,272.67){\rule{0.241pt}{0.400pt}}
\multiput(430.00,273.17)(0.500,-1.000){2}{\rule{0.120pt}{0.400pt}}
\put(430.0,274.0){\rule[-0.200pt]{0.400pt}{0.723pt}}
\put(431,267.67){\rule{0.241pt}{0.400pt}}
\multiput(431.00,268.17)(0.500,-1.000){2}{\rule{0.120pt}{0.400pt}}
\put(431.0,269.0){\rule[-0.200pt]{0.400pt}{0.964pt}}
\put(432,263.67){\rule{0.241pt}{0.400pt}}
\multiput(432.00,264.17)(0.500,-1.000){2}{\rule{0.120pt}{0.400pt}}
\put(432.0,265.0){\rule[-0.200pt]{0.400pt}{0.723pt}}
\put(433,258.67){\rule{0.241pt}{0.400pt}}
\multiput(433.00,259.17)(0.500,-1.000){2}{\rule{0.120pt}{0.400pt}}
\put(433.0,260.0){\rule[-0.200pt]{0.400pt}{0.964pt}}
\put(434,254.67){\rule{0.241pt}{0.400pt}}
\multiput(434.00,255.17)(0.500,-1.000){2}{\rule{0.120pt}{0.400pt}}
\put(434.0,256.0){\rule[-0.200pt]{0.400pt}{0.723pt}}
\put(435,248.67){\rule{0.241pt}{0.400pt}}
\multiput(435.00,249.17)(0.500,-1.000){2}{\rule{0.120pt}{0.400pt}}
\put(435.0,250.0){\rule[-0.200pt]{0.400pt}{1.204pt}}
\put(436,243.67){\rule{0.241pt}{0.400pt}}
\multiput(436.00,244.17)(0.500,-1.000){2}{\rule{0.120pt}{0.400pt}}
\put(436.0,245.0){\rule[-0.200pt]{0.400pt}{0.964pt}}
\put(437,238.67){\rule{0.241pt}{0.400pt}}
\multiput(437.00,239.17)(0.500,-1.000){2}{\rule{0.120pt}{0.400pt}}
\put(437.0,240.0){\rule[-0.200pt]{0.400pt}{0.964pt}}
\put(438,233.67){\rule{0.241pt}{0.400pt}}
\multiput(438.00,234.17)(0.500,-1.000){2}{\rule{0.120pt}{0.400pt}}
\put(438.0,235.0){\rule[-0.200pt]{0.400pt}{0.964pt}}
\put(439,228.67){\rule{0.241pt}{0.400pt}}
\multiput(439.00,229.17)(0.500,-1.000){2}{\rule{0.120pt}{0.400pt}}
\put(439.0,230.0){\rule[-0.200pt]{0.400pt}{0.964pt}}
\put(440,222.67){\rule{0.241pt}{0.400pt}}
\multiput(440.00,223.17)(0.500,-1.000){2}{\rule{0.120pt}{0.400pt}}
\put(440.0,224.0){\rule[-0.200pt]{0.400pt}{1.204pt}}
\put(441,218.67){\rule{0.241pt}{0.400pt}}
\multiput(441.00,219.17)(0.500,-1.000){2}{\rule{0.120pt}{0.400pt}}
\put(441.0,220.0){\rule[-0.200pt]{0.400pt}{0.723pt}}
\put(442,212.67){\rule{0.241pt}{0.400pt}}
\multiput(442.00,213.17)(0.500,-1.000){2}{\rule{0.120pt}{0.400pt}}
\put(442.0,214.0){\rule[-0.200pt]{0.400pt}{1.204pt}}
\put(443,207.67){\rule{0.241pt}{0.400pt}}
\multiput(443.00,208.17)(0.500,-1.000){2}{\rule{0.120pt}{0.400pt}}
\put(443.0,209.0){\rule[-0.200pt]{0.400pt}{0.964pt}}
\put(444,201.67){\rule{0.241pt}{0.400pt}}
\multiput(444.00,202.17)(0.500,-1.000){2}{\rule{0.120pt}{0.400pt}}
\put(444.0,203.0){\rule[-0.200pt]{0.400pt}{1.204pt}}
\put(445,195.67){\rule{0.241pt}{0.400pt}}
\multiput(445.00,196.17)(0.500,-1.000){2}{\rule{0.120pt}{0.400pt}}
\put(445.0,197.0){\rule[-0.200pt]{0.400pt}{1.204pt}}
\put(446,190.67){\rule{0.241pt}{0.400pt}}
\multiput(446.00,191.17)(0.500,-1.000){2}{\rule{0.120pt}{0.400pt}}
\put(446.0,192.0){\rule[-0.200pt]{0.400pt}{0.964pt}}
\put(446.67,185){\rule{0.400pt}{0.482pt}}
\multiput(446.17,186.00)(1.000,-1.000){2}{\rule{0.400pt}{0.241pt}}
\put(447.0,187.0){\rule[-0.200pt]{0.400pt}{0.964pt}}
\put(448,180.67){\rule{0.241pt}{0.400pt}}
\multiput(448.00,181.17)(0.500,-1.000){2}{\rule{0.120pt}{0.400pt}}
\put(448.0,182.0){\rule[-0.200pt]{0.400pt}{0.723pt}}
\put(449,174.67){\rule{0.241pt}{0.400pt}}
\multiput(449.00,175.17)(0.500,-1.000){2}{\rule{0.120pt}{0.400pt}}
\put(449.0,176.0){\rule[-0.200pt]{0.400pt}{1.204pt}}
\put(450,169.67){\rule{0.241pt}{0.400pt}}
\multiput(450.00,170.17)(0.500,-1.000){2}{\rule{0.120pt}{0.400pt}}
\put(450.0,171.0){\rule[-0.200pt]{0.400pt}{0.964pt}}
\put(450.67,164){\rule{0.400pt}{0.482pt}}
\multiput(450.17,165.00)(1.000,-1.000){2}{\rule{0.400pt}{0.241pt}}
\put(451.0,166.0){\rule[-0.200pt]{0.400pt}{0.964pt}}
\put(452,159.67){\rule{0.241pt}{0.400pt}}
\multiput(452.00,160.17)(0.500,-1.000){2}{\rule{0.120pt}{0.400pt}}
\put(452.0,161.0){\rule[-0.200pt]{0.400pt}{0.723pt}}
\put(452.67,154){\rule{0.400pt}{0.482pt}}
\multiput(452.17,155.00)(1.000,-1.000){2}{\rule{0.400pt}{0.241pt}}
\put(453.0,156.0){\rule[-0.200pt]{0.400pt}{0.964pt}}
\put(454,148.67){\rule{0.241pt}{0.400pt}}
\multiput(454.00,149.17)(0.500,-1.000){2}{\rule{0.120pt}{0.400pt}}
\put(454.0,150.0){\rule[-0.200pt]{0.400pt}{0.964pt}}
\put(455,144.67){\rule{0.241pt}{0.400pt}}
\multiput(455.00,145.17)(0.500,-1.000){2}{\rule{0.120pt}{0.400pt}}
\put(455.0,146.0){\rule[-0.200pt]{0.400pt}{0.723pt}}
\put(456,139.67){\rule{0.241pt}{0.400pt}}
\multiput(456.00,140.17)(0.500,-1.000){2}{\rule{0.120pt}{0.400pt}}
\put(456.0,141.0){\rule[-0.200pt]{0.400pt}{0.964pt}}
\put(457,135.67){\rule{0.241pt}{0.400pt}}
\multiput(457.00,136.17)(0.500,-1.000){2}{\rule{0.120pt}{0.400pt}}
\put(457.0,137.0){\rule[-0.200pt]{0.400pt}{0.723pt}}
\put(458,130.67){\rule{0.241pt}{0.400pt}}
\multiput(458.00,131.17)(0.500,-1.000){2}{\rule{0.120pt}{0.400pt}}
\put(458.0,132.0){\rule[-0.200pt]{0.400pt}{0.964pt}}
\put(459,126.67){\rule{0.241pt}{0.400pt}}
\multiput(459.00,127.17)(0.500,-1.000){2}{\rule{0.120pt}{0.400pt}}
\put(459.0,128.0){\rule[-0.200pt]{0.400pt}{0.723pt}}
\put(460,121.67){\rule{0.241pt}{0.400pt}}
\multiput(460.00,122.17)(0.500,-1.000){2}{\rule{0.120pt}{0.400pt}}
\put(460.0,123.0){\rule[-0.200pt]{0.400pt}{0.964pt}}
\put(461,118.67){\rule{0.241pt}{0.400pt}}
\multiput(461.00,119.17)(0.500,-1.000){2}{\rule{0.120pt}{0.400pt}}
\put(461.0,120.0){\rule[-0.200pt]{0.400pt}{0.482pt}}
\put(462.0,115.0){\rule[-0.200pt]{0.400pt}{0.964pt}}
\put(462.0,115.0){\usebox{\plotpoint}}
\put(463.0,111.0){\rule[-0.200pt]{0.400pt}{0.964pt}}
\put(463.0,111.0){\usebox{\plotpoint}}
\put(464.0,108.0){\rule[-0.200pt]{0.400pt}{0.723pt}}
\put(464.0,108.0){\usebox{\plotpoint}}
\put(465,103.67){\rule{0.241pt}{0.400pt}}
\multiput(465.00,104.17)(0.500,-1.000){2}{\rule{0.120pt}{0.400pt}}
\put(465.0,105.0){\rule[-0.200pt]{0.400pt}{0.723pt}}
\put(466,100.67){\rule{0.241pt}{0.400pt}}
\multiput(466.00,101.17)(0.500,-1.000){2}{\rule{0.120pt}{0.400pt}}
\put(466.0,102.0){\rule[-0.200pt]{0.400pt}{0.482pt}}
\put(467,101){\usebox{\plotpoint}}
\put(467,97.67){\rule{0.241pt}{0.400pt}}
\multiput(467.00,98.17)(0.500,-1.000){2}{\rule{0.120pt}{0.400pt}}
\put(467.0,99.0){\rule[-0.200pt]{0.400pt}{0.482pt}}
\put(468,98){\usebox{\plotpoint}}
\put(468.0,96.0){\rule[-0.200pt]{0.400pt}{0.482pt}}
\put(468.0,96.0){\usebox{\plotpoint}}
\put(469,92.67){\rule{0.241pt}{0.400pt}}
\multiput(469.00,93.17)(0.500,-1.000){2}{\rule{0.120pt}{0.400pt}}
\put(469.0,94.0){\rule[-0.200pt]{0.400pt}{0.482pt}}
\put(470,93){\usebox{\plotpoint}}
\put(470,90.67){\rule{0.241pt}{0.400pt}}
\multiput(470.00,91.17)(0.500,-1.000){2}{\rule{0.120pt}{0.400pt}}
\put(470.0,92.0){\usebox{\plotpoint}}
\put(471,91){\usebox{\plotpoint}}
\put(471,88.67){\rule{0.241pt}{0.400pt}}
\multiput(471.00,89.17)(0.500,-1.000){2}{\rule{0.120pt}{0.400pt}}
\put(471.0,90.0){\usebox{\plotpoint}}
\put(472,89){\usebox{\plotpoint}}
\put(472,86.67){\rule{0.241pt}{0.400pt}}
\multiput(472.00,87.17)(0.500,-1.000){2}{\rule{0.120pt}{0.400pt}}
\put(472.0,88.0){\usebox{\plotpoint}}
\put(473,87){\usebox{\plotpoint}}
\put(473,87){\usebox{\plotpoint}}
\put(473.0,86.0){\usebox{\plotpoint}}
\put(473.0,86.0){\usebox{\plotpoint}}
\put(474.0,85.0){\usebox{\plotpoint}}
\put(474.0,85.0){\usebox{\plotpoint}}
\put(475.0,84.0){\usebox{\plotpoint}}
\put(475.0,84.0){\usebox{\plotpoint}}
\put(476.0,83.0){\usebox{\plotpoint}}
\put(476.0,83.0){\usebox{\plotpoint}}
\put(477.0,82.0){\usebox{\plotpoint}}
\put(477.0,82.0){\rule[-0.200pt]{0.964pt}{0.400pt}}
\put(481.0,82.0){\usebox{\plotpoint}}
\put(481.0,83.0){\rule[-0.200pt]{0.482pt}{0.400pt}}
\put(483.0,83.0){\usebox{\plotpoint}}
\put(483.0,84.0){\usebox{\plotpoint}}
\put(484.0,84.0){\usebox{\plotpoint}}
\put(484.0,85.0){\usebox{\plotpoint}}
\put(485.0,85.0){\rule[-0.200pt]{0.400pt}{0.482pt}}
\put(485.0,87.0){\usebox{\plotpoint}}
\put(486.0,87.0){\usebox{\plotpoint}}
\put(486.0,88.0){\usebox{\plotpoint}}
\put(487,89.67){\rule{0.241pt}{0.400pt}}
\multiput(487.00,89.17)(0.500,1.000){2}{\rule{0.120pt}{0.400pt}}
\put(487.0,88.0){\rule[-0.200pt]{0.400pt}{0.482pt}}
\put(488,91){\usebox{\plotpoint}}
\put(488,91){\usebox{\plotpoint}}
\put(488.0,91.0){\usebox{\plotpoint}}
\put(488.0,92.0){\usebox{\plotpoint}}
\put(489,93.67){\rule{0.241pt}{0.400pt}}
\multiput(489.00,93.17)(0.500,1.000){2}{\rule{0.120pt}{0.400pt}}
\put(489.0,92.0){\rule[-0.200pt]{0.400pt}{0.482pt}}
\put(490,96.67){\rule{0.241pt}{0.400pt}}
\multiput(490.00,96.17)(0.500,1.000){2}{\rule{0.120pt}{0.400pt}}
\put(490.0,95.0){\rule[-0.200pt]{0.400pt}{0.482pt}}
\put(491,98){\usebox{\plotpoint}}
\put(491.0,98.0){\rule[-0.200pt]{0.400pt}{0.482pt}}
\put(491.0,100.0){\usebox{\plotpoint}}
\put(492,102.67){\rule{0.241pt}{0.400pt}}
\multiput(492.00,102.17)(0.500,1.000){2}{\rule{0.120pt}{0.400pt}}
\put(492.0,100.0){\rule[-0.200pt]{0.400pt}{0.723pt}}
\put(493,104){\usebox{\plotpoint}}
\put(493.0,104.0){\rule[-0.200pt]{0.400pt}{0.482pt}}
\put(493.0,106.0){\usebox{\plotpoint}}
\put(494,108.67){\rule{0.241pt}{0.400pt}}
\multiput(494.00,108.17)(0.500,1.000){2}{\rule{0.120pt}{0.400pt}}
\put(494.0,106.0){\rule[-0.200pt]{0.400pt}{0.723pt}}
\put(495,111.67){\rule{0.241pt}{0.400pt}}
\multiput(495.00,111.17)(0.500,1.000){2}{\rule{0.120pt}{0.400pt}}
\put(495.0,110.0){\rule[-0.200pt]{0.400pt}{0.482pt}}
\put(496.0,113.0){\rule[-0.200pt]{0.400pt}{0.964pt}}
\put(496.0,117.0){\usebox{\plotpoint}}
\put(497,119.67){\rule{0.241pt}{0.400pt}}
\multiput(497.00,119.17)(0.500,1.000){2}{\rule{0.120pt}{0.400pt}}
\put(497.0,117.0){\rule[-0.200pt]{0.400pt}{0.723pt}}
\put(498,123.67){\rule{0.241pt}{0.400pt}}
\multiput(498.00,123.17)(0.500,1.000){2}{\rule{0.120pt}{0.400pt}}
\put(498.0,121.0){\rule[-0.200pt]{0.400pt}{0.723pt}}
\put(499,128.67){\rule{0.241pt}{0.400pt}}
\multiput(499.00,128.17)(0.500,1.000){2}{\rule{0.120pt}{0.400pt}}
\put(499.0,125.0){\rule[-0.200pt]{0.400pt}{0.964pt}}
\put(500,132.67){\rule{0.241pt}{0.400pt}}
\multiput(500.00,132.17)(0.500,1.000){2}{\rule{0.120pt}{0.400pt}}
\put(500.0,130.0){\rule[-0.200pt]{0.400pt}{0.723pt}}
\put(501,137.67){\rule{0.241pt}{0.400pt}}
\multiput(501.00,137.17)(0.500,1.000){2}{\rule{0.120pt}{0.400pt}}
\put(501.0,134.0){\rule[-0.200pt]{0.400pt}{0.964pt}}
\put(502,141.67){\rule{0.241pt}{0.400pt}}
\multiput(502.00,141.17)(0.500,1.000){2}{\rule{0.120pt}{0.400pt}}
\put(502.0,139.0){\rule[-0.200pt]{0.400pt}{0.723pt}}
\put(503,146.67){\rule{0.241pt}{0.400pt}}
\multiput(503.00,146.17)(0.500,1.000){2}{\rule{0.120pt}{0.400pt}}
\put(503.0,143.0){\rule[-0.200pt]{0.400pt}{0.964pt}}
\put(503.67,151){\rule{0.400pt}{0.482pt}}
\multiput(503.17,151.00)(1.000,1.000){2}{\rule{0.400pt}{0.241pt}}
\put(504.0,148.0){\rule[-0.200pt]{0.400pt}{0.723pt}}
\put(505,156.67){\rule{0.241pt}{0.400pt}}
\multiput(505.00,156.17)(0.500,1.000){2}{\rule{0.120pt}{0.400pt}}
\put(505.0,153.0){\rule[-0.200pt]{0.400pt}{0.964pt}}
\put(506,160.67){\rule{0.241pt}{0.400pt}}
\multiput(506.00,160.17)(0.500,1.000){2}{\rule{0.120pt}{0.400pt}}
\put(506.0,158.0){\rule[-0.200pt]{0.400pt}{0.723pt}}
\put(507,166.67){\rule{0.241pt}{0.400pt}}
\multiput(507.00,166.17)(0.500,1.000){2}{\rule{0.120pt}{0.400pt}}
\put(507.0,162.0){\rule[-0.200pt]{0.400pt}{1.204pt}}
\put(508,172.67){\rule{0.241pt}{0.400pt}}
\multiput(508.00,172.17)(0.500,1.000){2}{\rule{0.120pt}{0.400pt}}
\put(508.0,168.0){\rule[-0.200pt]{0.400pt}{1.204pt}}
\put(508.67,177){\rule{0.400pt}{0.482pt}}
\multiput(508.17,177.00)(1.000,1.000){2}{\rule{0.400pt}{0.241pt}}
\put(509.0,174.0){\rule[-0.200pt]{0.400pt}{0.723pt}}
\put(510,182.67){\rule{0.241pt}{0.400pt}}
\multiput(510.00,182.17)(0.500,1.000){2}{\rule{0.120pt}{0.400pt}}
\put(510.0,179.0){\rule[-0.200pt]{0.400pt}{0.964pt}}
\put(511,187.67){\rule{0.241pt}{0.400pt}}
\multiput(511.00,187.17)(0.500,1.000){2}{\rule{0.120pt}{0.400pt}}
\put(511.0,184.0){\rule[-0.200pt]{0.400pt}{0.964pt}}
\put(512,193.67){\rule{0.241pt}{0.400pt}}
\multiput(512.00,193.17)(0.500,1.000){2}{\rule{0.120pt}{0.400pt}}
\put(512.0,189.0){\rule[-0.200pt]{0.400pt}{1.204pt}}
\put(513,198.67){\rule{0.241pt}{0.400pt}}
\multiput(513.00,198.17)(0.500,1.000){2}{\rule{0.120pt}{0.400pt}}
\put(513.0,195.0){\rule[-0.200pt]{0.400pt}{0.964pt}}
\put(514,204.67){\rule{0.241pt}{0.400pt}}
\multiput(514.00,204.17)(0.500,1.000){2}{\rule{0.120pt}{0.400pt}}
\put(514.0,200.0){\rule[-0.200pt]{0.400pt}{1.204pt}}
\put(514.67,209){\rule{0.400pt}{0.482pt}}
\multiput(514.17,209.00)(1.000,1.000){2}{\rule{0.400pt}{0.241pt}}
\put(515.0,206.0){\rule[-0.200pt]{0.400pt}{0.723pt}}
\put(516,214.67){\rule{0.241pt}{0.400pt}}
\multiput(516.00,214.17)(0.500,1.000){2}{\rule{0.120pt}{0.400pt}}
\put(516.0,211.0){\rule[-0.200pt]{0.400pt}{0.964pt}}
\put(517,220.67){\rule{0.241pt}{0.400pt}}
\multiput(517.00,220.17)(0.500,1.000){2}{\rule{0.120pt}{0.400pt}}
\put(517.0,216.0){\rule[-0.200pt]{0.400pt}{1.204pt}}
\put(518,225.67){\rule{0.241pt}{0.400pt}}
\multiput(518.00,225.17)(0.500,1.000){2}{\rule{0.120pt}{0.400pt}}
\put(518.0,222.0){\rule[-0.200pt]{0.400pt}{0.964pt}}
\put(519,231.67){\rule{0.241pt}{0.400pt}}
\multiput(519.00,231.17)(0.500,1.000){2}{\rule{0.120pt}{0.400pt}}
\put(519.0,227.0){\rule[-0.200pt]{0.400pt}{1.204pt}}
\put(520,235.67){\rule{0.241pt}{0.400pt}}
\multiput(520.00,235.17)(0.500,1.000){2}{\rule{0.120pt}{0.400pt}}
\put(520.0,233.0){\rule[-0.200pt]{0.400pt}{0.723pt}}
\put(521,241.67){\rule{0.241pt}{0.400pt}}
\multiput(521.00,241.17)(0.500,1.000){2}{\rule{0.120pt}{0.400pt}}
\put(521.0,237.0){\rule[-0.200pt]{0.400pt}{1.204pt}}
\put(522,245.67){\rule{0.241pt}{0.400pt}}
\multiput(522.00,245.17)(0.500,1.000){2}{\rule{0.120pt}{0.400pt}}
\put(522.0,243.0){\rule[-0.200pt]{0.400pt}{0.723pt}}
\put(523,251.67){\rule{0.241pt}{0.400pt}}
\multiput(523.00,251.17)(0.500,1.000){2}{\rule{0.120pt}{0.400pt}}
\put(523.0,247.0){\rule[-0.200pt]{0.400pt}{1.204pt}}
\put(524,255.67){\rule{0.241pt}{0.400pt}}
\multiput(524.00,255.17)(0.500,1.000){2}{\rule{0.120pt}{0.400pt}}
\put(524.0,253.0){\rule[-0.200pt]{0.400pt}{0.723pt}}
\put(525,260.67){\rule{0.241pt}{0.400pt}}
\multiput(525.00,260.17)(0.500,1.000){2}{\rule{0.120pt}{0.400pt}}
\put(525.0,257.0){\rule[-0.200pt]{0.400pt}{0.964pt}}
\put(526,265.67){\rule{0.241pt}{0.400pt}}
\multiput(526.00,265.17)(0.500,1.000){2}{\rule{0.120pt}{0.400pt}}
\put(526.0,262.0){\rule[-0.200pt]{0.400pt}{0.964pt}}
\put(527,269.67){\rule{0.241pt}{0.400pt}}
\multiput(527.00,269.17)(0.500,1.000){2}{\rule{0.120pt}{0.400pt}}
\put(527.0,267.0){\rule[-0.200pt]{0.400pt}{0.723pt}}
\put(528,274.67){\rule{0.241pt}{0.400pt}}
\multiput(528.00,274.17)(0.500,1.000){2}{\rule{0.120pt}{0.400pt}}
\put(528.0,271.0){\rule[-0.200pt]{0.400pt}{0.964pt}}
\put(529,278.67){\rule{0.241pt}{0.400pt}}
\multiput(529.00,278.17)(0.500,1.000){2}{\rule{0.120pt}{0.400pt}}
\put(529.0,276.0){\rule[-0.200pt]{0.400pt}{0.723pt}}
\put(530,280){\usebox{\plotpoint}}
\put(530,282.67){\rule{0.241pt}{0.400pt}}
\multiput(530.00,282.17)(0.500,1.000){2}{\rule{0.120pt}{0.400pt}}
\put(530.0,280.0){\rule[-0.200pt]{0.400pt}{0.723pt}}
\put(531,285.67){\rule{0.241pt}{0.400pt}}
\multiput(531.00,285.17)(0.500,1.000){2}{\rule{0.120pt}{0.400pt}}
\put(531.0,284.0){\rule[-0.200pt]{0.400pt}{0.482pt}}
\put(532,289.67){\rule{0.241pt}{0.400pt}}
\multiput(532.00,289.17)(0.500,1.000){2}{\rule{0.120pt}{0.400pt}}
\put(532.0,287.0){\rule[-0.200pt]{0.400pt}{0.723pt}}
\put(533.0,291.0){\rule[-0.200pt]{0.400pt}{0.723pt}}
\put(533.0,294.0){\usebox{\plotpoint}}
\put(534,296.67){\rule{0.241pt}{0.400pt}}
\multiput(534.00,296.17)(0.500,1.000){2}{\rule{0.120pt}{0.400pt}}
\put(534.0,294.0){\rule[-0.200pt]{0.400pt}{0.723pt}}
\put(535.0,298.0){\rule[-0.200pt]{0.400pt}{0.723pt}}
\put(535.0,301.0){\usebox{\plotpoint}}
\put(536,302.67){\rule{0.241pt}{0.400pt}}
\multiput(536.00,302.17)(0.500,1.000){2}{\rule{0.120pt}{0.400pt}}
\put(536.0,301.0){\rule[-0.200pt]{0.400pt}{0.482pt}}
\put(537,304){\usebox{\plotpoint}}
\put(537,305.67){\rule{0.241pt}{0.400pt}}
\multiput(537.00,305.17)(0.500,1.000){2}{\rule{0.120pt}{0.400pt}}
\put(537.0,304.0){\rule[-0.200pt]{0.400pt}{0.482pt}}
\put(538,307){\usebox{\plotpoint}}
\put(538,307.67){\rule{0.241pt}{0.400pt}}
\multiput(538.00,307.17)(0.500,1.000){2}{\rule{0.120pt}{0.400pt}}
\put(538.0,307.0){\usebox{\plotpoint}}
\put(539,309){\usebox{\plotpoint}}
\put(539.0,309.0){\usebox{\plotpoint}}
\end{picture}
\normalsize 
    \scriptsize Zeigerbild: \normalsize \\
    \input{Physik/Schwingungen/Zeigerbild.pstex_t} 
    \scriptsize Phasenkurve: \normalsize \\
    \input{Physik/Schwingungen/Phasenkurve.pstex_t}
  }

  \Formelbox{
      \footnotesize Funktion: \normalsize  \[ y = A \sin(\omega t + \varphi) \]
      \[ T = \frac{2 \pi}{\omega} \]
      \[ f = \frac{1}{T} \]
      \[ \omega = 2 \pi f \]
      \[ \ddot y + \omega ^2 y = 0 \] 
      \footnotesize Bei einer harmonischen Schwingung ist die Beschleunigung proportional zur Auslenkung: \normalsize
      \[ a(t) = \ddot y = -A \omega _0 ^2 \sin (\omega _0 t ) \]
      \[v(t) = \dot y = A  \omega_0 \cos (\omega _0 t ) \]
      \footnotesize Energie bleibt konstant: \normalsize 
      \[ E_{ges} = \frac{1}{2} c A^2  = E_{pot} + E_{kin} \]
      \[ E_{pot} = \frac{1}{2} c A^2 \cos ^2 (\omega t + \varphi ) \]
      \[ E_{kin} = \frac{1}{2} c A^2 \sin ^2 (\omega t + \varphi ) \] 
      }
  }
  {\Groessenbox{
     $ y$                & schwingende Gr�sse                    & $ [m]\footnotemark $ \\
     $ A$                & Amplitude                             & $ [1] $              \\
     $ \omega$           & Kreisfrequenz                         & $ [\frac{1}{s}] $    \\
     $ \varphi$          & Nullphasen"-winkel                    & $ [rad] $        \\
     $ t$                & Zeit                                  & $ [s] $          \\
     $T$                 & Periode                               & $ [s] $ \\
     $f$                 & Frequenz                              & $ [\frac{1}{s}] $ \\
     $a$                 & Beschleuni"-gung                      & $ [\frac{m}{s^2} ] $ \\
     $v$                 & Geschw.                               & $ [\frac{m}{s} ] $ \\
     $E$                 & Energie                               & $ [ J ] $ \\
     $c$                 & Konstante                             & $ [\frac{N}{m}] $\\
}}
\vfill


 \footnotetext{m gilt nur bei mechanischen Schwingungen} 

\subsection{Unged�mpfte, periodische Schwingung}

\index{Schwingungen!periodische Schwingung}

\Hauptbox{
  \Bildbox{
    \input{Physik/Schwingungen/spektrum.pstex_t}
  }
  \Formelbox{
      \footnotesize Fourierreihe: \normalsize   
      \[ y = A_0 + \sum _{k=1} ^\infty A_k \sin (\omega _k t + \varphi _k ) \]      
      \[ \omega _k = k \omega_1 \]
      \[ \omega _1 = \frac{2 \pi}{T} \]
      \scriptsize \[ y = A_0 + \sum _{k=1} ^\infty [a_k \sin (\omega_k t) + b_k \cos (\omega _k t)] \] 
      \[ A_k = \sqrt{a_k ^2 + b_k ^2} \]
      \[ \varphi _k = \arctan \left( \frac{b_k}{a_k} \right) \] \normalsize
      }
  }
  {\Groessenbox{
     $ y$                & schwingende Gr�sse                    & $ [m] $               \\
     $ A$                & Amplitude                             & $ [1] $               \\
     $ \omega$           & Kreisfrequenz                         & $ [\frac{1}{s}] $     \\
     $ \varphi$          & Nullphasen"-winkel                    & $ [rad] $             \\
     $T$                 & Periode der Grundschwingung           & $ [s] $ \\
}}


\subsection{Unged�mpfte, nicht periodische Schwingung}

\index{Schwingungen!aperiodische Schwingung}

\Hauptbox{
  \Bildbox{
    \boxdump
  }
  \Formelbox{
      \footnotesize Fourierreihe: \normalsize   
      \[ y = \int _\infty ^\infty A(\omega) e^{j\omega t} d \omega \] 
      \vspace{0.9cm}
      }
  }
  {\Groessenbox{
     $ y$                & schwingende Gr�sse                    & $ [m] $               \\
     $ A$                & Amplitude                             & $ [1] $               \\
     $ \omega$           & Kreisfrequenz                         & $ [\frac{1}{s}] $     \\
     $ t$                & Zeit                                  & $ [s] $               \\
}}
\vfill



\subsection{Federpendel}

\index{Schwingungen!Federpendel}
\index{Pendel!Federpendel}

\Hauptbox{
  \Bildbox{
    \input{Physik/Schwingungen/federpendel1.pstex_t}
  }
  \Formelbox{
      \footnotesize Federmasse vernachl�ssigt: \normalsize   
      \[ m \ddot y + c y = 0 \] 
      \[ y = A \sin ( \omega_0 t + \varphi ) \]
      \[ \omega_0 = \sqrt{\frac{c}{m}} \]
      \[ T = 2 \pi \sqrt{\frac{m}{c}} \]
      \[ a(t) = -\left ( \frac{c}{m} \right) y(t) \]
      \footnotesize Energiesatz: \normalsize  \[ \frac{1}{2} c A^2 = \frac{1}{2} c y^2(t) + \frac{1}{2} m v^2(t) \]
      }\\ \hline
    \Bildbox{
      \input{Physik/Schwingungen/federpendel2.pstex_t}
      }
    \Formelbox{
      \footnotesize Mit Federmasse: \normalsize   
      \[ T = 2 \pi \sqrt{\frac{m + \frac{m_F}{3}}{c}} \]
      }
    }
  {\Groessenbox{
     $ y$                & schwingende Gr�sse                    & $ [m] $ \\
     $ A$                & Amplitude                             & $ [1] $              \\
     $ \omega$           & Kreisfrequenz                         & $ [\frac{1}{s}] $    \\
     $ \varphi$          & Nullphasen"-winkel                    & $ [rad] $        \\
     $ t$                & Zeit                                  & $ [s] $          \\
     $T$                 & Periode                               & $ [s] $ \\
     $f$                 & Frequenz                              & $ [\frac{1}{s}] $ \\
     $m$                 & Bewegte Masse                         & $ [ kg ] $ \\
     $m_F$               & Masse der Feder                       & $ [ kg ] $ \\
     $c$                 & Feder"-kon"-stan"-te (siehe S. \pageref{cfeder} )   & $ [ \frac{N}{m} ] $ \\
     $a$                 & Beschleuni"-gung                      & $ [\frac{m}{s^2} ] $ \\
}}


\subsection{Drehpendel}

\index{Schwingungen!Drehpendel}
\index{Pendel!Drehpendel}

\Hauptbox{
  \Bildbox{
    \input{Physik/Schwingungen/drehpendel.pstex_t}
  }
  \Formelbox{
    \[ J \ddot y + c \varphi = 0 \] 
    \[ \omega _0 = \sqrt{\frac{c}{J}} \]
    \[ T = 2 \pi \sqrt{\frac{J}{c}} \]
    \footnotesize bestimmung von Massentr�gheitsmomenten: \normalsize
    \[ J_{unbek} = J_{bek} \frac{T_{unbek} ^2}{T_{bek} ^2} \]
    }
  }
  {\Groessenbox{
     $ y$                & schwingende Gr�sse                    & $ [m] $ \\
     $ \omega$           & Kreisfrequenz                         & $ [\frac{1}{s}] $    \\
     $ \varphi$          & Nullphasen"-winkel                    & $ [rad] $        \\
     $T$                 & Periode                               & $ [s] $ \\
     $c$                 & Feder"-kon"-stan"-te                  & $ [ \frac{N}{m} ] $ \\
     $J$                 & Massen"-tr�g"-heits"-mo"-ment         & $ [ \frac{kg}{m^2} ] $ \\
}}
\vfill

\subsection{Mathematisches Pendel}

\index{Schwingungen!Mathematisches Pendel}
\index{Pendel!Mathematisches Pendel}

\Hauptbox{
  \Bildbox{
    \input{Physik/Schwingungen/mathpendel.pstex_t}
  }
  \Formelbox{
    \[ l \ddot y + g \sin (\varphi ) = 0 \] 
    \[ l \ddot y + g \varphi = 0 \]
    \[ \omega _0 = \sqrt{\frac{g}{l}} \]
    \[ T = 2 \pi \sqrt{\frac{l}{g}} \]
    }
  }
  {\Groessenbox{
     $ y$                & schwingende Gr�sse                    & $ [m] $ \\
     $ \omega$           & Kreisfrequenz                         & $ [\frac{1}{s}] $    \\
     $ \varphi$          & Winkel                                & $ [rad] $        \\
     $T$                 & Periode                               & $ [s] $ \\
     $l$                 & Pendell�nge                           & $ [m] $ \\
     $g$                 & Erdbeschl. $=9.81$                    & $ [ \frac{m}{s^2} ] $ \\
}}


\subsection{Physikalisches Pendel}

\index{Schwingungen!Physikalisches Pendel}
\index{Pendel!Physikalisches Pendel}

\Hauptbox{
  \Bildbox{
    \input{Physik/Schwingungen/physpendel.pstex_t} 
    \vspace{5cm}
    \footnotesize Perkussionszentrum \normalsize
    \input{Physik/Schwingungen/perkussion.pstex_t}
  }
  \Formelbox{
    \[ J_A \ddot \varphi  + m g a \sin (\varphi ) = 0 \] 
    \[ l^* = \frac{J_A}{ma}  \]
    \[ J_A \ddot \varphi + m g a \varphi = 0 \]
    \[ \omega _0 = \sqrt{\frac{mga}{J_A}} \]
    \[ T = 2 \pi \sqrt{\frac{J_A}{mga}} =2 \pi \sqrt{\frac{J_S + m a^2}{mga}}\] 
    \[ T = 2 \pi \sqrt{\frac{J_{A1} + J_{A2}}{g (m_1 a_1 + m_2 a_2)}} \]
    \[ J_A = J_S + m a^2 \]
    \[ \omega _{max} = \sqrt{\frac{g}{2} \sqrt{\frac{m}{J_S}}} \]
    \[ \omega (A) = \omega(M) \]
    \[ x = \frac{J_S}{m a} \]
    \[ E = \frac{ m \omega_0 ^2 A ^2}{2} \]
    } 
  }
  {\Groessenbox{
     $ J_A$              & Massetr�gheit bez. A-Achse            & $ [kgm^2] $ \\
     $ J_S$              & Massetr�gheit bez. Achse $\|$ a (siehe S. \pageref{J})     & $ [kgm^2] $ \\
     $ S$                & Schwerpunkt                           & $ [1] $    \\
     $ a$                & Anstand zu S                          & $ [m] $        \\
     $T$                 & Periode                               & $ [s] $ \\
     $l^*$               & Pendell�nge des entsp. math. Pendel   & $ [m] $ \\
     $ y$                & schwingende Gr�sse                    & $ [m] $ \\
     $ A$                & Amplitude                             & $ [1] $              \\
     $ \omega$           & Kreisfrequenz                         & $ [\frac{1}{s}] $    \\
     $ \varphi$          & Winkel                                & $ [rad] $        \\
     $ t$                & Zeit                                  & $ [s] $          \\
     $f$                 & Frequenz                              & $ [\frac{1}{s}] $ \\
     $m$                 & Bewegte Masse                         & $ [ kg ] $ \\
     $E$                 & Energie                               & $ [ J ] $ \\
     $M$                 & Schwing-Mit"-tel"-punkt, Per"-kus"-sions"-zen"-trum & \\ 
     $x$                 & Abstand M,S                           & $[m]$\\
     $g$                 & Erdbeschl. $ = 9.81$                  & $ [\frac{m}{s^2} ] $ \\
}}
\vfill

\subsection{Ged�mpfte Schwingung mit konstanter Reibung}

\index{Schwingung!Ged�mpfte Schwingung}
\index{Ged�mpfte Schwingung}

\Hauptbox{
  \Bildbox{
    % GNUPLOT: LaTeX picture
\setlength{\unitlength}{0.240900pt}
\ifx\plotpoint\undefined\newsavebox{\plotpoint}\fi
\sbox{\plotpoint}{\rule[-0.200pt]{0.400pt}{0.400pt}}%
\scriptsize
\begin{picture}(600,360)(60,0)
\font\gnuplot=cmr10 at 10pt
\gnuplot
\sbox{\plotpoint}{\rule[-0.200pt]{0.400pt}{0.400pt}}%
\put(120.0,82.0){\rule[-0.200pt]{4.818pt}{0.400pt}}
%\put(100,82){\makebox(0,0)[r]{-50}}
\put(519.0,82.0){\rule[-0.200pt]{4.818pt}{0.400pt}}
\put(120.0,106.0){\rule[-0.200pt]{4.818pt}{0.400pt}}
\put(100,106){\makebox(0,0)[r]{-0.8}}
\put(519.0,106.0){\rule[-0.200pt]{4.818pt}{0.400pt}}
\put(120.0,130.0){\rule[-0.200pt]{4.818pt}{0.400pt}}
%\put(100,130){\makebox(0,0)[r]{-30}}
\put(519.0,130.0){\rule[-0.200pt]{4.818pt}{0.400pt}}
\put(120.0,153.0){\rule[-0.200pt]{4.818pt}{0.400pt}}
\put(100,153){\makebox(0,0)[r]{-0.4}}
\put(519.0,153.0){\rule[-0.200pt]{4.818pt}{0.400pt}}
\put(120.0,177.0){\rule[-0.200pt]{4.818pt}{0.400pt}}
%\put(100,177){\makebox(0,0)[r]{-10}}
\put(519.0,177.0){\rule[-0.200pt]{4.818pt}{0.400pt}}
\put(120.0,201.0){\rule[-0.200pt]{4.818pt}{0.400pt}}
\put(100,201){\makebox(0,0)[r]{0}}
\put(519.0,201.0){\rule[-0.200pt]{4.818pt}{0.400pt}}
\put(120.0,225.0){\rule[-0.200pt]{4.818pt}{0.400pt}}
%\put(100,225){\makebox(0,0)[r]{10}}
\put(519.0,225.0){\rule[-0.200pt]{4.818pt}{0.400pt}}
\put(120.0,249.0){\rule[-0.200pt]{4.818pt}{0.400pt}}
\put(100,249){\makebox(0,0)[r]{0.4}}
\put(519.0,249.0){\rule[-0.200pt]{4.818pt}{0.400pt}}
\put(120.0,272.0){\rule[-0.200pt]{4.818pt}{0.400pt}}
%\put(100,272){\makebox(0,0)[r]{30}}
\put(519.0,272.0){\rule[-0.200pt]{4.818pt}{0.400pt}}
\put(120.0,296.0){\rule[-0.200pt]{4.818pt}{0.400pt}}
\put(100,296){\makebox(0,0)[r]{0.8}}
\put(519.0,296.0){\rule[-0.200pt]{4.818pt}{0.400pt}}
\put(120.0,320.0){\rule[-0.200pt]{4.818pt}{0.400pt}}
%\put(100,320){\makebox(0,0)[r]{50}}
\put(519.0,320.0){\rule[-0.200pt]{4.818pt}{0.400pt}}
\put(120.0,82.0){\rule[-0.200pt]{0.400pt}{4.818pt}}
\put(120,41){\makebox(0,0){0}}
\put(120.0,300.0){\rule[-0.200pt]{0.400pt}{4.818pt}}
\put(162.0,82.0){\rule[-0.200pt]{0.400pt}{4.818pt}}
%\put(162,41){\makebox(0,0){5}}
\put(162.0,300.0){\rule[-0.200pt]{0.400pt}{4.818pt}}
\put(204.0,82.0){\rule[-0.200pt]{0.400pt}{4.818pt}}
\put(204,41){\makebox(0,0){10}}
\put(204.0,300.0){\rule[-0.200pt]{0.400pt}{4.818pt}}
\put(246.0,82.0){\rule[-0.200pt]{0.400pt}{4.818pt}}
%\put(246,41){\makebox(0,0){15}}
\put(246.0,300.0){\rule[-0.200pt]{0.400pt}{4.818pt}}
\put(288.0,82.0){\rule[-0.200pt]{0.400pt}{4.818pt}}
\put(288,41){\makebox(0,0){20}}
\put(288.0,300.0){\rule[-0.200pt]{0.400pt}{4.818pt}}
\put(330.0,82.0){\rule[-0.200pt]{0.400pt}{4.818pt}}
%\put(330,41){\makebox(0,0){25}}
\put(330.0,300.0){\rule[-0.200pt]{0.400pt}{4.818pt}}
\put(371.0,82.0){\rule[-0.200pt]{0.400pt}{4.818pt}}
\put(371,41){\makebox(0,0){30}}
\put(371.0,300.0){\rule[-0.200pt]{0.400pt}{4.818pt}}
\put(413.0,82.0){\rule[-0.200pt]{0.400pt}{4.818pt}}
%\put(413,41){\makebox(0,0){35}}
\put(413.0,300.0){\rule[-0.200pt]{0.400pt}{4.818pt}}
\put(455.0,82.0){\rule[-0.200pt]{0.400pt}{4.818pt}}
\put(455,41){\makebox(0,0){40}}
\put(455.0,300.0){\rule[-0.200pt]{0.400pt}{4.818pt}}
\put(497.0,82.0){\rule[-0.200pt]{0.400pt}{4.818pt}}
%\put(497,41){\makebox(0,0){45}}
\put(497.0,300.0){\rule[-0.200pt]{0.400pt}{4.818pt}}
\put(539.0,82.0){\rule[-0.200pt]{0.400pt}{4.818pt}}
\put(539,41){\makebox(0,0){50}}
\put(539.0,300.0){\rule[-0.200pt]{0.400pt}{4.818pt}}
\put(120.0,82.0){\rule[-0.200pt]{100.937pt}{0.400pt}}
\put(539.0,82.0){\rule[-0.200pt]{0.400pt}{57.334pt}}
\put(120.0,320.0){\rule[-0.200pt]{100.937pt}{0.400pt}}
\put(120.0,82.0){\rule[-0.200pt]{0.400pt}{57.334pt}}
\put(120,201){\usebox{\plotpoint}}
\put(120,206.67){\rule{0.241pt}{0.400pt}}
\multiput(120.00,206.17)(0.500,1.000){2}{\rule{0.120pt}{0.400pt}}
\put(120.0,201.0){\rule[-0.200pt]{0.400pt}{1.445pt}}
\put(121,220.67){\rule{0.241pt}{0.400pt}}
\multiput(121.00,220.17)(0.500,1.000){2}{\rule{0.120pt}{0.400pt}}
\put(121.0,208.0){\rule[-0.200pt]{0.400pt}{3.132pt}}
\put(122,234.67){\rule{0.241pt}{0.400pt}}
\multiput(122.00,234.17)(0.500,1.000){2}{\rule{0.120pt}{0.400pt}}
\put(122.0,222.0){\rule[-0.200pt]{0.400pt}{3.132pt}}
\put(123,247.67){\rule{0.241pt}{0.400pt}}
\multiput(123.00,247.17)(0.500,1.000){2}{\rule{0.120pt}{0.400pt}}
\put(123.0,236.0){\rule[-0.200pt]{0.400pt}{2.891pt}}
\put(124,260.67){\rule{0.241pt}{0.400pt}}
\multiput(124.00,260.17)(0.500,1.000){2}{\rule{0.120pt}{0.400pt}}
\put(124.0,249.0){\rule[-0.200pt]{0.400pt}{2.891pt}}
\put(125,271.67){\rule{0.241pt}{0.400pt}}
\multiput(125.00,271.17)(0.500,1.000){2}{\rule{0.120pt}{0.400pt}}
\put(125.0,262.0){\rule[-0.200pt]{0.400pt}{2.409pt}}
\put(126.0,273.0){\rule[-0.200pt]{0.400pt}{2.409pt}}
\put(126.0,283.0){\usebox{\plotpoint}}
\put(127,291.67){\rule{0.241pt}{0.400pt}}
\multiput(127.00,291.17)(0.500,1.000){2}{\rule{0.120pt}{0.400pt}}
\put(127.0,283.0){\rule[-0.200pt]{0.400pt}{2.168pt}}
\put(128,293){\usebox{\plotpoint}}
\put(128.0,293.0){\rule[-0.200pt]{0.400pt}{1.686pt}}
\put(128.0,300.0){\usebox{\plotpoint}}
\put(129,305.67){\rule{0.241pt}{0.400pt}}
\multiput(129.00,305.17)(0.500,1.000){2}{\rule{0.120pt}{0.400pt}}
\put(129.0,300.0){\rule[-0.200pt]{0.400pt}{1.445pt}}
\put(130,307){\usebox{\plotpoint}}
\put(130.0,307.0){\rule[-0.200pt]{0.400pt}{0.964pt}}
\put(130.0,311.0){\usebox{\plotpoint}}
\put(131,313.67){\rule{0.241pt}{0.400pt}}
\multiput(131.00,313.17)(0.500,1.000){2}{\rule{0.120pt}{0.400pt}}
\put(131.0,311.0){\rule[-0.200pt]{0.400pt}{0.723pt}}
\put(132,315){\usebox{\plotpoint}}
\put(132,315){\usebox{\plotpoint}}
\put(132,315){\usebox{\plotpoint}}
\put(132,315){\usebox{\plotpoint}}
\put(132,315){\usebox{\plotpoint}}
\put(132.0,315.0){\usebox{\plotpoint}}
\put(132.0,316.0){\rule[-0.200pt]{0.482pt}{0.400pt}}
\put(134.0,314.0){\rule[-0.200pt]{0.400pt}{0.482pt}}
\put(134.0,314.0){\usebox{\plotpoint}}
\put(135.0,311.0){\rule[-0.200pt]{0.400pt}{0.723pt}}
\put(135.0,311.0){\usebox{\plotpoint}}
\put(136,305.67){\rule{0.241pt}{0.400pt}}
\multiput(136.00,306.17)(0.500,-1.000){2}{\rule{0.120pt}{0.400pt}}
\put(136.0,307.0){\rule[-0.200pt]{0.400pt}{0.964pt}}
\put(137,306){\usebox{\plotpoint}}
\put(137,299.67){\rule{0.241pt}{0.400pt}}
\multiput(137.00,300.17)(0.500,-1.000){2}{\rule{0.120pt}{0.400pt}}
\put(137.0,301.0){\rule[-0.200pt]{0.400pt}{1.204pt}}
\put(138,291.67){\rule{0.241pt}{0.400pt}}
\multiput(138.00,292.17)(0.500,-1.000){2}{\rule{0.120pt}{0.400pt}}
\put(138.0,293.0){\rule[-0.200pt]{0.400pt}{1.686pt}}
\put(139,292){\usebox{\plotpoint}}
\put(139,282.67){\rule{0.241pt}{0.400pt}}
\multiput(139.00,283.17)(0.500,-1.000){2}{\rule{0.120pt}{0.400pt}}
\put(139.0,284.0){\rule[-0.200pt]{0.400pt}{1.927pt}}
\put(140,272.67){\rule{0.241pt}{0.400pt}}
\multiput(140.00,273.17)(0.500,-1.000){2}{\rule{0.120pt}{0.400pt}}
\put(140.0,274.0){\rule[-0.200pt]{0.400pt}{2.168pt}}
\put(141,261.67){\rule{0.241pt}{0.400pt}}
\multiput(141.00,262.17)(0.500,-1.000){2}{\rule{0.120pt}{0.400pt}}
\put(141.0,263.0){\rule[-0.200pt]{0.400pt}{2.409pt}}
\put(142,249.67){\rule{0.241pt}{0.400pt}}
\multiput(142.00,250.17)(0.500,-1.000){2}{\rule{0.120pt}{0.400pt}}
\put(142.0,251.0){\rule[-0.200pt]{0.400pt}{2.650pt}}
\put(143,237.67){\rule{0.241pt}{0.400pt}}
\multiput(143.00,238.17)(0.500,-1.000){2}{\rule{0.120pt}{0.400pt}}
\put(143.0,239.0){\rule[-0.200pt]{0.400pt}{2.650pt}}
\put(144,224.67){\rule{0.241pt}{0.400pt}}
\multiput(144.00,225.17)(0.500,-1.000){2}{\rule{0.120pt}{0.400pt}}
\put(144.0,226.0){\rule[-0.200pt]{0.400pt}{2.891pt}}
\put(145,210.67){\rule{0.241pt}{0.400pt}}
\multiput(145.00,211.17)(0.500,-1.000){2}{\rule{0.120pt}{0.400pt}}
\put(145.0,212.0){\rule[-0.200pt]{0.400pt}{3.132pt}}
\put(146,197.67){\rule{0.241pt}{0.400pt}}
\multiput(146.00,198.17)(0.500,-1.000){2}{\rule{0.120pt}{0.400pt}}
\put(146.0,199.0){\rule[-0.200pt]{0.400pt}{2.891pt}}
\put(147,184.67){\rule{0.241pt}{0.400pt}}
\multiput(147.00,185.17)(0.500,-1.000){2}{\rule{0.120pt}{0.400pt}}
\put(147.0,186.0){\rule[-0.200pt]{0.400pt}{2.891pt}}
\put(148,171.67){\rule{0.241pt}{0.400pt}}
\multiput(148.00,172.17)(0.500,-1.000){2}{\rule{0.120pt}{0.400pt}}
\put(148.0,173.0){\rule[-0.200pt]{0.400pt}{2.891pt}}
\put(149,158.67){\rule{0.241pt}{0.400pt}}
\multiput(149.00,159.17)(0.500,-1.000){2}{\rule{0.120pt}{0.400pt}}
\put(149.0,160.0){\rule[-0.200pt]{0.400pt}{2.891pt}}
\put(150,147.67){\rule{0.241pt}{0.400pt}}
\multiput(150.00,148.17)(0.500,-1.000){2}{\rule{0.120pt}{0.400pt}}
\put(150.0,149.0){\rule[-0.200pt]{0.400pt}{2.409pt}}
\put(151,136.67){\rule{0.241pt}{0.400pt}}
\multiput(151.00,137.17)(0.500,-1.000){2}{\rule{0.120pt}{0.400pt}}
\put(151.0,138.0){\rule[-0.200pt]{0.400pt}{2.409pt}}
\put(152,126.67){\rule{0.241pt}{0.400pt}}
\multiput(152.00,127.17)(0.500,-1.000){2}{\rule{0.120pt}{0.400pt}}
\put(152.0,128.0){\rule[-0.200pt]{0.400pt}{2.168pt}}
\put(153,117.67){\rule{0.241pt}{0.400pt}}
\multiput(153.00,118.17)(0.500,-1.000){2}{\rule{0.120pt}{0.400pt}}
\put(153.0,119.0){\rule[-0.200pt]{0.400pt}{1.927pt}}
\put(154,109.67){\rule{0.241pt}{0.400pt}}
\multiput(154.00,110.17)(0.500,-1.000){2}{\rule{0.120pt}{0.400pt}}
\put(154.0,111.0){\rule[-0.200pt]{0.400pt}{1.686pt}}
\put(155,110){\usebox{\plotpoint}}
\put(155.0,104.0){\rule[-0.200pt]{0.400pt}{1.445pt}}
\put(155.0,104.0){\usebox{\plotpoint}}
\put(156.0,99.0){\rule[-0.200pt]{0.400pt}{1.204pt}}
\put(156.0,99.0){\usebox{\plotpoint}}
\put(157.0,96.0){\rule[-0.200pt]{0.400pt}{0.723pt}}
\put(157.0,96.0){\usebox{\plotpoint}}
\put(158.0,94.0){\rule[-0.200pt]{0.400pt}{0.482pt}}
\put(158.0,94.0){\usebox{\plotpoint}}
\put(159.0,93.0){\usebox{\plotpoint}}
\put(159.0,93.0){\usebox{\plotpoint}}
\put(160.0,93.0){\usebox{\plotpoint}}
\put(160.0,94.0){\usebox{\plotpoint}}
\put(161.0,94.0){\rule[-0.200pt]{0.400pt}{0.723pt}}
\put(161.0,97.0){\usebox{\plotpoint}}
\put(162.0,97.0){\rule[-0.200pt]{0.400pt}{0.964pt}}
\put(162.0,101.0){\usebox{\plotpoint}}
\put(163,105.67){\rule{0.241pt}{0.400pt}}
\multiput(163.00,105.17)(0.500,1.000){2}{\rule{0.120pt}{0.400pt}}
\put(163.0,101.0){\rule[-0.200pt]{0.400pt}{1.204pt}}
\put(164,107){\usebox{\plotpoint}}
\put(164,112.67){\rule{0.241pt}{0.400pt}}
\multiput(164.00,112.17)(0.500,1.000){2}{\rule{0.120pt}{0.400pt}}
\put(164.0,107.0){\rule[-0.200pt]{0.400pt}{1.445pt}}
\put(165,114){\usebox{\plotpoint}}
\put(165,119.67){\rule{0.241pt}{0.400pt}}
\multiput(165.00,119.17)(0.500,1.000){2}{\rule{0.120pt}{0.400pt}}
\put(165.0,114.0){\rule[-0.200pt]{0.400pt}{1.445pt}}
\put(166,128.67){\rule{0.241pt}{0.400pt}}
\multiput(166.00,128.17)(0.500,1.000){2}{\rule{0.120pt}{0.400pt}}
\put(166.0,121.0){\rule[-0.200pt]{0.400pt}{1.927pt}}
\put(167,138.67){\rule{0.241pt}{0.400pt}}
\multiput(167.00,138.17)(0.500,1.000){2}{\rule{0.120pt}{0.400pt}}
\put(167.0,130.0){\rule[-0.200pt]{0.400pt}{2.168pt}}
\put(168,149.67){\rule{0.241pt}{0.400pt}}
\multiput(168.00,149.17)(0.500,1.000){2}{\rule{0.120pt}{0.400pt}}
\put(168.0,140.0){\rule[-0.200pt]{0.400pt}{2.409pt}}
\put(169,161.67){\rule{0.241pt}{0.400pt}}
\multiput(169.00,161.17)(0.500,1.000){2}{\rule{0.120pt}{0.400pt}}
\put(169.0,151.0){\rule[-0.200pt]{0.400pt}{2.650pt}}
\put(170,173.67){\rule{0.241pt}{0.400pt}}
\multiput(170.00,173.17)(0.500,1.000){2}{\rule{0.120pt}{0.400pt}}
\put(170.0,163.0){\rule[-0.200pt]{0.400pt}{2.650pt}}
\put(171,185.67){\rule{0.241pt}{0.400pt}}
\multiput(171.00,185.17)(0.500,1.000){2}{\rule{0.120pt}{0.400pt}}
\put(171.0,175.0){\rule[-0.200pt]{0.400pt}{2.650pt}}
\put(172,198.67){\rule{0.241pt}{0.400pt}}
\multiput(172.00,198.17)(0.500,1.000){2}{\rule{0.120pt}{0.400pt}}
\put(172.0,187.0){\rule[-0.200pt]{0.400pt}{2.891pt}}
\put(173,210.67){\rule{0.241pt}{0.400pt}}
\multiput(173.00,210.17)(0.500,1.000){2}{\rule{0.120pt}{0.400pt}}
\put(173.0,200.0){\rule[-0.200pt]{0.400pt}{2.650pt}}
\put(174,222.67){\rule{0.241pt}{0.400pt}}
\multiput(174.00,222.17)(0.500,1.000){2}{\rule{0.120pt}{0.400pt}}
\put(174.0,212.0){\rule[-0.200pt]{0.400pt}{2.650pt}}
\put(175,234.67){\rule{0.241pt}{0.400pt}}
\multiput(175.00,234.17)(0.500,1.000){2}{\rule{0.120pt}{0.400pt}}
\put(175.0,224.0){\rule[-0.200pt]{0.400pt}{2.650pt}}
\put(176,245.67){\rule{0.241pt}{0.400pt}}
\multiput(176.00,245.17)(0.500,1.000){2}{\rule{0.120pt}{0.400pt}}
\put(176.0,236.0){\rule[-0.200pt]{0.400pt}{2.409pt}}
\put(177,256.67){\rule{0.241pt}{0.400pt}}
\multiput(177.00,256.17)(0.500,1.000){2}{\rule{0.120pt}{0.400pt}}
\put(177.0,247.0){\rule[-0.200pt]{0.400pt}{2.409pt}}
\put(178.0,258.0){\rule[-0.200pt]{0.400pt}{2.168pt}}
\put(178.0,267.0){\usebox{\plotpoint}}
\put(179,274.67){\rule{0.241pt}{0.400pt}}
\multiput(179.00,274.17)(0.500,1.000){2}{\rule{0.120pt}{0.400pt}}
\put(179.0,267.0){\rule[-0.200pt]{0.400pt}{1.927pt}}
\put(180,281.67){\rule{0.241pt}{0.400pt}}
\multiput(180.00,281.17)(0.500,1.000){2}{\rule{0.120pt}{0.400pt}}
\put(180.0,276.0){\rule[-0.200pt]{0.400pt}{1.445pt}}
\put(181.0,283.0){\rule[-0.200pt]{0.400pt}{1.445pt}}
\put(181.0,289.0){\usebox{\plotpoint}}
\put(182.0,289.0){\rule[-0.200pt]{0.400pt}{1.204pt}}
\put(182.0,294.0){\usebox{\plotpoint}}
\put(183.0,294.0){\rule[-0.200pt]{0.400pt}{0.964pt}}
\put(183.0,298.0){\usebox{\plotpoint}}
\put(184.0,298.0){\rule[-0.200pt]{0.400pt}{0.482pt}}
\put(184.0,300.0){\usebox{\plotpoint}}
\put(185.0,300.0){\usebox{\plotpoint}}
\put(185.0,301.0){\rule[-0.200pt]{0.482pt}{0.400pt}}
\put(187.0,299.0){\rule[-0.200pt]{0.400pt}{0.482pt}}
\put(187.0,299.0){\usebox{\plotpoint}}
\put(188,294.67){\rule{0.241pt}{0.400pt}}
\multiput(188.00,295.17)(0.500,-1.000){2}{\rule{0.120pt}{0.400pt}}
\put(188.0,296.0){\rule[-0.200pt]{0.400pt}{0.723pt}}
\put(189,295){\usebox{\plotpoint}}
\put(189,295){\usebox{\plotpoint}}
\put(189.0,291.0){\rule[-0.200pt]{0.400pt}{0.964pt}}
\put(189.0,291.0){\usebox{\plotpoint}}
\put(190.0,285.0){\rule[-0.200pt]{0.400pt}{1.445pt}}
\put(190.0,285.0){\usebox{\plotpoint}}
\put(191,276.67){\rule{0.241pt}{0.400pt}}
\multiput(191.00,277.17)(0.500,-1.000){2}{\rule{0.120pt}{0.400pt}}
\put(191.0,278.0){\rule[-0.200pt]{0.400pt}{1.686pt}}
\put(192,277){\usebox{\plotpoint}}
\put(192,268.67){\rule{0.241pt}{0.400pt}}
\multiput(192.00,269.17)(0.500,-1.000){2}{\rule{0.120pt}{0.400pt}}
\put(192.0,270.0){\rule[-0.200pt]{0.400pt}{1.686pt}}
\put(193,259.67){\rule{0.241pt}{0.400pt}}
\multiput(193.00,260.17)(0.500,-1.000){2}{\rule{0.120pt}{0.400pt}}
\put(193.0,261.0){\rule[-0.200pt]{0.400pt}{1.927pt}}
\put(194,249.67){\rule{0.241pt}{0.400pt}}
\multiput(194.00,250.17)(0.500,-1.000){2}{\rule{0.120pt}{0.400pt}}
\put(194.0,251.0){\rule[-0.200pt]{0.400pt}{2.168pt}}
\put(195,239.67){\rule{0.241pt}{0.400pt}}
\multiput(195.00,240.17)(0.500,-1.000){2}{\rule{0.120pt}{0.400pt}}
\put(195.0,241.0){\rule[-0.200pt]{0.400pt}{2.168pt}}
\put(196,228.67){\rule{0.241pt}{0.400pt}}
\multiput(196.00,229.17)(0.500,-1.000){2}{\rule{0.120pt}{0.400pt}}
\put(196.0,230.0){\rule[-0.200pt]{0.400pt}{2.409pt}}
\put(197,217.67){\rule{0.241pt}{0.400pt}}
\multiput(197.00,218.17)(0.500,-1.000){2}{\rule{0.120pt}{0.400pt}}
\put(197.0,219.0){\rule[-0.200pt]{0.400pt}{2.409pt}}
\put(198,205.67){\rule{0.241pt}{0.400pt}}
\multiput(198.00,206.17)(0.500,-1.000){2}{\rule{0.120pt}{0.400pt}}
\put(198.0,207.0){\rule[-0.200pt]{0.400pt}{2.650pt}}
\put(199,194.67){\rule{0.241pt}{0.400pt}}
\multiput(199.00,195.17)(0.500,-1.000){2}{\rule{0.120pt}{0.400pt}}
\put(199.0,196.0){\rule[-0.200pt]{0.400pt}{2.409pt}}
\put(200,182.67){\rule{0.241pt}{0.400pt}}
\multiput(200.00,183.17)(0.500,-1.000){2}{\rule{0.120pt}{0.400pt}}
\put(200.0,184.0){\rule[-0.200pt]{0.400pt}{2.650pt}}
\put(201,171.67){\rule{0.241pt}{0.400pt}}
\multiput(201.00,172.17)(0.500,-1.000){2}{\rule{0.120pt}{0.400pt}}
\put(201.0,173.0){\rule[-0.200pt]{0.400pt}{2.409pt}}
\put(202.0,162.0){\rule[-0.200pt]{0.400pt}{2.409pt}}
\put(202.0,162.0){\usebox{\plotpoint}}
\put(203.0,152.0){\rule[-0.200pt]{0.400pt}{2.409pt}}
\put(203.0,152.0){\usebox{\plotpoint}}
\put(204,141.67){\rule{0.241pt}{0.400pt}}
\multiput(204.00,142.17)(0.500,-1.000){2}{\rule{0.120pt}{0.400pt}}
\put(204.0,143.0){\rule[-0.200pt]{0.400pt}{2.168pt}}
\put(205,142){\usebox{\plotpoint}}
\put(205,133.67){\rule{0.241pt}{0.400pt}}
\multiput(205.00,134.17)(0.500,-1.000){2}{\rule{0.120pt}{0.400pt}}
\put(205.0,135.0){\rule[-0.200pt]{0.400pt}{1.686pt}}
\put(206.0,127.0){\rule[-0.200pt]{0.400pt}{1.686pt}}
\put(206.0,127.0){\usebox{\plotpoint}}
\put(207.0,121.0){\rule[-0.200pt]{0.400pt}{1.445pt}}
\put(207.0,121.0){\usebox{\plotpoint}}
\put(208.0,116.0){\rule[-0.200pt]{0.400pt}{1.204pt}}
\put(208.0,116.0){\usebox{\plotpoint}}
\put(209.0,112.0){\rule[-0.200pt]{0.400pt}{0.964pt}}
\put(209.0,112.0){\usebox{\plotpoint}}
\put(210,108.67){\rule{0.241pt}{0.400pt}}
\multiput(210.00,109.17)(0.500,-1.000){2}{\rule{0.120pt}{0.400pt}}
\put(210.0,110.0){\rule[-0.200pt]{0.400pt}{0.482pt}}
\put(211,109){\usebox{\plotpoint}}
\put(211,109){\usebox{\plotpoint}}
\put(211,109){\usebox{\plotpoint}}
\put(211,109){\usebox{\plotpoint}}
\put(211,109){\usebox{\plotpoint}}
\put(211,109){\usebox{\plotpoint}}
\put(211,109){\usebox{\plotpoint}}
\put(211,109){\usebox{\plotpoint}}
\put(211.0,108.0){\usebox{\plotpoint}}
\put(211.0,108.0){\rule[-0.200pt]{0.482pt}{0.400pt}}
\put(213.0,108.0){\rule[-0.200pt]{0.400pt}{0.482pt}}
\put(213.0,110.0){\usebox{\plotpoint}}
\put(214,111.67){\rule{0.241pt}{0.400pt}}
\multiput(214.00,111.17)(0.500,1.000){2}{\rule{0.120pt}{0.400pt}}
\put(214.0,110.0){\rule[-0.200pt]{0.400pt}{0.482pt}}
\put(215,113){\usebox{\plotpoint}}
\put(215,113){\usebox{\plotpoint}}
\put(215,113){\usebox{\plotpoint}}
\put(215,115.67){\rule{0.241pt}{0.400pt}}
\multiput(215.00,115.17)(0.500,1.000){2}{\rule{0.120pt}{0.400pt}}
\put(215.0,113.0){\rule[-0.200pt]{0.400pt}{0.723pt}}
\put(216,117){\usebox{\plotpoint}}
\put(216,117){\usebox{\plotpoint}}
\put(216,120.67){\rule{0.241pt}{0.400pt}}
\multiput(216.00,120.17)(0.500,1.000){2}{\rule{0.120pt}{0.400pt}}
\put(216.0,117.0){\rule[-0.200pt]{0.400pt}{0.964pt}}
\put(217,122){\usebox{\plotpoint}}
\put(217,126.67){\rule{0.241pt}{0.400pt}}
\multiput(217.00,126.17)(0.500,1.000){2}{\rule{0.120pt}{0.400pt}}
\put(217.0,122.0){\rule[-0.200pt]{0.400pt}{1.204pt}}
\put(218.0,128.0){\rule[-0.200pt]{0.400pt}{1.686pt}}
\put(218.0,135.0){\usebox{\plotpoint}}
\put(219,142.67){\rule{0.241pt}{0.400pt}}
\multiput(219.00,142.17)(0.500,1.000){2}{\rule{0.120pt}{0.400pt}}
\put(219.0,135.0){\rule[-0.200pt]{0.400pt}{1.927pt}}
\put(220,144){\usebox{\plotpoint}}
\put(220.0,144.0){\rule[-0.200pt]{0.400pt}{1.927pt}}
\put(220.0,152.0){\usebox{\plotpoint}}
\put(221,160.67){\rule{0.241pt}{0.400pt}}
\multiput(221.00,160.17)(0.500,1.000){2}{\rule{0.120pt}{0.400pt}}
\put(221.0,152.0){\rule[-0.200pt]{0.400pt}{2.168pt}}
\put(222,170.67){\rule{0.241pt}{0.400pt}}
\multiput(222.00,170.17)(0.500,1.000){2}{\rule{0.120pt}{0.400pt}}
\put(222.0,162.0){\rule[-0.200pt]{0.400pt}{2.168pt}}
\put(223,181.67){\rule{0.241pt}{0.400pt}}
\multiput(223.00,181.17)(0.500,1.000){2}{\rule{0.120pt}{0.400pt}}
\put(223.0,172.0){\rule[-0.200pt]{0.400pt}{2.409pt}}
\put(224,191.67){\rule{0.241pt}{0.400pt}}
\multiput(224.00,191.17)(0.500,1.000){2}{\rule{0.120pt}{0.400pt}}
\put(224.0,183.0){\rule[-0.200pt]{0.400pt}{2.168pt}}
\put(225,201.67){\rule{0.241pt}{0.400pt}}
\multiput(225.00,201.17)(0.500,1.000){2}{\rule{0.120pt}{0.400pt}}
\put(225.0,193.0){\rule[-0.200pt]{0.400pt}{2.168pt}}
\put(226,212.67){\rule{0.241pt}{0.400pt}}
\multiput(226.00,212.17)(0.500,1.000){2}{\rule{0.120pt}{0.400pt}}
\put(226.0,203.0){\rule[-0.200pt]{0.400pt}{2.409pt}}
\put(227,222.67){\rule{0.241pt}{0.400pt}}
\multiput(227.00,222.17)(0.500,1.000){2}{\rule{0.120pt}{0.400pt}}
\put(227.0,214.0){\rule[-0.200pt]{0.400pt}{2.168pt}}
\put(228,232.67){\rule{0.241pt}{0.400pt}}
\multiput(228.00,232.17)(0.500,1.000){2}{\rule{0.120pt}{0.400pt}}
\put(228.0,224.0){\rule[-0.200pt]{0.400pt}{2.168pt}}
\put(229.0,234.0){\rule[-0.200pt]{0.400pt}{2.168pt}}
\put(229.0,243.0){\usebox{\plotpoint}}
\put(230,250.67){\rule{0.241pt}{0.400pt}}
\multiput(230.00,250.17)(0.500,1.000){2}{\rule{0.120pt}{0.400pt}}
\put(230.0,243.0){\rule[-0.200pt]{0.400pt}{1.927pt}}
\put(231,258.67){\rule{0.241pt}{0.400pt}}
\multiput(231.00,258.17)(0.500,1.000){2}{\rule{0.120pt}{0.400pt}}
\put(231.0,252.0){\rule[-0.200pt]{0.400pt}{1.686pt}}
\put(232.0,260.0){\rule[-0.200pt]{0.400pt}{1.686pt}}
\put(232.0,267.0){\usebox{\plotpoint}}
\put(233.0,267.0){\rule[-0.200pt]{0.400pt}{1.445pt}}
\put(233.0,273.0){\usebox{\plotpoint}}
\put(234.0,273.0){\rule[-0.200pt]{0.400pt}{1.204pt}}
\put(234.0,278.0){\usebox{\plotpoint}}
\put(235.0,278.0){\rule[-0.200pt]{0.400pt}{0.964pt}}
\put(235.0,282.0){\usebox{\plotpoint}}
\put(236.0,282.0){\rule[-0.200pt]{0.400pt}{0.723pt}}
\put(236.0,285.0){\usebox{\plotpoint}}
\put(237.0,285.0){\usebox{\plotpoint}}
\put(237.0,286.0){\rule[-0.200pt]{0.482pt}{0.400pt}}
\put(239.0,285.0){\usebox{\plotpoint}}
\put(239.0,285.0){\usebox{\plotpoint}}
\put(240,282.67){\rule{0.241pt}{0.400pt}}
\multiput(240.00,283.17)(0.500,-1.000){2}{\rule{0.120pt}{0.400pt}}
\put(240.0,284.0){\usebox{\plotpoint}}
\put(241,283){\usebox{\plotpoint}}
\put(241,283){\usebox{\plotpoint}}
\put(241,283){\usebox{\plotpoint}}
\put(241.0,280.0){\rule[-0.200pt]{0.400pt}{0.723pt}}
\put(241.0,280.0){\usebox{\plotpoint}}
\put(242.0,276.0){\rule[-0.200pt]{0.400pt}{0.964pt}}
\put(242.0,276.0){\usebox{\plotpoint}}
\put(243,269.67){\rule{0.241pt}{0.400pt}}
\multiput(243.00,270.17)(0.500,-1.000){2}{\rule{0.120pt}{0.400pt}}
\put(243.0,271.0){\rule[-0.200pt]{0.400pt}{1.204pt}}
\put(244,270){\usebox{\plotpoint}}
\put(244.0,264.0){\rule[-0.200pt]{0.400pt}{1.445pt}}
\put(244.0,264.0){\usebox{\plotpoint}}
\put(245,255.67){\rule{0.241pt}{0.400pt}}
\multiput(245.00,256.17)(0.500,-1.000){2}{\rule{0.120pt}{0.400pt}}
\put(245.0,257.0){\rule[-0.200pt]{0.400pt}{1.686pt}}
\put(246,256){\usebox{\plotpoint}}
\put(246,247.67){\rule{0.241pt}{0.400pt}}
\multiput(246.00,248.17)(0.500,-1.000){2}{\rule{0.120pt}{0.400pt}}
\put(246.0,249.0){\rule[-0.200pt]{0.400pt}{1.686pt}}
\put(247,248){\usebox{\plotpoint}}
\put(247,239.67){\rule{0.241pt}{0.400pt}}
\multiput(247.00,240.17)(0.500,-1.000){2}{\rule{0.120pt}{0.400pt}}
\put(247.0,241.0){\rule[-0.200pt]{0.400pt}{1.686pt}}
\put(248.0,231.0){\rule[-0.200pt]{0.400pt}{2.168pt}}
\put(248.0,231.0){\usebox{\plotpoint}}
\put(249,220.67){\rule{0.241pt}{0.400pt}}
\multiput(249.00,221.17)(0.500,-1.000){2}{\rule{0.120pt}{0.400pt}}
\put(249.0,222.0){\rule[-0.200pt]{0.400pt}{2.168pt}}
\put(250,210.67){\rule{0.241pt}{0.400pt}}
\multiput(250.00,211.17)(0.500,-1.000){2}{\rule{0.120pt}{0.400pt}}
\put(250.0,212.0){\rule[-0.200pt]{0.400pt}{2.168pt}}
\put(251,211){\usebox{\plotpoint}}
\put(251.0,202.0){\rule[-0.200pt]{0.400pt}{2.168pt}}
\put(251.0,202.0){\usebox{\plotpoint}}
\put(252,191.67){\rule{0.241pt}{0.400pt}}
\multiput(252.00,192.17)(0.500,-1.000){2}{\rule{0.120pt}{0.400pt}}
\put(252.0,193.0){\rule[-0.200pt]{0.400pt}{2.168pt}}
\put(253,181.67){\rule{0.241pt}{0.400pt}}
\multiput(253.00,182.17)(0.500,-1.000){2}{\rule{0.120pt}{0.400pt}}
\put(253.0,183.0){\rule[-0.200pt]{0.400pt}{2.168pt}}
\put(254,182){\usebox{\plotpoint}}
\put(254,172.67){\rule{0.241pt}{0.400pt}}
\multiput(254.00,173.17)(0.500,-1.000){2}{\rule{0.120pt}{0.400pt}}
\put(254.0,174.0){\rule[-0.200pt]{0.400pt}{1.927pt}}
\put(255,164.67){\rule{0.241pt}{0.400pt}}
\multiput(255.00,165.17)(0.500,-1.000){2}{\rule{0.120pt}{0.400pt}}
\put(255.0,166.0){\rule[-0.200pt]{0.400pt}{1.686pt}}
\put(256,156.67){\rule{0.241pt}{0.400pt}}
\multiput(256.00,157.17)(0.500,-1.000){2}{\rule{0.120pt}{0.400pt}}
\put(256.0,158.0){\rule[-0.200pt]{0.400pt}{1.686pt}}
\put(257,148.67){\rule{0.241pt}{0.400pt}}
\multiput(257.00,149.17)(0.500,-1.000){2}{\rule{0.120pt}{0.400pt}}
\put(257.0,150.0){\rule[-0.200pt]{0.400pt}{1.686pt}}
\put(258,149){\usebox{\plotpoint}}
\put(258.0,143.0){\rule[-0.200pt]{0.400pt}{1.445pt}}
\put(258.0,143.0){\usebox{\plotpoint}}
\put(259.0,137.0){\rule[-0.200pt]{0.400pt}{1.445pt}}
\put(259.0,137.0){\usebox{\plotpoint}}
\put(260.0,132.0){\rule[-0.200pt]{0.400pt}{1.204pt}}
\put(260.0,132.0){\usebox{\plotpoint}}
\put(261.0,128.0){\rule[-0.200pt]{0.400pt}{0.964pt}}
\put(261.0,128.0){\usebox{\plotpoint}}
\put(262.0,125.0){\rule[-0.200pt]{0.400pt}{0.723pt}}
\put(262.0,125.0){\usebox{\plotpoint}}
\put(263.0,124.0){\usebox{\plotpoint}}
\put(263.0,124.0){\usebox{\plotpoint}}
\put(264.0,123.0){\usebox{\plotpoint}}
\put(264.0,123.0){\usebox{\plotpoint}}
\put(265.0,123.0){\usebox{\plotpoint}}
\put(265.0,124.0){\usebox{\plotpoint}}
\put(266.0,124.0){\usebox{\plotpoint}}
\put(266.0,125.0){\usebox{\plotpoint}}
\put(267.0,125.0){\rule[-0.200pt]{0.400pt}{0.723pt}}
\put(267.0,128.0){\usebox{\plotpoint}}
\put(268.0,128.0){\rule[-0.200pt]{0.400pt}{0.964pt}}
\put(268.0,132.0){\usebox{\plotpoint}}
\put(269,135.67){\rule{0.241pt}{0.400pt}}
\multiput(269.00,135.17)(0.500,1.000){2}{\rule{0.120pt}{0.400pt}}
\put(269.0,132.0){\rule[-0.200pt]{0.400pt}{0.964pt}}
\put(270,137){\usebox{\plotpoint}}
\put(270,140.67){\rule{0.241pt}{0.400pt}}
\multiput(270.00,140.17)(0.500,1.000){2}{\rule{0.120pt}{0.400pt}}
\put(270.0,137.0){\rule[-0.200pt]{0.400pt}{0.964pt}}
\put(271,142){\usebox{\plotpoint}}
\put(271.0,142.0){\rule[-0.200pt]{0.400pt}{1.445pt}}
\put(271.0,148.0){\usebox{\plotpoint}}
\put(272.0,148.0){\rule[-0.200pt]{0.400pt}{1.686pt}}
\put(272.0,155.0){\usebox{\plotpoint}}
\put(273,161.67){\rule{0.241pt}{0.400pt}}
\multiput(273.00,161.17)(0.500,1.000){2}{\rule{0.120pt}{0.400pt}}
\put(273.0,155.0){\rule[-0.200pt]{0.400pt}{1.686pt}}
\put(274,169.67){\rule{0.241pt}{0.400pt}}
\multiput(274.00,169.17)(0.500,1.000){2}{\rule{0.120pt}{0.400pt}}
\put(274.0,163.0){\rule[-0.200pt]{0.400pt}{1.686pt}}
\put(275,178.67){\rule{0.241pt}{0.400pt}}
\multiput(275.00,178.17)(0.500,1.000){2}{\rule{0.120pt}{0.400pt}}
\put(275.0,171.0){\rule[-0.200pt]{0.400pt}{1.927pt}}
\put(276,180){\usebox{\plotpoint}}
\put(276.0,180.0){\rule[-0.200pt]{0.400pt}{1.927pt}}
\put(276.0,188.0){\usebox{\plotpoint}}
\put(277.0,188.0){\rule[-0.200pt]{0.400pt}{2.168pt}}
\put(277.0,197.0){\usebox{\plotpoint}}
\put(278,204.67){\rule{0.241pt}{0.400pt}}
\multiput(278.00,204.17)(0.500,1.000){2}{\rule{0.120pt}{0.400pt}}
\put(278.0,197.0){\rule[-0.200pt]{0.400pt}{1.927pt}}
\put(279,213.67){\rule{0.241pt}{0.400pt}}
\multiput(279.00,213.17)(0.500,1.000){2}{\rule{0.120pt}{0.400pt}}
\put(279.0,206.0){\rule[-0.200pt]{0.400pt}{1.927pt}}
\put(280.0,215.0){\rule[-0.200pt]{0.400pt}{1.927pt}}
\put(280.0,223.0){\usebox{\plotpoint}}
\put(281,230.67){\rule{0.241pt}{0.400pt}}
\multiput(281.00,230.17)(0.500,1.000){2}{\rule{0.120pt}{0.400pt}}
\put(281.0,223.0){\rule[-0.200pt]{0.400pt}{1.927pt}}
\put(282,232){\usebox{\plotpoint}}
\put(282,237.67){\rule{0.241pt}{0.400pt}}
\multiput(282.00,237.17)(0.500,1.000){2}{\rule{0.120pt}{0.400pt}}
\put(282.0,232.0){\rule[-0.200pt]{0.400pt}{1.445pt}}
\put(283.0,239.0){\rule[-0.200pt]{0.400pt}{1.686pt}}
\put(283.0,246.0){\usebox{\plotpoint}}
\put(284.0,246.0){\rule[-0.200pt]{0.400pt}{1.445pt}}
\put(284.0,252.0){\usebox{\plotpoint}}
\put(285.0,252.0){\rule[-0.200pt]{0.400pt}{1.204pt}}
\put(285.0,257.0){\usebox{\plotpoint}}
\put(286.0,257.0){\rule[-0.200pt]{0.400pt}{1.204pt}}
\put(286.0,262.0){\usebox{\plotpoint}}
\put(287.0,262.0){\rule[-0.200pt]{0.400pt}{0.964pt}}
\put(287.0,266.0){\usebox{\plotpoint}}
\put(288.0,266.0){\rule[-0.200pt]{0.400pt}{0.723pt}}
\put(288.0,269.0){\usebox{\plotpoint}}
\put(289,269.67){\rule{0.241pt}{0.400pt}}
\multiput(289.00,269.17)(0.500,1.000){2}{\rule{0.120pt}{0.400pt}}
\put(289.0,269.0){\usebox{\plotpoint}}
\put(290,271){\usebox{\plotpoint}}
\put(290,271){\usebox{\plotpoint}}
\put(290,271){\usebox{\plotpoint}}
\put(290,271){\usebox{\plotpoint}}
\put(290,271){\usebox{\plotpoint}}
\put(290,271){\usebox{\plotpoint}}
\put(290,271){\usebox{\plotpoint}}
\put(290,271){\usebox{\plotpoint}}
\put(290,271){\usebox{\plotpoint}}
\put(290,271){\usebox{\plotpoint}}
\put(290,271){\usebox{\plotpoint}}
\put(290.0,271.0){\rule[-0.200pt]{0.482pt}{0.400pt}}
\put(292.0,270.0){\usebox{\plotpoint}}
\put(292.0,270.0){\usebox{\plotpoint}}
\put(293.0,268.0){\rule[-0.200pt]{0.400pt}{0.482pt}}
\put(293.0,268.0){\usebox{\plotpoint}}
\put(294.0,265.0){\rule[-0.200pt]{0.400pt}{0.723pt}}
\put(294.0,265.0){\usebox{\plotpoint}}
\put(295.0,261.0){\rule[-0.200pt]{0.400pt}{0.964pt}}
\put(295.0,261.0){\usebox{\plotpoint}}
\put(296.0,256.0){\rule[-0.200pt]{0.400pt}{1.204pt}}
\put(296.0,256.0){\usebox{\plotpoint}}
\put(297,249.67){\rule{0.241pt}{0.400pt}}
\multiput(297.00,250.17)(0.500,-1.000){2}{\rule{0.120pt}{0.400pt}}
\put(297.0,251.0){\rule[-0.200pt]{0.400pt}{1.204pt}}
\put(298,250){\usebox{\plotpoint}}
\put(298,243.67){\rule{0.241pt}{0.400pt}}
\multiput(298.00,244.17)(0.500,-1.000){2}{\rule{0.120pt}{0.400pt}}
\put(298.0,245.0){\rule[-0.200pt]{0.400pt}{1.204pt}}
\put(299,236.67){\rule{0.241pt}{0.400pt}}
\multiput(299.00,237.17)(0.500,-1.000){2}{\rule{0.120pt}{0.400pt}}
\put(299.0,238.0){\rule[-0.200pt]{0.400pt}{1.445pt}}
\put(300,237){\usebox{\plotpoint}}
\put(300.0,231.0){\rule[-0.200pt]{0.400pt}{1.445pt}}
\put(300.0,231.0){\usebox{\plotpoint}}
\put(301,222.67){\rule{0.241pt}{0.400pt}}
\multiput(301.00,223.17)(0.500,-1.000){2}{\rule{0.120pt}{0.400pt}}
\put(301.0,224.0){\rule[-0.200pt]{0.400pt}{1.686pt}}
\put(302,214.67){\rule{0.241pt}{0.400pt}}
\multiput(302.00,215.17)(0.500,-1.000){2}{\rule{0.120pt}{0.400pt}}
\put(302.0,216.0){\rule[-0.200pt]{0.400pt}{1.686pt}}
\put(303,206.67){\rule{0.241pt}{0.400pt}}
\multiput(303.00,207.17)(0.500,-1.000){2}{\rule{0.120pt}{0.400pt}}
\put(303.0,208.0){\rule[-0.200pt]{0.400pt}{1.686pt}}
\put(304,198.67){\rule{0.241pt}{0.400pt}}
\multiput(304.00,199.17)(0.500,-1.000){2}{\rule{0.120pt}{0.400pt}}
\put(304.0,200.0){\rule[-0.200pt]{0.400pt}{1.686pt}}
\put(305,190.67){\rule{0.241pt}{0.400pt}}
\multiput(305.00,191.17)(0.500,-1.000){2}{\rule{0.120pt}{0.400pt}}
\put(305.0,192.0){\rule[-0.200pt]{0.400pt}{1.686pt}}
\put(306,191){\usebox{\plotpoint}}
\put(306,182.67){\rule{0.241pt}{0.400pt}}
\multiput(306.00,183.17)(0.500,-1.000){2}{\rule{0.120pt}{0.400pt}}
\put(306.0,184.0){\rule[-0.200pt]{0.400pt}{1.686pt}}
\put(307,183){\usebox{\plotpoint}}
\put(307,175.67){\rule{0.241pt}{0.400pt}}
\multiput(307.00,176.17)(0.500,-1.000){2}{\rule{0.120pt}{0.400pt}}
\put(307.0,177.0){\rule[-0.200pt]{0.400pt}{1.445pt}}
\put(308,168.67){\rule{0.241pt}{0.400pt}}
\multiput(308.00,169.17)(0.500,-1.000){2}{\rule{0.120pt}{0.400pt}}
\put(308.0,170.0){\rule[-0.200pt]{0.400pt}{1.445pt}}
\put(309,169){\usebox{\plotpoint}}
\put(309.0,163.0){\rule[-0.200pt]{0.400pt}{1.445pt}}
\put(309.0,163.0){\usebox{\plotpoint}}
\put(310.0,157.0){\rule[-0.200pt]{0.400pt}{1.445pt}}
\put(310.0,157.0){\usebox{\plotpoint}}
\put(311.0,152.0){\rule[-0.200pt]{0.400pt}{1.204pt}}
\put(311.0,152.0){\usebox{\plotpoint}}
\put(312,146.67){\rule{0.241pt}{0.400pt}}
\multiput(312.00,147.17)(0.500,-1.000){2}{\rule{0.120pt}{0.400pt}}
\put(312.0,148.0){\rule[-0.200pt]{0.400pt}{0.964pt}}
\put(313,147){\usebox{\plotpoint}}
\put(313,147){\usebox{\plotpoint}}
\put(313.0,144.0){\rule[-0.200pt]{0.400pt}{0.723pt}}
\put(313.0,144.0){\usebox{\plotpoint}}
\put(314.0,141.0){\rule[-0.200pt]{0.400pt}{0.723pt}}
\put(314.0,141.0){\usebox{\plotpoint}}
\put(315.0,139.0){\rule[-0.200pt]{0.400pt}{0.482pt}}
\put(315.0,139.0){\usebox{\plotpoint}}
\put(316.0,138.0){\usebox{\plotpoint}}
\put(316.0,138.0){\rule[-0.200pt]{0.482pt}{0.400pt}}
\put(318.0,138.0){\usebox{\plotpoint}}
\put(318.0,139.0){\usebox{\plotpoint}}
\put(319,139.67){\rule{0.241pt}{0.400pt}}
\multiput(319.00,139.17)(0.500,1.000){2}{\rule{0.120pt}{0.400pt}}
\put(319.0,139.0){\usebox{\plotpoint}}
\put(320,141){\usebox{\plotpoint}}
\put(320,141){\usebox{\plotpoint}}
\put(320,141){\usebox{\plotpoint}}
\put(320,141){\usebox{\plotpoint}}
\put(320.0,141.0){\rule[-0.200pt]{0.400pt}{0.482pt}}
\put(320.0,143.0){\usebox{\plotpoint}}
\put(321.0,143.0){\rule[-0.200pt]{0.400pt}{0.723pt}}
\put(321.0,146.0){\usebox{\plotpoint}}
\put(322,149.67){\rule{0.241pt}{0.400pt}}
\multiput(322.00,149.17)(0.500,1.000){2}{\rule{0.120pt}{0.400pt}}
\put(322.0,146.0){\rule[-0.200pt]{0.400pt}{0.964pt}}
\put(323,151){\usebox{\plotpoint}}
\put(323,151){\usebox{\plotpoint}}
\put(323.0,151.0){\rule[-0.200pt]{0.400pt}{0.964pt}}
\put(323.0,155.0){\usebox{\plotpoint}}
\put(324,159.67){\rule{0.241pt}{0.400pt}}
\multiput(324.00,159.17)(0.500,1.000){2}{\rule{0.120pt}{0.400pt}}
\put(324.0,155.0){\rule[-0.200pt]{0.400pt}{1.204pt}}
\put(325,161){\usebox{\plotpoint}}
\put(325,165.67){\rule{0.241pt}{0.400pt}}
\multiput(325.00,165.17)(0.500,1.000){2}{\rule{0.120pt}{0.400pt}}
\put(325.0,161.0){\rule[-0.200pt]{0.400pt}{1.204pt}}
\put(326,167){\usebox{\plotpoint}}
\put(326,171.67){\rule{0.241pt}{0.400pt}}
\multiput(326.00,171.17)(0.500,1.000){2}{\rule{0.120pt}{0.400pt}}
\put(326.0,167.0){\rule[-0.200pt]{0.400pt}{1.204pt}}
\put(327,178.67){\rule{0.241pt}{0.400pt}}
\multiput(327.00,178.17)(0.500,1.000){2}{\rule{0.120pt}{0.400pt}}
\put(327.0,173.0){\rule[-0.200pt]{0.400pt}{1.445pt}}
\put(328,180){\usebox{\plotpoint}}
\put(328,185.67){\rule{0.241pt}{0.400pt}}
\multiput(328.00,185.17)(0.500,1.000){2}{\rule{0.120pt}{0.400pt}}
\put(328.0,180.0){\rule[-0.200pt]{0.400pt}{1.445pt}}
\put(329,187){\usebox{\plotpoint}}
\put(329.0,187.0){\rule[-0.200pt]{0.400pt}{1.445pt}}
\put(329.0,193.0){\usebox{\plotpoint}}
\put(330.0,193.0){\rule[-0.200pt]{0.400pt}{1.686pt}}
\put(330.0,200.0){\usebox{\plotpoint}}
\put(331.0,200.0){\rule[-0.200pt]{0.400pt}{1.686pt}}
\put(331.0,207.0){\usebox{\plotpoint}}
\put(332.0,207.0){\rule[-0.200pt]{0.400pt}{1.686pt}}
\put(332.0,214.0){\usebox{\plotpoint}}
\put(333,219.67){\rule{0.241pt}{0.400pt}}
\multiput(333.00,219.17)(0.500,1.000){2}{\rule{0.120pt}{0.400pt}}
\put(333.0,214.0){\rule[-0.200pt]{0.400pt}{1.445pt}}
\put(334,221){\usebox{\plotpoint}}
\put(334.0,221.0){\rule[-0.200pt]{0.400pt}{1.445pt}}
\put(334.0,227.0){\usebox{\plotpoint}}
\put(335.0,227.0){\rule[-0.200pt]{0.400pt}{1.445pt}}
\put(335.0,233.0){\usebox{\plotpoint}}
\put(336.0,233.0){\rule[-0.200pt]{0.400pt}{1.204pt}}
\put(336.0,238.0){\usebox{\plotpoint}}
\put(337.0,238.0){\rule[-0.200pt]{0.400pt}{1.204pt}}
\put(337.0,243.0){\usebox{\plotpoint}}
\put(338.0,243.0){\rule[-0.200pt]{0.400pt}{0.964pt}}
\put(338.0,247.0){\usebox{\plotpoint}}
\put(339,249.67){\rule{0.241pt}{0.400pt}}
\multiput(339.00,249.17)(0.500,1.000){2}{\rule{0.120pt}{0.400pt}}
\put(339.0,247.0){\rule[-0.200pt]{0.400pt}{0.723pt}}
\put(340,251){\usebox{\plotpoint}}
\put(340,251){\usebox{\plotpoint}}
\put(340,251){\usebox{\plotpoint}}
\put(340.0,251.0){\rule[-0.200pt]{0.400pt}{0.482pt}}
\put(340.0,253.0){\usebox{\plotpoint}}
\put(341.0,253.0){\rule[-0.200pt]{0.400pt}{0.482pt}}
\put(341.0,255.0){\usebox{\plotpoint}}
\put(342.0,255.0){\usebox{\plotpoint}}
\put(342.0,256.0){\rule[-0.200pt]{0.723pt}{0.400pt}}
\put(345.0,255.0){\usebox{\plotpoint}}
\put(345.0,255.0){\usebox{\plotpoint}}
\put(346.0,253.0){\rule[-0.200pt]{0.400pt}{0.482pt}}
\put(346.0,253.0){\usebox{\plotpoint}}
\put(347.0,250.0){\rule[-0.200pt]{0.400pt}{0.723pt}}
\put(347.0,250.0){\usebox{\plotpoint}}
\put(348.0,247.0){\rule[-0.200pt]{0.400pt}{0.723pt}}
\put(348.0,247.0){\usebox{\plotpoint}}
\put(349.0,243.0){\rule[-0.200pt]{0.400pt}{0.964pt}}
\put(349.0,243.0){\usebox{\plotpoint}}
\put(350.0,238.0){\rule[-0.200pt]{0.400pt}{1.204pt}}
\put(350.0,238.0){\usebox{\plotpoint}}
\put(351.0,233.0){\rule[-0.200pt]{0.400pt}{1.204pt}}
\put(351.0,233.0){\usebox{\plotpoint}}
\put(352.0,228.0){\rule[-0.200pt]{0.400pt}{1.204pt}}
\put(352.0,228.0){\usebox{\plotpoint}}
\put(353.0,222.0){\rule[-0.200pt]{0.400pt}{1.445pt}}
\put(353.0,222.0){\usebox{\plotpoint}}
\put(354.0,216.0){\rule[-0.200pt]{0.400pt}{1.445pt}}
\put(354.0,216.0){\usebox{\plotpoint}}
\put(355.0,210.0){\rule[-0.200pt]{0.400pt}{1.445pt}}
\put(355.0,210.0){\usebox{\plotpoint}}
\put(356,202.67){\rule{0.241pt}{0.400pt}}
\multiput(356.00,203.17)(0.500,-1.000){2}{\rule{0.120pt}{0.400pt}}
\put(356.0,204.0){\rule[-0.200pt]{0.400pt}{1.445pt}}
\put(357,203){\usebox{\plotpoint}}
\put(357,196.67){\rule{0.241pt}{0.400pt}}
\multiput(357.00,197.17)(0.500,-1.000){2}{\rule{0.120pt}{0.400pt}}
\put(357.0,198.0){\rule[-0.200pt]{0.400pt}{1.204pt}}
\put(358,197){\usebox{\plotpoint}}
\put(358,190.67){\rule{0.241pt}{0.400pt}}
\multiput(358.00,191.17)(0.500,-1.000){2}{\rule{0.120pt}{0.400pt}}
\put(358.0,192.0){\rule[-0.200pt]{0.400pt}{1.204pt}}
\put(359,191){\usebox{\plotpoint}}
\put(359.0,186.0){\rule[-0.200pt]{0.400pt}{1.204pt}}
\put(359.0,186.0){\usebox{\plotpoint}}
\put(360,179.67){\rule{0.241pt}{0.400pt}}
\multiput(360.00,180.17)(0.500,-1.000){2}{\rule{0.120pt}{0.400pt}}
\put(360.0,181.0){\rule[-0.200pt]{0.400pt}{1.204pt}}
\put(361,180){\usebox{\plotpoint}}
\put(361.0,175.0){\rule[-0.200pt]{0.400pt}{1.204pt}}
\put(361.0,175.0){\usebox{\plotpoint}}
\put(362.0,170.0){\rule[-0.200pt]{0.400pt}{1.204pt}}
\put(362.0,170.0){\usebox{\plotpoint}}
\put(363.0,166.0){\rule[-0.200pt]{0.400pt}{0.964pt}}
\put(363.0,166.0){\usebox{\plotpoint}}
\put(364.0,162.0){\rule[-0.200pt]{0.400pt}{0.964pt}}
\put(364.0,162.0){\usebox{\plotpoint}}
\put(365.0,159.0){\rule[-0.200pt]{0.400pt}{0.723pt}}
\put(365.0,159.0){\usebox{\plotpoint}}
\put(366,155.67){\rule{0.241pt}{0.400pt}}
\multiput(366.00,156.17)(0.500,-1.000){2}{\rule{0.120pt}{0.400pt}}
\put(366.0,157.0){\rule[-0.200pt]{0.400pt}{0.482pt}}
\put(367,156){\usebox{\plotpoint}}
\put(367,156){\usebox{\plotpoint}}
\put(367,156){\usebox{\plotpoint}}
\put(367,156){\usebox{\plotpoint}}
\put(367,156){\usebox{\plotpoint}}
\put(367.0,155.0){\usebox{\plotpoint}}
\put(367.0,155.0){\usebox{\plotpoint}}
\put(368.0,153.0){\rule[-0.200pt]{0.400pt}{0.482pt}}
\put(368.0,153.0){\rule[-0.200pt]{0.723pt}{0.400pt}}
\put(371.0,153.0){\usebox{\plotpoint}}
\put(371.0,154.0){\usebox{\plotpoint}}
\put(372.0,154.0){\rule[-0.200pt]{0.400pt}{0.482pt}}
\put(372.0,156.0){\usebox{\plotpoint}}
\put(373.0,156.0){\rule[-0.200pt]{0.400pt}{0.482pt}}
\put(373.0,158.0){\usebox{\plotpoint}}
\put(374,159.67){\rule{0.241pt}{0.400pt}}
\multiput(374.00,159.17)(0.500,1.000){2}{\rule{0.120pt}{0.400pt}}
\put(374.0,158.0){\rule[-0.200pt]{0.400pt}{0.482pt}}
\put(375,161){\usebox{\plotpoint}}
\put(375,161){\usebox{\plotpoint}}
\put(375,161){\usebox{\plotpoint}}
\put(375.0,161.0){\rule[-0.200pt]{0.400pt}{0.723pt}}
\put(375.0,164.0){\usebox{\plotpoint}}
\put(376,166.67){\rule{0.241pt}{0.400pt}}
\multiput(376.00,166.17)(0.500,1.000){2}{\rule{0.120pt}{0.400pt}}
\put(376.0,164.0){\rule[-0.200pt]{0.400pt}{0.723pt}}
\put(377,168){\usebox{\plotpoint}}
\put(377,168){\usebox{\plotpoint}}
\put(377.0,168.0){\rule[-0.200pt]{0.400pt}{0.964pt}}
\put(377.0,172.0){\usebox{\plotpoint}}
\put(378,175.67){\rule{0.241pt}{0.400pt}}
\multiput(378.00,175.17)(0.500,1.000){2}{\rule{0.120pt}{0.400pt}}
\put(378.0,172.0){\rule[-0.200pt]{0.400pt}{0.964pt}}
\put(379,177){\usebox{\plotpoint}}
\put(379,177){\usebox{\plotpoint}}
\put(379.0,177.0){\rule[-0.200pt]{0.400pt}{0.964pt}}
\put(379.0,181.0){\usebox{\plotpoint}}
\put(380,185.67){\rule{0.241pt}{0.400pt}}
\multiput(380.00,185.17)(0.500,1.000){2}{\rule{0.120pt}{0.400pt}}
\put(380.0,181.0){\rule[-0.200pt]{0.400pt}{1.204pt}}
\put(381,187){\usebox{\plotpoint}}
\put(381,187){\usebox{\plotpoint}}
\put(381,190.67){\rule{0.241pt}{0.400pt}}
\multiput(381.00,190.17)(0.500,1.000){2}{\rule{0.120pt}{0.400pt}}
\put(381.0,187.0){\rule[-0.200pt]{0.400pt}{0.964pt}}
\put(382,192){\usebox{\plotpoint}}
\put(382.0,192.0){\rule[-0.200pt]{0.400pt}{1.204pt}}
\put(382.0,197.0){\usebox{\plotpoint}}
\put(383,201.67){\rule{0.241pt}{0.400pt}}
\multiput(383.00,201.17)(0.500,1.000){2}{\rule{0.120pt}{0.400pt}}
\put(383.0,197.0){\rule[-0.200pt]{0.400pt}{1.204pt}}
\put(384,203){\usebox{\plotpoint}}
\put(384,203){\usebox{\plotpoint}}
\put(384,206.67){\rule{0.241pt}{0.400pt}}
\multiput(384.00,206.17)(0.500,1.000){2}{\rule{0.120pt}{0.400pt}}
\put(384.0,203.0){\rule[-0.200pt]{0.400pt}{0.964pt}}
\put(385,208){\usebox{\plotpoint}}
\put(385,211.67){\rule{0.241pt}{0.400pt}}
\multiput(385.00,211.17)(0.500,1.000){2}{\rule{0.120pt}{0.400pt}}
\put(385.0,208.0){\rule[-0.200pt]{0.400pt}{0.964pt}}
\put(386,213){\usebox{\plotpoint}}
\put(386,216.67){\rule{0.241pt}{0.400pt}}
\multiput(386.00,216.17)(0.500,1.000){2}{\rule{0.120pt}{0.400pt}}
\put(386.0,213.0){\rule[-0.200pt]{0.400pt}{0.964pt}}
\put(387,218){\usebox{\plotpoint}}
\put(387,218){\usebox{\plotpoint}}
\put(387.0,218.0){\rule[-0.200pt]{0.400pt}{0.964pt}}
\put(387.0,222.0){\usebox{\plotpoint}}
\put(388.0,222.0){\rule[-0.200pt]{0.400pt}{0.964pt}}
\put(388.0,226.0){\usebox{\plotpoint}}
\put(389,228.67){\rule{0.241pt}{0.400pt}}
\multiput(389.00,228.17)(0.500,1.000){2}{\rule{0.120pt}{0.400pt}}
\put(389.0,226.0){\rule[-0.200pt]{0.400pt}{0.723pt}}
\put(390,230){\usebox{\plotpoint}}
\put(390,230){\usebox{\plotpoint}}
\put(390.0,230.0){\rule[-0.200pt]{0.400pt}{0.723pt}}
\put(390.0,233.0){\usebox{\plotpoint}}
\put(391.0,233.0){\rule[-0.200pt]{0.400pt}{0.723pt}}
\put(391.0,236.0){\usebox{\plotpoint}}
\put(392.0,236.0){\rule[-0.200pt]{0.400pt}{0.482pt}}
\put(392.0,238.0){\usebox{\plotpoint}}
\put(393.0,238.0){\rule[-0.200pt]{0.400pt}{0.482pt}}
\put(393.0,240.0){\usebox{\plotpoint}}
\put(394.0,240.0){\usebox{\plotpoint}}
\put(395,240.67){\rule{0.241pt}{0.400pt}}
\multiput(395.00,240.17)(0.500,1.000){2}{\rule{0.120pt}{0.400pt}}
\put(394.0,241.0){\usebox{\plotpoint}}
\put(396,242){\usebox{\plotpoint}}
\put(396,242){\usebox{\plotpoint}}
\put(396,242){\usebox{\plotpoint}}
\put(396,242){\usebox{\plotpoint}}
\put(396,242){\usebox{\plotpoint}}
\put(396,242){\usebox{\plotpoint}}
\put(396,242){\usebox{\plotpoint}}
\put(396,242){\usebox{\plotpoint}}
\put(396,242){\usebox{\plotpoint}}
\put(396,242){\usebox{\plotpoint}}
\put(396.0,241.0){\usebox{\plotpoint}}
\put(396.0,241.0){\rule[-0.200pt]{0.482pt}{0.400pt}}
\put(398.0,240.0){\usebox{\plotpoint}}
\put(398.0,240.0){\usebox{\plotpoint}}
\put(399.0,238.0){\rule[-0.200pt]{0.400pt}{0.482pt}}
\put(399.0,238.0){\usebox{\plotpoint}}
\put(400.0,236.0){\rule[-0.200pt]{0.400pt}{0.482pt}}
\put(400.0,236.0){\usebox{\plotpoint}}
\put(401.0,233.0){\rule[-0.200pt]{0.400pt}{0.723pt}}
\put(401.0,233.0){\usebox{\plotpoint}}
\put(402.0,230.0){\rule[-0.200pt]{0.400pt}{0.723pt}}
\put(402.0,230.0){\usebox{\plotpoint}}
\put(403,225.67){\rule{0.241pt}{0.400pt}}
\multiput(403.00,226.17)(0.500,-1.000){2}{\rule{0.120pt}{0.400pt}}
\put(403.0,227.0){\rule[-0.200pt]{0.400pt}{0.723pt}}
\put(404,226){\usebox{\plotpoint}}
\put(404,226){\usebox{\plotpoint}}
\put(404.0,223.0){\rule[-0.200pt]{0.400pt}{0.723pt}}
\put(404.0,223.0){\usebox{\plotpoint}}
\put(405.0,219.0){\rule[-0.200pt]{0.400pt}{0.964pt}}
\put(405.0,219.0){\usebox{\plotpoint}}
\put(406.0,215.0){\rule[-0.200pt]{0.400pt}{0.964pt}}
\put(406.0,215.0){\usebox{\plotpoint}}
\put(407,209.67){\rule{0.241pt}{0.400pt}}
\multiput(407.00,210.17)(0.500,-1.000){2}{\rule{0.120pt}{0.400pt}}
\put(407.0,211.0){\rule[-0.200pt]{0.400pt}{0.964pt}}
\put(408,210){\usebox{\plotpoint}}
\put(408.0,206.0){\rule[-0.200pt]{0.400pt}{0.964pt}}
\put(408.0,206.0){\usebox{\plotpoint}}
\put(409,200.67){\rule{0.241pt}{0.400pt}}
\multiput(409.00,201.17)(0.500,-1.000){2}{\rule{0.120pt}{0.400pt}}
\put(409.0,202.0){\rule[-0.200pt]{0.400pt}{0.964pt}}
\put(410,201){\usebox{\plotpoint}}
\put(410,201){\usebox{\plotpoint}}
\put(410.0,197.0){\rule[-0.200pt]{0.400pt}{0.964pt}}
\put(410.0,197.0){\usebox{\plotpoint}}
\put(411.0,193.0){\rule[-0.200pt]{0.400pt}{0.964pt}}
\put(411.0,193.0){\usebox{\plotpoint}}
\put(412.0,189.0){\rule[-0.200pt]{0.400pt}{0.964pt}}
\put(412.0,189.0){\usebox{\plotpoint}}
\put(413.0,185.0){\rule[-0.200pt]{0.400pt}{0.964pt}}
\put(413.0,185.0){\usebox{\plotpoint}}
\put(414,180.67){\rule{0.241pt}{0.400pt}}
\multiput(414.00,181.17)(0.500,-1.000){2}{\rule{0.120pt}{0.400pt}}
\put(414.0,182.0){\rule[-0.200pt]{0.400pt}{0.723pt}}
\put(415,181){\usebox{\plotpoint}}
\put(415,181){\usebox{\plotpoint}}
\put(415.0,178.0){\rule[-0.200pt]{0.400pt}{0.723pt}}
\put(415.0,178.0){\usebox{\plotpoint}}
\put(416,174.67){\rule{0.241pt}{0.400pt}}
\multiput(416.00,175.17)(0.500,-1.000){2}{\rule{0.120pt}{0.400pt}}
\put(416.0,176.0){\rule[-0.200pt]{0.400pt}{0.482pt}}
\put(417,175){\usebox{\plotpoint}}
\put(417,175){\usebox{\plotpoint}}
\put(417,175){\usebox{\plotpoint}}
\put(417.0,173.0){\rule[-0.200pt]{0.400pt}{0.482pt}}
\put(417.0,173.0){\usebox{\plotpoint}}
\put(418.0,171.0){\rule[-0.200pt]{0.400pt}{0.482pt}}
\put(418.0,171.0){\usebox{\plotpoint}}
\put(419.0,170.0){\usebox{\plotpoint}}
\put(419.0,170.0){\usebox{\plotpoint}}
\put(420.0,169.0){\usebox{\plotpoint}}
\put(420.0,169.0){\usebox{\plotpoint}}
\put(421.0,168.0){\usebox{\plotpoint}}
\put(421.0,168.0){\rule[-0.200pt]{0.723pt}{0.400pt}}
\put(424.0,168.0){\usebox{\plotpoint}}
\put(424.0,169.0){\usebox{\plotpoint}}
\put(425,169.67){\rule{0.241pt}{0.400pt}}
\multiput(425.00,169.17)(0.500,1.000){2}{\rule{0.120pt}{0.400pt}}
\put(425.0,169.0){\usebox{\plotpoint}}
\put(426,171){\usebox{\plotpoint}}
\put(426,171){\usebox{\plotpoint}}
\put(426,171){\usebox{\plotpoint}}
\put(426,171){\usebox{\plotpoint}}
\put(426,171){\usebox{\plotpoint}}
\put(426,171){\usebox{\plotpoint}}
\put(426,171){\usebox{\plotpoint}}
\put(426.0,171.0){\usebox{\plotpoint}}
\put(426.0,172.0){\usebox{\plotpoint}}
\put(427.0,172.0){\rule[-0.200pt]{0.400pt}{0.482pt}}
\put(427.0,174.0){\usebox{\plotpoint}}
\put(428.0,174.0){\rule[-0.200pt]{0.400pt}{0.723pt}}
\put(428.0,177.0){\usebox{\plotpoint}}
\put(429,178.67){\rule{0.241pt}{0.400pt}}
\multiput(429.00,178.17)(0.500,1.000){2}{\rule{0.120pt}{0.400pt}}
\put(429.0,177.0){\rule[-0.200pt]{0.400pt}{0.482pt}}
\put(430,180){\usebox{\plotpoint}}
\put(430,180){\usebox{\plotpoint}}
\put(430,180){\usebox{\plotpoint}}
\put(430,181.67){\rule{0.241pt}{0.400pt}}
\multiput(430.00,181.17)(0.500,1.000){2}{\rule{0.120pt}{0.400pt}}
\put(430.0,180.0){\rule[-0.200pt]{0.400pt}{0.482pt}}
\put(431,183){\usebox{\plotpoint}}
\put(431,183){\usebox{\plotpoint}}
\put(431,183){\usebox{\plotpoint}}
\put(431.0,183.0){\rule[-0.200pt]{0.400pt}{0.723pt}}
\put(431.0,186.0){\usebox{\plotpoint}}
\put(432.0,186.0){\rule[-0.200pt]{0.400pt}{0.723pt}}
\put(432.0,189.0){\usebox{\plotpoint}}
\put(433,191.67){\rule{0.241pt}{0.400pt}}
\multiput(433.00,191.17)(0.500,1.000){2}{\rule{0.120pt}{0.400pt}}
\put(433.0,189.0){\rule[-0.200pt]{0.400pt}{0.723pt}}
\put(434,193){\usebox{\plotpoint}}
\put(434,193){\usebox{\plotpoint}}
\put(434.0,193.0){\rule[-0.200pt]{0.400pt}{0.723pt}}
\put(434.0,196.0){\usebox{\plotpoint}}
\put(435,198.67){\rule{0.241pt}{0.400pt}}
\multiput(435.00,198.17)(0.500,1.000){2}{\rule{0.120pt}{0.400pt}}
\put(435.0,196.0){\rule[-0.200pt]{0.400pt}{0.723pt}}
\put(436,200){\usebox{\plotpoint}}
\put(436,200){\usebox{\plotpoint}}
\put(436,200){\usebox{\plotpoint}}
\put(436.0,200.0){\rule[-0.200pt]{0.400pt}{0.723pt}}
\put(436.0,203.0){\usebox{\plotpoint}}
\put(437.0,203.0){\rule[-0.200pt]{0.400pt}{0.723pt}}
\put(437.0,206.0){\usebox{\plotpoint}}
\put(438,208.67){\rule{0.241pt}{0.400pt}}
\multiput(438.00,208.17)(0.500,1.000){2}{\rule{0.120pt}{0.400pt}}
\put(438.0,206.0){\rule[-0.200pt]{0.400pt}{0.723pt}}
\put(439,210){\usebox{\plotpoint}}
\put(439,210){\usebox{\plotpoint}}
\put(439.0,210.0){\rule[-0.200pt]{0.400pt}{0.723pt}}
\put(439.0,213.0){\usebox{\plotpoint}}
\put(440,214.67){\rule{0.241pt}{0.400pt}}
\multiput(440.00,214.17)(0.500,1.000){2}{\rule{0.120pt}{0.400pt}}
\put(440.0,213.0){\rule[-0.200pt]{0.400pt}{0.482pt}}
\put(441,216){\usebox{\plotpoint}}
\put(441,216){\usebox{\plotpoint}}
\put(441,216){\usebox{\plotpoint}}
\put(441.0,216.0){\rule[-0.200pt]{0.400pt}{0.482pt}}
\put(441.0,218.0){\usebox{\plotpoint}}
\put(442.0,218.0){\rule[-0.200pt]{0.400pt}{0.482pt}}
\put(442.0,220.0){\usebox{\plotpoint}}
\put(443.0,220.0){\rule[-0.200pt]{0.400pt}{0.482pt}}
\put(443.0,222.0){\usebox{\plotpoint}}
\put(444.0,222.0){\rule[-0.200pt]{0.400pt}{0.482pt}}
\put(444.0,224.0){\usebox{\plotpoint}}
\put(445.0,224.0){\usebox{\plotpoint}}
\put(445.0,225.0){\usebox{\plotpoint}}
\put(446.0,225.0){\usebox{\plotpoint}}
\put(446.0,226.0){\usebox{\plotpoint}}
\put(447.0,226.0){\usebox{\plotpoint}}
\put(447.0,227.0){\rule[-0.200pt]{0.482pt}{0.400pt}}
\put(449.0,226.0){\usebox{\plotpoint}}
\put(449.0,226.0){\rule[-0.200pt]{0.482pt}{0.400pt}}
\put(451.0,225.0){\usebox{\plotpoint}}
\put(451.0,225.0){\usebox{\plotpoint}}
\put(452.0,224.0){\usebox{\plotpoint}}
\put(452.0,224.0){\usebox{\plotpoint}}
\put(453.0,222.0){\rule[-0.200pt]{0.400pt}{0.482pt}}
\put(453.0,222.0){\usebox{\plotpoint}}
\put(454.0,220.0){\rule[-0.200pt]{0.400pt}{0.482pt}}
\put(454.0,220.0){\usebox{\plotpoint}}
\put(455.0,218.0){\rule[-0.200pt]{0.400pt}{0.482pt}}
\put(455.0,218.0){\usebox{\plotpoint}}
\put(456.0,216.0){\rule[-0.200pt]{0.400pt}{0.482pt}}
\put(456.0,216.0){\usebox{\plotpoint}}
\put(457,212.67){\rule{0.241pt}{0.400pt}}
\multiput(457.00,213.17)(0.500,-1.000){2}{\rule{0.120pt}{0.400pt}}
\put(457.0,214.0){\rule[-0.200pt]{0.400pt}{0.482pt}}
\put(458,213){\usebox{\plotpoint}}
\put(458,213){\usebox{\plotpoint}}
\put(458,213){\usebox{\plotpoint}}
\put(458.0,211.0){\rule[-0.200pt]{0.400pt}{0.482pt}}
\put(458.0,211.0){\usebox{\plotpoint}}
\put(459.0,208.0){\rule[-0.200pt]{0.400pt}{0.723pt}}
\put(459.0,208.0){\usebox{\plotpoint}}
\put(460,204.67){\rule{0.241pt}{0.400pt}}
\multiput(460.00,205.17)(0.500,-1.000){2}{\rule{0.120pt}{0.400pt}}
\put(460.0,206.0){\rule[-0.200pt]{0.400pt}{0.482pt}}
\put(461,205){\usebox{\plotpoint}}
\put(461,205){\usebox{\plotpoint}}
\put(461,205){\usebox{\plotpoint}}
\put(461,205){\usebox{\plotpoint}}
\put(461.0,203.0){\rule[-0.200pt]{0.400pt}{0.482pt}}
\put(461.0,203.0){\usebox{\plotpoint}}
\put(462.0,200.0){\rule[-0.200pt]{0.400pt}{0.723pt}}
\put(462.0,200.0){\usebox{\plotpoint}}
\put(463.0,198.0){\rule[-0.200pt]{0.400pt}{0.482pt}}
\put(463.0,198.0){\usebox{\plotpoint}}
\put(464,194.67){\rule{0.241pt}{0.400pt}}
\multiput(464.00,195.17)(0.500,-1.000){2}{\rule{0.120pt}{0.400pt}}
\put(464.0,196.0){\rule[-0.200pt]{0.400pt}{0.482pt}}
\put(465,195){\usebox{\plotpoint}}
\put(465,195){\usebox{\plotpoint}}
\put(465,195){\usebox{\plotpoint}}
\put(465,195){\usebox{\plotpoint}}
\put(465.0,193.0){\rule[-0.200pt]{0.400pt}{0.482pt}}
\put(465.0,193.0){\usebox{\plotpoint}}
\put(466.0,191.0){\rule[-0.200pt]{0.400pt}{0.482pt}}
\put(466.0,191.0){\usebox{\plotpoint}}
\put(467.0,189.0){\rule[-0.200pt]{0.400pt}{0.482pt}}
\put(467.0,189.0){\usebox{\plotpoint}}
\put(468.0,187.0){\rule[-0.200pt]{0.400pt}{0.482pt}}
\put(468.0,187.0){\usebox{\plotpoint}}
\put(469.0,186.0){\usebox{\plotpoint}}
\put(469.0,186.0){\usebox{\plotpoint}}
\put(470.0,185.0){\usebox{\plotpoint}}
\put(470.0,185.0){\usebox{\plotpoint}}
\put(471.0,184.0){\usebox{\plotpoint}}
\put(471.0,184.0){\usebox{\plotpoint}}
\put(472.0,183.0){\usebox{\plotpoint}}
\put(472.0,183.0){\rule[-0.200pt]{1.204pt}{0.400pt}}
\put(477.0,183.0){\usebox{\plotpoint}}
\put(477.0,184.0){\usebox{\plotpoint}}
\put(478.0,184.0){\usebox{\plotpoint}}
\put(478.0,185.0){\usebox{\plotpoint}}
\put(479.0,185.0){\usebox{\plotpoint}}
\put(479.0,186.0){\usebox{\plotpoint}}
\put(480.0,186.0){\usebox{\plotpoint}}
\put(480.0,187.0){\usebox{\plotpoint}}
\put(481.0,187.0){\rule[-0.200pt]{0.400pt}{0.482pt}}
\put(481.0,189.0){\usebox{\plotpoint}}
\put(482.0,189.0){\usebox{\plotpoint}}
\put(482.0,190.0){\usebox{\plotpoint}}
\put(483.0,190.0){\rule[-0.200pt]{0.400pt}{0.482pt}}
\put(483.0,192.0){\usebox{\plotpoint}}
\put(484.0,192.0){\rule[-0.200pt]{0.400pt}{0.482pt}}
\put(484.0,194.0){\usebox{\plotpoint}}
\put(485,194.67){\rule{0.241pt}{0.400pt}}
\multiput(485.00,194.17)(0.500,1.000){2}{\rule{0.120pt}{0.400pt}}
\put(485.0,194.0){\usebox{\plotpoint}}
\put(486,196){\usebox{\plotpoint}}
\put(486,196){\usebox{\plotpoint}}
\put(486,196){\usebox{\plotpoint}}
\put(486,196){\usebox{\plotpoint}}
\put(486,196){\usebox{\plotpoint}}
\put(486.0,196.0){\usebox{\plotpoint}}
\put(486.0,197.0){\usebox{\plotpoint}}
\put(487.0,197.0){\rule[-0.200pt]{0.400pt}{0.482pt}}
\put(487.0,199.0){\usebox{\plotpoint}}
\put(488.0,199.0){\rule[-0.200pt]{0.400pt}{0.482pt}}
\put(488.0,201.0){\usebox{\plotpoint}}
\put(489.0,201.0){\rule[-0.200pt]{0.400pt}{0.482pt}}
\put(489.0,203.0){\usebox{\plotpoint}}
\put(490.0,203.0){\usebox{\plotpoint}}
\put(490.0,204.0){\usebox{\plotpoint}}
\put(491.0,204.0){\rule[-0.200pt]{0.400pt}{0.482pt}}
\put(491.0,206.0){\usebox{\plotpoint}}
\put(492.0,206.0){\usebox{\plotpoint}}
\put(492.0,207.0){\usebox{\plotpoint}}
\put(493.0,207.0){\usebox{\plotpoint}}
\put(493.0,208.0){\usebox{\plotpoint}}
\put(494.0,208.0){\usebox{\plotpoint}}
\put(494.0,209.0){\usebox{\plotpoint}}
\put(495.0,209.0){\usebox{\plotpoint}}
\put(495.0,210.0){\usebox{\plotpoint}}
\put(496.0,210.0){\usebox{\plotpoint}}
\put(496.0,211.0){\rule[-0.200pt]{0.482pt}{0.400pt}}
\put(498.0,211.0){\usebox{\plotpoint}}
\put(498.0,212.0){\rule[-0.200pt]{0.964pt}{0.400pt}}
\put(502.0,211.0){\usebox{\plotpoint}}
\put(502.0,211.0){\rule[-0.200pt]{0.482pt}{0.400pt}}
\put(504.0,210.0){\usebox{\plotpoint}}
\put(504.0,210.0){\rule[-0.200pt]{0.482pt}{0.400pt}}
\put(506.0,209.0){\usebox{\plotpoint}}
\put(506.0,209.0){\usebox{\plotpoint}}
\put(507.0,208.0){\usebox{\plotpoint}}
\put(507.0,208.0){\usebox{\plotpoint}}
\put(508.0,207.0){\usebox{\plotpoint}}
\put(508.0,207.0){\usebox{\plotpoint}}
\put(509.0,206.0){\usebox{\plotpoint}}
\put(509.0,206.0){\usebox{\plotpoint}}
\put(510.0,205.0){\usebox{\plotpoint}}
\put(510.0,205.0){\usebox{\plotpoint}}
\put(511.0,204.0){\usebox{\plotpoint}}
\put(511.0,204.0){\usebox{\plotpoint}}
\put(512.0,203.0){\usebox{\plotpoint}}
\put(512.0,203.0){\usebox{\plotpoint}}
\put(513.0,202.0){\usebox{\plotpoint}}
\put(513.0,202.0){\usebox{\plotpoint}}
\put(514.0,201.0){\usebox{\plotpoint}}
\put(515,199.67){\rule{0.241pt}{0.400pt}}
\multiput(515.00,200.17)(0.500,-1.000){2}{\rule{0.120pt}{0.400pt}}
\put(514.0,201.0){\usebox{\plotpoint}}
\put(516,200){\usebox{\plotpoint}}
\put(516,200){\usebox{\plotpoint}}
\put(516,200){\usebox{\plotpoint}}
\put(516,200){\usebox{\plotpoint}}
\put(516,200){\usebox{\plotpoint}}
\put(516,200){\usebox{\plotpoint}}
\put(516,200){\usebox{\plotpoint}}
\put(516,200){\usebox{\plotpoint}}
\put(516,200){\usebox{\plotpoint}}
\put(516,200){\usebox{\plotpoint}}
\put(516,200){\usebox{\plotpoint}}
\put(516.0,200.0){\usebox{\plotpoint}}
\put(517.0,199.0){\usebox{\plotpoint}}
\put(517.0,199.0){\rule[-0.200pt]{0.482pt}{0.400pt}}
\put(519.0,198.0){\usebox{\plotpoint}}
\put(519.0,198.0){\rule[-0.200pt]{0.482pt}{0.400pt}}
\put(521.0,197.0){\usebox{\plotpoint}}
\put(526,196.67){\rule{0.241pt}{0.400pt}}
\multiput(526.00,196.17)(0.500,1.000){2}{\rule{0.120pt}{0.400pt}}
\put(521.0,197.0){\rule[-0.200pt]{1.204pt}{0.400pt}}
\put(527,198){\usebox{\plotpoint}}
\put(527,198){\usebox{\plotpoint}}
\put(527,198){\usebox{\plotpoint}}
\put(527,198){\usebox{\plotpoint}}
\put(527,198){\usebox{\plotpoint}}
\put(527,198){\usebox{\plotpoint}}
\put(527,198){\usebox{\plotpoint}}
\put(527,198){\usebox{\plotpoint}}
\put(527,198){\usebox{\plotpoint}}
\put(527,198){\usebox{\plotpoint}}
\put(527,198){\usebox{\plotpoint}}
\put(527.0,198.0){\rule[-0.200pt]{0.723pt}{0.400pt}}
\put(530.0,198.0){\usebox{\plotpoint}}
\put(530.0,199.0){\rule[-0.200pt]{0.723pt}{0.400pt}}
\put(533.0,199.0){\usebox{\plotpoint}}
\put(533.0,200.0){\rule[-0.200pt]{0.723pt}{0.400pt}}
\put(536.0,200.0){\usebox{\plotpoint}}
\put(536.0,201.0){\rule[-0.200pt]{0.723pt}{0.400pt}}
\end{picture}
\normalsize
  }
  \Formelbox{
    \[ \Delta A = 4 \frac{F_R}{c} \] 
    \[ m \ddot y + c y + F_R = 0 \]
    }
  }
  {\Groessenbox{
     $ \Delta A$         & $\Delta$ Amplitude pro Periode        & $ [m] $ \\
     $ F_R $             & Reibkraft                             & $ [\frac{1}{s}] $    \\
     $ c$                & Feder"-kon"-stan"-te                  & $ [\frac{N}{m}] $        \\
}}


\subsection{Schwingung mit geschwindigkeitsproportionaler D�mpfung ($D < 1$ )}

\index{Schwingung!Ged�mpfte Schwingung}
\index{Ged�mpfte Schwingung}

\Hauptbox{
  \Bildbox{
    % GNUPLOT: LaTeX picture
\setlength{\unitlength}{0.240900pt}
\ifx\plotpoint\undefined\newsavebox{\plotpoint}\fi
\begin{picture}(600,360)(60,0)
\scriptsize
\font\gnuplot=cmr10 at 10pt
\gnuplot
\sbox{\plotpoint}{\rule[-0.200pt]{0.400pt}{0.400pt}}%
\put(140.0,82.0){\rule[-0.200pt]{4.818pt}{0.400pt}}
\put(120,82){\makebox(0,0)[r]{-0.8}}
\put(519.0,82.0){\rule[-0.200pt]{4.818pt}{0.400pt}}
\put(140.0,108.0){\rule[-0.200pt]{4.818pt}{0.400pt}}
%\put(120,108){\makebox(0,0)[r]{-0.6}}
\put(519.0,108.0){\rule[-0.200pt]{4.818pt}{0.400pt}}
\put(140.0,135.0){\rule[-0.200pt]{4.818pt}{0.400pt}}
\put(120,135){\makebox(0,0)[r]{-0.4}}
\put(519.0,135.0){\rule[-0.200pt]{4.818pt}{0.400pt}}
\put(140.0,161.0){\rule[-0.200pt]{4.818pt}{0.400pt}}
%\put(120,161){\makebox(0,0)[r]{-0.2}}
\put(519.0,161.0){\rule[-0.200pt]{4.818pt}{0.400pt}}
\put(140.0,188.0){\rule[-0.200pt]{4.818pt}{0.400pt}}
\put(120,188){\makebox(0,0)[r]{0}}
\put(519.0,188.0){\rule[-0.200pt]{4.818pt}{0.400pt}}
\put(140.0,214.0){\rule[-0.200pt]{4.818pt}{0.400pt}}
%\put(120,214){\makebox(0,0)[r]{0.2}}
\put(519.0,214.0){\rule[-0.200pt]{4.818pt}{0.400pt}}
\put(140.0,241.0){\rule[-0.200pt]{4.818pt}{0.400pt}}
\put(120,241){\makebox(0,0)[r]{0.4}}
\put(519.0,241.0){\rule[-0.200pt]{4.818pt}{0.400pt}}
\put(140.0,267.0){\rule[-0.200pt]{4.818pt}{0.400pt}}
%\put(120,267){\makebox(0,0)[r]{0.6}}
\put(519.0,267.0){\rule[-0.200pt]{4.818pt}{0.400pt}}
\put(140.0,294.0){\rule[-0.200pt]{4.818pt}{0.400pt}}
\put(120,294){\makebox(0,0)[r]{0.8}}
\put(519.0,294.0){\rule[-0.200pt]{4.818pt}{0.400pt}}
\put(140.0,320.0){\rule[-0.200pt]{4.818pt}{0.400pt}}
%\put(120,320){\makebox(0,0)[r]{1}}
\put(519.0,320.0){\rule[-0.200pt]{4.818pt}{0.400pt}}
\put(140.0,82.0){\rule[-0.200pt]{0.400pt}{4.818pt}}
\put(140,41){\makebox(0,0){0}}
\put(140.0,300.0){\rule[-0.200pt]{0.400pt}{4.818pt}}
\put(180.0,82.0){\rule[-0.200pt]{0.400pt}{4.818pt}}
%\put(180,41){\makebox(0,0){5}}
\put(180.0,300.0){\rule[-0.200pt]{0.400pt}{4.818pt}}
\put(220.0,82.0){\rule[-0.200pt]{0.400pt}{4.818pt}}
\put(220,41){\makebox(0,0){10}}
\put(220.0,300.0){\rule[-0.200pt]{0.400pt}{4.818pt}}
\put(260.0,82.0){\rule[-0.200pt]{0.400pt}{4.818pt}}
%\put(260,41){\makebox(0,0){15}}
\put(260.0,300.0){\rule[-0.200pt]{0.400pt}{4.818pt}}
\put(300.0,82.0){\rule[-0.200pt]{0.400pt}{4.818pt}}
\put(300,41){\makebox(0,0){20}}
\put(300.0,300.0){\rule[-0.200pt]{0.400pt}{4.818pt}}
\put(340.0,82.0){\rule[-0.200pt]{0.400pt}{4.818pt}}
%\put(340,41){\makebox(0,0){25}}
\put(340.0,300.0){\rule[-0.200pt]{0.400pt}{4.818pt}}
\put(379.0,82.0){\rule[-0.200pt]{0.400pt}{4.818pt}}
\put(379,41){\makebox(0,0){30}}
\put(379.0,300.0){\rule[-0.200pt]{0.400pt}{4.818pt}}
\put(419.0,82.0){\rule[-0.200pt]{0.400pt}{4.818pt}}
%\put(419,41){\makebox(0,0){35}}
\put(419.0,300.0){\rule[-0.200pt]{0.400pt}{4.818pt}}
\put(459.0,82.0){\rule[-0.200pt]{0.400pt}{4.818pt}}
\put(459,41){\makebox(0,0){40}}
\put(459.0,300.0){\rule[-0.200pt]{0.400pt}{4.818pt}}
\put(499.0,82.0){\rule[-0.200pt]{0.400pt}{4.818pt}}
%\put(499,41){\makebox(0,0){45}}
\put(499.0,300.0){\rule[-0.200pt]{0.400pt}{4.818pt}}
\put(539.0,82.0){\rule[-0.200pt]{0.400pt}{4.818pt}}
\put(539,41){\makebox(0,0){50}}
\put(539.0,300.0){\rule[-0.200pt]{0.400pt}{4.818pt}}
\put(140.0,82.0){\rule[-0.200pt]{96.119pt}{0.400pt}}
\put(539.0,82.0){\rule[-0.200pt]{0.400pt}{57.334pt}}
\put(140.0,320.0){\rule[-0.200pt]{96.119pt}{0.400pt}}
\put(140.0,82.0){\rule[-0.200pt]{0.400pt}{57.334pt}}
%\put(379,280){\makebox(0,0)[r]{line 1}}
%\put(399.0,280.0){\rule[-0.200pt]{24.090pt}{0.400pt}}
\put(140,188){\usebox{\plotpoint}}
\put(140,195.67){\rule{0.241pt}{0.400pt}}
\multiput(140.00,195.17)(0.500,1.000){2}{\rule{0.120pt}{0.400pt}}
\put(140.0,188.0){\rule[-0.200pt]{0.400pt}{1.927pt}}
\put(141,210.67){\rule{0.241pt}{0.400pt}}
\multiput(141.00,210.17)(0.500,1.000){2}{\rule{0.120pt}{0.400pt}}
\put(141.0,197.0){\rule[-0.200pt]{0.400pt}{3.373pt}}
\put(142,226.67){\rule{0.241pt}{0.400pt}}
\multiput(142.00,226.17)(0.500,1.000){2}{\rule{0.120pt}{0.400pt}}
\put(142.0,212.0){\rule[-0.200pt]{0.400pt}{3.613pt}}
\put(143,240.67){\rule{0.241pt}{0.400pt}}
\multiput(143.00,240.17)(0.500,1.000){2}{\rule{0.120pt}{0.400pt}}
\put(143.0,228.0){\rule[-0.200pt]{0.400pt}{3.132pt}}
\put(144,253.67){\rule{0.241pt}{0.400pt}}
\multiput(144.00,253.17)(0.500,1.000){2}{\rule{0.120pt}{0.400pt}}
\put(144.0,242.0){\rule[-0.200pt]{0.400pt}{2.891pt}}
\put(145,264.67){\rule{0.241pt}{0.400pt}}
\multiput(145.00,264.17)(0.500,1.000){2}{\rule{0.120pt}{0.400pt}}
\put(145.0,255.0){\rule[-0.200pt]{0.400pt}{2.409pt}}
\put(146,275.67){\rule{0.241pt}{0.400pt}}
\multiput(146.00,275.17)(0.500,1.000){2}{\rule{0.120pt}{0.400pt}}
\put(146.0,266.0){\rule[-0.200pt]{0.400pt}{2.409pt}}
\put(147,283.67){\rule{0.241pt}{0.400pt}}
\multiput(147.00,283.17)(0.500,1.000){2}{\rule{0.120pt}{0.400pt}}
\put(147.0,277.0){\rule[-0.200pt]{0.400pt}{1.686pt}}
\put(148.0,285.0){\rule[-0.200pt]{0.400pt}{1.686pt}}
\put(148.0,292.0){\usebox{\plotpoint}}
\put(149.0,292.0){\rule[-0.200pt]{0.400pt}{1.204pt}}
\put(149.0,297.0){\usebox{\plotpoint}}
\put(150.0,297.0){\rule[-0.200pt]{0.400pt}{0.723pt}}
\put(150.0,300.0){\usebox{\plotpoint}}
\put(151.0,300.0){\usebox{\plotpoint}}
\put(151.0,301.0){\rule[-0.200pt]{0.482pt}{0.400pt}}
\put(153,297.67){\rule{0.241pt}{0.400pt}}
\multiput(153.00,298.17)(0.500,-1.000){2}{\rule{0.120pt}{0.400pt}}
\put(153.0,299.0){\rule[-0.200pt]{0.400pt}{0.482pt}}
\put(154,298){\usebox{\plotpoint}}
\put(154,298){\usebox{\plotpoint}}
\put(154,298){\usebox{\plotpoint}}
\put(154.0,295.0){\rule[-0.200pt]{0.400pt}{0.723pt}}
\put(154.0,295.0){\usebox{\plotpoint}}
\put(155.0,289.0){\rule[-0.200pt]{0.400pt}{1.445pt}}
\put(155.0,289.0){\usebox{\plotpoint}}
\put(156,281.67){\rule{0.241pt}{0.400pt}}
\multiput(156.00,282.17)(0.500,-1.000){2}{\rule{0.120pt}{0.400pt}}
\put(156.0,283.0){\rule[-0.200pt]{0.400pt}{1.445pt}}
\put(157,282){\usebox{\plotpoint}}
\put(157.0,274.0){\rule[-0.200pt]{0.400pt}{1.927pt}}
\put(157.0,274.0){\usebox{\plotpoint}}
\put(158.0,265.0){\rule[-0.200pt]{0.400pt}{2.168pt}}
\put(158.0,265.0){\usebox{\plotpoint}}
\put(159,253.67){\rule{0.241pt}{0.400pt}}
\multiput(159.00,254.17)(0.500,-1.000){2}{\rule{0.120pt}{0.400pt}}
\put(159.0,255.0){\rule[-0.200pt]{0.400pt}{2.409pt}}
\put(160,242.67){\rule{0.241pt}{0.400pt}}
\multiput(160.00,243.17)(0.500,-1.000){2}{\rule{0.120pt}{0.400pt}}
\put(160.0,244.0){\rule[-0.200pt]{0.400pt}{2.409pt}}
\put(161,230.67){\rule{0.241pt}{0.400pt}}
\multiput(161.00,231.17)(0.500,-1.000){2}{\rule{0.120pt}{0.400pt}}
\put(161.0,232.0){\rule[-0.200pt]{0.400pt}{2.650pt}}
\put(162,218.67){\rule{0.241pt}{0.400pt}}
\multiput(162.00,219.17)(0.500,-1.000){2}{\rule{0.120pt}{0.400pt}}
\put(162.0,220.0){\rule[-0.200pt]{0.400pt}{2.650pt}}
\put(163,206.67){\rule{0.241pt}{0.400pt}}
\multiput(163.00,207.17)(0.500,-1.000){2}{\rule{0.120pt}{0.400pt}}
\put(163.0,208.0){\rule[-0.200pt]{0.400pt}{2.650pt}}
\put(164,193.67){\rule{0.241pt}{0.400pt}}
\multiput(164.00,194.17)(0.500,-1.000){2}{\rule{0.120pt}{0.400pt}}
\put(164.0,195.0){\rule[-0.200pt]{0.400pt}{2.891pt}}
\put(165,181.67){\rule{0.241pt}{0.400pt}}
\multiput(165.00,182.17)(0.500,-1.000){2}{\rule{0.120pt}{0.400pt}}
\put(165.0,183.0){\rule[-0.200pt]{0.400pt}{2.650pt}}
\put(166,169.67){\rule{0.241pt}{0.400pt}}
\multiput(166.00,170.17)(0.500,-1.000){2}{\rule{0.120pt}{0.400pt}}
\put(166.0,171.0){\rule[-0.200pt]{0.400pt}{2.650pt}}
\put(167,158.67){\rule{0.241pt}{0.400pt}}
\multiput(167.00,159.17)(0.500,-1.000){2}{\rule{0.120pt}{0.400pt}}
\put(167.0,160.0){\rule[-0.200pt]{0.400pt}{2.409pt}}
\put(168.0,149.0){\rule[-0.200pt]{0.400pt}{2.409pt}}
\put(168.0,149.0){\usebox{\plotpoint}}
\put(169,138.67){\rule{0.241pt}{0.400pt}}
\multiput(169.00,139.17)(0.500,-1.000){2}{\rule{0.120pt}{0.400pt}}
\put(169.0,140.0){\rule[-0.200pt]{0.400pt}{2.168pt}}
\put(170,139){\usebox{\plotpoint}}
\put(170.0,131.0){\rule[-0.200pt]{0.400pt}{1.927pt}}
\put(170.0,131.0){\usebox{\plotpoint}}
\put(171,122.67){\rule{0.241pt}{0.400pt}}
\multiput(171.00,123.17)(0.500,-1.000){2}{\rule{0.120pt}{0.400pt}}
\put(171.0,124.0){\rule[-0.200pt]{0.400pt}{1.686pt}}
\put(172,123){\usebox{\plotpoint}}
\put(172.0,117.0){\rule[-0.200pt]{0.400pt}{1.445pt}}
\put(172.0,117.0){\usebox{\plotpoint}}
\put(173.0,112.0){\rule[-0.200pt]{0.400pt}{1.204pt}}
\put(173.0,112.0){\usebox{\plotpoint}}
\put(174.0,108.0){\rule[-0.200pt]{0.400pt}{0.964pt}}
\put(174.0,108.0){\usebox{\plotpoint}}
\put(175.0,106.0){\rule[-0.200pt]{0.400pt}{0.482pt}}
\put(175.0,106.0){\usebox{\plotpoint}}
\put(176.0,105.0){\usebox{\plotpoint}}
\put(176.0,105.0){\rule[-0.200pt]{0.482pt}{0.400pt}}
\put(178.0,105.0){\rule[-0.200pt]{0.400pt}{0.482pt}}
\put(178.0,107.0){\usebox{\plotpoint}}
\put(179.0,107.0){\rule[-0.200pt]{0.400pt}{0.482pt}}
\put(179.0,109.0){\usebox{\plotpoint}}
\put(180,112.67){\rule{0.241pt}{0.400pt}}
\multiput(180.00,112.17)(0.500,1.000){2}{\rule{0.120pt}{0.400pt}}
\put(180.0,109.0){\rule[-0.200pt]{0.400pt}{0.964pt}}
\put(181,114){\usebox{\plotpoint}}
\put(181,114){\usebox{\plotpoint}}
\put(181,117.67){\rule{0.241pt}{0.400pt}}
\multiput(181.00,117.17)(0.500,1.000){2}{\rule{0.120pt}{0.400pt}}
\put(181.0,114.0){\rule[-0.200pt]{0.400pt}{0.964pt}}
\put(182,119){\usebox{\plotpoint}}
\put(182,123.67){\rule{0.241pt}{0.400pt}}
\multiput(182.00,123.17)(0.500,1.000){2}{\rule{0.120pt}{0.400pt}}
\put(182.0,119.0){\rule[-0.200pt]{0.400pt}{1.204pt}}
\put(183,125){\usebox{\plotpoint}}
\put(183,130.67){\rule{0.241pt}{0.400pt}}
\multiput(183.00,130.17)(0.500,1.000){2}{\rule{0.120pt}{0.400pt}}
\put(183.0,125.0){\rule[-0.200pt]{0.400pt}{1.445pt}}
\put(184,132){\usebox{\plotpoint}}
\put(184,137.67){\rule{0.241pt}{0.400pt}}
\multiput(184.00,137.17)(0.500,1.000){2}{\rule{0.120pt}{0.400pt}}
\put(184.0,132.0){\rule[-0.200pt]{0.400pt}{1.445pt}}
\put(185,139){\usebox{\plotpoint}}
\put(185.0,139.0){\rule[-0.200pt]{0.400pt}{1.927pt}}
\put(185.0,147.0){\usebox{\plotpoint}}
\put(186.0,147.0){\rule[-0.200pt]{0.400pt}{1.927pt}}
\put(186.0,155.0){\usebox{\plotpoint}}
\put(187,163.67){\rule{0.241pt}{0.400pt}}
\multiput(187.00,163.17)(0.500,1.000){2}{\rule{0.120pt}{0.400pt}}
\put(187.0,155.0){\rule[-0.200pt]{0.400pt}{2.168pt}}
\put(188,165){\usebox{\plotpoint}}
\put(188.0,165.0){\rule[-0.200pt]{0.400pt}{1.927pt}}
\put(188.0,173.0){\usebox{\plotpoint}}
\put(189,181.67){\rule{0.241pt}{0.400pt}}
\multiput(189.00,181.17)(0.500,1.000){2}{\rule{0.120pt}{0.400pt}}
\put(189.0,173.0){\rule[-0.200pt]{0.400pt}{2.168pt}}
\put(190,183){\usebox{\plotpoint}}
\put(190,189.67){\rule{0.241pt}{0.400pt}}
\multiput(190.00,189.17)(0.500,1.000){2}{\rule{0.120pt}{0.400pt}}
\put(190.0,183.0){\rule[-0.200pt]{0.400pt}{1.686pt}}
\put(191,198.67){\rule{0.241pt}{0.400pt}}
\multiput(191.00,198.17)(0.500,1.000){2}{\rule{0.120pt}{0.400pt}}
\put(191.0,191.0){\rule[-0.200pt]{0.400pt}{1.927pt}}
\put(192,206.67){\rule{0.241pt}{0.400pt}}
\multiput(192.00,206.17)(0.500,1.000){2}{\rule{0.120pt}{0.400pt}}
\put(192.0,200.0){\rule[-0.200pt]{0.400pt}{1.686pt}}
\put(193,208){\usebox{\plotpoint}}
\put(193,214.67){\rule{0.241pt}{0.400pt}}
\multiput(193.00,214.17)(0.500,1.000){2}{\rule{0.120pt}{0.400pt}}
\put(193.0,208.0){\rule[-0.200pt]{0.400pt}{1.686pt}}
\put(194,216){\usebox{\plotpoint}}
\put(194,221.67){\rule{0.241pt}{0.400pt}}
\multiput(194.00,221.17)(0.500,1.000){2}{\rule{0.120pt}{0.400pt}}
\put(194.0,216.0){\rule[-0.200pt]{0.400pt}{1.445pt}}
\put(195,223){\usebox{\plotpoint}}
\put(195.0,223.0){\rule[-0.200pt]{0.400pt}{1.445pt}}
\put(195.0,229.0){\usebox{\plotpoint}}
\put(196,233.67){\rule{0.241pt}{0.400pt}}
\multiput(196.00,233.17)(0.500,1.000){2}{\rule{0.120pt}{0.400pt}}
\put(196.0,229.0){\rule[-0.200pt]{0.400pt}{1.204pt}}
\put(197,235){\usebox{\plotpoint}}
\put(197.0,235.0){\rule[-0.200pt]{0.400pt}{0.964pt}}
\put(197.0,239.0){\usebox{\plotpoint}}
\put(198.0,239.0){\rule[-0.200pt]{0.400pt}{0.964pt}}
\put(198.0,243.0){\usebox{\plotpoint}}
\put(199,244.67){\rule{0.241pt}{0.400pt}}
\multiput(199.00,244.17)(0.500,1.000){2}{\rule{0.120pt}{0.400pt}}
\put(199.0,243.0){\rule[-0.200pt]{0.400pt}{0.482pt}}
\put(200,246){\usebox{\plotpoint}}
\put(200,246){\usebox{\plotpoint}}
\put(200,246){\usebox{\plotpoint}}
\put(200,246){\usebox{\plotpoint}}
\put(200,246.67){\rule{0.241pt}{0.400pt}}
\multiput(200.00,246.17)(0.500,1.000){2}{\rule{0.120pt}{0.400pt}}
\put(200.0,246.0){\usebox{\plotpoint}}
\put(201,248){\usebox{\plotpoint}}
\put(201,248){\usebox{\plotpoint}}
\put(201,248){\usebox{\plotpoint}}
\put(201,248){\usebox{\plotpoint}}
\put(201,248){\usebox{\plotpoint}}
\put(201,248){\usebox{\plotpoint}}
\put(201,248){\usebox{\plotpoint}}
\put(201,248){\usebox{\plotpoint}}
\put(201,248){\usebox{\plotpoint}}
\put(201,248){\usebox{\plotpoint}}
\put(201,248){\usebox{\plotpoint}}
\put(201.0,248.0){\rule[-0.200pt]{0.482pt}{0.400pt}}
\put(203.0,247.0){\usebox{\plotpoint}}
\put(203.0,247.0){\usebox{\plotpoint}}
\put(204.0,245.0){\rule[-0.200pt]{0.400pt}{0.482pt}}
\put(204.0,245.0){\usebox{\plotpoint}}
\put(205,241.67){\rule{0.241pt}{0.400pt}}
\multiput(205.00,242.17)(0.500,-1.000){2}{\rule{0.120pt}{0.400pt}}
\put(205.0,243.0){\rule[-0.200pt]{0.400pt}{0.482pt}}
\put(206,242){\usebox{\plotpoint}}
\put(206,242){\usebox{\plotpoint}}
\put(206,242){\usebox{\plotpoint}}
\put(206.0,239.0){\rule[-0.200pt]{0.400pt}{0.723pt}}
\put(206.0,239.0){\usebox{\plotpoint}}
\put(207,233.67){\rule{0.241pt}{0.400pt}}
\multiput(207.00,234.17)(0.500,-1.000){2}{\rule{0.120pt}{0.400pt}}
\put(207.0,235.0){\rule[-0.200pt]{0.400pt}{0.964pt}}
\put(208,234){\usebox{\plotpoint}}
\put(208,234){\usebox{\plotpoint}}
\put(208,228.67){\rule{0.241pt}{0.400pt}}
\multiput(208.00,229.17)(0.500,-1.000){2}{\rule{0.120pt}{0.400pt}}
\put(208.0,230.0){\rule[-0.200pt]{0.400pt}{0.964pt}}
\put(209,229){\usebox{\plotpoint}}
\put(209,223.67){\rule{0.241pt}{0.400pt}}
\multiput(209.00,224.17)(0.500,-1.000){2}{\rule{0.120pt}{0.400pt}}
\put(209.0,225.0){\rule[-0.200pt]{0.400pt}{0.964pt}}
\put(210,224){\usebox{\plotpoint}}
\put(210.0,218.0){\rule[-0.200pt]{0.400pt}{1.445pt}}
\put(210.0,218.0){\usebox{\plotpoint}}
\put(211.0,212.0){\rule[-0.200pt]{0.400pt}{1.445pt}}
\put(211.0,212.0){\usebox{\plotpoint}}
\put(212,204.67){\rule{0.241pt}{0.400pt}}
\multiput(212.00,205.17)(0.500,-1.000){2}{\rule{0.120pt}{0.400pt}}
\put(212.0,206.0){\rule[-0.200pt]{0.400pt}{1.445pt}}
\put(213,205){\usebox{\plotpoint}}
\put(213,197.67){\rule{0.241pt}{0.400pt}}
\multiput(213.00,198.17)(0.500,-1.000){2}{\rule{0.120pt}{0.400pt}}
\put(213.0,199.0){\rule[-0.200pt]{0.400pt}{1.445pt}}
\put(214,198){\usebox{\plotpoint}}
\put(214,191.67){\rule{0.241pt}{0.400pt}}
\multiput(214.00,192.17)(0.500,-1.000){2}{\rule{0.120pt}{0.400pt}}
\put(214.0,193.0){\rule[-0.200pt]{0.400pt}{1.204pt}}
\put(215,192){\usebox{\plotpoint}}
\put(215,184.67){\rule{0.241pt}{0.400pt}}
\multiput(215.00,185.17)(0.500,-1.000){2}{\rule{0.120pt}{0.400pt}}
\put(215.0,186.0){\rule[-0.200pt]{0.400pt}{1.445pt}}
\put(216,185){\usebox{\plotpoint}}
\put(216,178.67){\rule{0.241pt}{0.400pt}}
\multiput(216.00,179.17)(0.500,-1.000){2}{\rule{0.120pt}{0.400pt}}
\put(216.0,180.0){\rule[-0.200pt]{0.400pt}{1.204pt}}
\put(217,179){\usebox{\plotpoint}}
\put(217,172.67){\rule{0.241pt}{0.400pt}}
\multiput(217.00,173.17)(0.500,-1.000){2}{\rule{0.120pt}{0.400pt}}
\put(217.0,174.0){\rule[-0.200pt]{0.400pt}{1.204pt}}
\put(218,173){\usebox{\plotpoint}}
\put(218.0,168.0){\rule[-0.200pt]{0.400pt}{1.204pt}}
\put(218.0,168.0){\usebox{\plotpoint}}
\put(219,161.67){\rule{0.241pt}{0.400pt}}
\multiput(219.00,162.17)(0.500,-1.000){2}{\rule{0.120pt}{0.400pt}}
\put(219.0,163.0){\rule[-0.200pt]{0.400pt}{1.204pt}}
\put(220,162){\usebox{\plotpoint}}
\put(220,162){\usebox{\plotpoint}}
\put(220.0,158.0){\rule[-0.200pt]{0.400pt}{0.964pt}}
\put(220.0,158.0){\usebox{\plotpoint}}
\put(221.0,154.0){\rule[-0.200pt]{0.400pt}{0.964pt}}
\put(221.0,154.0){\usebox{\plotpoint}}
\put(222.0,151.0){\rule[-0.200pt]{0.400pt}{0.723pt}}
\put(222.0,151.0){\usebox{\plotpoint}}
\put(223.0,148.0){\rule[-0.200pt]{0.400pt}{0.723pt}}
\put(223.0,148.0){\usebox{\plotpoint}}
\put(224.0,146.0){\rule[-0.200pt]{0.400pt}{0.482pt}}
\put(224.0,146.0){\usebox{\plotpoint}}
\put(225.0,144.0){\rule[-0.200pt]{0.400pt}{0.482pt}}
\put(225.0,144.0){\rule[-0.200pt]{0.964pt}{0.400pt}}
\put(229.0,144.0){\rule[-0.200pt]{0.400pt}{0.482pt}}
\put(229.0,146.0){\usebox{\plotpoint}}
\put(230.0,146.0){\rule[-0.200pt]{0.400pt}{0.482pt}}
\put(230.0,148.0){\usebox{\plotpoint}}
\put(231.0,148.0){\rule[-0.200pt]{0.400pt}{0.482pt}}
\put(231.0,150.0){\usebox{\plotpoint}}
\put(232,152.67){\rule{0.241pt}{0.400pt}}
\multiput(232.00,152.17)(0.500,1.000){2}{\rule{0.120pt}{0.400pt}}
\put(232.0,150.0){\rule[-0.200pt]{0.400pt}{0.723pt}}
\put(233,154){\usebox{\plotpoint}}
\put(233,154){\usebox{\plotpoint}}
\put(233,154){\usebox{\plotpoint}}
\put(233.0,154.0){\rule[-0.200pt]{0.400pt}{0.723pt}}
\put(233.0,157.0){\usebox{\plotpoint}}
\put(234.0,157.0){\rule[-0.200pt]{0.400pt}{0.964pt}}
\put(234.0,161.0){\usebox{\plotpoint}}
\put(235.0,161.0){\rule[-0.200pt]{0.400pt}{0.964pt}}
\put(235.0,165.0){\usebox{\plotpoint}}
\put(236.0,165.0){\rule[-0.200pt]{0.400pt}{1.204pt}}
\put(236.0,170.0){\usebox{\plotpoint}}
\put(237,173.67){\rule{0.241pt}{0.400pt}}
\multiput(237.00,173.17)(0.500,1.000){2}{\rule{0.120pt}{0.400pt}}
\put(237.0,170.0){\rule[-0.200pt]{0.400pt}{0.964pt}}
\put(238,175){\usebox{\plotpoint}}
\put(238,175){\usebox{\plotpoint}}
\put(238,178.67){\rule{0.241pt}{0.400pt}}
\multiput(238.00,178.17)(0.500,1.000){2}{\rule{0.120pt}{0.400pt}}
\put(238.0,175.0){\rule[-0.200pt]{0.400pt}{0.964pt}}
\put(239,180){\usebox{\plotpoint}}
\put(239,180){\usebox{\plotpoint}}
\put(239.0,180.0){\rule[-0.200pt]{0.400pt}{0.964pt}}
\put(239.0,184.0){\usebox{\plotpoint}}
\put(240.0,184.0){\rule[-0.200pt]{0.400pt}{1.204pt}}
\put(240.0,189.0){\usebox{\plotpoint}}
\put(241.0,189.0){\rule[-0.200pt]{0.400pt}{0.964pt}}
\put(241.0,193.0){\usebox{\plotpoint}}
\put(242.0,193.0){\rule[-0.200pt]{0.400pt}{1.204pt}}
\put(242.0,198.0){\usebox{\plotpoint}}
\put(243.0,198.0){\rule[-0.200pt]{0.400pt}{0.964pt}}
\put(243.0,202.0){\usebox{\plotpoint}}
\put(244.0,202.0){\rule[-0.200pt]{0.400pt}{0.964pt}}
\put(244.0,206.0){\usebox{\plotpoint}}
\put(245.0,206.0){\rule[-0.200pt]{0.400pt}{0.723pt}}
\put(245.0,209.0){\usebox{\plotpoint}}
\put(246.0,209.0){\rule[-0.200pt]{0.400pt}{0.723pt}}
\put(246.0,212.0){\usebox{\plotpoint}}
\put(247.0,212.0){\rule[-0.200pt]{0.400pt}{0.723pt}}
\put(247.0,215.0){\usebox{\plotpoint}}
\put(248.0,215.0){\rule[-0.200pt]{0.400pt}{0.482pt}}
\put(248.0,217.0){\usebox{\plotpoint}}
\put(249,217.67){\rule{0.241pt}{0.400pt}}
\multiput(249.00,217.17)(0.500,1.000){2}{\rule{0.120pt}{0.400pt}}
\put(249.0,217.0){\usebox{\plotpoint}}
\put(250,219){\usebox{\plotpoint}}
\put(250,219){\usebox{\plotpoint}}
\put(250,219){\usebox{\plotpoint}}
\put(250,219){\usebox{\plotpoint}}
\put(250,219){\usebox{\plotpoint}}
\put(250,219){\usebox{\plotpoint}}
\put(250,219){\usebox{\plotpoint}}
\put(250,219){\usebox{\plotpoint}}
\put(250,219){\usebox{\plotpoint}}
\put(250,219){\usebox{\plotpoint}}
\put(250,219){\usebox{\plotpoint}}
\put(250,218.67){\rule{0.241pt}{0.400pt}}
\multiput(250.00,218.17)(0.500,1.000){2}{\rule{0.120pt}{0.400pt}}
\put(251,220){\usebox{\plotpoint}}
\put(251,220){\usebox{\plotpoint}}
\put(251,220){\usebox{\plotpoint}}
\put(251,220){\usebox{\plotpoint}}
\put(251,220){\usebox{\plotpoint}}
\put(251,220){\usebox{\plotpoint}}
\put(251,220){\usebox{\plotpoint}}
\put(251,220){\usebox{\plotpoint}}
\put(251,220){\usebox{\plotpoint}}
\put(251,220){\usebox{\plotpoint}}
\put(251,220){\usebox{\plotpoint}}
\put(251,220){\usebox{\plotpoint}}
\put(251.0,220.0){\rule[-0.200pt]{0.723pt}{0.400pt}}
\put(254.0,219.0){\usebox{\plotpoint}}
\put(254.0,219.0){\usebox{\plotpoint}}
\put(255.0,217.0){\rule[-0.200pt]{0.400pt}{0.482pt}}
\put(255.0,217.0){\usebox{\plotpoint}}
\put(256.0,215.0){\rule[-0.200pt]{0.400pt}{0.482pt}}
\put(256.0,215.0){\usebox{\plotpoint}}
\put(257.0,213.0){\rule[-0.200pt]{0.400pt}{0.482pt}}
\put(257.0,213.0){\usebox{\plotpoint}}
\put(258,209.67){\rule{0.241pt}{0.400pt}}
\multiput(258.00,210.17)(0.500,-1.000){2}{\rule{0.120pt}{0.400pt}}
\put(258.0,211.0){\rule[-0.200pt]{0.400pt}{0.482pt}}
\put(259,210){\usebox{\plotpoint}}
\put(259,210){\usebox{\plotpoint}}
\put(259,210){\usebox{\plotpoint}}
\put(259,210){\usebox{\plotpoint}}
\put(259,206.67){\rule{0.241pt}{0.400pt}}
\multiput(259.00,207.17)(0.500,-1.000){2}{\rule{0.120pt}{0.400pt}}
\put(259.0,208.0){\rule[-0.200pt]{0.400pt}{0.482pt}}
\put(260,207){\usebox{\plotpoint}}
\put(260,207){\usebox{\plotpoint}}
\put(260,207){\usebox{\plotpoint}}
\put(260.0,204.0){\rule[-0.200pt]{0.400pt}{0.723pt}}
\put(260.0,204.0){\usebox{\plotpoint}}
\put(261.0,201.0){\rule[-0.200pt]{0.400pt}{0.723pt}}
\put(261.0,201.0){\usebox{\plotpoint}}
\put(262,196.67){\rule{0.241pt}{0.400pt}}
\multiput(262.00,197.17)(0.500,-1.000){2}{\rule{0.120pt}{0.400pt}}
\put(262.0,198.0){\rule[-0.200pt]{0.400pt}{0.723pt}}
\put(263,197){\usebox{\plotpoint}}
\put(263,197){\usebox{\plotpoint}}
\put(263,197){\usebox{\plotpoint}}
\put(263.0,194.0){\rule[-0.200pt]{0.400pt}{0.723pt}}
\put(263.0,194.0){\usebox{\plotpoint}}
\put(264,189.67){\rule{0.241pt}{0.400pt}}
\multiput(264.00,190.17)(0.500,-1.000){2}{\rule{0.120pt}{0.400pt}}
\put(264.0,191.0){\rule[-0.200pt]{0.400pt}{0.723pt}}
\put(265,190){\usebox{\plotpoint}}
\put(265,190){\usebox{\plotpoint}}
\put(265,190){\usebox{\plotpoint}}
\put(265.0,187.0){\rule[-0.200pt]{0.400pt}{0.723pt}}
\put(265.0,187.0){\usebox{\plotpoint}}
\put(266.0,184.0){\rule[-0.200pt]{0.400pt}{0.723pt}}
\put(266.0,184.0){\usebox{\plotpoint}}
\put(267.0,181.0){\rule[-0.200pt]{0.400pt}{0.723pt}}
\put(267.0,181.0){\usebox{\plotpoint}}
\put(268,176.67){\rule{0.241pt}{0.400pt}}
\multiput(268.00,177.17)(0.500,-1.000){2}{\rule{0.120pt}{0.400pt}}
\put(268.0,178.0){\rule[-0.200pt]{0.400pt}{0.723pt}}
\put(269,177){\usebox{\plotpoint}}
\put(269,177){\usebox{\plotpoint}}
\put(269,177){\usebox{\plotpoint}}
\put(269,177){\usebox{\plotpoint}}
\put(269.0,175.0){\rule[-0.200pt]{0.400pt}{0.482pt}}
\put(269.0,175.0){\usebox{\plotpoint}}
\put(270.0,172.0){\rule[-0.200pt]{0.400pt}{0.723pt}}
\put(270.0,172.0){\usebox{\plotpoint}}
\put(271.0,170.0){\rule[-0.200pt]{0.400pt}{0.482pt}}
\put(271.0,170.0){\usebox{\plotpoint}}
\put(272.0,168.0){\rule[-0.200pt]{0.400pt}{0.482pt}}
\put(272.0,168.0){\usebox{\plotpoint}}
\put(273.0,167.0){\usebox{\plotpoint}}
\put(273.0,167.0){\usebox{\plotpoint}}
\put(274.0,165.0){\rule[-0.200pt]{0.400pt}{0.482pt}}
\put(274.0,165.0){\rule[-0.200pt]{0.482pt}{0.400pt}}
\put(276.0,164.0){\usebox{\plotpoint}}
\put(276.0,164.0){\rule[-0.200pt]{0.482pt}{0.400pt}}
\put(278.0,164.0){\usebox{\plotpoint}}
\put(278.0,165.0){\rule[-0.200pt]{0.482pt}{0.400pt}}
\put(280.0,165.0){\usebox{\plotpoint}}
\put(280.0,166.0){\usebox{\plotpoint}}
\put(281.0,166.0){\rule[-0.200pt]{0.400pt}{0.482pt}}
\put(281.0,168.0){\usebox{\plotpoint}}
\put(282.0,168.0){\usebox{\plotpoint}}
\put(282.0,169.0){\usebox{\plotpoint}}
\put(283.0,169.0){\rule[-0.200pt]{0.400pt}{0.482pt}}
\put(283.0,171.0){\usebox{\plotpoint}}
\put(284.0,171.0){\rule[-0.200pt]{0.400pt}{0.482pt}}
\put(284.0,173.0){\usebox{\plotpoint}}
\put(285,174.67){\rule{0.241pt}{0.400pt}}
\multiput(285.00,174.17)(0.500,1.000){2}{\rule{0.120pt}{0.400pt}}
\put(285.0,173.0){\rule[-0.200pt]{0.400pt}{0.482pt}}
\put(286,176){\usebox{\plotpoint}}
\put(286,176){\usebox{\plotpoint}}
\put(286,176){\usebox{\plotpoint}}
\put(286,176){\usebox{\plotpoint}}
\put(286.0,176.0){\rule[-0.200pt]{0.400pt}{0.482pt}}
\put(286.0,178.0){\usebox{\plotpoint}}
\put(287.0,178.0){\rule[-0.200pt]{0.400pt}{0.482pt}}
\put(287.0,180.0){\usebox{\plotpoint}}
\put(288.0,180.0){\rule[-0.200pt]{0.400pt}{0.723pt}}
\put(288.0,183.0){\usebox{\plotpoint}}
\put(289,184.67){\rule{0.241pt}{0.400pt}}
\multiput(289.00,184.17)(0.500,1.000){2}{\rule{0.120pt}{0.400pt}}
\put(289.0,183.0){\rule[-0.200pt]{0.400pt}{0.482pt}}
\put(290,186){\usebox{\plotpoint}}
\put(290,186){\usebox{\plotpoint}}
\put(290,186){\usebox{\plotpoint}}
\put(290,186){\usebox{\plotpoint}}
\put(290.0,186.0){\rule[-0.200pt]{0.400pt}{0.482pt}}
\put(290.0,188.0){\usebox{\plotpoint}}
\put(291,189.67){\rule{0.241pt}{0.400pt}}
\multiput(291.00,189.17)(0.500,1.000){2}{\rule{0.120pt}{0.400pt}}
\put(291.0,188.0){\rule[-0.200pt]{0.400pt}{0.482pt}}
\put(292,191){\usebox{\plotpoint}}
\put(292,191){\usebox{\plotpoint}}
\put(292,191){\usebox{\plotpoint}}
\put(292,191){\usebox{\plotpoint}}
\put(292.0,191.0){\rule[-0.200pt]{0.400pt}{0.482pt}}
\put(292.0,193.0){\usebox{\plotpoint}}
\put(293.0,193.0){\rule[-0.200pt]{0.400pt}{0.482pt}}
\put(293.0,195.0){\usebox{\plotpoint}}
\put(294.0,195.0){\rule[-0.200pt]{0.400pt}{0.482pt}}
\put(294.0,197.0){\usebox{\plotpoint}}
\put(295.0,197.0){\rule[-0.200pt]{0.400pt}{0.482pt}}
\put(295.0,199.0){\usebox{\plotpoint}}
\put(296.0,199.0){\rule[-0.200pt]{0.400pt}{0.482pt}}
\put(296.0,201.0){\usebox{\plotpoint}}
\put(297.0,201.0){\usebox{\plotpoint}}
\put(297.0,202.0){\usebox{\plotpoint}}
\put(298.0,202.0){\usebox{\plotpoint}}
\put(298.0,203.0){\usebox{\plotpoint}}
\put(299.0,203.0){\usebox{\plotpoint}}
\put(299.0,204.0){\usebox{\plotpoint}}
\put(300.0,204.0){\usebox{\plotpoint}}
\put(300.0,205.0){\rule[-0.200pt]{0.964pt}{0.400pt}}
\put(304.0,204.0){\usebox{\plotpoint}}
\put(304.0,204.0){\rule[-0.200pt]{0.482pt}{0.400pt}}
\put(306.0,203.0){\usebox{\plotpoint}}
\put(306.0,203.0){\usebox{\plotpoint}}
\put(307,200.67){\rule{0.241pt}{0.400pt}}
\multiput(307.00,201.17)(0.500,-1.000){2}{\rule{0.120pt}{0.400pt}}
\put(307.0,202.0){\usebox{\plotpoint}}
\put(308,201){\usebox{\plotpoint}}
\put(308,201){\usebox{\plotpoint}}
\put(308,201){\usebox{\plotpoint}}
\put(308,201){\usebox{\plotpoint}}
\put(308,201){\usebox{\plotpoint}}
\put(308,201){\usebox{\plotpoint}}
\put(308,201){\usebox{\plotpoint}}
\put(308,201){\usebox{\plotpoint}}
\put(308.0,200.0){\usebox{\plotpoint}}
\put(308.0,200.0){\usebox{\plotpoint}}
\put(309,197.67){\rule{0.241pt}{0.400pt}}
\multiput(309.00,198.17)(0.500,-1.000){2}{\rule{0.120pt}{0.400pt}}
\put(309.0,199.0){\usebox{\plotpoint}}
\put(310,198){\usebox{\plotpoint}}
\put(310,198){\usebox{\plotpoint}}
\put(310,198){\usebox{\plotpoint}}
\put(310,198){\usebox{\plotpoint}}
\put(310,198){\usebox{\plotpoint}}
\put(310,198){\usebox{\plotpoint}}
\put(310,198){\usebox{\plotpoint}}
\put(310.0,197.0){\usebox{\plotpoint}}
\put(310.0,197.0){\usebox{\plotpoint}}
\put(311.0,195.0){\rule[-0.200pt]{0.400pt}{0.482pt}}
\put(311.0,195.0){\usebox{\plotpoint}}
\put(312.0,193.0){\rule[-0.200pt]{0.400pt}{0.482pt}}
\put(312.0,193.0){\usebox{\plotpoint}}
\put(313,190.67){\rule{0.241pt}{0.400pt}}
\multiput(313.00,191.17)(0.500,-1.000){2}{\rule{0.120pt}{0.400pt}}
\put(313.0,192.0){\usebox{\plotpoint}}
\put(314,191){\usebox{\plotpoint}}
\put(314,191){\usebox{\plotpoint}}
\put(314,191){\usebox{\plotpoint}}
\put(314,191){\usebox{\plotpoint}}
\put(314,191){\usebox{\plotpoint}}
\put(314.0,190.0){\usebox{\plotpoint}}
\put(314.0,190.0){\usebox{\plotpoint}}
\put(315.0,188.0){\rule[-0.200pt]{0.400pt}{0.482pt}}
\put(315.0,188.0){\usebox{\plotpoint}}
\put(316.0,186.0){\rule[-0.200pt]{0.400pt}{0.482pt}}
\put(316.0,186.0){\usebox{\plotpoint}}
\put(317.0,184.0){\rule[-0.200pt]{0.400pt}{0.482pt}}
\put(317.0,184.0){\usebox{\plotpoint}}
\put(318.0,183.0){\usebox{\plotpoint}}
\put(318.0,183.0){\usebox{\plotpoint}}
\put(319.0,181.0){\rule[-0.200pt]{0.400pt}{0.482pt}}
\put(319.0,181.0){\usebox{\plotpoint}}
\put(320.0,180.0){\usebox{\plotpoint}}
\put(320.0,180.0){\usebox{\plotpoint}}
\put(321,177.67){\rule{0.241pt}{0.400pt}}
\multiput(321.00,178.17)(0.500,-1.000){2}{\rule{0.120pt}{0.400pt}}
\put(321.0,179.0){\usebox{\plotpoint}}
\put(322,178){\usebox{\plotpoint}}
\put(322,178){\usebox{\plotpoint}}
\put(322,178){\usebox{\plotpoint}}
\put(322,178){\usebox{\plotpoint}}
\put(322,178){\usebox{\plotpoint}}
\put(322,178){\usebox{\plotpoint}}
\put(322,178){\usebox{\plotpoint}}
\put(322,178){\usebox{\plotpoint}}
\put(322,178){\usebox{\plotpoint}}
\put(322,178){\usebox{\plotpoint}}
\put(322,178){\usebox{\plotpoint}}
\put(322,176.67){\rule{0.241pt}{0.400pt}}
\multiput(322.00,177.17)(0.500,-1.000){2}{\rule{0.120pt}{0.400pt}}
\put(323,177){\usebox{\plotpoint}}
\put(323,177){\usebox{\plotpoint}}
\put(323,177){\usebox{\plotpoint}}
\put(323,177){\usebox{\plotpoint}}
\put(323,177){\usebox{\plotpoint}}
\put(323,177){\usebox{\plotpoint}}
\put(323,177){\usebox{\plotpoint}}
\put(323,177){\usebox{\plotpoint}}
\put(323,177){\usebox{\plotpoint}}
\put(323,177){\usebox{\plotpoint}}
\put(323,177){\usebox{\plotpoint}}
\put(323,177){\usebox{\plotpoint}}
\put(323.0,177.0){\usebox{\plotpoint}}
\put(324.0,176.0){\usebox{\plotpoint}}
\put(325,174.67){\rule{0.241pt}{0.400pt}}
\multiput(325.00,175.17)(0.500,-1.000){2}{\rule{0.120pt}{0.400pt}}
\put(324.0,176.0){\usebox{\plotpoint}}
\put(326,175){\usebox{\plotpoint}}
\put(326,175){\usebox{\plotpoint}}
\put(326,175){\usebox{\plotpoint}}
\put(326,175){\usebox{\plotpoint}}
\put(326,175){\usebox{\plotpoint}}
\put(326,175){\usebox{\plotpoint}}
\put(326,175){\usebox{\plotpoint}}
\put(326,175){\usebox{\plotpoint}}
\put(326,175){\usebox{\plotpoint}}
\put(326,175){\usebox{\plotpoint}}
\put(326,175){\usebox{\plotpoint}}
\put(326,175){\usebox{\plotpoint}}
\put(326.0,175.0){\rule[-0.200pt]{0.723pt}{0.400pt}}
\put(329.0,175.0){\usebox{\plotpoint}}
\put(329.0,176.0){\rule[-0.200pt]{0.482pt}{0.400pt}}
\put(331.0,176.0){\usebox{\plotpoint}}
\put(331.0,177.0){\usebox{\plotpoint}}
\put(332.0,177.0){\usebox{\plotpoint}}
\put(332.0,178.0){\usebox{\plotpoint}}
\put(333.0,178.0){\usebox{\plotpoint}}
\put(333.0,179.0){\usebox{\plotpoint}}
\put(334.0,179.0){\usebox{\plotpoint}}
\put(334.0,180.0){\usebox{\plotpoint}}
\put(335.0,180.0){\usebox{\plotpoint}}
\put(335.0,181.0){\usebox{\plotpoint}}
\put(336.0,181.0){\usebox{\plotpoint}}
\put(336.0,182.0){\usebox{\plotpoint}}
\put(337.0,182.0){\rule[-0.200pt]{0.400pt}{0.482pt}}
\put(337.0,184.0){\usebox{\plotpoint}}
\put(338.0,184.0){\usebox{\plotpoint}}
\put(338.0,185.0){\usebox{\plotpoint}}
\put(339.0,185.0){\usebox{\plotpoint}}
\put(339.0,186.0){\usebox{\plotpoint}}
\put(340.0,186.0){\rule[-0.200pt]{0.400pt}{0.482pt}}
\put(340.0,188.0){\usebox{\plotpoint}}
\put(341.0,188.0){\usebox{\plotpoint}}
\put(341.0,189.0){\usebox{\plotpoint}}
\put(342.0,189.0){\usebox{\plotpoint}}
\put(342.0,190.0){\usebox{\plotpoint}}
\put(343,190.67){\rule{0.241pt}{0.400pt}}
\multiput(343.00,190.17)(0.500,1.000){2}{\rule{0.120pt}{0.400pt}}
\put(343.0,190.0){\usebox{\plotpoint}}
\put(344,192){\usebox{\plotpoint}}
\put(344,192){\usebox{\plotpoint}}
\put(344,192){\usebox{\plotpoint}}
\put(344,192){\usebox{\plotpoint}}
\put(344,192){\usebox{\plotpoint}}
\put(344,192){\usebox{\plotpoint}}
\put(344,192){\usebox{\plotpoint}}
\put(344,192){\usebox{\plotpoint}}
\put(344,192){\usebox{\plotpoint}}
\put(344,192){\usebox{\plotpoint}}
\put(344.0,192.0){\usebox{\plotpoint}}
\put(344.0,193.0){\usebox{\plotpoint}}
\put(345.0,193.0){\usebox{\plotpoint}}
\put(346,193.67){\rule{0.241pt}{0.400pt}}
\multiput(346.00,193.17)(0.500,1.000){2}{\rule{0.120pt}{0.400pt}}
\put(345.0,194.0){\usebox{\plotpoint}}
\put(347,195){\usebox{\plotpoint}}
\put(347,195){\usebox{\plotpoint}}
\put(347,195){\usebox{\plotpoint}}
\put(347,195){\usebox{\plotpoint}}
\put(347,195){\usebox{\plotpoint}}
\put(347,195){\usebox{\plotpoint}}
\put(347,195){\usebox{\plotpoint}}
\put(347,195){\usebox{\plotpoint}}
\put(347,195){\usebox{\plotpoint}}
\put(347,195){\usebox{\plotpoint}}
\put(347,195){\usebox{\plotpoint}}
\put(347,195){\usebox{\plotpoint}}
\put(347.0,195.0){\usebox{\plotpoint}}
\put(348.0,195.0){\usebox{\plotpoint}}
\put(348.0,196.0){\rule[-0.200pt]{0.482pt}{0.400pt}}
\put(350.0,196.0){\usebox{\plotpoint}}
\put(350.0,197.0){\rule[-0.200pt]{1.204pt}{0.400pt}}
\put(355.0,196.0){\usebox{\plotpoint}}
\put(355.0,196.0){\rule[-0.200pt]{0.482pt}{0.400pt}}
\put(357.0,195.0){\usebox{\plotpoint}}
\put(357.0,195.0){\usebox{\plotpoint}}
\put(358.0,194.0){\usebox{\plotpoint}}
\put(358.0,194.0){\rule[-0.200pt]{0.482pt}{0.400pt}}
\put(360.0,193.0){\usebox{\plotpoint}}
\put(360.0,193.0){\usebox{\plotpoint}}
\put(361.0,192.0){\usebox{\plotpoint}}
\put(361.0,192.0){\usebox{\plotpoint}}
\put(362.0,191.0){\usebox{\plotpoint}}
\put(362.0,191.0){\usebox{\plotpoint}}
\put(363.0,190.0){\usebox{\plotpoint}}
\put(363.0,190.0){\usebox{\plotpoint}}
\put(364.0,189.0){\usebox{\plotpoint}}
\put(364.0,189.0){\usebox{\plotpoint}}
\put(365.0,188.0){\usebox{\plotpoint}}
\put(365.0,188.0){\usebox{\plotpoint}}
\put(366.0,187.0){\usebox{\plotpoint}}
\put(366.0,187.0){\usebox{\plotpoint}}
\put(367.0,186.0){\usebox{\plotpoint}}
\put(367.0,186.0){\usebox{\plotpoint}}
\put(368.0,185.0){\usebox{\plotpoint}}
\put(368.0,185.0){\usebox{\plotpoint}}
\put(369.0,184.0){\usebox{\plotpoint}}
\put(369.0,184.0){\rule[-0.200pt]{0.482pt}{0.400pt}}
\put(371.0,183.0){\usebox{\plotpoint}}
\put(371.0,183.0){\usebox{\plotpoint}}
\put(372.0,182.0){\usebox{\plotpoint}}
\put(374,180.67){\rule{0.241pt}{0.400pt}}
\multiput(374.00,181.17)(0.500,-1.000){2}{\rule{0.120pt}{0.400pt}}
\put(372.0,182.0){\rule[-0.200pt]{0.482pt}{0.400pt}}
\put(375,181){\usebox{\plotpoint}}
\put(375,181){\usebox{\plotpoint}}
\put(375,181){\usebox{\plotpoint}}
\put(375,181){\usebox{\plotpoint}}
\put(375,181){\usebox{\plotpoint}}
\put(375,181){\usebox{\plotpoint}}
\put(375,181){\usebox{\plotpoint}}
\put(375,181){\usebox{\plotpoint}}
\put(375,181){\usebox{\plotpoint}}
\put(375,181){\usebox{\plotpoint}}
\put(375,181){\usebox{\plotpoint}}
\put(375,181){\usebox{\plotpoint}}
\put(375.0,181.0){\rule[-0.200pt]{1.204pt}{0.400pt}}
\put(380.0,181.0){\usebox{\plotpoint}}
\put(380.0,182.0){\rule[-0.200pt]{0.723pt}{0.400pt}}
\put(383.0,182.0){\usebox{\plotpoint}}
\put(383.0,183.0){\rule[-0.200pt]{0.482pt}{0.400pt}}
\put(385.0,183.0){\usebox{\plotpoint}}
\put(385.0,184.0){\usebox{\plotpoint}}
\put(386.0,184.0){\usebox{\plotpoint}}
\put(386.0,185.0){\rule[-0.200pt]{0.482pt}{0.400pt}}
\put(388.0,185.0){\usebox{\plotpoint}}
\put(388.0,186.0){\usebox{\plotpoint}}
\put(389.0,186.0){\usebox{\plotpoint}}
\put(389.0,187.0){\usebox{\plotpoint}}
\put(390.0,187.0){\usebox{\plotpoint}}
\put(390.0,188.0){\rule[-0.200pt]{0.482pt}{0.400pt}}
\put(392.0,188.0){\usebox{\plotpoint}}
\put(392.0,189.0){\usebox{\plotpoint}}
\put(393.0,189.0){\usebox{\plotpoint}}
\put(393.0,190.0){\rule[-0.200pt]{0.482pt}{0.400pt}}
\put(395.0,190.0){\usebox{\plotpoint}}
\put(395.0,191.0){\rule[-0.200pt]{0.482pt}{0.400pt}}
\put(397.0,191.0){\usebox{\plotpoint}}
\put(397.0,192.0){\rule[-0.200pt]{0.723pt}{0.400pt}}
\put(400.0,192.0){\usebox{\plotpoint}}
\put(400.0,193.0){\rule[-0.200pt]{1.204pt}{0.400pt}}
\put(405.0,192.0){\usebox{\plotpoint}}
\put(405.0,192.0){\rule[-0.200pt]{0.723pt}{0.400pt}}
\put(408.0,191.0){\usebox{\plotpoint}}
\put(408.0,191.0){\rule[-0.200pt]{0.723pt}{0.400pt}}
\put(411.0,190.0){\usebox{\plotpoint}}
\put(411.0,190.0){\rule[-0.200pt]{0.482pt}{0.400pt}}
\put(413.0,189.0){\usebox{\plotpoint}}
\put(413.0,189.0){\usebox{\plotpoint}}
\put(414.0,188.0){\usebox{\plotpoint}}
\put(414.0,188.0){\rule[-0.200pt]{0.482pt}{0.400pt}}
\put(416.0,187.0){\usebox{\plotpoint}}
\put(416.0,187.0){\rule[-0.200pt]{0.482pt}{0.400pt}}
\put(418.0,186.0){\usebox{\plotpoint}}
\put(418.0,186.0){\rule[-0.200pt]{0.723pt}{0.400pt}}
\put(421.0,185.0){\usebox{\plotpoint}}
\put(421.0,185.0){\rule[-0.200pt]{0.723pt}{0.400pt}}
\put(424.0,184.0){\usebox{\plotpoint}}
\put(424.0,184.0){\rule[-0.200pt]{1.686pt}{0.400pt}}
\put(431.0,184.0){\usebox{\plotpoint}}
\put(431.0,185.0){\rule[-0.200pt]{0.964pt}{0.400pt}}
\put(435.0,185.0){\usebox{\plotpoint}}
\put(437,185.67){\rule{0.241pt}{0.400pt}}
\multiput(437.00,185.17)(0.500,1.000){2}{\rule{0.120pt}{0.400pt}}
\put(435.0,186.0){\rule[-0.200pt]{0.482pt}{0.400pt}}
\put(438,187){\usebox{\plotpoint}}
\put(438,187){\usebox{\plotpoint}}
\put(438,187){\usebox{\plotpoint}}
\put(438,187){\usebox{\plotpoint}}
\put(438,187){\usebox{\plotpoint}}
\put(438,187){\usebox{\plotpoint}}
\put(438,187){\usebox{\plotpoint}}
\put(438,187){\usebox{\plotpoint}}
\put(438,187){\usebox{\plotpoint}}
\put(438,187){\usebox{\plotpoint}}
\put(438,187){\usebox{\plotpoint}}
\put(438.0,187.0){\rule[-0.200pt]{0.482pt}{0.400pt}}
\put(440.0,187.0){\usebox{\plotpoint}}
\put(440.0,188.0){\rule[-0.200pt]{0.723pt}{0.400pt}}
\put(443.0,188.0){\usebox{\plotpoint}}
\put(443.0,189.0){\rule[-0.200pt]{0.723pt}{0.400pt}}
\put(446.0,189.0){\usebox{\plotpoint}}
\put(446.0,190.0){\rule[-0.200pt]{3.373pt}{0.400pt}}
\put(460.0,189.0){\usebox{\plotpoint}}
\put(460.0,189.0){\rule[-0.200pt]{0.723pt}{0.400pt}}
\put(463.0,188.0){\usebox{\plotpoint}}
\put(463.0,188.0){\rule[-0.200pt]{0.964pt}{0.400pt}}
\put(467.0,187.0){\usebox{\plotpoint}}
\put(467.0,187.0){\rule[-0.200pt]{0.964pt}{0.400pt}}
\put(471.0,186.0){\usebox{\plotpoint}}
\put(484,185.67){\rule{0.241pt}{0.400pt}}
\multiput(484.00,185.17)(0.500,1.000){2}{\rule{0.120pt}{0.400pt}}
\put(471.0,186.0){\rule[-0.200pt]{3.132pt}{0.400pt}}
\put(485,187){\usebox{\plotpoint}}
\put(485,187){\usebox{\plotpoint}}
\put(485,187){\usebox{\plotpoint}}
\put(485,187){\usebox{\plotpoint}}
\put(485,187){\usebox{\plotpoint}}
\put(485,187){\usebox{\plotpoint}}
\put(485,187){\usebox{\plotpoint}}
\put(485,187){\usebox{\plotpoint}}
\put(485,187){\usebox{\plotpoint}}
\put(485,187){\usebox{\plotpoint}}
\put(485,187){\usebox{\plotpoint}}
\put(485.0,187.0){\rule[-0.200pt]{1.204pt}{0.400pt}}
\put(490.0,187.0){\usebox{\plotpoint}}
\put(490.0,188.0){\rule[-0.200pt]{1.204pt}{0.400pt}}
\put(495.0,188.0){\usebox{\plotpoint}}
\put(495.0,189.0){\rule[-0.200pt]{3.854pt}{0.400pt}}
\put(511.0,188.0){\usebox{\plotpoint}}
\put(511.0,188.0){\rule[-0.200pt]{1.686pt}{0.400pt}}
\put(518.0,187.0){\usebox{\plotpoint}}
\put(518.0,187.0){\rule[-0.200pt]{5.059pt}{0.400pt}}
\put(539.0,187.0){\usebox{\plotpoint}}
\end{picture}
\normalsize
  }
  \Formelbox{
    \[ m \ddot y + b \dot y + cy = 0 \] 
    \[ y = A e^{-\delta t} \sin ( \omega _d t + \phi _0 ) \]
    \[ \delta = \frac{b}{2m} \mbox{\hspace{1cm}} F_R = -b \dot y \]
    \[ \omega _d = \sqrt{\omega _0 ^2 - \delta ^2 } \]
    \[ \omega _0 = \sqrt{\frac{c}{m}} \]
    \[ D = \frac {\delta}{\omega _0 } \]
    \[ D = \frac{ \frac{\Lambda}{2 \pi}}{\sqrt{1 + \left( \frac{\Lambda}{2 \pi} \right )^2}} \]
    \[ \omega _d = \omega _0 \sqrt{ 1 - D^2} \]
    \[ \Lambda = \frac{2 \pi D}{\sqrt{1 - D^2}} \]  
    \[ \Lambda = \delta T  \]  
    \[ \Lambda = \ln \frac{A_n}{A_{n + 1}} \mbox{\hspace{1cm}} \frac{A_n}{A_{n + 1}}  = e ^{\delta T} \] 
    \[ \frac{E_1}{E_2} = \frac{A_1 ^2}{A_2 ^2} \]
    }
  }
  {\Groessenbox{
     $ y$                & schwingende Gr�sse                    & $ [m] $ \\
     $ \omega$           & Kreisfre"-quenz                       & $ [\frac{1}{s}] $    \\
     $ \varphi$          & Win"-kel                              & $ [rad] $        \\
     $T$                 & Periode                               & $ [s] $ \\
     $\delta$            & Abkling"-konst.                       & $ [1] $ \\
     $b$                 & D�mpfungs"-konst.                     & $ [ \frac{kg}{s}] $ \\
     $m$                 & Masse                                 & $ [kg] $\\  
     $c$                 & Feder"-konst.                         & $ [ \frac{N}{m} ]$\\
     $D$                 & D�mpfungs"-grad                       & $ [ 1 ] $ \\
     $\Lambda$           & log. Dekre"-ment                      & $ [ 1] $ \\
     $ A $               & Amplitude                             & $ [ 1 ]$ \\   
     $ E $               & Energie                               & $ [ J ]$ \\ 
}}
\vfill


\subsection{Aperiodeische L�sung ($D > 1$ )}

\index{Schwingung!Aperiodische Schwingung}
\index{Aperiodische Schwingung}

\Hauptbox{
  \Bildbox{
    % GNUPLOT: LaTeX picture
\setlength{\unitlength}{0.240900pt}
\ifx\plotpoint\undefined\newsavebox{\plotpoint}\fi
\sbox{\plotpoint}{\rule[-0.200pt]{0.400pt}{0.400pt}}%
\scriptsize
\begin{picture}(600,360)(60,0)
\font\gnuplot=cmr10 at 10pt
\gnuplot
\sbox{\plotpoint}{\rule[-0.200pt]{0.400pt}{0.400pt}}%
\put(140.0,82.0){\rule[-0.200pt]{4.818pt}{0.400pt}}
\put(120,82){\makebox(0,0)[r]{-0.2}}
\put(519.0,82.0){\rule[-0.200pt]{4.818pt}{0.400pt}}
\put(140.0,122.0){\rule[-0.200pt]{4.818pt}{0.400pt}}
\put(120,122){\makebox(0,0)[r]{0}}
\put(519.0,122.0){\rule[-0.200pt]{4.818pt}{0.400pt}}
\put(140.0,161.0){\rule[-0.200pt]{4.818pt}{0.400pt}}
\put(120,161){\makebox(0,0)[r]{0.2}}
\put(519.0,161.0){\rule[-0.200pt]{4.818pt}{0.400pt}}
\put(140.0,201.0){\rule[-0.200pt]{4.818pt}{0.400pt}}
\put(120,201){\makebox(0,0)[r]{0.4}}
\put(519.0,201.0){\rule[-0.200pt]{4.818pt}{0.400pt}}
\put(140.0,241.0){\rule[-0.200pt]{4.818pt}{0.400pt}}
\put(120,241){\makebox(0,0)[r]{0.6}}
\put(519.0,241.0){\rule[-0.200pt]{4.818pt}{0.400pt}}
\put(140.0,280.0){\rule[-0.200pt]{4.818pt}{0.400pt}}
\put(120,280){\makebox(0,0)[r]{0.8}}
\put(519.0,280.0){\rule[-0.200pt]{4.818pt}{0.400pt}}
\put(140.0,320.0){\rule[-0.200pt]{4.818pt}{0.400pt}}
\put(120,320){\makebox(0,0)[r]{1}}
\put(519.0,320.0){\rule[-0.200pt]{4.818pt}{0.400pt}}
\put(140.0,82.0){\rule[-0.200pt]{0.400pt}{4.818pt}}
\put(140,41){\makebox(0,0){0}}
\put(140.0,300.0){\rule[-0.200pt]{0.400pt}{4.818pt}}
\put(207.0,82.0){\rule[-0.200pt]{0.400pt}{4.818pt}}
\put(207,41){\makebox(0,0){1}}
\put(207.0,300.0){\rule[-0.200pt]{0.400pt}{4.818pt}}
\put(273.0,82.0){\rule[-0.200pt]{0.400pt}{4.818pt}}
\put(273,41){\makebox(0,0){2}}
\put(273.0,300.0){\rule[-0.200pt]{0.400pt}{4.818pt}}
\put(340.0,82.0){\rule[-0.200pt]{0.400pt}{4.818pt}}
\put(340,41){\makebox(0,0){3}}
\put(340.0,300.0){\rule[-0.200pt]{0.400pt}{4.818pt}}
\put(406.0,82.0){\rule[-0.200pt]{0.400pt}{4.818pt}}
\put(406,41){\makebox(0,0){4}}
\put(406.0,300.0){\rule[-0.200pt]{0.400pt}{4.818pt}}
\put(473.0,82.0){\rule[-0.200pt]{0.400pt}{4.818pt}}
\put(473,41){\makebox(0,0){5}}
\put(473.0,300.0){\rule[-0.200pt]{0.400pt}{4.818pt}}
\put(539.0,82.0){\rule[-0.200pt]{0.400pt}{4.818pt}}
\put(539,41){\makebox(0,0){6}}
\put(539.0,300.0){\rule[-0.200pt]{0.400pt}{4.818pt}}
\put(140.0,82.0){\rule[-0.200pt]{96.119pt}{0.400pt}}
\put(539.0,82.0){\rule[-0.200pt]{0.400pt}{57.334pt}}
\put(140.0,320.0){\rule[-0.200pt]{96.119pt}{0.400pt}}
\put(140.0,82.0){\rule[-0.200pt]{0.400pt}{57.334pt}}
\put(379,280){\makebox(0,0)[r]{$D = 1.2$}}
\put(399.0,280.0){\rule[-0.200pt]{24.090pt}{0.400pt}}
\put(140,320){\usebox{\plotpoint}}
\put(139.67,308){\rule{0.400pt}{2.891pt}}
\multiput(139.17,314.00)(1.000,-6.000){2}{\rule{0.400pt}{1.445pt}}
\put(140.67,286){\rule{0.400pt}{2.650pt}}
\multiput(140.17,291.50)(1.000,-5.500){2}{\rule{0.400pt}{1.325pt}}
\put(141.67,276){\rule{0.400pt}{2.409pt}}
\multiput(141.17,281.00)(1.000,-5.000){2}{\rule{0.400pt}{1.204pt}}
\put(141.0,297.0){\rule[-0.200pt]{0.400pt}{2.650pt}}
\put(142.67,258){\rule{0.400pt}{2.168pt}}
\multiput(142.17,262.50)(1.000,-4.500){2}{\rule{0.400pt}{1.084pt}}
\put(143.67,249){\rule{0.400pt}{2.168pt}}
\multiput(143.17,253.50)(1.000,-4.500){2}{\rule{0.400pt}{1.084pt}}
\put(143.0,267.0){\rule[-0.200pt]{0.400pt}{2.168pt}}
\put(144.67,234){\rule{0.400pt}{1.927pt}}
\multiput(144.17,238.00)(1.000,-4.000){2}{\rule{0.400pt}{0.964pt}}
\put(145.67,227){\rule{0.400pt}{1.686pt}}
\multiput(145.17,230.50)(1.000,-3.500){2}{\rule{0.400pt}{0.843pt}}
\put(145.0,242.0){\rule[-0.200pt]{0.400pt}{1.686pt}}
\put(146.67,215){\rule{0.400pt}{1.445pt}}
\multiput(146.17,218.00)(1.000,-3.000){2}{\rule{0.400pt}{0.723pt}}
\put(147.67,209){\rule{0.400pt}{1.445pt}}
\multiput(147.17,212.00)(1.000,-3.000){2}{\rule{0.400pt}{0.723pt}}
\put(147.0,221.0){\rule[-0.200pt]{0.400pt}{1.445pt}}
\put(148.67,199){\rule{0.400pt}{1.204pt}}
\multiput(148.17,201.50)(1.000,-2.500){2}{\rule{0.400pt}{0.602pt}}
\put(149.67,194){\rule{0.400pt}{1.204pt}}
\multiput(149.17,196.50)(1.000,-2.500){2}{\rule{0.400pt}{0.602pt}}
\put(149.0,204.0){\rule[-0.200pt]{0.400pt}{1.204pt}}
\put(150.67,186){\rule{0.400pt}{0.964pt}}
\multiput(150.17,188.00)(1.000,-2.000){2}{\rule{0.400pt}{0.482pt}}
\put(151.67,182){\rule{0.400pt}{0.964pt}}
\multiput(151.17,184.00)(1.000,-2.000){2}{\rule{0.400pt}{0.482pt}}
\put(151.0,190.0){\rule[-0.200pt]{0.400pt}{0.964pt}}
\put(152.67,175){\rule{0.400pt}{0.723pt}}
\multiput(152.17,176.50)(1.000,-1.500){2}{\rule{0.400pt}{0.361pt}}
\put(153.67,171){\rule{0.400pt}{0.964pt}}
\multiput(153.17,173.00)(1.000,-2.000){2}{\rule{0.400pt}{0.482pt}}
\put(153.0,178.0){\rule[-0.200pt]{0.400pt}{0.964pt}}
\put(154.67,166){\rule{0.400pt}{0.482pt}}
\multiput(154.17,167.00)(1.000,-1.000){2}{\rule{0.400pt}{0.241pt}}
\put(155.67,163){\rule{0.400pt}{0.723pt}}
\multiput(155.17,164.50)(1.000,-1.500){2}{\rule{0.400pt}{0.361pt}}
\put(155.0,168.0){\rule[-0.200pt]{0.400pt}{0.723pt}}
\put(156.67,158){\rule{0.400pt}{0.482pt}}
\multiput(156.17,159.00)(1.000,-1.000){2}{\rule{0.400pt}{0.241pt}}
\put(157.67,156){\rule{0.400pt}{0.482pt}}
\multiput(157.17,157.00)(1.000,-1.000){2}{\rule{0.400pt}{0.241pt}}
\put(157.0,160.0){\rule[-0.200pt]{0.400pt}{0.723pt}}
\put(158.67,152){\rule{0.400pt}{0.482pt}}
\multiput(158.17,153.00)(1.000,-1.000){2}{\rule{0.400pt}{0.241pt}}
\put(159.67,150){\rule{0.400pt}{0.482pt}}
\multiput(159.17,151.00)(1.000,-1.000){2}{\rule{0.400pt}{0.241pt}}
\put(159.0,154.0){\rule[-0.200pt]{0.400pt}{0.482pt}}
\put(161,146.67){\rule{0.241pt}{0.400pt}}
\multiput(161.00,147.17)(0.500,-1.000){2}{\rule{0.120pt}{0.400pt}}
\put(161.67,145){\rule{0.400pt}{0.482pt}}
\multiput(161.17,146.00)(1.000,-1.000){2}{\rule{0.400pt}{0.241pt}}
\put(161.0,148.0){\rule[-0.200pt]{0.400pt}{0.482pt}}
\put(162.67,142){\rule{0.400pt}{0.482pt}}
\multiput(162.17,143.00)(1.000,-1.000){2}{\rule{0.400pt}{0.241pt}}
\put(164,140.67){\rule{0.241pt}{0.400pt}}
\multiput(164.00,141.17)(0.500,-1.000){2}{\rule{0.120pt}{0.400pt}}
\put(163.0,144.0){\usebox{\plotpoint}}
\put(165,138.67){\rule{0.241pt}{0.400pt}}
\multiput(165.00,139.17)(0.500,-1.000){2}{\rule{0.120pt}{0.400pt}}
\put(166,137.67){\rule{0.241pt}{0.400pt}}
\multiput(166.00,138.17)(0.500,-1.000){2}{\rule{0.120pt}{0.400pt}}
\put(165.0,140.0){\usebox{\plotpoint}}
\put(167,135.67){\rule{0.241pt}{0.400pt}}
\multiput(167.00,136.17)(0.500,-1.000){2}{\rule{0.120pt}{0.400pt}}
\put(168,134.67){\rule{0.241pt}{0.400pt}}
\multiput(168.00,135.17)(0.500,-1.000){2}{\rule{0.120pt}{0.400pt}}
\put(167.0,137.0){\usebox{\plotpoint}}
\put(169,132.67){\rule{0.241pt}{0.400pt}}
\multiput(169.00,133.17)(0.500,-1.000){2}{\rule{0.120pt}{0.400pt}}
\put(169.0,134.0){\usebox{\plotpoint}}
\put(170.0,133.0){\usebox{\plotpoint}}
\put(171,130.67){\rule{0.241pt}{0.400pt}}
\multiput(171.00,131.17)(0.500,-1.000){2}{\rule{0.120pt}{0.400pt}}
\put(171.0,132.0){\usebox{\plotpoint}}
\put(172.0,131.0){\usebox{\plotpoint}}
\put(173.0,130.0){\usebox{\plotpoint}}
\put(174,128.67){\rule{0.241pt}{0.400pt}}
\multiput(174.00,129.17)(0.500,-1.000){2}{\rule{0.120pt}{0.400pt}}
\put(173.0,130.0){\usebox{\plotpoint}}
\put(175,129){\usebox{\plotpoint}}
\put(175,127.67){\rule{0.241pt}{0.400pt}}
\multiput(175.00,128.17)(0.500,-1.000){2}{\rule{0.120pt}{0.400pt}}
\put(177,126.67){\rule{0.241pt}{0.400pt}}
\multiput(177.00,127.17)(0.500,-1.000){2}{\rule{0.120pt}{0.400pt}}
\put(176.0,128.0){\usebox{\plotpoint}}
\put(179,125.67){\rule{0.241pt}{0.400pt}}
\multiput(179.00,126.17)(0.500,-1.000){2}{\rule{0.120pt}{0.400pt}}
\put(178.0,127.0){\usebox{\plotpoint}}
\put(181,124.67){\rule{0.241pt}{0.400pt}}
\multiput(181.00,125.17)(0.500,-1.000){2}{\rule{0.120pt}{0.400pt}}
\put(180.0,126.0){\usebox{\plotpoint}}
\put(182.0,125.0){\rule[-0.200pt]{0.723pt}{0.400pt}}
\put(185.0,124.0){\usebox{\plotpoint}}
\put(189,122.67){\rule{0.241pt}{0.400pt}}
\multiput(189.00,123.17)(0.500,-1.000){2}{\rule{0.120pt}{0.400pt}}
\put(185.0,124.0){\rule[-0.200pt]{0.964pt}{0.400pt}}
\put(198,121.67){\rule{0.241pt}{0.400pt}}
\multiput(198.00,122.17)(0.500,-1.000){2}{\rule{0.120pt}{0.400pt}}
\put(190.0,123.0){\rule[-0.200pt]{1.927pt}{0.400pt}}
\put(199,122){\usebox{\plotpoint}}
\put(199.0,122.0){\rule[-0.200pt]{81.906pt}{0.400pt}}
\put(379,239){\makebox(0,0)[r]{$D = 1.0$}}
\multiput(399,239)(20.756,0.000){5}{\usebox{\plotpoint}}
\put(499,239){\usebox{\plotpoint}}
\put(140,320){\usebox{\plotpoint}}
\put(140.00,320.00){\usebox{\plotpoint}}
\put(141.00,299.28){\usebox{\plotpoint}}
\put(142.22,278.58){\usebox{\plotpoint}}
\put(143.35,257.88){\usebox{\plotpoint}}
\put(145.00,237.22){\usebox{\plotpoint}}
\put(147.08,216.62){\usebox{\plotpoint}}
\put(150.68,196.27){\usebox{\plotpoint}}
\put(154.94,176.19){\usebox{\plotpoint}}
\put(161.43,157.14){\usebox{\plotpoint}}
\put(172.24,141.76){\usebox{\plotpoint}}
\put(186.79,131.00){\usebox{\plotpoint}}
\put(204.63,125.37){\usebox{\plotpoint}}
\put(223.81,123.00){\usebox{\plotpoint}}
\put(243.57,122.00){\usebox{\plotpoint}}
\put(264.32,122.00){\usebox{\plotpoint}}
\put(285.08,122.00){\usebox{\plotpoint}}
\put(305.84,122.00){\usebox{\plotpoint}}
\put(326.59,122.00){\usebox{\plotpoint}}
\put(347.35,122.00){\usebox{\plotpoint}}
\put(368.10,122.00){\usebox{\plotpoint}}
\put(388.86,122.00){\usebox{\plotpoint}}
\put(409.61,122.00){\usebox{\plotpoint}}
\put(430.37,122.00){\usebox{\plotpoint}}
\put(451.12,122.00){\usebox{\plotpoint}}
\put(471.88,122.00){\usebox{\plotpoint}}
\put(492.63,122.00){\usebox{\plotpoint}}
\put(513.39,122.00){\usebox{\plotpoint}}
\put(534.15,122.00){\usebox{\plotpoint}}
\put(539,122){\usebox{\plotpoint}}
\sbox{\plotpoint}{\rule[-0.400pt]{0.800pt}{0.800pt}}%
\put(379,198){\makebox(0,0)[r]{$D = 0.8$}}
\put(399.0,198.0){\rule[-0.400pt]{24.090pt}{0.800pt}}
\put(140,320){\usebox{\plotpoint}}
\put(138.84,318){\rule{0.800pt}{0.482pt}}
\multiput(138.34,319.00)(1.000,-1.000){2}{\rule{0.800pt}{0.241pt}}
\put(139.84,314){\rule{0.800pt}{0.482pt}}
\multiput(139.34,315.00)(1.000,-1.000){2}{\rule{0.800pt}{0.241pt}}
\put(140.84,312){\rule{0.800pt}{0.482pt}}
\multiput(140.34,313.00)(1.000,-1.000){2}{\rule{0.800pt}{0.241pt}}
\put(141.0,316.0){\usebox{\plotpoint}}
\put(141.84,308){\rule{0.800pt}{0.482pt}}
\multiput(141.34,309.00)(1.000,-1.000){2}{\rule{0.800pt}{0.241pt}}
\put(142.84,306){\rule{0.800pt}{0.482pt}}
\multiput(142.34,307.00)(1.000,-1.000){2}{\rule{0.800pt}{0.241pt}}
\put(143.0,310.0){\usebox{\plotpoint}}
\put(143.84,302){\rule{0.800pt}{0.482pt}}
\multiput(143.34,303.00)(1.000,-1.000){2}{\rule{0.800pt}{0.241pt}}
\put(144.84,300){\rule{0.800pt}{0.482pt}}
\multiput(144.34,301.00)(1.000,-1.000){2}{\rule{0.800pt}{0.241pt}}
\put(145.0,304.0){\usebox{\plotpoint}}
\put(145.84,296){\rule{0.800pt}{0.482pt}}
\multiput(145.34,297.00)(1.000,-1.000){2}{\rule{0.800pt}{0.241pt}}
\put(146.84,294){\rule{0.800pt}{0.482pt}}
\multiput(146.34,295.00)(1.000,-1.000){2}{\rule{0.800pt}{0.241pt}}
\put(147.0,298.0){\usebox{\plotpoint}}
\put(147.84,290){\rule{0.800pt}{0.482pt}}
\multiput(147.34,291.00)(1.000,-1.000){2}{\rule{0.800pt}{0.241pt}}
\put(150,287.84){\rule{0.241pt}{0.800pt}}
\multiput(150.00,288.34)(0.500,-1.000){2}{\rule{0.120pt}{0.800pt}}
\put(149.0,292.0){\usebox{\plotpoint}}
\put(149.84,285){\rule{0.800pt}{0.482pt}}
\multiput(149.34,286.00)(1.000,-1.000){2}{\rule{0.800pt}{0.241pt}}
\put(150.84,283){\rule{0.800pt}{0.482pt}}
\multiput(150.34,284.00)(1.000,-1.000){2}{\rule{0.800pt}{0.241pt}}
\put(151.0,287.0){\usebox{\plotpoint}}
\put(151.84,279){\rule{0.800pt}{0.482pt}}
\multiput(151.34,280.00)(1.000,-1.000){2}{\rule{0.800pt}{0.241pt}}
\put(152.84,277){\rule{0.800pt}{0.482pt}}
\multiput(152.34,278.00)(1.000,-1.000){2}{\rule{0.800pt}{0.241pt}}
\put(153.0,281.0){\usebox{\plotpoint}}
\put(153.84,273){\rule{0.800pt}{0.482pt}}
\multiput(153.34,274.00)(1.000,-1.000){2}{\rule{0.800pt}{0.241pt}}
\put(154.84,271){\rule{0.800pt}{0.482pt}}
\multiput(154.34,272.00)(1.000,-1.000){2}{\rule{0.800pt}{0.241pt}}
\put(155.0,275.0){\usebox{\plotpoint}}
\put(157,266.84){\rule{0.241pt}{0.800pt}}
\multiput(157.00,267.34)(0.500,-1.000){2}{\rule{0.120pt}{0.800pt}}
\put(156.84,266){\rule{0.800pt}{0.482pt}}
\multiput(156.34,267.00)(1.000,-1.000){2}{\rule{0.800pt}{0.241pt}}
\put(157.0,269.0){\usebox{\plotpoint}}
\put(157.84,262){\rule{0.800pt}{0.482pt}}
\multiput(157.34,263.00)(1.000,-1.000){2}{\rule{0.800pt}{0.241pt}}
\put(158.84,260){\rule{0.800pt}{0.482pt}}
\multiput(158.34,261.00)(1.000,-1.000){2}{\rule{0.800pt}{0.241pt}}
\put(159.0,264.0){\usebox{\plotpoint}}
\put(161,255.84){\rule{0.241pt}{0.800pt}}
\multiput(161.00,256.34)(0.500,-1.000){2}{\rule{0.120pt}{0.800pt}}
\put(160.84,255){\rule{0.800pt}{0.482pt}}
\multiput(160.34,256.00)(1.000,-1.000){2}{\rule{0.800pt}{0.241pt}}
\put(161.0,258.0){\usebox{\plotpoint}}
\put(161.84,251){\rule{0.800pt}{0.482pt}}
\multiput(161.34,252.00)(1.000,-1.000){2}{\rule{0.800pt}{0.241pt}}
\put(162.84,249){\rule{0.800pt}{0.482pt}}
\multiput(162.34,250.00)(1.000,-1.000){2}{\rule{0.800pt}{0.241pt}}
\put(163.0,253.0){\usebox{\plotpoint}}
\put(163.84,246){\rule{0.800pt}{0.482pt}}
\multiput(163.34,247.00)(1.000,-1.000){2}{\rule{0.800pt}{0.241pt}}
\put(164.84,244){\rule{0.800pt}{0.482pt}}
\multiput(164.34,245.00)(1.000,-1.000){2}{\rule{0.800pt}{0.241pt}}
\put(165.0,248.0){\usebox{\plotpoint}}
\put(167,239.84){\rule{0.241pt}{0.800pt}}
\multiput(167.00,240.34)(0.500,-1.000){2}{\rule{0.120pt}{0.800pt}}
\put(166.84,239){\rule{0.800pt}{0.482pt}}
\multiput(166.34,240.00)(1.000,-1.000){2}{\rule{0.800pt}{0.241pt}}
\put(167.0,242.0){\usebox{\plotpoint}}
\put(169,234.84){\rule{0.241pt}{0.800pt}}
\multiput(169.00,235.34)(0.500,-1.000){2}{\rule{0.120pt}{0.800pt}}
\put(168.84,234){\rule{0.800pt}{0.482pt}}
\multiput(168.34,235.00)(1.000,-1.000){2}{\rule{0.800pt}{0.241pt}}
\put(169.0,237.0){\usebox{\plotpoint}}
\put(171,229.84){\rule{0.241pt}{0.800pt}}
\multiput(171.00,230.34)(0.500,-1.000){2}{\rule{0.120pt}{0.800pt}}
\put(170.84,229){\rule{0.800pt}{0.482pt}}
\multiput(170.34,230.00)(1.000,-1.000){2}{\rule{0.800pt}{0.241pt}}
\put(171.0,232.0){\usebox{\plotpoint}}
\put(173,224.84){\rule{0.241pt}{0.800pt}}
\multiput(173.00,225.34)(0.500,-1.000){2}{\rule{0.120pt}{0.800pt}}
\put(172.84,224){\rule{0.800pt}{0.482pt}}
\multiput(172.34,225.00)(1.000,-1.000){2}{\rule{0.800pt}{0.241pt}}
\put(173.0,227.0){\usebox{\plotpoint}}
\put(175,219.84){\rule{0.241pt}{0.800pt}}
\multiput(175.00,220.34)(0.500,-1.000){2}{\rule{0.120pt}{0.800pt}}
\put(174.84,219){\rule{0.800pt}{0.482pt}}
\multiput(174.34,220.00)(1.000,-1.000){2}{\rule{0.800pt}{0.241pt}}
\put(175.0,222.0){\usebox{\plotpoint}}
\put(175.84,216){\rule{0.800pt}{0.482pt}}
\multiput(175.34,217.00)(1.000,-1.000){2}{\rule{0.800pt}{0.241pt}}
\put(178,213.84){\rule{0.241pt}{0.800pt}}
\multiput(178.00,214.34)(0.500,-1.000){2}{\rule{0.120pt}{0.800pt}}
\put(177.0,218.0){\usebox{\plotpoint}}
\put(179,210.84){\rule{0.241pt}{0.800pt}}
\multiput(179.00,211.34)(0.500,-1.000){2}{\rule{0.120pt}{0.800pt}}
\put(178.84,210){\rule{0.800pt}{0.482pt}}
\multiput(178.34,211.00)(1.000,-1.000){2}{\rule{0.800pt}{0.241pt}}
\put(179.0,213.0){\usebox{\plotpoint}}
\put(179.84,207){\rule{0.800pt}{0.482pt}}
\multiput(179.34,208.00)(1.000,-1.000){2}{\rule{0.800pt}{0.241pt}}
\put(182,204.84){\rule{0.241pt}{0.800pt}}
\multiput(182.00,205.34)(0.500,-1.000){2}{\rule{0.120pt}{0.800pt}}
\put(181.0,209.0){\usebox{\plotpoint}}
\put(183,201.84){\rule{0.241pt}{0.800pt}}
\multiput(183.00,202.34)(0.500,-1.000){2}{\rule{0.120pt}{0.800pt}}
\put(182.84,201){\rule{0.800pt}{0.482pt}}
\multiput(182.34,202.00)(1.000,-1.000){2}{\rule{0.800pt}{0.241pt}}
\put(183.0,204.0){\usebox{\plotpoint}}
\put(183.84,198){\rule{0.800pt}{0.482pt}}
\multiput(183.34,199.00)(1.000,-1.000){2}{\rule{0.800pt}{0.241pt}}
\put(186,195.84){\rule{0.241pt}{0.800pt}}
\multiput(186.00,196.34)(0.500,-1.000){2}{\rule{0.120pt}{0.800pt}}
\put(185.0,200.0){\usebox{\plotpoint}}
\put(185.84,194){\rule{0.800pt}{0.482pt}}
\multiput(185.34,195.00)(1.000,-1.000){2}{\rule{0.800pt}{0.241pt}}
\put(188,191.84){\rule{0.241pt}{0.800pt}}
\multiput(188.00,192.34)(0.500,-1.000){2}{\rule{0.120pt}{0.800pt}}
\put(187.0,196.0){\usebox{\plotpoint}}
\put(187.84,190){\rule{0.800pt}{0.482pt}}
\multiput(187.34,191.00)(1.000,-1.000){2}{\rule{0.800pt}{0.241pt}}
\put(190,187.84){\rule{0.241pt}{0.800pt}}
\multiput(190.00,188.34)(0.500,-1.000){2}{\rule{0.120pt}{0.800pt}}
\put(189.0,192.0){\usebox{\plotpoint}}
\put(189.84,186){\rule{0.800pt}{0.482pt}}
\multiput(189.34,187.00)(1.000,-1.000){2}{\rule{0.800pt}{0.241pt}}
\put(192,183.84){\rule{0.241pt}{0.800pt}}
\multiput(192.00,184.34)(0.500,-1.000){2}{\rule{0.120pt}{0.800pt}}
\put(191.0,188.0){\usebox{\plotpoint}}
\put(193,181.84){\rule{0.241pt}{0.800pt}}
\multiput(193.00,182.34)(0.500,-1.000){2}{\rule{0.120pt}{0.800pt}}
\put(192.84,181){\rule{0.800pt}{0.482pt}}
\multiput(192.34,182.00)(1.000,-1.000){2}{\rule{0.800pt}{0.241pt}}
\put(193.0,184.0){\usebox{\plotpoint}}
\put(195,177.84){\rule{0.241pt}{0.800pt}}
\multiput(195.00,178.34)(0.500,-1.000){2}{\rule{0.120pt}{0.800pt}}
\put(196,176.84){\rule{0.241pt}{0.800pt}}
\multiput(196.00,177.34)(0.500,-1.000){2}{\rule{0.120pt}{0.800pt}}
\put(195.0,180.0){\usebox{\plotpoint}}
\put(197,173.84){\rule{0.241pt}{0.800pt}}
\multiput(197.00,174.34)(0.500,-1.000){2}{\rule{0.120pt}{0.800pt}}
\put(198,172.84){\rule{0.241pt}{0.800pt}}
\multiput(198.00,173.34)(0.500,-1.000){2}{\rule{0.120pt}{0.800pt}}
\put(197.0,176.0){\usebox{\plotpoint}}
\put(199,170.84){\rule{0.241pt}{0.800pt}}
\multiput(199.00,171.34)(0.500,-1.000){2}{\rule{0.120pt}{0.800pt}}
\put(200,169.84){\rule{0.241pt}{0.800pt}}
\multiput(200.00,170.34)(0.500,-1.000){2}{\rule{0.120pt}{0.800pt}}
\put(199.0,173.0){\usebox{\plotpoint}}
\put(199.84,168){\rule{0.800pt}{0.482pt}}
\multiput(199.34,169.00)(1.000,-1.000){2}{\rule{0.800pt}{0.241pt}}
\put(202,165.84){\rule{0.241pt}{0.800pt}}
\multiput(202.00,166.34)(0.500,-1.000){2}{\rule{0.120pt}{0.800pt}}
\put(201.0,170.0){\usebox{\plotpoint}}
\put(203,163.84){\rule{0.241pt}{0.800pt}}
\multiput(203.00,164.34)(0.500,-1.000){2}{\rule{0.120pt}{0.800pt}}
\put(204,162.84){\rule{0.241pt}{0.800pt}}
\multiput(204.00,163.34)(0.500,-1.000){2}{\rule{0.120pt}{0.800pt}}
\put(203.0,166.0){\usebox{\plotpoint}}
\put(205,160.84){\rule{0.241pt}{0.800pt}}
\multiput(205.00,161.34)(0.500,-1.000){2}{\rule{0.120pt}{0.800pt}}
\put(206,159.84){\rule{0.241pt}{0.800pt}}
\multiput(206.00,160.34)(0.500,-1.000){2}{\rule{0.120pt}{0.800pt}}
\put(205.0,163.0){\usebox{\plotpoint}}
\put(207,157.84){\rule{0.241pt}{0.800pt}}
\multiput(207.00,158.34)(0.500,-1.000){2}{\rule{0.120pt}{0.800pt}}
\put(207.0,160.0){\usebox{\plotpoint}}
\put(208,155.84){\rule{0.241pt}{0.800pt}}
\multiput(208.00,156.34)(0.500,-1.000){2}{\rule{0.120pt}{0.800pt}}
\put(209,154.84){\rule{0.241pt}{0.800pt}}
\multiput(209.00,155.34)(0.500,-1.000){2}{\rule{0.120pt}{0.800pt}}
\put(208.0,158.0){\usebox{\plotpoint}}
\put(210,152.84){\rule{0.241pt}{0.800pt}}
\multiput(210.00,153.34)(0.500,-1.000){2}{\rule{0.120pt}{0.800pt}}
\put(211,151.84){\rule{0.241pt}{0.800pt}}
\multiput(211.00,152.34)(0.500,-1.000){2}{\rule{0.120pt}{0.800pt}}
\put(210.0,155.0){\usebox{\plotpoint}}
\put(212,153){\usebox{\plotpoint}}
\put(212,150.84){\rule{0.241pt}{0.800pt}}
\multiput(212.00,151.34)(0.500,-1.000){2}{\rule{0.120pt}{0.800pt}}
\put(213,149.84){\rule{0.241pt}{0.800pt}}
\multiput(213.00,150.34)(0.500,-1.000){2}{\rule{0.120pt}{0.800pt}}
\put(214,147.84){\rule{0.241pt}{0.800pt}}
\multiput(214.00,148.34)(0.500,-1.000){2}{\rule{0.120pt}{0.800pt}}
\put(215,146.84){\rule{0.241pt}{0.800pt}}
\multiput(215.00,147.34)(0.500,-1.000){2}{\rule{0.120pt}{0.800pt}}
\put(214.0,150.0){\usebox{\plotpoint}}
\put(216,144.84){\rule{0.241pt}{0.800pt}}
\multiput(216.00,145.34)(0.500,-1.000){2}{\rule{0.120pt}{0.800pt}}
\put(216.0,147.0){\usebox{\plotpoint}}
\put(217.0,146.0){\usebox{\plotpoint}}
\put(218,142.84){\rule{0.241pt}{0.800pt}}
\multiput(218.00,143.34)(0.500,-1.000){2}{\rule{0.120pt}{0.800pt}}
\put(219,141.84){\rule{0.241pt}{0.800pt}}
\multiput(219.00,142.34)(0.500,-1.000){2}{\rule{0.120pt}{0.800pt}}
\put(218.0,145.0){\usebox{\plotpoint}}
\put(220,143){\usebox{\plotpoint}}
\put(220,140.84){\rule{0.241pt}{0.800pt}}
\multiput(220.00,141.34)(0.500,-1.000){2}{\rule{0.120pt}{0.800pt}}
\put(221,139.84){\rule{0.241pt}{0.800pt}}
\multiput(221.00,140.34)(0.500,-1.000){2}{\rule{0.120pt}{0.800pt}}
\put(222.0,140.0){\usebox{\plotpoint}}
\put(223,137.84){\rule{0.241pt}{0.800pt}}
\multiput(223.00,138.34)(0.500,-1.000){2}{\rule{0.120pt}{0.800pt}}
\put(222.0,140.0){\usebox{\plotpoint}}
\put(224,135.84){\rule{0.241pt}{0.800pt}}
\multiput(224.00,136.34)(0.500,-1.000){2}{\rule{0.120pt}{0.800pt}}
\put(224.0,138.0){\usebox{\plotpoint}}
\put(225.0,137.0){\usebox{\plotpoint}}
\put(226,133.84){\rule{0.241pt}{0.800pt}}
\multiput(226.00,134.34)(0.500,-1.000){2}{\rule{0.120pt}{0.800pt}}
\put(226.0,136.0){\usebox{\plotpoint}}
\put(227.0,135.0){\usebox{\plotpoint}}
\put(228.0,134.0){\usebox{\plotpoint}}
\put(229,131.84){\rule{0.241pt}{0.800pt}}
\multiput(229.00,132.34)(0.500,-1.000){2}{\rule{0.120pt}{0.800pt}}
\put(228.0,134.0){\usebox{\plotpoint}}
\put(230.0,132.0){\usebox{\plotpoint}}
\put(231,129.84){\rule{0.241pt}{0.800pt}}
\multiput(231.00,130.34)(0.500,-1.000){2}{\rule{0.120pt}{0.800pt}}
\put(230.0,132.0){\usebox{\plotpoint}}
\put(232,131){\usebox{\plotpoint}}
\put(232,128.84){\rule{0.241pt}{0.800pt}}
\multiput(232.00,129.34)(0.500,-1.000){2}{\rule{0.120pt}{0.800pt}}
\put(233,127.84){\rule{0.241pt}{0.800pt}}
\multiput(233.00,128.34)(0.500,-1.000){2}{\rule{0.120pt}{0.800pt}}
\put(234,129){\usebox{\plotpoint}}
\put(234,126.84){\rule{0.241pt}{0.800pt}}
\multiput(234.00,127.34)(0.500,-1.000){2}{\rule{0.120pt}{0.800pt}}
\put(235.0,128.0){\usebox{\plotpoint}}
\put(236.0,127.0){\usebox{\plotpoint}}
\put(237,124.84){\rule{0.241pt}{0.800pt}}
\multiput(237.00,125.34)(0.500,-1.000){2}{\rule{0.120pt}{0.800pt}}
\put(236.0,127.0){\usebox{\plotpoint}}
\put(238,126){\usebox{\plotpoint}}
\put(238,123.84){\rule{0.241pt}{0.800pt}}
\multiput(238.00,124.34)(0.500,-1.000){2}{\rule{0.120pt}{0.800pt}}
\put(239.0,125.0){\usebox{\plotpoint}}
\put(240.0,124.0){\usebox{\plotpoint}}
\put(241,121.84){\rule{0.241pt}{0.800pt}}
\multiput(241.00,122.34)(0.500,-1.000){2}{\rule{0.120pt}{0.800pt}}
\put(240.0,124.0){\usebox{\plotpoint}}
\put(242,123){\usebox{\plotpoint}}
\put(243,120.84){\rule{0.241pt}{0.800pt}}
\multiput(243.00,121.34)(0.500,-1.000){2}{\rule{0.120pt}{0.800pt}}
\put(242.0,123.0){\usebox{\plotpoint}}
\put(244,122){\usebox{\plotpoint}}
\put(244,119.84){\rule{0.241pt}{0.800pt}}
\multiput(244.00,120.34)(0.500,-1.000){2}{\rule{0.120pt}{0.800pt}}
\put(245.0,121.0){\usebox{\plotpoint}}
\put(246.0,120.0){\usebox{\plotpoint}}
\put(246.0,120.0){\usebox{\plotpoint}}
\put(248.0,119.0){\usebox{\plotpoint}}
\put(248.0,119.0){\usebox{\plotpoint}}
\put(250.0,118.0){\usebox{\plotpoint}}
\put(250.0,118.0){\usebox{\plotpoint}}
\put(252.0,117.0){\usebox{\plotpoint}}
\put(252.0,117.0){\usebox{\plotpoint}}
\put(254.0,116.0){\usebox{\plotpoint}}
\put(256,113.84){\rule{0.241pt}{0.800pt}}
\multiput(256.00,114.34)(0.500,-1.000){2}{\rule{0.120pt}{0.800pt}}
\put(254.0,116.0){\usebox{\plotpoint}}
\put(258,112.84){\rule{0.241pt}{0.800pt}}
\multiput(258.00,113.34)(0.500,-1.000){2}{\rule{0.120pt}{0.800pt}}
\put(257.0,115.0){\usebox{\plotpoint}}
\put(259.0,114.0){\usebox{\plotpoint}}
\put(262.0,113.0){\usebox{\plotpoint}}
\put(264,110.84){\rule{0.241pt}{0.800pt}}
\multiput(264.00,111.34)(0.500,-1.000){2}{\rule{0.120pt}{0.800pt}}
\put(262.0,113.0){\usebox{\plotpoint}}
\put(268,109.84){\rule{0.241pt}{0.800pt}}
\multiput(268.00,110.34)(0.500,-1.000){2}{\rule{0.120pt}{0.800pt}}
\put(265.0,112.0){\usebox{\plotpoint}}
\put(272,108.84){\rule{0.241pt}{0.800pt}}
\multiput(272.00,109.34)(0.500,-1.000){2}{\rule{0.120pt}{0.800pt}}
\put(269.0,111.0){\usebox{\plotpoint}}
\put(278,107.84){\rule{0.241pt}{0.800pt}}
\multiput(278.00,108.34)(0.500,-1.000){2}{\rule{0.120pt}{0.800pt}}
\put(273.0,110.0){\rule[-0.400pt]{1.204pt}{0.800pt}}
\put(290,106.84){\rule{0.241pt}{0.800pt}}
\multiput(290.00,107.34)(0.500,-1.000){2}{\rule{0.120pt}{0.800pt}}
\put(279.0,109.0){\rule[-0.400pt]{2.650pt}{0.800pt}}
\put(303,106.84){\rule{0.241pt}{0.800pt}}
\multiput(303.00,106.34)(0.500,1.000){2}{\rule{0.120pt}{0.800pt}}
\put(291.0,108.0){\rule[-0.400pt]{2.891pt}{0.800pt}}
\put(304,109){\usebox{\plotpoint}}
\put(318,107.84){\rule{0.241pt}{0.800pt}}
\multiput(318.00,107.34)(0.500,1.000){2}{\rule{0.120pt}{0.800pt}}
\put(304.0,109.0){\rule[-0.400pt]{3.373pt}{0.800pt}}
\put(319.0,110.0){\rule[-0.400pt]{2.168pt}{0.800pt}}
\put(328.0,110.0){\usebox{\plotpoint}}
\put(336,109.84){\rule{0.241pt}{0.800pt}}
\multiput(336.00,109.34)(0.500,1.000){2}{\rule{0.120pt}{0.800pt}}
\put(328.0,111.0){\rule[-0.400pt]{1.927pt}{0.800pt}}
\put(344,110.84){\rule{0.241pt}{0.800pt}}
\multiput(344.00,110.34)(0.500,1.000){2}{\rule{0.120pt}{0.800pt}}
\put(337.0,112.0){\rule[-0.400pt]{1.686pt}{0.800pt}}
\put(345,113){\usebox{\plotpoint}}
\put(351,111.84){\rule{0.241pt}{0.800pt}}
\multiput(351.00,111.34)(0.500,1.000){2}{\rule{0.120pt}{0.800pt}}
\put(345.0,113.0){\rule[-0.400pt]{1.445pt}{0.800pt}}
\put(359,112.84){\rule{0.241pt}{0.800pt}}
\multiput(359.00,112.34)(0.500,1.000){2}{\rule{0.120pt}{0.800pt}}
\put(352.0,114.0){\rule[-0.400pt]{1.686pt}{0.800pt}}
\put(367,113.84){\rule{0.241pt}{0.800pt}}
\multiput(367.00,113.34)(0.500,1.000){2}{\rule{0.120pt}{0.800pt}}
\put(360.0,115.0){\rule[-0.400pt]{1.686pt}{0.800pt}}
\put(376,114.84){\rule{0.241pt}{0.800pt}}
\multiput(376.00,114.34)(0.500,1.000){2}{\rule{0.120pt}{0.800pt}}
\put(368.0,116.0){\rule[-0.400pt]{1.927pt}{0.800pt}}
\put(377,117){\usebox{\plotpoint}}
\put(385,115.84){\rule{0.241pt}{0.800pt}}
\multiput(385.00,115.34)(0.500,1.000){2}{\rule{0.120pt}{0.800pt}}
\put(377.0,117.0){\rule[-0.400pt]{1.927pt}{0.800pt}}
\put(396,116.84){\rule{0.241pt}{0.800pt}}
\multiput(396.00,116.34)(0.500,1.000){2}{\rule{0.120pt}{0.800pt}}
\put(386.0,118.0){\rule[-0.400pt]{2.409pt}{0.800pt}}
\put(397,119){\usebox{\plotpoint}}
\put(397.0,119.0){\rule[-0.400pt]{2.891pt}{0.800pt}}
\put(409.0,119.0){\usebox{\plotpoint}}
\put(424,118.84){\rule{0.241pt}{0.800pt}}
\multiput(424.00,118.34)(0.500,1.000){2}{\rule{0.120pt}{0.800pt}}
\put(409.0,120.0){\rule[-0.400pt]{3.613pt}{0.800pt}}
\put(425,121){\usebox{\plotpoint}}
\put(447,119.84){\rule{0.241pt}{0.800pt}}
\multiput(447.00,119.34)(0.500,1.000){2}{\rule{0.120pt}{0.800pt}}
\put(425.0,121.0){\rule[-0.400pt]{5.300pt}{0.800pt}}
\put(448.0,122.0){\rule[-0.400pt]{21.922pt}{0.800pt}}
\end{picture}
\normalsize
  }
  \Formelbox{
    \[ y = b_1 e^{\lambda_1 t} + b_2 e^{\lambda_2 t}  \]
    \[ \lambda_1 = - \omega _0 ( D + \sqrt{D^2 -1 } ) \]
    \[ \lambda_2 = - \omega _0 ( D - \sqrt{D^2 -1 } ) \]
    \footnotesize Grenzfall $D = 1$\normalsize
    \[ \frac{c}{m} = \frac {b^2}{4 m^2} \]
    \[y = ( b_1 + b_2 t ) e ^{- \delta t} \]
    }
  }
  {\Groessenbox{
     $ y$                & schwingende Gr�sse                    & $ [m] $ \\
     $ \omega$           & Kreisfre"-quenz                       & $ [\frac{1}{s}] $    \\
     $\delta$            & Abkling"-konst.                       & $ [1] $ \\
     $b$                 & D�mpfungs"-konst.                     & $ [ \frac{kg}{s}] $ \\
     $D$                 & D�mpfungs"-grad                       & $ [ 1 ] $ \\
     $m$                 & Masse                                 & $ [ kg ] $ \\
     $c$                 & Federkonstante                        & $ [ \frac{N}{m} ] $ \\
}}

\subsection{Elektrischer Schwingkreis}

\index{Elektrischer Schwingkreis}
\index{Schwingungen!Elektrischer Schwingkreis}

\Hauptbox{
  \Bildbox{
    \input{Physik/Schwingungen/elsk.pstex_t}
  }
  \Formelbox{
    \[ I = I_0 e^{-\delta t} \sin(\omega_d t + \phi_0) \]
    \[ \delta = \frac{R}{2L} \]
    \[ \omega_d = \omega_0 \sqrt{1 -D^2} \]
    \[ \omega _0 = \frac{1}{\sqrt{LC}} \]
    \[ D = \frac{R}{2} \sqrt{\frac{C}{L}} \]
    \[ \omega _d = \frac{1}{\sqrt{LC}} \sqrt{1- \frac{R^2 C}{4 L}} \]
    }
  }
  {\Groessenbox{
     $ I$                & Strom                                 & $ [A] $ \\
     $ R$                & Widerstand                            & $ [\Omega ]$ \\
     $ L$                & Induktivit�t                          & $ [ H ] $ \\
     $ \omega$           & Kreisfre"-quenz                       & $ [\frac{1}{s}] $    \\
     $\delta$            & Abkling"-konst.                       & $ [1] $ \\
     $t$                 & Zeit                                  & $ [s] $ \\
     $D$                 & D�mpfungs"-grad                       & $ [ 1 ] $ \\ 
}}
\vfill



%%% Local Variables: 
%%% mode: latex
%%% TeX-master: "../../FoSaHSR"
%%% End: 
 