%==========================================================
% Titel:    Formelsammlung-HSR
% Autor:    Thomas K�ng, Adrian Freihofer
% Erstellt: 7.11.2001
% Ge�ndert:
%==========================================================

\index{Vorwort}

\begin{titlepage}
\title{Formelsammlung HSR}
\author{Thomas K�ng \and Urs Winiger \and Adrian Freihofer}
\date{Version 1.0.2 \\ \today}
\end{titlepage}

\maketitle
%@Book{,
%ALTauthor = {Adrian Freihofer, Thomas K�ng, Urs Winiger},
%ALTeditor = {},
%title = {Formelsammlung f�r Elektrotechnik},
%publisher = {},
%year = {2003},
%OPTkey = {},
%OPTvolume = {1},
%OPTnumber = {},
%OPTseries = {},
%OPTaddress = {},
%OPTedition = {},
%OPTmonth = {},
%OPTnote = {},
%OPTannote = {}
%}

\chapter{Vorwort}
Die vorliegende Formelsammlung wurde w�hrend unserem Studium f�r Elektrotechnik (2000 --- 2003) an der Fachhochschule in Rapperswil mit \LaTeX\ erstelltgeschrieben. Ziel war es, den Inhalt an den Pr�ffungsstoff anzupassen, aber auch ein Werk zu schreiben, das wir sp�ter im Berufsleben verwenden k�nnen. Obwohl wir den Inhalt sorgf�lltig zusammengestellt haben, sind Fehler nicht ausschliessbar. Als wir von einer Studentin im Jahr 2024 darauf angesprochen wurden, haben wir uns dazu entschlossen diese Formelsammlung als GitHub Repository unter \url{https://www.github.com/tmkueng/Formelsammlung} unter Einhaltung der GNU Public Licence zur allgemeinen Verf�gung zu stellen. Wir sind nat�rlich sehr stolz, dass die Formelsammlung nach �ber 20 Jahren immer noch sehr gefragt ist und hoffen auf viele neue Inhalte welche durch die Gemeinschaft kreiert werden. 
\\
\\
In der Formelsammlung sind die folgenden F�cher enthalten:
\begin{itemize}
\item Physik
\item Elektrizit�tslehre
\item Energie und Antriebstechnik 
\item Elektronik
\item Digitale Signalverarbeitung
\item Mathematik
\end{itemize} 

%\includegraphics{HSR-logo.eps}

%%% Local Variables: 
%%% mode: latex
%%% TeX-master: "FoSaHSR"
%%% End: 
