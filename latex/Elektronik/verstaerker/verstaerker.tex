%%% Local Variables: 
%%% mode: latex
%%% TeX-master: "../../FoSaHSR"
%%% End: 
%==========================================================
% Titel:    Elektronik, Verst�rker
% Autor:    Adrian Freihofer
% Erstellt: 19.03.2002
% Ge�ndert:


\chapter{Gegengekoppelte Verst�rker} \index{Verst�rker}
                                \index{Gegenkopplung}
				\index{R�ckkopplung}
				\index{Verst�rker!Gegenkopplung}
				\index{Verst�rker!R�ckkopplung}

\section{Mit- und Gegenkopplung} \index{Verst�rker!Mitkopplung}
				\index{Mitkopplung}

\Hauptbox{
  \Bildbox{
	\input{Elektronik/verstaerker/rueckGegenKoppl.pstex_t}
    }
  \Formelbox{
	Gegenkopplung:    
	\[ A_{CL} = \frac{U_{out}}{U_{in}} = \frac{A_o}{1 + k A_{o}} \]
	Mitkopplung:    
	\[ A_{CL} = \frac{U_{out}}{U_{in}} = \frac{A_o}{1 - k A_{o}} \]
    }} {
  \Groessenbox{
    $A_{CL}$   & Closed Loop Verst�rkung & $[1]$ \\
    $A_o$      & Open Loop Verst�rkung   & $[1]$\\
    $U$        & Spannung                & $[V]$ \\
    $k$        & Faktor                  & $[1]$ \\
}}

\subsection{Gegenkopplung beim OP} \index{Gegenkopplung}

\Hauptbox{
  \Bildbox{
	\input{Elektronik/verstaerker/gkOp.pstex_t}\\ \\
	\begin{small}Bodeplot:\end{small}\\ \\
	\input{Elektronik/verstaerker/gkBode.pstex_t}
    }
  \Formelbox{ 
	Ideal:
	\[ A_{CL} = \frac{n A_o}{1 + k A_{o}} \]
	\begin{footnotesize}Nicht invertierend:\end{footnotesize}
	\[ n=1 \]
	\[ | A_{CLideal}| = \frac{R_F + R_1}{R_1} = \frac{1}{k} \]
	\begin{footnotesize}Inveriterend:\end{footnotesize}
	\[ n = \frac{R_F}{R_1 + R_F} \, \mbox{\hspace{0.5cm}} k = \frac{R_1}{R_1+R_F} \]
	\[ |A_{CLideal}| = \frac{R_F}{R_1} \]
	Real:
	\[A_{CLreal} = n A_o \| A_{CLideal}  \]
    }} {
  \Groessenbox{
    $A_{CL}$   & Closed Loop Verst�rkung & $[1]$ \\
    $A_o$      & Open Loop Verst�rkung   & $[1]$\\
    $k$        & GK-Faktor               & $[1]$ \\
    $n$        & Faktor                  & $[1]$ \\
    $R$        & Widerstand              & $[\Omega]$ \\
}}
\vfill

\section{Gegenkopplungsarten}

\index{Gegenkopplungsarten}

\Hauptbox{
  \Bildbox{
	\begin{small}Serie-Parallel\end{small}\\ \\
	\input{Elektronik/verstaerker/spGk.pstex_t}
    }
  \Formelbox{ 
	\begin{footnotesize}Eingang: Seriell\\
	Ausgang: Parallel\end{footnotesize}
	\[ r_{iCL} \rightarrow \infty \mbox{\hspace{1cm}} r_{oCL} \rightarrow 0 \]
	\[ r_{iCL} = r_i(1+kA_o) \]
	\[ r_{oCL} = \frac{u_{out}}{i_{out}} = \frac{r_{o}}{1 + kA_o} \]
    }\hline
 \Bildbox{
	\begin{small}Parallel-Parallel\end{small}\\ \\
	\input{Elektronik/verstaerker/ppGk.pstex_t}
    }
  \Formelbox{ 
	\begin{footnotesize}Eingang: Parallel\\
	Ausgang: Parallel\end{footnotesize}
	\[ r_{iCL} \rightarrow 0 \mbox{\hspace{1cm}} r_{oCL} \rightarrow 0 \]
	\[ r_{iCL} = \frac{r_{i}}{1+kA_o}\]
	\[ r_{oCL} = \frac{u_{out}}{i_{out}} = \frac{r_o}{1 + kA_o} \]
    }\hline
 \Bildbox{
	\begin{small}Parallel-Serie\end{small}\\ \\
	\input{Elektronik/verstaerker/psGk.pstex_t}
    }
  \Formelbox{ 
	\begin{footnotesize}Eingang: Parallel\\
	Ausgang: Seriell\end{footnotesize}
	\[ r_{iCL} \rightarrow 0 \mbox{\hspace{1cm}} r_{oCL} \rightarrow \infty \]
	\[ r_{iCL} = \frac{r_{i}}{1+kA_o}\]
	\[ r_{oCL} = \frac{u_{out}}{i_{out}} = r_o(1+kA_o) \]
    }\hline
 \Bildbox{
	\begin{small}Serie-Serie\end{small}\\ \\
	\input{Elektronik/verstaerker/ssGk.pstex_t}
    }
  \Formelbox{ 
	\begin{footnotesize}Eingang: Seriell\\
	Ausgang: Seriell\end{footnotesize}
	\[ r_{iCL} \rightarrow \infty \mbox{\hspace{1cm}} r_{oCL} \rightarrow \infty \]
	\[ r_{iCL} = r_i(1+kA_o) \]
	\[ r_{oCL} = \frac{u_{out}}{i_{out}} = r_o(1+kA_o) \]
    }
} {
  \Groessenbox{
    $A_o$   & Open Loop Verst�rkung   & $[1]$ \\
    $k$       & Faktor                  & $[1]$ \\
    $r_i$     & Open Loop r-Eingang     & $[\Omega]$ \\
    $r_o$     & Open Loop r-Ausgang     & $[\Omega]$ \\
    $r_{iCL}$ & Closed Loop r-Eingang   & $[\Omega]$ \\
    $r_{oCL}$ & Closed Loop r-Ausgang   & $[\Omega]$ \\
    $u_{out}$ & u-Ausgang               & $[V]$ \\
    $i_{out}$ & i-Ausgang               & $[A]$\\
    $R_S$     & Quell-Widerst.          & $[\Omega]$ \\
    $R_L$     & Last-Widerst.           & $[\Omega]$ \\
}}
 \vfill 

\subsection{Bestimmung der Gegenkopplungsart}
\begin{enumerate}
\item Forw�rtspfad, R�ckw�rtspfad und Gegenkopplugsschleife einzeichnen.
\item Anzahl Inversionen im Vorw�rtspfad ($\Rightarrow $ Invertierend oder nicht invertierend) bzw. in der Schleife bestimmen ($\Rightarrow $ Gegenkopplung bei ungerade Anzahl bzw. Mittkopplung bei gerader Anzahl).
\item Knoten (out, in$+$ und in$-$ ) der �quivalenten OP-Schaltung bestimmen.
\item �quivalenten OP-Schaltung zeichnen.
\end{enumerate} 

\subsection{Eingangsschaltungen} \index{Verst�rker!Eingangschaltungen}
Eingangsschaltungen bei Serieschaltung (Spannungsaddition)\\ 
\input{Elektronik/verstaerker/eingSchalt.pstex_t} \\
Eingangsschaltungen bei Parallelschaltung von Verst�rkereingnag und Ausgang (Stromaddition)\\ 
\input{Elektronik/verstaerker/eingSchalt2.pstex_t}

\subsection{Ausgangsschaltungen}\index{Verst�rker!Ausgangschaltungen}
Ausgangsschaltungen bei Parallelschaltung von Last und Eingang (Spannungsabnahme am Ausgang)\\ 
\input{Elektronik/verstaerker/ausgSchalt.pstex_t} \\
Ausgangsschaltungen bei Serieschaltung von Last und Eingang (Stromabnahme am Ausgang)\\ 
\input{Elektronik/verstaerker/ausgSchalt2.pstex_t}
\vfill

\section{Schleifenverst�rkung}

\index{Schleifenverst�rkung}
\index{Verst�rker!Schleifenverst�rkung}

\Hauptbox{
  \Bildbox{
	\begin{small}Beispiel:\end{small} \\
	\input{Elektronik/verstaerker/schleifenVerst.pstex_t}
    }
  \Formelbox{    
	\[ A_L = k A_o = \frac{u_{xout}}{u_{xin}}\]
	\begin{small}Gegenkopplungsgrad:\end{small}
	\[ 1 + A_L = 1 + k A_L \]
	\begin{small}$U_{Bias} $ legt den Arbeitspunkt fest. \end{small} \\
	\begin{small}Es soll eine Trennstelle gew�hlt werden bei der $ r_{loop out} \gg r_{loop in} $ gilt $ \Rightarrow $ Belastung des Schleifenausganges kann vernachl�ssigt werden.\end{small} \vspace{0.1cm}	
    }} {
  \Groessenbox{
    $A_L$      & Schliefen- Verst�rkung & $[1]$ \\
    $A_o$      & Open Loop Verst�rkung   & $[1]$\\
    $U$        & Spannung                & $[V]$ \\
    $k$        & Faktor                  & $[1]$ \\
    $R$        & Widerstand              & $[\Omega]$ \\
}}


\section{Wirkung der GK auf die Sensivit�t der Verst�rkung}

\index{Sensivit�t}

\Hauptbox{
  \Bildbox{
	\begin{footnotesize}Die Sensitivit�t $ S_x^N $ ist ein Mass f�r die Empfindlichkeit einer Schaltungseigenschaft N gegen�ber Schwankungen eines Parameters x.\end{footnotesize}
    }
  \Formelbox{    
	\[ S_x^N = \frac{\frac{dN}{N}}{\frac{dx}{x}} \]
	\[ S_{A_o}^{A_{CL}} = \frac{\frac{dA_{CL}}{A_{CL}}}{\frac{dA_o}{A_o}} = \frac{A_o}{A_{CL}} \frac{dA_{CL}}{dA_o}\]
	\[ S_{A_o}^{A_{CL}} =\frac{1}{1 + k A_o} \]
	\vspace{0.3cm}
    }} {
  \Groessenbox{
    $ S $      & Sensitivit�t            & $[1]$\\
    $A_L$      & Schliefen- Verst�rkung  & $[1]$ \\
    $A_o$      & Open Loop Verst�rkung   & $[1]$\\   
    $k$        & Faktor                  & $[1]$ \\
    $x$        & ver�nderter Parameter   & $[...]$ \\
    $N$        & Beeinflusste Gr�sse     & $[...]$ \\
}}

\section{Das Verst�rkungs-Bandbreiten-Produkt}

\index{Verst�rkungs Bandbreiten Produkt}
\index{GBP}
\index{Verst�rker!GBP}
\index{Transitfrequenz}
\index{Verst�rker!Transitfrequenz}

\Hauptbox{
  \Bildbox{
	\begin{small}F�r alle Punkte die auf einer Amplitudengeraden mit einer Neignung von $ \pm 20 \frac{dB}{Dek} $ liegen gilt das Gesetz vom konstanten Verst�rkungs-Bandbreiten-Produkt. Siehe auch S. \pageref{GBP}\end{small} \\ \\
\input{Elektronik/verstaerker/gbp.pstex_t}
    }
  \Formelbox{    \label{GBPElektronik}
	\[ A f = f_T = GBP \]
	\[ A_1 f_1 = A_2 f_2 \]
	\[ A_{oDC} = f_o = GBP \]
    }} {
  \Groessenbox{
    $ f_T $    & Transitfrequenz $ = $ Amplitude $ \cap $ $ 0dB$-Achse      & $[\frac{1}{s}]$\\
    $f$        & Frequenz     & $[\frac{1}{s}]$ \\
    $A$        & Verst�rkung  & $[1]$ \\
    $A_{oDC}$  & Open-Loop DC-Gain & $[1]$ \\
}}
%==========================================================