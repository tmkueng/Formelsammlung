%==========================================================
% Titel:    Physik, Optik 
% Autor:    Adrian Freihofer
% Erstellt: 7.11.2001
% Ge�ndert:
%==========================================================

\chapter{Geometrische Optik}                               \index{Optik} \index{Geometrische Optik}

\section{Sichtbares Licht}
\begin{tabular}{|l|l|}
\hline Wellenbereich $ \lambda  / nm$ & Farbe \\ 
\hline 380 - 435 & violett \\ 
\hline 435 - 465 & blau \\ 
\hline 465 - 485 & blaugr�n \\ 
\hline 485 - 565 & gr�n \\ 
\hline 565 - 590 & gelb \\ 
\hline 590 - 630 & orange \\ 
\hline 630 - 780 & rot \\ 
\hline 
\end{tabular} 

\section{Reflexionsgesetz}                \index{Optik!Reflexionsgesetz}
                                          \index{Reflexionsgesetz}
 \Hauptbox{
  \Bildbox{
  \input{Physik/Optik/reflexion.pstex_t}
  }

  \Formelbox{
      \[ \alpha_1 = \alpha_2 \]
      }
  }
  {\Groessenbox{
     $ \alpha_1$          & Einfalls"-win"-kel            & $ [rad] $                 \\
     $ \alpha_2$          & Ausfalls"-win"-kel            & $ [rad] $                \\
}}

\section{Brechung}                \index{Optik!Brechung}
                                          \index{Brechung}
 \Hauptbox{
  \Bildbox{
  \input{Physik/Optik/brechung.pstex_t}
  }

  \Formelbox{
	falls $ n_2 > n_1 $: 
	\[ n_1 \sin(\alpha_1) = n_2 \sin(\alpha_2) \]
	\[ n = \frac{c}{u} \]
	\[ n_2 > n_1 \Rightarrow \alpha_1 > \alpha_2 \] 
      }
  }
  {\Groessenbox{
     $ \alpha_1$          & Einfalls"-win"-kel            & $ [rad] $                 \\
     $ \alpha_2$          & Brechungs"-win"-kel           & $ [rad] $                \\
     $ n_{1, 2}$          & Brechungs"-index              & $ [1] $ \\     
}}
\vfill

\section{Totalreflexion}                \index{Optik!Totalreflexion}
                                          \index{Totalreflexion}
 \Hauptbox{
  \Bildbox{
  \input{Physik/Optik/totalreflexion.pstex_t}
  }

  \Formelbox{
	falls $ n_2 > n_1 $: 
	\[ \alpha_g = \arcsin \left(\frac{n_1}{n_2} \right) \]
      }
  }
  {\Groessenbox{
     $ \alpha_g$          & Grenz"-win"-kel            & $ [rad] $                 \\
     $ n_{1, 2}$          & Brechungs"-in"-dex             & $ [1] $ \\     
}}


\subsection{Prisma}                \index{Optik!Prisma}
                                          \index{Prisma}
 \Hauptbox{
  \Bildbox{
  \input{Physik/Optik/prisma.pstex_t}
  }

  \Formelbox{
	falls $ n_2 = 1 $: 
	\[ n_1 = \frac{\sin \frac{\delta + \gamma}{2}}{\frac{\delta}{2}} \]
	\[ n_2 = n_1 \frac{\sin \frac{\delta}{2}}{\frac{\gamma_{min} + \delta}{2}} \]
	\[ \gamma = \alpha - \delta + \arcsin \left\lbrace \sin (\delta) z \right\rbrace \]
	\begin{footnotesize}\[ z = \sqrt{\left[ \left( \frac{n_1}{n_2}\right) ^2 - \sin^2 \alpha \right] - \cos \delta \sin \alpha  } \]\end{footnotesize}
	\[ \gamma_{min} = 2 \arcsin \left( \frac{n_1}{n_2} \sin \frac{\delta}{2} \right) - \delta \]
      }
  }
  {\Groessenbox{
     $ \alpha$          & Einfalls"-win"-kel            & $ [rad] $ \\ 
     $ \beta$           & Ausfalls"-win"-kel            & $ [rad] $ \\
     $ \delta$          & Scheitel"-win"-kel            & $ [rad] $ \\
     $ \gamma$          & Ablenk"-win"-kel              & $ [rad] $ \\
     $ \gamma_{min}$    & min Ablenk"-win"-kel          & $ [rad] $ \\
     $ n_1$             & n-Prisma                      & $ [1] $ \\     
     $ n_2$             & n-Medium                      & $ [1] $ \\      
}}



\subsection{Lichtwellenleiter}                \index{Optik!Lichtwellenleiter}
                                          \index{Lichtwellenleiter}
 \Hauptbox{
  \Bildbox{
  \input{Physik/Optik/lwleiter.pstex_t}
  }

  \Formelbox{
	falls $ n_1 > n_2 $: 
	\begin{small}\[ \alpha_{1max} = \arcsin \frac{n_1 \cos \left[ \arcsin \left( \frac{n_2}{n_1}\right) \right] }{n_0} \]\end{small}
	\[ n_0 \sin \alpha_1 = n_1 \sqrt{1- \cos^2 \alpha_2} \]
	\[ n_0 \sin \alpha_1 = n_1 \sqrt{1-\left( \frac{n_2}{n_1} \right)^2} \]
	\[ n_0 \sin \alpha_1 = \sqrt{n_1 ^2 - n_2 ^2 } \]
      }
  }
  {\Groessenbox{
     $ \alpha_1$        & Einfalls"-win"-kel            & $ [rad] $ \\ 
     $ \alpha_{1max}$   & max Einfalls"-win"-kel        & $ [rad] $ \\
     $ n_0$             & n-Medium                      & $ [1] $ \\  
     $ n_1$             & n-Kern                        & $ [1] $ \\     
     $ n_2$             & n-Mantel                      & $ [1] $ \\      
}}
\vfill


\section{Abbildungen}                \index{Optik!Abbildungen}
                                          \index{Abbildungen}
\subsection{Allgemein}
 \Hauptbox{
  \Bildbox{
  \input{Physik/Optik/kardinalspunkte.pstex_t}
	\begin{small}Vorzeichen:
	\begin{itemize}
	\item F�r sammelde optische Bauelemente ist $ f>0$.
	\item F�r zerstreuende optische Bau"-elemente $ f<0$.
	\item F�r virtuelle Bilder ist $ b<0$ und $ B < 0$. 
	\item F�r vortuelle Gegenst�nde ist $ g < 0 $ und $ G < 0$. 
	\end{itemize} \end{small}
  }

  \Formelbox{
	\[ g = \overline{H_1 G} \]
	\[ b = \overline{H_2 B} \]
	\[ f = \overline{H_1 F_1} \]
	\[ f = \overline{H_2 F_2} \]
	\[ \frac{1}{f} = \frac{1}{g} + \frac{1}{b} \]
	\[ \frac{B}{G} = \frac{b}{g} \]
	\[ \beta = \frac{B}{G}\]
      }
  }
  {\Groessenbox{
     $ g$        & Gegenstands"-weite            & $ [m] $ \\ 
     $ b$        & Bildweite                     & $ [m] $ \\
     $ f$        & Brennweite                    & $ [m] $ \\  
     $ H_1$      & vorderer Hauptpunkt           &  \\     
     $ H_2$      & hinterer Hauptpunkt           &  \\ 
     $ F_1$      & vorderer Brennpunkt           &  \\ 
     $ F_2$      & hinterer Brennpunkt           &  \\
     $ G$        & Gegenstands"-gr�sse           & $ [m] $ \\  
     $ B$        & Bildgr�sse                    & $ [m] $ \\     
     $ \beta$    & Abbildungs"-ver"-h�lt"-nis    & $ [1] $ \\ 
}}


\subsection{Spiegel}               \index{Optik!Spiegel}
                                          \index{Spiegel}

\subsubsection{Parabolspiegel}               \index{Optik!Parabolspiegel}
                                          \index{Parabolspiegel}

Bei Parabolspiegeln treffen sich alle paralell einfallenden Strahlen in einem Punkt (Brennpunkt) auf der optische Achse.

\subsubsection{Elliptische Spiegel}               \index{Optik!Elliptische Spiegel}
                                          \index{Elliptische Spiegel}
Alle Strahlen die vom einen Brennpunkt ausgesendet werden, treffen auf den zweiten Brennpunkt. (Ellipse ist der geometrische Ort aller Punkte einer Ebene, f�r die die Summe ihrer Abst�nde von zwei festen Punkten $ F_1 $ und $ F_2 $ konstant ist.)

\subsubsection{Hyperbolische Spiegel}               \index{Optik!Hyperbolische Spiegel}
                                          \index{Hyperbolische Spiegel}
Alle Strahlen, die von einem Brennpunkt ausgesendet werden, verlaufen nach der Reflexion so, als w�ren sie vom anderen der beiden Brennpunkte ausgesendet worden.(Hyperbel ist der geometrische Ort aller Punkte einer Ebene, f�r die die Differenz ihrer Abst�nde von zwei festen Punkten $ F_1 $ und $ F_2 $ konstant ist.)

\vfill

\subsubsection{Sph�rische Spiegel}               \index{Optik!Sph�rische Spiegel}
                                          \index{Sph�rische Spiegel}
Die Spiegelnde Fl�che ist ein Teil einer Kugel. Wenn nur ein kleiner Ausschnit der Kugelfl�che verwendet wird, gehen parallel einfallende Strahlen n�herungsweise durch einen Brennpunkt: $ f= r / 2 $.

\subsection{Abbildungen durch Spiegel}

\subsubsection{Konkavspiegel}               \index{Optik!Konkavspiegel}
                                          \index{Konkavspiegel}

 \Hauptbox{
  \Bildbox{
  \input{Physik/Optik/konkavspiegel.pstex_t}
  }
  \Formelbox{
	Befindet sich der Gegnstand ausserhalb der Brennweite, so entsteht ein reelles Bild, anderseits ist das Bild virtuell.	
      }
  }
  {\Groessenbox{
     $ G$        & Gegenstand           & $ [m] $ \\ 
     $ B$        & Bild                 & $ [m] $ \\
     $ F$        & Brennpunkt           &  \\  
}}


\subsubsection{Konvexspiegel}              \index{Optik!Konvexspiegel}
                                          \index{Konvexspiegel}

 \Hauptbox{
  \Bildbox{
  \input{Physik/Optik/konvexspiegel.pstex_t}
  }
  \Formelbox{
	Konvexspiegel haben stets virtuelle Bilder bei reellen Gegenst�nden.
      }
  }
  {\Groessenbox{
     $ G$        & Gegenstand           & $ [m] $ \\ 
     $ B$        & Bild                 & $ [m] $ \\
     $ F$        & Brennpunkt           &  \\  
}}


\subsubsection{Planspiegel}              \index{Optik!Planspiegel}
                                          \index{Planspiegel}
 \Hauptbox{
  \Bildbox{
  \input{Physik/Optik/planspiegel.pstex_t}
  }
  \Formelbox{
	Das Virtuelle Bild ist gleich gross wie der Gegenstand. Der Brennpunkt F liegt im Unendlichen. Der Planspiegel ist ein Spezialfall des Konvexspiegels.
      }
  }
  {\Groessenbox{
     $ G$        & Gegenstand           & $ [m] $ \\ 
     $ B$        & Bild                 & $ [m] $ \\
     $ F$        & Brennpunkt           &  \\  
}}
\vfill


\subsection{Linsen}              \index{Optik!Linsen}
                                          \index{Linsen}

\subsubsection{Linsentypen}              \index{Optik!Linsentypen}
                                          \index{Linsentypen}

 \Hauptbox{
  \Bildbox{
  \input{Physik/Optik/linsen.pstex_t}
  }
  \Formelbox{
	\[ q = \frac{d}{f} \]
	\[ D = \frac{1}{f} \]
	\begin{footnotesize}Linsenschleifergleichung:\end{footnotesize}
	\[ D = \left( \frac{n_2}{n_1}-1 \right) \left( \frac{1}{r_1} + \frac{1}{r_2} \right) \]
	\begin{footnotesize}Falls das Linsenmaterial optisch dichter ist als das umgebende Medium, zeigt die obere Abbildung eine Sammel- und die untere eine Streulinse.\end{footnotesize} \vspace{0.1cm}
      }
  }
  {\Groessenbox{
     $ q$        & �ffnungs"-ver"-h�lt"-nis    & $ [m] $ \\ 
     $ d$        & effektiver Durchm.          & $ [m] $ \\
     $ f$        & Brennweite                  & $ [m] $ \\ 
     $ D$        & Brechkraft                  & $ [dpt] $ \\ 
     $ n_1$      & n-Umgebung                  & $ [1] $ \\ 
     $ n_2$      & n-Linse                     & $ [1] $ \\
     $ r_{1,2}$  & Linsenradien                & $ [m] $ \\     
}}


\subsubsection{Linsensysteme}              \index{Optik!Linsensysteme}
                                          \index{Linsensysteme}
 \Hauptbox{
  \Bildbox{
  Zwei Linsen mit Brennweiten $ f_1 $ , $ f_2 $ auf einer Achse ergeben eine Linse mit Brennweite f, falls ihr Abstand d kleiner $ f_1 $ ist.
  }
  \Formelbox{
	\[ \frac{1}{f} = \frac{1}{f_1} + \frac{1}{f_2} - \frac{d}{f_1 f_2} \]
	\[ D = D_1 + D_2 - d D_1 D_2 \]
      }
  }
  {\Groessenbox{
     $ f_{1,2}$  & Brennweiten                 & $ [m] $ \\ 
     $ D$        & Brechkraft                  & $ [dpt] $ \\ 
     $ d$        & Linsenabstand               & $ [m] $ \\     
}}


\subsection{Abbildungen durch Linsen}    

\subsubsection{Sammellinse}              \index{Optik!Sammellinse}
                                          \index{Sammellinse}

 \Hauptbox{
  \Bildbox{
  \input{Physik/Optik/linseInnerhalb.pstex_t}
  }
  \Formelbox{
	Der Gegenstand ist innerhalb der Brennweite $ \Rightarrow $ reelles Bild.
      } 
  \Bildbox{
  \input{Physik/Optik/linseAusserhalb.pstex_t}
  }
  \Formelbox{
	Der Gegenstand ist ausserhalb der Brennweite $ \Rightarrow $ virtuelles Bild.
      }
  }
  {\Groessenbox{
     $ G$        & Gegenstand           & $ [m] $ \\ 
     $ B$        & Bild                 & $ [m] $ \\
     $ F$        & Brennpunkt           &  \\  
     $ f_{1,2}$  & Brennweiten          & $ [m] $ \\ 
}}
\vfill


\subsubsection{Zerstreuungslinse}              \index{Optik!Zerstreuungslinse}
                                          \index{Zerstreuungslinse}
 \Hauptbox{
  \Bildbox{
  \input{Physik/Optik/zerstreulinse.pstex_t}
  }
  \Formelbox{
	Bei Zerstreungslinsen haben reelle Gegenst�nde stets virtuelle Bilder, unabh�ngig von ihrer Position.
	\vspace{0.3cm}
      } 
  }
  {\Groessenbox{
     $ G$        & Gegenstand           & $ [m] $ \\ 
     $ B$        & Bild                 & $ [m] $ \\
     $ F$        & Brennpunkt           &  \\  
     $ f_{1,2}$  & Brennweiten          & $ [m] $ \\ 
}}

\subsection{Optische Ger�te}

\subsubsection{Fotoapparat}              \index{Optik!Fotoapparat}
                                          \index{Fotoapparat}
\Hauptbox{
  \Bildbox{
  \input{Physik/Optik/fotoapparat.pstex_t}
  }
  \Formelbox{
	\[ B = \frac{f}{g-f} G \]
	\[ I \approx d^2 \]
	\[ H \approx \frac{1}{B^2} \approx \frac{d^2}{f^2} \]
	\[ E = H t \]
	\[ q = \frac{d}{f} \mbox{\hspace{1cm}} Z = \frac{1}{q} \]
	\[ E \approx q^2 t \]
	\[ \frac{u}{d} = \frac{b - b_0}{b} \]
	\[ \frac{1}{g} = \frac{1}{g_0} \pm \frac{u}{q f^2} \]
	\[ g>g_0 \Rightarrow - \mbox{\hspace{0.5cm}} g<g_0 \Rightarrow + \]
      } 
  }
  {\Groessenbox{
     $ G$        & Gegenstand                         & $ [m] $ \\ 
     $ g$        & Gegen"-stands"-wei"-te             & $ [m] $ \\ 
     $ g_0$      & Sch�rfen"-tiefen"-be"-reich        & $ [m] $ \\ 
     $ B$        & Bildweite                          & $ [m] $ \\
     $ b$        & Bild                               & $ [m] $ \\
     $ f$        & Brennweite                         & $ [m] $ \\ 
     $ I$        & Lichtstrom                         & $ [W] $ \\ 
     $ d$        & Durchm. Eintritts"-pu"-pil"-le     & $ [m] $ \\ 
     $ H$        & Helligkeit                         & $ [\frac{W}{m^2}] $ \\ 
     $ q$        & �ffnungs"-ver"-h�lt"-nis           & $ [1] $ \\ 
     $ Z$        & Brendenein"-stel"-lung             & $ [1] $ \\ 
     $ E$        & Belichtung                         & $ [1] $ \\ 
     $ t$        & Belichtungs"-zeit                  & $ [s] $ \\ 
     $ u$        & Durchm. Un"-sch�r"-fen"-kreis      & $ [m] $ \\
}}

\subsubsection{Projektor}              \index{Optik!Projektor}
                                          \index{Projektor}
\Hauptbox{
  \Bildbox{
  \input{Physik/Optik/projektor.pstex_t}
  }
  \Formelbox{
	Das Dia wird im Objektiv abgebidet $ \Rightarrow g_2 = b_1$ \\
	Das Bild der Lampe muss im Objektiv sein.
      } 
  }
  {\Groessenbox{     
     $ g_2$      & g-Objektiv                 & $ [m] $ \\      
     $ b_1$      & b-Konden"-sa"-tor          & $ [m] $ \\
}}
\vfill


\subsubsection{Lupe}              \index{Optik!Lupe}
                                          \index{Lupe}
\Hauptbox{
  \Bildbox{
  \input{Physik/Optik/lupe.pstex_t}
  \begin{footnotesize}Sammellinse zur Vergr�sserung des Sehwinkels (Bild im Unendlichen) \end{footnotesize}
  }
  \Formelbox{
	\begin{footnotesize}Gegenstand in Brennweite $ \Rightarrow $ Sehwinkel $ \epsilon $ ist unabh�ngig von der Augenposition\end{footnotesize}
	\[ V = \frac{\tan \epsilon}{\tan \epsilon_0} \]
        \[ V =  \frac{s}{g} > V_{normal} \]	
	\[ \tan \epsilon ' = \frac{G}{f} \]
	\[ \tan \epsilon _0 = \frac{G}{s} \]
	\[ V = \frac{s}{f} \]
      } 
  }
  {\Groessenbox{
     $ G$        & Gegenstand                         & $ [m] $ \\ 
     $ g$        & Gegen"-stands"-wei"-te             & $ [m] $ \\ 
     $ B$        & Bildweite                          & $ [m] $ \\
     $ b$        & Bild                               & $ [m] $ \\
     $ f$        & Brennweite                         & $ [m] $ \\     
     $ \epsilon$ & Sehwinkel durch Lupe               & $ [rad] $ \\ 
     $ \epsilon_0$ & Sehwinkel ohne Lupe              & $ [rad] $ \\ 
     $ s$        & deutliche Sehweite                 & $ [m] $ \\ 
     $ V$        & Vergr�sserung (max. ca. 25)        & $ [1] $ \\ 
}}

\subsubsection{Mikroprojektor}              \index{Optik!Mikroprojektor}
                                          \index{Mikroprojektor}
\Hauptbox{
  \Bildbox{
  \input{Physik/Optik/microprojektor.pstex_t}
  \begin{footnotesize}Das reelle Bild einer Sammellinse wird verwendet und auf einer Mattscheibe abgebildet\end{footnotesize}
  }
  \Formelbox{
	\begin{footnotesize}Bild aus deutlicher Sehweite betrachtet:\end{footnotesize}
	\[ V = \frac{B}{G} = \frac{b}{g} \] 
	\begin{footnotesize}Stahlengang siehe Projektor\end{footnotesize}
      } 
  }
  {\Groessenbox{
     $ G$        & Gegenstand                         & $ [m] $ \\ 
     $ g$        & Gegen"-stands"-wei"-te             & $ [m] $ \\ 
     $ B$        & Bildweite                          & $ [m] $ \\
     $ b$        & Bild                               & $ [m] $ \\
     $ V$        & Vergr�sserung                      & $ [1] $ \\  
}}
\vfill


\subsubsection{Mikroskop}              \index{Optik!Mikroskop}
                                          \index{Mikroskop}
\Hauptbox{
  \Bildbox{
  \input{Physik/Optik/mikroskop.pstex_t}
  \begin{footnotesize}Das Objektiv verh�lt sich wie ein Mikroprojektor. Sein Bild wird durch das Okular, welches sich wie eine Lupe verh�lt, betrachtet.\end{footnotesize}
  }
  \Formelbox{
	\[ V = V_1 V_2 \]
	\[ V = \frac{\tan \epsilon}{\tan \epsilon _0} \]
	\[ V = \frac{B}{G} \frac{f}{f_2} \]
	\[ V = \frac{\Delta}{f_1} \frac{s}{f_2} \]
	\[ V_1 = \frac{\Delta}{f_1} \]
	\[ V_2 = \frac{s}{f_2} \]
	\[ \Delta = b_1 - f_1 \]
	\vspace{2cm}
      } 
  }
  {\Groessenbox{
     $ G$        & Gegenstand                         & $ [m] $ \\ 
     $ g_1$      & Gegen"-stands"-wei"-te             & $ [m] $ \\ 
     $ B$        & Bild                               & $ [m] $ \\
     $ b_1$      & Bildweite                          & $ [m] $ \\
     $ F_1$      & Brennpunkt Objektiv                &  \\ 
     $ F_2$      & Brennpunkt Okular                  &  \\ 
     $ f_1$      & Brennweite Objektiv                & $ [m] $ \\    
     $ f_2$      & Brennweite Okular                  & $ [m] $ \\   
     $ \Delta $  & Tubusl�nge                         & $ [m] $ \\ 
     $ \epsilon$ & Sehwinkel                          & $ [rad] $ \\ 
     $ s$        & deutliche Sehweite                 & $ [m] $ \\ 
     $ V$        & Vergr�sserung total                & $ [1] $ \\  
     $ V_1$      & V-Objektiv                         & $ [1] $ \\  
     $ V_2$      & V-Okular                           & $ [1] $ \\ 
}}
\vfill


\subsubsection{Fernrohre}              \index{Optik!Fernrohre}
                                          \index{Fernrohre}
\Hauptbox{
  \Bildbox{
  \input{Physik/Optik/fernrohr.pstex_t}
  \begin{footnotesize}Ein Fernglas mit den Daten $ 10 \times 50 $ hat eine Vergr�sserung von 10 und einen Objektivdurchmesser von 50 mm.\end{footnotesize}
  }
  \Formelbox{
	\[ V = \frac{\tan \epsilon}{\tan \epsilon_0} \]
	\[ \epsilon = V \epsilon ' \]
	\[ V = \frac{f_1}{f_2} \]
	\[ \frac{1}{f_1 + f_2} + \frac{1}{a} = \frac{1}{f_2} \]
	\[ \frac{D}{d} = \frac{f_1 + f_2}{a} = V \]
	\[ a = \frac{l}{V} \mbox{\hspace{1cm}} d = \frac{D}{V} \]
	\[ L = d^2 \]
	\[ L = \left( \frac{D}{V} \right) ^2 \]
	\[ l = f_1 + f_2 \]
      } 
  }
  {\Groessenbox{     
     $ B$        & Bildweite                          & $ [m] $ \\    
     $ f_1$      & Brennweite Objektiv                & $ [m] $ \\    
     $ f_2$      & Brennweite Okular                  & $ [m] $ \\   
     $ l $       & Fernrohrl�nge                      & $ [m] $ \\ 
     $ \epsilon$ & Ausfallswinkel                     & $ [rad] $ \\ 
     $ \epsilon'$& Einfallswinkel                     & $ [rad] $ \\ 
     $ s$        & deutliche Sehweite                 & $ [m] $ \\ 
     $ V$        & Vergr�sserung total                & $ [1] $ \\  
     $ L$        & Lichtst�rke                        & $ [1] $ \\  
     $ D$        & Durchm. Objektiv                   & $ [m] $ \\  
     $ d$        & Durchm. Austrittspupille           & $ [mm] $ \\
     $ a$        & Abstand Okular Astrittspupille     & $ [m] $ \\    
}}
\vfill


%%% Local Variables: 
%%% mode: latex
%%% TeX-master: "../../FoSaHSR"
%%% End: 