%====================================================================
% Titel: Global Settings f�r Formelbuch
% Autor(en):von Adrian Freihofer und Thomas K�ng
% Erstellt: 5.11.2001
% Ge�ndert:
% ===================================================================

\documentclass[12pt,a4paper,ngerman,openany,fleqn]{scrbook}
%\usepackage{a4,german}
\usepackage{noChapapp}
\usepackage{fancyhdr}
\usepackage[latin1]{inputenc}
\usepackage{graphicx}
\usepackage{color}
\usepackage[T1]{fontenc}                                %T1 fontencoding
\usepackage{multicol}                                   %Package for Multicolums
%\usepackage[bookmarks=true, bookmarksnumbered=true, linkbordercolor={0 0 1}, pdfborder={0 0 10.0}]{hyperref}                                   %hyperlinks and bookmarks in DVI
\usepackage[dvipdfm, bookmarks=true, bookmarksnumbered=true]{hyperref}                                   %hyperlinks and bookmarks in PDF
\hypersetup{
pdfauthor = {Thomas K�ng, Urs Winiger und Adrian Freihofer},
pdftitle = {HSR Formelsammlung},
pdfsubject = {Freie Formalsammlung},
pdfkeywords = {Elektrizit�tslehre, Elektronik, Physik, Mathematik, Energietechnik, Antriebstechnik},
pdfcreator = {LaTeX with hyperref package},
pdfproducer = {dvipdfm}}

\usepackage{ae}                                         %Almost European Computer Modern
\usepackage[slantedGreek]{mathpple}                     %Palatino with slanted greek uppercase letters
\usepackage{amssymb}
\usepackage{babel}                                      %German Package
\usepackage{amsfonts}                                   %AMS-Fonts
\usepackage{textcomp}
\usepackage{latexsym}                                   %additional Symbols
\makeindex
\usepackage{makeidx}
\usepackage{multicol}

\renewcommand{\sectfont}{\rmfamily\bfseries}            %Select rmfamily for section-font
%\renewcommand{\descfont}{\sffamily\bfseries}
%\renewcommand{\pnumfont}{\normalfont}
%\renewcommand{\headfont}{\slshape}
%\renewcommand{\capfont}{\normalfont}
%\renewcommand{\caplabelfont}{\normalfont}
\renewcommand{\sfdefault}{bfr}                          %Frutiger SansSerif
%\renewcommand{\sfdefault}{phv}                         %Helvetica SansSerif
%\renewcommand{\ttdefault}{pcr}                         %Courier Typewriter
%\renewcommand{\rmdefault}{put}                         %Utopia Roman

\newcommand*{\zb}{z.\,B.\xspace}
\newcommand{\D}{\displaystyle}
\newcommand{\T}{\textstyle}
\newcommand{\E}{\displaystyle \rm}

%\setlength{\voffset}{-10mm}                              %type area
%\setlength{\hoffset}{0cm}
\setlength{\textheight}{243mm}
\addtolength{\textwidth}{1cm}
\addtolength{\evensidemargin}{-1cm}
%\setlength{\oddsidemargin}{}
%\setlength{\topmargin}{0cm}
%\setlength{\columnsep}{5mm}
%\setlength{\headsep}{1cm}
\setlength{\mathindent}{0pt}
%\setlength{\unitlength}{0.75mm}
%\setlength{\parindent}{0em}                            %setzt einen anderen Einzug am Absatzanfang
%\setlength{\parskip}{1.5ex}                                                    %setzt einen anderen Zeilenabstandzwischen zwei Absaetzen
%\renewcommand{\baselinestretch}{1.2}                   %setzt einen anderen Zeilenabstand innerhalb eines Absatzes
%\renewcommand{\arraystretch}{1.5}


%\pagestyle{headings}



\newcommand{\Hauptbox}[2]{
\noindent
\begin{tabular}[t]{@{\vline}l@{\extracolsep\fill}@{\vline}l|}
  \hline
  \begin{tabular}[t]{l|l}
    #1
  \end{tabular}&
    #2 \\
  \hline
\end{tabular}
}

\newcommand{\Bildbox}[2][0.25]{
 
  \begin{minipage}[t][\totalheight][c]{#1\textwidth}
    #2
    \vspace{1ex}

  \end{minipage} &}


\newcommand{\Formelbox}[2][0.35]{
  \begin{minipage}[t][\totalheight][c]{#1\textwidth}
%    \begin{tabular*}{#1\textwidth}{[t]{l}
      #2 %[1ex]
%    \end{tabular*}
\vspace{0ex}
  \end{minipage} \\ }

\newlength{\temp}
\newlength{\tempint}
\newcommand{\Groessenbox}[2][0.4]{
\setlength{\temp}{#1\textwidth}
\addtolength{\temp}{-8\tabcolsep}
\setlength{\tempint}{1\temp}
\addtolength{\tempint}{-1\tabcolsep}
%\addtolength{\tempint}{-1em}
 \begin{minipage}[t][\totalheight][c]{\temp}
  \begin{tabular*}{\temp}[t]{@{}p{0.2\tempint}@{\hspace{0.5em}}p{0.55\tempint}@{\hspace{0.5em}}p{0.25\tempint}@{}}
   #2 [1ex]
  \end{tabular*}
 \end{minipage}}

\newcommand{\boxdump}{\rule{2cm}{0mm}}




%%% Local Variables: 
%%% mode: latex
%%% TeX-master: "FoSaHSR"
%%% End: 
