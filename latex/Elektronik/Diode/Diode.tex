%==========================================================
% Titel:    Formelsammlung-HSR
% Autor:    Adrian Freihofer
% Erstellt: 7.11.2001
% Ge�ndert:
%==========================================================

\chapter{Diode}

\section{Ideale Diode}

\index{ideale Diode}
\index{Diode!ideale}

\Hauptbox{                                                                    
  \Bildbox{
  \input{Elektronik/Diode/idealeDiode.pstex_t}
  }
  \Formelbox{
    \footnotesize Sperrbereich (SB) : \normalsize \[ i_D = 0 \mbox{, } u_D < 0 \] 
    \footnotesize Durchlassbereich (DB): \normalsize \[ i_D \ge 0 \mbox{, } u_D = 0 \]
    }
  }
  {\Groessenbox{
     $ u_D$          & Spannungs �ber Diode        & $ [V] $                 \\
     $ i_D$          & Strom durch Diode            & $ [A] $                \\
}}

\section{Konstantspannungsmodel}

\index{Diode!Konstantspannungsmodell}

\label{kmod}

\Hauptbox{                                                                    
  \Bildbox{
  \input{Elektronik/Diode/konstModell.pstex_t}
  }
  \Formelbox{
           \[u_{F SI Diode} = 0.6 \]
           \[u_{F Germanium Diode} = 0.3 \]
           \[u_{F Shottky Diode} = 0.1 \]
           \[u_{F LED rot} = 1.6 \]
      }
  }
  {\Groessenbox{
     $ u_F$          & Forw�rts-Spannungs        & $ [V] $                 \\
}}


\section{Arbeitspunktberechnung}

\index{Diode!Arbeitspunktberechnung}

\Hauptbox{                                                                    
  \Bildbox{
  \input{Elektronik/Diode/arbeitspunkt.pstex_t}
  }
  \Formelbox{
         \footnotesize Teilweise ist es nur durch ausprobieren m�glich zu sagen, ob die Diode leitet oder sperrt. \normalsize
      }
  }
  {\Groessenbox{
    $ i_D$          & Strom durch Diode            & $ [A] $                \\  
}}
\vfill


\section{Kennlinie}

\index{Diode!Kennlinie}
\index{Diode!Durchlassbereich}
\index{Diode!Sperrbereich}
\index{Diode!�bergangsbereich}
\index{Diode!Durchbruchbereich}

\Hauptbox{                                                                    
  \Bildbox{
    \scriptsize Kennlinie allgemein: \normalsize
    \input{Elektronik/Diode/kennlinie2.pstex_t}
    \scriptsize 
    \begin{tabular}{ll}
      I & Durchlassbereich \\
      II & �bergangsbereich\\
      III & Sperrbereich\\
      IV & Durchbruchbereich\\
      & \\
    \end{tabular}
    Kennlinie Ge- und Si-Diode: \normalsize
    % GNUPLOT: LaTeX picture
\setlength{\unitlength}{0.240900pt}
\ifx\plotpoint\undefined\newsavebox{\plotpoint}\fi
\begin{picture}(600,360)(80,0)
\font\gnuplot=cmr10 at 10pt
\gnuplot
\tiny
\sbox{\plotpoint}{\rule[-0.200pt]{0.400pt}{0.400pt}}%
\put(140.0,82.0){\rule[-0.200pt]{4.818pt}{0.400pt}}
\put(120,82){\makebox(0,0)[r]{0}}
\put(519.0,82.0){\rule[-0.200pt]{4.818pt}{0.400pt}}
\put(140.0,130.0){\rule[-0.200pt]{4.818pt}{0.400pt}}
\put(120,130){\makebox(0,0)[r]{0.05}}
\put(519.0,130.0){\rule[-0.200pt]{4.818pt}{0.400pt}}
\put(140.0,177.0){\rule[-0.200pt]{4.818pt}{0.400pt}}
\put(120,177){\makebox(0,0)[r]{0.1}}
\put(519.0,177.0){\rule[-0.200pt]{4.818pt}{0.400pt}}
\put(140.0,225.0){\rule[-0.200pt]{4.818pt}{0.400pt}}
\put(120,225){\makebox(0,0)[r]{0.15}}
\put(519.0,225.0){\rule[-0.200pt]{4.818pt}{0.400pt}}
\put(140.0,272.0){\rule[-0.200pt]{4.818pt}{0.400pt}}
\put(120,272){\makebox(0,0)[r]{0.2}}
\put(519.0,272.0){\rule[-0.200pt]{4.818pt}{0.400pt}}
\put(140.0,320.0){\rule[-0.200pt]{4.818pt}{0.400pt}}
\put(120,320){\makebox(0,0)[r]{0.25}}
\put(519.0,320.0){\rule[-0.200pt]{4.818pt}{0.400pt}}
\put(140.0,82.0){\rule[-0.200pt]{0.400pt}{4.818pt}}
\put(140,41){\makebox(0,0){0}}
\put(140.0,300.0){\rule[-0.200pt]{0.400pt}{4.818pt}}
\put(197.0,82.0){\rule[-0.200pt]{0.400pt}{4.818pt}}
\put(197,41){\makebox(0,0){0.1}}
\put(197.0,300.0){\rule[-0.200pt]{0.400pt}{4.818pt}}
\put(254.0,82.0){\rule[-0.200pt]{0.400pt}{4.818pt}}
\put(254,41){\makebox(0,0){0.2}}
\put(254.0,300.0){\rule[-0.200pt]{0.400pt}{4.818pt}}
\put(311.0,82.0){\rule[-0.200pt]{0.400pt}{4.818pt}}
\put(311,41){\makebox(0,0){0.3}}
\put(311.0,300.0){\rule[-0.200pt]{0.400pt}{4.818pt}}
\put(368.0,82.0){\rule[-0.200pt]{0.400pt}{4.818pt}}
\put(368,41){\makebox(0,0){0.4}}
\put(368.0,300.0){\rule[-0.200pt]{0.400pt}{4.818pt}}
\put(425.0,82.0){\rule[-0.200pt]{0.400pt}{4.818pt}}
\put(425,41){\makebox(0,0){0.5}}
\put(425.0,300.0){\rule[-0.200pt]{0.400pt}{4.818pt}}
\put(482.0,82.0){\rule[-0.200pt]{0.400pt}{4.818pt}}
\put(482,41){\makebox(0,0){0.6}}
\put(482.0,300.0){\rule[-0.200pt]{0.400pt}{4.818pt}}
\put(539.0,82.0){\rule[-0.200pt]{0.400pt}{4.818pt}}
\put(539,41){\makebox(0,0){0.7}}
\put(539.0,300.0){\rule[-0.200pt]{0.400pt}{4.818pt}}
\put(140.0,82.0){\rule[-0.200pt]{96.119pt}{0.400pt}}
\put(539.0,82.0){\rule[-0.200pt]{0.400pt}{57.334pt}}
\put(140.0,320.0){\rule[-0.200pt]{96.119pt}{0.400pt}}
\put(140.0,82.0){\rule[-0.200pt]{0.400pt}{57.334pt}}
\put(379,280){\makebox(0,0)[r]{Si}}
\put(399.0,280.0){\rule[-0.200pt]{24.090pt}{0.400pt}}
\put(140,82){\usebox{\plotpoint}}
\put(140,81.67){\rule{73.956pt}{0.400pt}}
\multiput(140.00,81.17)(153.500,1.000){2}{\rule{36.978pt}{0.400pt}}
\put(447,82.67){\rule{2.650pt}{0.400pt}}
\multiput(447.00,82.17)(5.500,1.000){2}{\rule{1.325pt}{0.400pt}}
\put(458,83.67){\rule{1.445pt}{0.400pt}}
\multiput(458.00,83.17)(3.000,1.000){2}{\rule{0.723pt}{0.400pt}}
\put(464,84.67){\rule{0.964pt}{0.400pt}}
\multiput(464.00,84.17)(2.000,1.000){2}{\rule{0.482pt}{0.400pt}}
\put(468,85.67){\rule{0.723pt}{0.400pt}}
\multiput(468.00,85.17)(1.500,1.000){2}{\rule{0.361pt}{0.400pt}}
\put(471,86.67){\rule{0.723pt}{0.400pt}}
\multiput(471.00,86.17)(1.500,1.000){2}{\rule{0.361pt}{0.400pt}}
\put(474,87.67){\rule{0.482pt}{0.400pt}}
\multiput(474.00,87.17)(1.000,1.000){2}{\rule{0.241pt}{0.400pt}}
\put(476,88.67){\rule{0.482pt}{0.400pt}}
\multiput(476.00,88.17)(1.000,1.000){2}{\rule{0.241pt}{0.400pt}}
\put(478,89.67){\rule{0.482pt}{0.400pt}}
\multiput(478.00,89.17)(1.000,1.000){2}{\rule{0.241pt}{0.400pt}}
\put(480,90.67){\rule{0.482pt}{0.400pt}}
\multiput(480.00,90.17)(1.000,1.000){2}{\rule{0.241pt}{0.400pt}}
\put(483,91.67){\rule{0.482pt}{0.400pt}}
\multiput(483.00,91.17)(1.000,1.000){2}{\rule{0.241pt}{0.400pt}}
\put(485,92.67){\rule{0.241pt}{0.400pt}}
\multiput(485.00,92.17)(0.500,1.000){2}{\rule{0.120pt}{0.400pt}}
\put(486,93.67){\rule{0.241pt}{0.400pt}}
\multiput(486.00,93.17)(0.500,1.000){2}{\rule{0.120pt}{0.400pt}}
\put(487,94.67){\rule{0.241pt}{0.400pt}}
\multiput(487.00,94.17)(0.500,1.000){2}{\rule{0.120pt}{0.400pt}}
\put(488,95.67){\rule{0.241pt}{0.400pt}}
\multiput(488.00,95.17)(0.500,1.000){2}{\rule{0.120pt}{0.400pt}}
\put(489,96.67){\rule{0.241pt}{0.400pt}}
\multiput(489.00,96.17)(0.500,1.000){2}{\rule{0.120pt}{0.400pt}}
\put(490,97.67){\rule{0.241pt}{0.400pt}}
\multiput(490.00,97.17)(0.500,1.000){2}{\rule{0.120pt}{0.400pt}}
\put(491,98.67){\rule{0.241pt}{0.400pt}}
\multiput(491.00,98.17)(0.500,1.000){2}{\rule{0.120pt}{0.400pt}}
\put(492,99.67){\rule{0.241pt}{0.400pt}}
\multiput(492.00,99.17)(0.500,1.000){2}{\rule{0.120pt}{0.400pt}}
\put(482.0,92.0){\usebox{\plotpoint}}
\put(493,101.67){\rule{0.241pt}{0.400pt}}
\multiput(493.00,101.17)(0.500,1.000){2}{\rule{0.120pt}{0.400pt}}
\put(494,102.67){\rule{0.241pt}{0.400pt}}
\multiput(494.00,102.17)(0.500,1.000){2}{\rule{0.120pt}{0.400pt}}
\put(495,103.67){\rule{0.241pt}{0.400pt}}
\multiput(495.00,103.17)(0.500,1.000){2}{\rule{0.120pt}{0.400pt}}
\put(493.0,101.0){\usebox{\plotpoint}}
\put(496,105.67){\rule{0.241pt}{0.400pt}}
\multiput(496.00,105.17)(0.500,1.000){2}{\rule{0.120pt}{0.400pt}}
\put(497,106.67){\rule{0.241pt}{0.400pt}}
\multiput(497.00,106.17)(0.500,1.000){2}{\rule{0.120pt}{0.400pt}}
\put(496.0,105.0){\usebox{\plotpoint}}
\put(498,108.67){\rule{0.241pt}{0.400pt}}
\multiput(498.00,108.17)(0.500,1.000){2}{\rule{0.120pt}{0.400pt}}
\put(498.0,108.0){\usebox{\plotpoint}}
\put(499,110.67){\rule{0.241pt}{0.400pt}}
\multiput(499.00,110.17)(0.500,1.000){2}{\rule{0.120pt}{0.400pt}}
\put(499.0,110.0){\usebox{\plotpoint}}
\put(500,112){\usebox{\plotpoint}}
\put(500,111.67){\rule{0.241pt}{0.400pt}}
\multiput(500.00,111.17)(0.500,1.000){2}{\rule{0.120pt}{0.400pt}}
\put(501,113.67){\rule{0.241pt}{0.400pt}}
\multiput(501.00,113.17)(0.500,1.000){2}{\rule{0.120pt}{0.400pt}}
\put(501.0,113.0){\usebox{\plotpoint}}
\put(502,115.67){\rule{0.241pt}{0.400pt}}
\multiput(502.00,115.17)(0.500,1.000){2}{\rule{0.120pt}{0.400pt}}
\put(502.0,115.0){\usebox{\plotpoint}}
\put(503,117.67){\rule{0.241pt}{0.400pt}}
\multiput(503.00,117.17)(0.500,1.000){2}{\rule{0.120pt}{0.400pt}}
\put(503.0,117.0){\usebox{\plotpoint}}
\put(504,120.67){\rule{0.241pt}{0.400pt}}
\multiput(504.00,120.17)(0.500,1.000){2}{\rule{0.120pt}{0.400pt}}
\put(504.0,119.0){\rule[-0.200pt]{0.400pt}{0.482pt}}
\put(505,122.67){\rule{0.241pt}{0.400pt}}
\multiput(505.00,122.17)(0.500,1.000){2}{\rule{0.120pt}{0.400pt}}
\put(505.0,122.0){\usebox{\plotpoint}}
\put(506,125.67){\rule{0.241pt}{0.400pt}}
\multiput(506.00,125.17)(0.500,1.000){2}{\rule{0.120pt}{0.400pt}}
\put(506.0,124.0){\rule[-0.200pt]{0.400pt}{0.482pt}}
\put(507,127.67){\rule{0.241pt}{0.400pt}}
\multiput(507.00,127.17)(0.500,1.000){2}{\rule{0.120pt}{0.400pt}}
\put(507.0,127.0){\usebox{\plotpoint}}
\put(508,130.67){\rule{0.241pt}{0.400pt}}
\multiput(508.00,130.17)(0.500,1.000){2}{\rule{0.120pt}{0.400pt}}
\put(508.0,129.0){\rule[-0.200pt]{0.400pt}{0.482pt}}
\put(509,132){\usebox{\plotpoint}}
\put(509,132.67){\rule{0.241pt}{0.400pt}}
\multiput(509.00,132.17)(0.500,1.000){2}{\rule{0.120pt}{0.400pt}}
\put(509.0,132.0){\usebox{\plotpoint}}
\put(510,135.67){\rule{0.241pt}{0.400pt}}
\multiput(510.00,135.17)(0.500,1.000){2}{\rule{0.120pt}{0.400pt}}
\put(510.0,134.0){\rule[-0.200pt]{0.400pt}{0.482pt}}
\put(511,138.67){\rule{0.241pt}{0.400pt}}
\multiput(511.00,138.17)(0.500,1.000){2}{\rule{0.120pt}{0.400pt}}
\put(511.0,137.0){\rule[-0.200pt]{0.400pt}{0.482pt}}
\put(512,142.67){\rule{0.241pt}{0.400pt}}
\multiput(512.00,142.17)(0.500,1.000){2}{\rule{0.120pt}{0.400pt}}
\put(512.0,140.0){\rule[-0.200pt]{0.400pt}{0.723pt}}
\put(513,145.67){\rule{0.241pt}{0.400pt}}
\multiput(513.00,145.17)(0.500,1.000){2}{\rule{0.120pt}{0.400pt}}
\put(513.0,144.0){\rule[-0.200pt]{0.400pt}{0.482pt}}
\put(514,149.67){\rule{0.241pt}{0.400pt}}
\multiput(514.00,149.17)(0.500,1.000){2}{\rule{0.120pt}{0.400pt}}
\put(514.0,147.0){\rule[-0.200pt]{0.400pt}{0.723pt}}
\put(515,151){\usebox{\plotpoint}}
\put(515,152.67){\rule{0.241pt}{0.400pt}}
\multiput(515.00,152.17)(0.500,1.000){2}{\rule{0.120pt}{0.400pt}}
\put(515.0,151.0){\rule[-0.200pt]{0.400pt}{0.482pt}}
\put(516,156.67){\rule{0.241pt}{0.400pt}}
\multiput(516.00,156.17)(0.500,1.000){2}{\rule{0.120pt}{0.400pt}}
\put(516.0,154.0){\rule[-0.200pt]{0.400pt}{0.723pt}}
\put(517,160.67){\rule{0.241pt}{0.400pt}}
\multiput(517.00,160.17)(0.500,1.000){2}{\rule{0.120pt}{0.400pt}}
\put(517.0,158.0){\rule[-0.200pt]{0.400pt}{0.723pt}}
\put(518,165.67){\rule{0.241pt}{0.400pt}}
\multiput(518.00,165.17)(0.500,1.000){2}{\rule{0.120pt}{0.400pt}}
\put(518.0,162.0){\rule[-0.200pt]{0.400pt}{0.964pt}}
\put(519,169.67){\rule{0.241pt}{0.400pt}}
\multiput(519.00,169.17)(0.500,1.000){2}{\rule{0.120pt}{0.400pt}}
\put(519.0,167.0){\rule[-0.200pt]{0.400pt}{0.723pt}}
\put(520,171){\usebox{\plotpoint}}
\put(520,173.67){\rule{0.241pt}{0.400pt}}
\multiput(520.00,173.17)(0.500,1.000){2}{\rule{0.120pt}{0.400pt}}
\put(520.0,171.0){\rule[-0.200pt]{0.400pt}{0.723pt}}
\put(521,178.67){\rule{0.241pt}{0.400pt}}
\multiput(521.00,178.17)(0.500,1.000){2}{\rule{0.120pt}{0.400pt}}
\put(521.0,175.0){\rule[-0.200pt]{0.400pt}{0.964pt}}
\put(522,183.67){\rule{0.241pt}{0.400pt}}
\multiput(522.00,183.17)(0.500,1.000){2}{\rule{0.120pt}{0.400pt}}
\put(522.0,180.0){\rule[-0.200pt]{0.400pt}{0.964pt}}
\put(523,188.67){\rule{0.241pt}{0.400pt}}
\multiput(523.00,188.17)(0.500,1.000){2}{\rule{0.120pt}{0.400pt}}
\put(523.0,185.0){\rule[-0.200pt]{0.400pt}{0.964pt}}
\put(524,192.67){\rule{0.241pt}{0.400pt}}
\multiput(524.00,192.17)(0.500,1.000){2}{\rule{0.120pt}{0.400pt}}
\put(524.0,190.0){\rule[-0.200pt]{0.400pt}{0.723pt}}
\put(525,198.67){\rule{0.241pt}{0.400pt}}
\multiput(525.00,198.17)(0.500,1.000){2}{\rule{0.120pt}{0.400pt}}
\put(525.0,194.0){\rule[-0.200pt]{0.400pt}{1.204pt}}
\put(526,204.67){\rule{0.241pt}{0.400pt}}
\multiput(526.00,204.17)(0.500,1.000){2}{\rule{0.120pt}{0.400pt}}
\put(526.0,200.0){\rule[-0.200pt]{0.400pt}{1.204pt}}
\put(527,209.67){\rule{0.241pt}{0.400pt}}
\multiput(527.00,209.17)(0.500,1.000){2}{\rule{0.120pt}{0.400pt}}
\put(527.0,206.0){\rule[-0.200pt]{0.400pt}{0.964pt}}
\put(528,211){\usebox{\plotpoint}}
\put(528,215.67){\rule{0.241pt}{0.400pt}}
\multiput(528.00,215.17)(0.500,1.000){2}{\rule{0.120pt}{0.400pt}}
\put(528.0,211.0){\rule[-0.200pt]{0.400pt}{1.204pt}}
\put(529,221.67){\rule{0.241pt}{0.400pt}}
\multiput(529.00,221.17)(0.500,1.000){2}{\rule{0.120pt}{0.400pt}}
\put(529.0,217.0){\rule[-0.200pt]{0.400pt}{1.204pt}}
\put(530,227.67){\rule{0.241pt}{0.400pt}}
\multiput(530.00,227.17)(0.500,1.000){2}{\rule{0.120pt}{0.400pt}}
\put(530.0,223.0){\rule[-0.200pt]{0.400pt}{1.204pt}}
\put(531,233.67){\rule{0.241pt}{0.400pt}}
\multiput(531.00,233.17)(0.500,1.000){2}{\rule{0.120pt}{0.400pt}}
\put(531.0,229.0){\rule[-0.200pt]{0.400pt}{1.204pt}}
\put(532,240.67){\rule{0.241pt}{0.400pt}}
\multiput(532.00,240.17)(0.500,1.000){2}{\rule{0.120pt}{0.400pt}}
\put(532.0,235.0){\rule[-0.200pt]{0.400pt}{1.445pt}}
\put(533,247.67){\rule{0.241pt}{0.400pt}}
\multiput(533.00,247.17)(0.500,1.000){2}{\rule{0.120pt}{0.400pt}}
\put(533.0,242.0){\rule[-0.200pt]{0.400pt}{1.445pt}}
\put(534,253.67){\rule{0.241pt}{0.400pt}}
\multiput(534.00,253.17)(0.500,1.000){2}{\rule{0.120pt}{0.400pt}}
\put(534.0,249.0){\rule[-0.200pt]{0.400pt}{1.204pt}}
\put(535,260.67){\rule{0.241pt}{0.400pt}}
\multiput(535.00,260.17)(0.500,1.000){2}{\rule{0.120pt}{0.400pt}}
\put(535.0,255.0){\rule[-0.200pt]{0.400pt}{1.445pt}}
\put(536,267.67){\rule{0.241pt}{0.400pt}}
\multiput(536.00,267.17)(0.500,1.000){2}{\rule{0.120pt}{0.400pt}}
\put(536.0,262.0){\rule[-0.200pt]{0.400pt}{1.445pt}}
\put(537.0,269.0){\rule[-0.200pt]{0.400pt}{0.723pt}}
\put(379,239){\makebox(0,0)[r]{Ge}}
\multiput(399,239)(20.756,0.000){5}{\usebox{\plotpoint}}
\put(499,239){\usebox{\plotpoint}}
\put(140,82){\usebox{\plotpoint}}
\multiput(140,82)(20.755,0.151){7}{\usebox{\plotpoint}}
\put(285.24,83.82){\usebox{\plotpoint}}
\put(305.30,88.77){\usebox{\plotpoint}}
\put(321.58,100.58){\usebox{\plotpoint}}
\put(332.00,116.43){\usebox{\plotpoint}}
\put(339.00,134.29){\usebox{\plotpoint}}
\put(345.00,152.56){\usebox{\plotpoint}}
\put(349.00,171.66){\usebox{\plotpoint}}
\put(353.00,190.76){\usebox{\plotpoint}}
\put(357.00,209.85){\usebox{\plotpoint}}
\put(360.00,229.37){\usebox{\plotpoint}}
\put(363.00,248.88){\usebox{\plotpoint}}
\put(366.00,268.39){\usebox{\plotpoint}}
\put(366,272){\usebox{\plotpoint}}
\end{picture}

  }
  \Formelbox{
        \[ i_D = I_S ( e^\frac{u_D}{U_T} - 1 ) \]
        \[ i_D = I_S ( e^\frac{u_D-r_b i_D}{m U_T} -1 ) \]
        \[ U_T = \frac{k T}{e} = 8.6 \cdot 10^{-5} \cdot T\]
        \scriptsize $U_T(300K) = 26mV $ , $U_T(348K) = 30mV$ , $U_T(393K) = 34mV$ \normalsize
        \scriptsize \\ F�r normale Si-Diode gilt: \\
        $I_S = 10^{-12} A $ , $ r_b = 0.1 \Omega $ , $m = 1$ \normalsize
        \scriptsize \\ \\ Die vier Bereiche der Kennlinie: 
        \begin{tabbing}
          I:  \quad\= $ -0.1V < u_D < 0.1V $ \\
              \> \vspace{0.5cm} $\Rightarrow $ Diodengleichung exakt verwenden \\
          II: \> $u_D > 0.1V $ \\
              \> \vspace{0.5cm} $\Rightarrow $ Diodengleichung wird: $ i_D = | I_S| e^{\frac{u_D}{U_T}}$\\
              \> Verh�ltnis zweier Spannungen: \\
              \> \vspace{0.5cm} $ \frac{I_{D2}}{I_{D1}} = e^{\frac{U_{D2}-U_{D1}}{U_T}} \rightarrow U_{D2}- U_{D1} = U_T \ln \frac{I_{D2}}{I_{D1}} $ \\
          III:\> $ u_D < - 0.1 V $ \\
          \> $ i_D = -I_S $ oder $ i_D = | I_S |$ \\
          IV: \> Siehe Zehner-Diode 
        \end{tabbing}
        \normalsize
      }
  }
  {\Groessenbox{
    $ i_D$          & Strom durch Diode            & $ [A] $                \\
    $ u_D$          & Spannung �ber Diode          & $ [V] $                \\
    $ U_T$          & Temperatur"-spannung         & $ [V] $                \\ 
    $ T $           & Temperatur                   & $ [K] $                \\  
    $ I_S $         & S�tti"-gungs"-strom          & $ [A] $ \\
    $ I_R $         & Sperr"-strom                 & $ [A] $ \\ 
    $ r_b $         & Bahnwider"-stand             & $ [\Omega]$ \\
    $ m $           & Korrekturfakt.              & $ [1] $ \\
    $ k$            & Bolzmann"-konst. $=1.38 \cdot 10^{-23} $          & $ [\frac{Ws}{K}] $ \\
    $ e$            & Elemen"-tar"-la"-dung $ = 1.602 \cdot 10^{-19}$           & $ [As] $   \\
}}

\subsection{Differentieller Widerstand}

\label{rdiff}

\index{Diode!Differentieller Widerstand}

\Hauptbox{                                                                    
  \Bildbox{ 
    \input{Elektronik/Diode/tangente.pstex_t}
    \footnotesize F�r kleine Signale wird die Kennlinie der Diode durch eine Tangente $(= r_d)$  approximiert. \normalsize
  }
  \Formelbox{
         \[ r_d = \frac{d u_D}{d i_D} = \frac{1}{g_d} \approx \frac{U_T}{I_{D0}} \]
         \footnotesize Falls $ m=1 $ und $ r_b = 0 $ gilt: \normalsize
         \[ g_d = \frac{d i_D}{d u_D} = I_S e^{\frac{u_D}{U_T}} \frac{1}{U_T} = \frac{i_D}{U_T} \]
         \vspace{3.1cm}
      }
  }
  {\Groessenbox{
    $ r_d$          & Differentieller Widerstand   & $ [\Omega] $ \\
    $ d_d$          & Differentieller Leitwert     & $ [S] $ \\
    $ i_D$          & Strom durch Diode            & $ [A] $                \\  
    $ u_D$          & Spannung �ber Diode          & $ [V] $                \\ 
    $ U_T$          & Temperatur"-spannung         & $ [V] $                \\   
    $ m $           & Korrekturfakt.              & $ [1] $ \\
    $ I_{D0} $      & DC-Strom im Arbeitspunkt     & $ [A] $ \\
}}
\vfill

\section{DC- und AC-Analyse von Diodenschaltungen}

\index{Diode!DC-Analyse}
\index{Diode!AC-Analyse}

\subsection{Vorgehen}
\begin{enumerate}
\item Schaltung aufteilen in AC- und DC-Ersatzschltbild
\item In DC-Ersatzschaltung den Arbeitspunkt bestimmen (Konstantspannungsmodell)
\item Berechnen der dynamischen Widerst�nde im Arbeitspunkt (approximieren der Diodenkennlinie)
\item Kleinsignalanalyse (Lineare Netzwerktheorie)
\item Gesamtl�sung setzt sich aus Arbeitspunkt und Wechselstroml�sung zusammen
\end{enumerate}


\subsection{Kleinsignalanalyse}

\index{Diode!Kleinsignalanalyse}

\Hauptbox{                                                                    
  \Bildbox{ 
    \input{Elektronik/Diode/kleinsignal.pstex_t}
    \footnotesize Arbeitspunktbestimmung \normalsize
  }
  \Formelbox{
    \footnotesize 
    DC-Ersatzschaltung: Konstantspannungsmodell (siehe S. \pageref{kmod}) \\
    AC-Ersatzschaltung: Differentieller Widerstand (siehe S. \pageref{rdiff}) \\
    \\Resultierendes Gleichungssystem:
    \normalsize
    \[ \left| \begin{array}{l} i_D = \frac{U_{Q0}-u_D}{R_V} \\ i_D = I_S e^{}\frac{u_D}{U_T} \end{array} \right| \]
    \begin{tabular}{lll}
      $i_D = 0$                        & $u_D <= U_E $ & \\ 
      $i_D = \frac{1}{r_D (u_D -U_E)}$ & $u_D => U_E$ & \\
    \end{tabular}
    \[ U_E = U_{D0} -I_{D0} rd \]
    \vspace{0.2cm}
    }
  }
  {\Groessenbox{
    $ U_F$          & Flussspan.                  & $ [V] $ \\
    $ R_V$          & Vorwiderstand                  & $ [\Omega] $ \\
    $ U_{Q0}$       & Quellspan.                     & $ [V] $ \\  
    $ I_{D0}$       & Arbeitsstrom                   & $ [A] $ \\ 
    $ U_{D0}$       & Arbeitspan.                    & $ [V] $ \\ 
    $ U_T$          & Temperatur"-spannung           & $ [V] $ \\   
    $ u_D $         & Spannung �ber Diode            & $ [V] $ \\
    $ i_D $         & Strom durch Diode              & $ [A] $ \\
    $ I_S $         & S�ttigungs"-strom              & $ [A] $ \\
    $ U_E $         & Gleichspan.                    & $ [V] $ \\
}}


\subsection{Grosssignalanalyse}

\index{Diode!Grosssignalanalyse}

\Hauptbox{                                                                    
  \Bildbox{ 
    \input{Elektronik/Diode/grosssignal.pstex_t}
    \footnotesize F�r Grosssignalanalyse wird die Kennlinie durch eine Gerade durch die Punkte $ 0.1 i_{Dmax} $ und $ 0.9 i_{Dmax} $ approximiert. \normalsize
  }
  \Formelbox{
    \[ U_E = u_D(0.1 I_{Dmax}) - 0.1 I_{Dmax} r_F  \]
    \[ r_F = \frac{\Delta u_D}{0.8 I_{Dmax}} \] 
    }  
  }
  {\Groessenbox{
    $ u_D$          & Spannung �ber Diode          & $ [V] $       \\  
    $ i_D$          & Strom durch Diode            & $ [A] $       \\ 
    $ r_F$          & Diodenwider"-stand             & $ [\Omega] $  \\   
}}
\vfill

\section{Z-Dioden}

\index{Diode!Z-Diode}
\index{Z-Diode}
\index{Diode!Temperaturkoeffizient}

\Hauptbox{                                                                    
  \Bildbox{ 
    \input{Elektronik/Diode/zdiode.pstex_t}
  }
  \Formelbox{        
         \[ r_Z = \frac{d u_Z}{d i_Z} \]
         \footnotesize Teperaturkoeffizient: \normalsize
         \[ \alpha = \frac{\frac{dU_Z}{dT}}{U_Z} \] 
         \footnotesize
         $ \alpha < 0 $ bei $ U_Z < 5.6V $ \\
         $ \alpha \approx 0 $ bei $ U_Z \approx 5.6V $ \\
         $ \alpha > 0 $ bei $ U_Z > 5.6V $ \\ 
         \normalsize
         \footnotesize
         Temperaturkompensation durch Serieschaltung: $ \alpha_1 U_{Z1} = - \alpha_2 U_{Z2} $
         \normalsize
      }
  }
  {\Groessenbox{
    $ r_Z$          & Z-Widerstand                 & $ [\Omega] $ \\
    $ r_d$          & Differentieller Widerstand   & $ [\Omega] $ \\
    $ U_Z$          & Zehnersp.                    & $ [V] $ \\
    $ U_F$          & Flusssp.                  & $ [V] $                \\  
    $ T $           & Temperatur                   & $ [K] $ \\
}}


\subsection{Z-Dioden zur Spannungsstabilisierung}

\index{Diode!Spannungsstabilisierung}
\index{Spannungsstabilisierung}

\Hauptbox{                                                                    
  \Bildbox{ 
    \input{Elektronik/Diode/spannungsstab.pstex_t}
  }
  \Formelbox{        
         \[ I_{totmin} = \frac{U_{Smin}-U_{outmin}}{R} \]
         \[ I_{totmax} = \frac{U_{Smax}-U_{outmin}}{R} \]
         \[ I_{outmin} = I_{totmin} - I_{Lmax} \] 
         \[ I_{outmax} = I_{totmax} - I_{Lmin} \]
         \[ P_{Zmax} = U_{outnom} I_{outmax} \]
         \footnotesize Rippelunterdr�ckung: \normalsize 
         \[ u_{out} = u_S \frac{r_Z \| R_L}{R + (r_Z \| R_L)} \]
      }
  }
  {\Groessenbox{
    $ I_{tot} $     & I-Eingang                  & $ [A] $ \\
    $ U_S $         & Speisesp.                  & $ [V] $ \\
    $ I_L$          & Laststrom                  & $ [A] $ \\
    $ U_{out}$      & Ausgangssp.                & $ [V] $ \\  
    $ R $           & Vorwiderstand              & $ [\Omega] $ \\
    $ R_L $         & Lastwiderstand             & $ [\Omega] $ \\
    $ P_Z$          & P-Verslust                 & $ [W] $ \\
    $ r_Z$          & Differentieller Widerstand & $ [\Omega] $ \\ 
    $ u_S$          & Rippel am Eingang          & $ [V] $\\
    $ u_{out} $     & Rippel am Ausgang          & $ [V] $ \\
}}


%%% Local Variables: 
%%% mode: latex
%%% TeX-master: "../../FoSaHSR"
%%% End: 